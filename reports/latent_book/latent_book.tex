\documentclass[12pt,a4paper]{article}

\usepackage{michel}
\usepackage[dcucite,abbr]{harvard}
\harvardparenthesis{none}\harvardyearparenthesis{round}
\usepackage{hyperref}
\usepackage{varioref}
\usepackage{longtable}
\usepackage{siunitx}
\sisetup{
  parse-numbers=false,      % Prevents automatic parsing (needed for parentheses & superscripts)
  detect-inline-weight=math,% Ensures proper formatting in tables
  tight-spacing=true        % Keeps spacing consistent
}
% Package to include code
\usepackage{listings}
\usepackage{color}
\lstset{language=Python}
\lstset{numbers=none, basicstyle=\footnotesize,
  numberstyle=\tiny,keywordstyle=\color{blue},stringstyle=\ttfamily,showstringspaces=false}
\lstset{backgroundcolor=\color[rgb]{0.95 0.95 0.95}}
\lstdefinestyle{numbers}{numbers=left, stepnumber=1,
  numberstyle=\tiny,basicstyle=\tiny, numbersep=10pt}
\lstdefinestyle{nonumbers}{numbers=none}
\lstset{
  breaklines=true,
  breakatwhitespace=true,
}

\title{Estimating hybrid choice models  with Biogeme}
\author{Michel Bierlaire \and Moshe Ben-Akiva \and Joan Walker} 
\date{\today}


\begin{document}


\begin{titlepage}
\pagestyle{empty}

\maketitle
\vspace{2cm}

\begin{center}
\small Report TRANSP-OR xxxxxx  \\ Transport and Mobility Laboratory \\ School of Architecture, Civil and Environmental Engineering \\ Ecole Polytechnique F\'ed\'erale de Lausanne \\ \verb+transp-or.epfl.ch+
\begin{center}
\textsc{Series on Biogeme}
\end{center}
\end{center}


\clearpage
\end{titlepage}

\begin{titlepage}
\tableofcontents
\end{titlepage}

The package Biogeme (\texttt{biogeme.epfl.ch}) is designed to estimate
the parameters of various models using maximum likelihood
estimation. It is particularly designed for discrete choice
models.  In this document, we present how to estimate  choice
models involving latent variables: hybrid choice models.

We assume that the reader is already familiar with discrete choice
models, and has successfully installed Biogeme. This document has
been written using Biogeme 3.3.2.



\section{Models and notations}
\label{sec:model}
The literature on discrete choice models with latent variables is vast
(\cite{walker2001extended}, \cite{ashok2002extending},
\cite{greene2003latent}, \cite{ben2002integration}, to cite just a
few). We start this document by a short introduction to the models and
the notations. 

\subsection{Structural equations}
A \emph{latent variable} is a variable that cannot be directly
observed. It is typically modeled using a \textbf{structural
  equation}, which expresses the latent variable as a function of
observed (explanatory) variables and an error term. A general form of
such a structural equation is:
\begin{equation}
\label{eq:structural}
x_{nk}^* = x^*(x_n; \psi_k) + \omega_{nk},
\end{equation}
where $n$ indexes individuals, $x_{nk}^*$ is the $k$th latent variable of interest, $x_n$ is a vector of observed explanatory variables, $\psi_k$ is a vector of parameters to be estimated, and $\omega_{nk}$ is a stochastic error term.

A common specification assumes a linear functional form   with i.i.d. normally distributed  error terms:
\begin{equation}
\label{eq:linearStructural}
 x_{nk}^* = \sum_s \psi_{sk} x_{ns} + \sigma_{\omega k} \omega_{nk},
\end{equation}
where $\omega_{nk} \sim N(0, 1)$, and $\sigma_{\omega k}$ is a scaling parameter for the error term. 

Information about latent variables is obtained indirectly through
\emph{measurements}, which are observable manifestations of the
underlying latent constructs. For example, in discrete choice models,
utility is not directly observed but is inferred from the choices
individuals make. The relationship between a latent variable and its
associated measurements is described by \textbf{measurement
  equations}. The specific form of these equations depends on the
nature of the observed measurements (e.g., continuous, or ordinal).


\subsection{Measurement equations: the continuous case}
\label{sec:continuous}

Since latent variables cannot be directly observed, analysts rely on
indirect measurements to infer their values. A common approach
involves asking respondents to rate the perceived magnitude of the
latent construct on an arbitrary scale. For example: \emph{``How would
you rate the level of pain that you are experiencing, from 0 (no pain)
to 10 (worst pain imaginable)?''}

Each such rating is referred to as an \emph{indicator}, indexed by
$\ell=1, \ldots, L_n$, and is modeled using a \textbf{measurement equation}. This
equation relates the observed indicator to the latent variables and, sometimes,
other explanatory variables:
\begin{equation}
\label{eq:continuousMeasurement}
I_{n\ell} = I_\ell(x_n, x_n^*;\lambda_\ell) + \upsilon_{n\ell}, \; \forall \ell=1, \ldots, L_n, \forall n,
\end{equation}
where $I_{n\ell}$ denotes the response provided by individual $n$ for
indicator $\ell$, $x_n^*$ is the latent variable of interest (e.g., pain
perception), $x_n$ is a vector of observed explanatory variables (such as
socio-demographic characteristics), $\lambda_\ell$ is a vector of
parameters to be estimated, and $\upsilon_{n\ell}$ is the random error term.

A common specification of the measurement function assumes linearity
and i.i.d. normally distributed errors:
\begin{equation}
\label{eq:linearMeasurement}
I_{n\ell} = \lambda_{\ell 0} + \sum_k \lambda_{\ell k}  x_{nk}^* + \sigma_{\upsilon \ell} \upsilon_{n\ell}, \quad \forall \ell,
\end{equation}
where $\lambda_{\ell k}$ are unknown parameters to be estimated, $\sigma_{\upsilon \ell}$ is an indicator-specific scale parameter, and $\upsilon_{n\ell} \sim N(0, 1)$.

If we observe a vector of continuous indicators \( I_n = (I_{n1}, \ldots, I_{nL_n}) \) for individual \( n \), the contribution to the likelihood function, \emph{conditional on the latent variables} \( x^*_{n} \), is given by the product:
\begin{equation}
  \label{eq:conditional_likelihood_continuous}
\prod_{\ell=1}^{L_n} \phi\left( \frac{I_{n\ell} - \lambda_{\ell 0} - \sum_k \lambda_{\ell k} x_{nk}^*}{\sigma_{\upsilon \ell}} \right),
\end{equation}
where \( \phi(\cdot) \) denotes the probability density function (pdf) of the standard normal distribution.

If other types of observations are available for the same individual (such as discrete choices), the corresponding components of the likelihood can be multiplied with the expression above. Once all relevant components are combined, the latent variables must be integrated out, as discussed later.

If the continuous indicators are the only data available for individual \( n \), the contribution to the unconditional likelihood becomes:
\begin{equation}
  \label{eq:likelihood_continuous}
\int_{x_n^*} \left[ \prod_{\ell=1}^{L_n} \phi\left( \frac{I_{n\ell} - \lambda_{\ell 0} - \sum_k \lambda_{\ell k} x_{nk}^*}{\sigma_{\upsilon \ell}} \right) \right] f(x_n^*) \, dx_n^*,
\end{equation}
where \( f(x_n^*) \) is the pdf of the vector of latent variables \( x_n^* \).
As this integral does not have a closed-form expression, it is approximated using Monte Carlo integration (see \cite{Bier19} for a discussion about performing Monte-Carlo integration with Biogeme).




\subsection{Measurement equation: the ordinal case}
\label{sec:likert}

Another type of indicator arises when respondents are asked to
evaluate a statement using an ordinal scale. A typical context for
this type of measurement is the use of a Likert scale
(\cite{likert1932technique}), where individuals express their degree of
agreement or disagreement with a given statement. For example:
\begin{quote}
\emph{``I believe that my own actions have an impact on the planet.''} \\
Response options: strongly agree (2), agree (1), neutral (0), disagree ($-1$), strongly disagree ($-2$).
\end{quote}


To model these types of indicators, we represent the observed
measurement as an \emph{ordered discrete variable} $I_{n\ell}$, which
takes values in a finite, ordered set $\{j_1, j_2, \ldots, j_{M_\ell}\}$. The
measurement equation involves two stages:

\paragraph{Step 1: Latent response formulation.} We first define a continuous response variable, as explained in Section~\ref{sec:continuous}, except that it happens to be unobserved (latent) in this case:
\begin{equation}
I^*_{n\ell} = I^*_\ell(x_n, x_n^*; \lambda_\ell) + \upsilon_{n\ell},
\end{equation}
where $I^*_{n\ell}$ is a continuous latent variable underlying the
reported response, $x_n^*$ is a vector of  relevant latent variables (e.g.,
environmental concern), $x_n$ is a vector of observed explanatory
variables (e.g., age, income), $\lambda$ is a vector of parameters to
be estimated, and $\upsilon_{n\ell}$ is a random error term.

\paragraph{Step 2: Discretization via thresholds.} Since $I^*_{n\ell}$ is not observed, we relate it to the reported discrete measurement $I_{n\ell}$ through a set of threshold parameters:
\begin{equation}
\label{eq:discreteMeas-b}
I_{n\ell} = \left\{
\begin{array}{ll}
j_1 & \text{if } I^*_{n\ell} < \tau_1, \\
j_2 & \text{if } \tau_1 \leq I^*_{n\ell} < \tau_2, \\
\vdots \\
j_m & \text{if } \tau_{m-1} \leq I^*_{n\ell} < \tau_m, \\
\vdots \\
j_M & \text{if } \tau_{M_\ell-1} \leq I^*_{n\ell},
\end{array}
\right.
\end{equation}
where $\tau_1, \ldots, \tau_{M_\ell-1}$ are threshold parameters to be estimated, satisfying the ordering constraint:
\begin{equation}
\label{eq:discreteMeas-c}
\tau_1 \leq \tau_2 \leq \cdots \leq \tau_{M_\ell-1}.
\end{equation}
Note that it is customary to use the same set of parameters for all
individuals $n$ and all indicators $\ell$, which explains the absence
of these indices on the parameter $\tau$.

Defining $\tau_0=-\infty$ and $\tau_{M_\ell}=+\infty$, it simplifies to
\begin{equation}
  \label{eq:prob_indicator}
I_{n\ell} = j_m  \text{ if } \tau_{m-1} \leq I^*_{n\ell} < \tau_m, \; m=1, \ldots, M_\ell. 
\end{equation}

It is often advantageous to impose a symmetric structure on the
definition of the thresholds. In addition, it is more convenient from
an estimation standpoint to parameterize the thresholds in terms of
differences and to constrain these differences to be positive. For
example, when \( M_\ell = 4 \), the thresholds can be defined as
follows:
\[
\begin{aligned}
  \tau_1 &= -\delta_1 -\delta_2, \\
  \tau_2 &= -\delta_1, \\
  \tau_3 &= \phantom{-}\delta_1, \\
  \tau_4 &= \phantom{-}\delta_1 + \delta_2,
\end{aligned}
\]
where \( \delta_1 > 0 \) and \( \delta_2 > 0 \) are the parameters to
be estimated.
This parameterization guarantees that the thresholds are strictly ordered 
and symmetrically centered around zero, which facilitates both 
identification and interpretation. 


If we consider a linear specification,
\begin{equation}
  \label{eq:measurement_latent}
I^*_{n\ell} = \lambda_{\ell 0}  + \sum_k \lambda_{\ell k}  x_{nk}^* + \sigma_{\upsilon \ell} \upsilon_{n\ell}, \quad \forall \ell,
\end{equation}
where the error term $\upsilon_{n\ell} \sim N(0,1)$,
the contribution of each indicator $\ell$ for each observation $n$ to the likelihood function, \emph{conditional on the latent variables}, is defined as follows:
\begin{equation}
  \label{eq:discreteMeas-d}
  \begin{aligned}
    \prob(I_{n\ell} = j_m| x^*_n, x_n ; \lambda_\ell, \Sigma_{\upsilon \ell})
    =& \prob(\tau_{m-1} \leq I^*_{n\ell} \leq \tau_m) \\
    =& \prob(I^*_{n\ell} \leq \tau_m) -  \prob(I^*_{n\ell} \leq \tau_{m-1}),\\
    =& \prob\left( \upsilon_{n\ell} \leq \frac{\tau_m -\lambda_{\ell 0}  - \sum_k \lambda_{\ell k}  x_{nk}^*}{\sigma_{\upsilon \ell}} \right) \\
    &-  \prob \left( \upsilon_{n\ell} \leq \frac{\tau_{m-1}- \lambda_{\ell 0}  - \sum_k \lambda_{\ell k}  x_{nk}^*}{\sigma_{\upsilon \ell}}\right), \\
    =& \Phi\left(  \frac{\tau_m -\lambda_{\ell 0}  - \sum_k \lambda_{\ell k}  x_{nk}^*}{\sigma_{\upsilon \ell}} \right) \\
    &-\Phi\left(\frac{\tau_{m-1}- \lambda_{\ell 0}  - \sum_k \lambda_{\ell k}  x_{nk}^*}{\sigma_{\upsilon \ell}}\right)
  \end{aligned}
\end{equation}
where $j_m$ is the observed category for respondent $n$ and indicator $\ell$.

This specification is known as the \emph{ordered probit model} and is
widely used for modeling ordinal responses that depend on latent
constructs.

If we observe a vector of continuous indicators \( I_n = (I_{n1}, \ldots, I_{nL_n}) \) for individual \( n \), the contribution to the likelihood function, \emph{conditional on the latent variables} \( x^*_{n} \), is given by:
\begin{equation}
\prod_{\ell=1}^{L_n} \prob(I_{n\ell} = j_m| x^*_n, x_n ; \lambda_\ell, \Sigma_{\upsilon \ell}).
\end{equation}

As in the continous case, if other types of observations are available
for the same individual (such as  choices), the corresponding
components of the likelihood can be multiplied with the expression
above. Once all relevant components are combined, the latent variables
must be integrated out, as discussed later.

If the continuous indicators are the only data available for individual \( n \), the contribution to the unconditional likelihood becomes:
\begin{equation}
\int_{x_n^*} \left[ \prod_{\ell=1}^{L_n} \prob(I_{n\ell} = j_m| x^*_n, x_n ; \lambda_\ell, \Sigma_{\upsilon \ell})
\right] f(x_n^*) \, dx_n^*,
\end{equation}
where \( f(x_n^*) \) is the pdf of the vector of latent variables \( x_n^* \).
Again, this integral is approximated using Monte-Carlo integration.


\subsection{Normalization and identification in latent-variable models}
\label{sec:normalization}

Models with latent variables involve parameters that are not directly
identified from the data without additional restrictions. These
restrictions, known as \emph{normalizations}, are not substantive
assumptions about behavior; rather, they fix the arbitrary units of
measurement (location and scale) that are inherent to latent constructs.
This section explains \emph{why} normalization is needed, \emph{where}
non-identification arises in our notation, and provides practical
guidelines. We emphasize the \emph{reference-indicator} strategy because
it yields parameters with a direct interpretation and tends to be
numerically stable, although it is not the only valid approach.

Latent-variable models are invariant to certain transformations of the
latent variables and associated parameters. Without normalization,
multiple parameter vectors generate exactly the same probability of the
observed data, hence the same likelihood. In such a case, maximization is
ill-posed (flat directions), standard errors are meaningless, and
numerical optimization or Bayesian sampling may become unstable.

Two systematic sources of non-identification arise:

\begin{itemize}
  \item \textbf{Location (translation) invariance}: the origin of a latent
  variable is arbitrary.
  \item \textbf{Scale invariance}: the unit (measurement scale) of a latent
  variable is arbitrary.
\end{itemize}

Both must be addressed to obtain a well-identified model.

We first consider the continuous measurement equation
\eqref{eq:linearMeasurement}:
\[
I_{n\ell} = \lambda_{\ell 0} + \sum_k \lambda_{\ell k} x_{nk}^*
+ \sigma_{\upsilon \ell}\upsilon_{n\ell},
\qquad \upsilon_{n\ell}\sim\mathcal N(0,1).
\]
The structural equation is \eqref{eq:linearStructural}:
\[
x_{nk}^* = \sum_s \psi_{sk}x_{ns}+\sigma_{\omega k}\omega_{nk},
\qquad \omega_{nk}\sim\mathcal N(0,1).
\]

Suppose that the intercept $\lambda_{\ell 0}$ is estimated for at least
one indicator that loads on latent variable $k$. Then, for any constant
$c_k$, define
\[
x_{nk}^{*'} = x_{nk}^* + c_k.
\]
In the measurement equation, the term $\lambda_{\ell k}x_{nk}^*$ can be
rewritten as $\lambda_{\ell k}(x_{nk}^{*'}-c_k)$, which is the same as
keeping $x_{nk}^{*'}$ but shifting the intercept:
\[
\lambda_{\ell 0}' = \lambda_{\ell 0}-\lambda_{\ell k}c_k.
\]
Therefore, the distribution of $I_{n\ell}$ (and hence the likelihood)
is unchanged: the model cannot distinguish a shift in the latent variable
from a compensating shift in intercepts. This is why the \emph{origin} of
each latent variable must be fixed.

Also, for any nonzero constant $a_k$, define
\[
x_{nk}^{*'} = a_k x_{nk}^*.
\]
Then $\lambda_{\ell k}x_{nk}^* = (\lambda_{\ell k}/a_k)x_{nk}^{*'}$.
Hence, scaling the latent variable can be compensated by inversely
scaling all associated loadings. On the structural side, scaling $x_{nk}^*$
can be compensated by scaling $\sigma_{\omega k}$ and the coefficients
$\psi_{sk}$. Again, the likelihood is unchanged: the model cannot infer
the \emph{unit} in which $x_{nk}^*$ is measured unless we fix it.

Consequently, for each latent variable $k$, the analyst must impose
\emph{one} location normalization and \emph{one} scale normalization.
These two restrictions remove two degrees of freedom per latent
variable: one for translation, one for rescaling.

A practical and interpretable approach is to choose, for each latent
variable $k$, a \emph{reference indicator} $\ell(k)$ among the indicators
intended to measure $x_{nk}^*$. The key idea is to anchor the latent
variable to a concrete observed scale.

First, we fix the location by fixing one intercept.
For each $k$, impose
\[
\boxed{\lambda_{\ell(k)0}=0.}
\]
Because $\lambda_{\ell 0}$ can absorb any shift of
$x_{nk}^*$, setting the intercept of one measurement equation to zero
declares where the latent variable ``zero'' is, in a way that is tied to
an observed indicator.

Second, fix the scale by fixing one loading.
For each $k$, impose
\[
\boxed{\lambda_{\ell(k)k}=1.}
\]
Because the product $\lambda_{\ell k}x_{nk}^*$ is all that
matters, fixing $\lambda_{\ell(k)k}=1$ defines the unit of $x_{nk}^*$ as
the unit that makes the reference indicator respond one-for-one (in its
systematic part) to changes in the latent variable.

Importantly, fixing the loading to $+1$ is a \emph{convention}. Since the sign of a latent variable is itself
arbitrary, it may be equally appropriate to fix
$\lambda_{\ell(k)k}=-1$. This choice simply reverses the orientation of
the latent scale, without affecting the likelihood or the model fit.
The advantage of using $-1$ arises when the natural interpretation of
the reference indicator runs opposite to the intended interpretation of
the latent construct.

For instance, suppose $x_{nk}^*$ represents \emph{environmental
concern}, with higher values intended to mean \emph{stronger concern}.
Consider a reference indicator based on agreement with the statement
``I do not care about the environmental impact of my travel,'' coded so
that higher responses correspond to stronger agreement. In this case,
higher values of the indicator imply \emph{lower} environmental concern.
Fixing $\lambda_{\ell(k)k}=+1$ would force increases in $x_{nk}^*$ to
increase the indicator, leading to a latent variable whose numerical
interpretation runs counter to its conceptual meaning. By fixing
$\lambda_{\ell(k)k}=-1$ instead, higher values of $x_{nk}^*$ decrease the
expected indicator response, restoring a coherent and intuitive
interpretation: larger $x_{nk}^*$ corresponds to stronger environmental
concern.

Thus, choosing $\lambda_{\ell(k)k}=1$ or $-1$ does not affect
identification or statistical fit, but it greatly improves the semantic
consistency and interpretability of the model parameters.

With $\lambda_{\ell(k)0}=0$ and $\lambda_{\ell(k)k}=1$ fixed, it is
meaningful to estimate $\sigma_{\upsilon\ell(k)}$: it quantifies the
amount of noise in that indicator relative to the anchored latent scale.
In other words, it is now a genuine signal-to-noise parameter rather than
a normalization device.

If the model contains multiple latent variables, the reference indicator
for $k$ should ideally load only on $k$:
\[
\lambda_{\ell(k)h}=0 \quad \text{for } h\neq k.
\]
Inded, if the reference indicator mixes several latent variables,
then fixing a single loading to one does not define a clean unit for a
single construct; it anchors a linear combination instead, which
complicates interpretation and may reintroduce weak identification.

We now consider ordinal indicators (Likert-type) modeled through an
ordered probit mechanism (Section~\ref{sec:likert}). The essential
difference with the continuous case is that the observed indicator
$I_{n\ell}$ is not the latent response $I^*_{n\ell}$ itself, but only
the interval in which it lies:
\[
I_{n\ell}=j_m \quad \text{if} \quad
\tau_{m-1}\le I_{n\ell}^*<\tau_m,
\qquad m=1,\ldots,M_\ell,
\]
with $\tau_0=-\infty$ and $\tau_{M_\ell}=+\infty$.

The latent response is given by
\[
I^*_{n\ell}
= \lambda_{\ell 0}
+ \sum_k \lambda_{\ell k}x_{nk}^*
+ \sigma_{\upsilon\ell}\upsilon_{n\ell},
\qquad \upsilon_{n\ell}\sim\mathcal N(0,1).
\]

What changes compared to the continuous case? In the ordered probit likelihood \eqref{eq:discreteMeas-d}, probabilities
depend only on \emph{standardized differences}
\[
\frac{\tau_m
- \lambda_{\ell 0}
- \sum_k \lambda_{\ell k}x_{nk}^*}
{\sigma_{\upsilon\ell}}.
\]
As a consequence, the ordered probit layer introduces \emph{two additional
invariances} that must be addressed explicitly:

\begin{itemize}
  \item a \emph{location invariance}: adding the same constant to
  $I^*_{n\ell}$ and to all thresholds leaves probabilities unchanged;
  \item a \emph{scale invariance}: multiplying $I^*_{n\ell}$,
  all thresholds, and $\sigma_{\upsilon\ell}$ by the same positive
  constant leaves probabilities unchanged.
\end{itemize}

These invariances are specific to ordinal models and are \emph{in
addition} to the invariances already present in the latent-variable
structure discussed in the continuous case.

We adopt the same reference-indicator philosophy as in the continuous
case. For each latent variable $k$, one ordinal indicator $\ell(k)$ is
chosen as its reference indicator.

\begin{itemize}
\item Step 1: anchor the latent variable.
Exactly as in the continuous case, we fix
\[
\boxed{
\lambda_{\ell(k)0}=0,
\qquad
\lambda_{\ell(k)k}=1
\quad (\text{or } -1).
}
\]
This fixes the \emph{location} and \emph{orientation} of the latent
variable $x_{nk}^*$ and defines its unit through the response of the
reference indicator. This step is conceptually identical to the
continuous case and should always be applied first.

\item Step 2: fix the location of the threshold system.
Because only differences $\tau_m-I^*_{n\ell}$ matter, the threshold
system has an arbitrary origin. This must be fixed \emph{once}, and only
once.

There are two principled ways to do so. For non-symmetric thresholds, fix one threshold to zero,
  for example $\tau_c=0$. This is a pure location normalization. For symmetric thresholds, impose symmetry around zero.   In this case, location is fixed by construction.

When the Likert scale is designed to be symmetric (which is typically
the case), the symmetric parameterization is recommended, because it
aligns the statistical model with the semantics of the survey scale and
reduces the number of free parameters.

Note that, in  the symmetric case, it is advised to re-parametrize the model.
Let $M_\ell$ be the number of ordered categories. There are
$M_\ell-1$ finite thresholds.
When $M_\ell$ is odd (e.g., 5-point Likert), there is a natural central category. For
$M_\ell=5$, there are four thresholds. A symmetric parameterization is:
\[
\boxed{
\begin{aligned}
\tau_1 &= -(\delta_1+\delta_2),\\
\tau_2 &= -\delta_1,\\
\tau_3 &= \phantom{-}\delta_1,\\
\tau_4 &= \phantom{-}(\delta_1+\delta_2),
\end{aligned}
\qquad
\delta_1>0,\ \delta_2>0.
}
\]
This construction:
\begin{itemize}
  \item enforces strict ordering automatically;
  \item fixes the location (centered at zero);
  \item reduces dimensionality (two parameters instead of four).
\end{itemize}

When $M_\ell$ is even, there is no central category. In this case,
$M_\ell-1$ is odd, and symmetry necessarily implies that one threshold
lies exactly at zero. For $M_\ell=4$:
\[
\boxed{
\tau_1=-\delta_1,
\qquad
\tau_2=0,
\qquad
\tau_3=\delta_1,
\qquad
\delta_1>0.
}
\]
Here again, location is fixed by symmetry, and ordering is guaranteed.

\item Step 3: fix the scale of the ordered probit layer.
In ordered probit models, scale is not identified because
\[
(\tau_m,\ I^*_{n\ell},\ \sigma_{\upsilon\ell})
\mapsto
(a\tau_m,\ a I^*_{n\ell},\ a\sigma_{\upsilon\ell})
\]
leaves probabilities unchanged. Therefore, one scale normalization is
required \emph{within the ordinal layer}.

To remain consistent with the reference-indicator philosophy adopted
for latent variables, we fix
 the scale of the latent variable  by
  $\lambda_{\ell(k)k}=1$; and  the scale of the ordinal response by fixing
  $\sigma_{\upsilon\ell}=1$ for ordinal indicators.

This choice has three advantages:
\begin{enumerate}
  \item it follows the standard ordered probit convention;
  \item it avoids entangling threshold spacings with noise variance;
  \item it preserves a clear interpretation of thresholds as cutpoints
  on a standardized latent response scale.
\end{enumerate}
\end{itemize}

In summary, for an ordinal indicator $\ell$ measuring latent variable $k$, the
recommended normalization strategy is:

\begin{enumerate}
  \item choose $\ell(k)$ as reference indicator;
  \item fix $\lambda_{\ell(k)0}=0$ and $\lambda_{\ell(k)k}=\pm1$;
  \item impose an ordered threshold parameterization;
  \item fix threshold location (preferably via symmetry);
  \item fix $\sigma_{\upsilon\ell}=1$.
\end{enumerate}

These steps jointly remove all location and scale indeterminacies,
without redundancy, and yield a model whose parameters have a clear,
stable, and interpretable meaning.


%================================================================
\section{Hybrid choice models}
\label{sec:hcm}
%================================================================

This section builds on the notation and model components introduced in
Section~\ref{sec:model}. We combine the structural equations for the
latent variables, the measurement
equations for the indicators (continuous or ordinal), and  a
discrete choice model, into two frameworks of increasing scope: the
\emph{MIMIC} model and the \emph{hybrid choice model}.

We use the same notations as in Section~\ref{sec:model}: $n$ indexes individuals, $x_n$ denotes the
observed explanatory variables, $x_n^*$ the vector of latent
variables, $I_n$ the vector of observed indicators, and
$i_n\in\mathcal C_n$ the observed choice.

%----------------------------------------------------------------
\subsection{The MIMIC model}
\label{sec:mimic}
%----------------------------------------------------------------

A \emph{Multiple Indicators Multiple Causes} (MIMIC) model is obtained by
combining:
\begin{itemize}
  \item the structural part (``multiple causes''), which explains
  each latent variable as a function of observed covariates through the
  structural equations \req{eq:linearStructural}; and
  \item the measurement part (``multiple indicators''), which
  explains each indicator as a function of the latent variables through
  measurement equations (continuous indicators: \req{eq:linearMeasurement};
  ordinal indicators: ordered probit, \req{eq:discreteMeas-d}).
\end{itemize}

The purpose of the MIMIC model is to infer latent variables from their
observable manifestations (the indicators) while simultaneously
explaining how these latent constructs vary with observed covariates.

For a given individual $n$, the contribution of the indicators to the
likelihood \emph{conditional on the latent variables} $x_n^*$
is obtained by multiplying the indicator-specific contributions:
\begin{equation}
  \label{eq:mimic_conditional_indicator_likelihood}
  L_n^{I}\!\left(I_n \mid x_n^*,x_n\right)
  \;=\;
  \prod_{\ell=1}^{L_n}
  \begin{cases}
    \text{density of Eq.~\eqref{eq:conditional_likelihood_continuous}}, &
    \text{if indicator $\ell$ is continuous},\\
    \text{probability of Eq.~\eqref{eq:discreteMeas-d}}, &
    \text{if indicator $\ell$ is ordinal}.
  \end{cases}
\end{equation}

Because $x_n^*$ is not observed, the individual likelihood
contribution integrates the conditional indicator likelihood over the
distribution of the latent variables implied by the structural equations:
\begin{equation}
  \label{eq:mimic_likelihood}
  L_n^{\text{MIMIC}}
  \;=\;
  \int_{x_n^*}
  L_n^{I}\!\left(I_n \mid x_n^*,x_n\right)\;
  f(x_n^*\mid x_n)\;
  dx_n^*,
\end{equation}
where $f(x_n^*\mid x_n)$ is the conditional
density implied by Eq.~\eqref{eq:linearStructural}.

%----------------------------------------------------------------
\subsection{Hybrid choice model (choice model with latent variables)}
\label{sec:hybrid_choice_model}
%----------------------------------------------------------------

A \emph{hybrid choice model} extends the MIMIC model by adding a discrete
choice component in which latent variables enter the systematic utilities.
The key modeling idea is that attitudes or perceptions (latent variables)
may influence choices, while being measured only indirectly through the
indicators.

Let $V_{in}(x_n,x_n^*;\beta)$ denote the
systematic utility of alternative $i$ for individual $n$. The probability of the
observed choice $i_n$ conditional on the latent variables is
\begin{equation}
  \label{eq:hcm_choice_conditional}
  P(i_n \mid x_n^*,x_n;\beta)
  \;=\;
  \frac{\exp\!\left(\mu\,V_{in}(x_n,x_n^*;\beta)\right)}
  {\sum_{j\in\mathcal C_n}\exp\!\left(\mu\,V_{jn}(x_n,x_n^*;\beta)\right)},
\end{equation}
where $\mu$ is the scale parameter of the logit model and $\beta$ is the
vector of utility parameters. Note that the logit model is adopted here for expositional convenience and because it is a widely used specification. However, the framework is fully general and can accommodate alternative discrete choice models, such as nested logit or cross-nested logit models, without any conceptual modification.



For a given individual $n$, the full observation is
$(i_n,I_n)$. Conditional on $x_n^*$, the hybrid
choice model contribution to the likelihood is the product of:
\begin{itemize}
  \item the conditional choice probability from Eq.~\eqref{eq:hcm_choice_conditional},
  \item the conditional indicator likelihood from
  Eq.~\eqref{eq:mimic_conditional_indicator_likelihood}.
\end{itemize}
That is,
\begin{equation}
  \label{eq:hcm_joint_conditional_likelihood}
  L_n\!\left(i_n,I_n \mid x_n^*,x_n\right)
  \;=\;
  P(i_n\mid x_n^*,x_n;\beta)\;
  L_n^{I}\!\left(I_n \mid x_n^*,x_n\right).
\end{equation}

Because $x_n^*$ is latent, the individual likelihood
contribution integrates \eqref{eq:hcm_joint_conditional_likelihood} over
$f(x_n^*\mid x_n)$ implied by the structural
equations \req{eq:linearStructural}:
\begin{equation}
  \label{eq:hcm_likelihood}
  L_n^{\text{HCM}}
  \;=\;
  \int_{x_n^*}
  P(i_n\mid x_n^*,x_n;\beta)\;
  L_n^{I}\!\left(I_n \mid x_n^*,x_n\right)\;
  f(x_n^*\mid x_n)\;
  dx_n^*.
\end{equation}

Let $\theta$ denote the full parameter vector, including the parameters
of the choice model, structural equations, and measurement equations (and
thresholds for ordinal indicators). The sample log-likelihood is
\begin{equation}
  \label{eq:hcm_loglikelihood}
  \begin{aligned}
  \LL(\theta) & =
  \sum_n \ln L_n^{\text{HCM}} \\
  &=
  \sum_n \ln \left[
  \int_{x_n^*}
  P(i_n\mid x_n^*,x_n;\beta)\;
  L_n^{I}\!\left(I_n \mid x_n^*,x_n\right)\;
  f(x_n^*\mid x_n)\;
  dx_n^*
  \right].
  \end{aligned}
\end{equation}

The integral in \eqref{eq:hcm_loglikelihood} typically has no closed form
and is evaluated numerically, most often by Monte-Carlo integration.

\section{A case study}
\label{sec:example}
This example focuses on the estimation of a mode choice model for
residents of Switzerland, using revealed preference data. The data
were collected as part of a research project aimed to assess the market potential
of combined mobility solutions --- particularly in urban agglomerations --- by
identifying the factors that influence individuals in their choice of
transport mode (\cite{OptimaRP2011}).

The survey was conducted between 2009 and 2010 on behalf of CarPostal,
the public transport operator of the Swiss Postal Service. Its primary
objective was to collect data on travel behavior in low-density areas,
which represent the typical service environment of CarPostal. In
addition to revealed preference data, the survey includes several
psychometric indicators, enabling the incorporation of latent
variables into the model specification.

The data file as well as its description is available on the \href{http://biogeme.epfl.ch/#data}{Biogeme webpage}. A description of the variables is also available in Appendix~\ref{sec:variables}.

We consider a  model involving two latent variables. The
first one captures a ``car-centric'' attitude. The second one captures
an ``environmental attitude''. The car-centric attitude captures the
extent to which individuals exhibit a strong preference for private
car use as their primary mode of transportation. This latent construct
reflects values such as independence, flexibility, comfort, and
perceived status associated with driving. Individuals with a high
car-centric attitude are more likely to perceive cars as the most
practical and desirable means of travel, often resisting modal shift
to public transport or active mobility.
The environmental attitude represents the degree to which individuals
value environmental protection and sustainability in their mobility
choices. It reflects concerns about issues such as climate change,
air pollution, energy consumption, and the broader environmental
impacts of transportation. Individuals with a strong environmental
attitude are more likely to favor low-emission travel options, to
support policies that reduce car use, and to accept constraints on
private mobility when these contribute to environmental goals.


\subsection{Psychometric indicators}
\label{sec:psycho}
The psychmometric indicators selected to be used in the model are the following:
\begin{description}
  \item[Envir01] Fuel price should be increased to reduce congestion and air pollution.
  \item[Envir02] More public transportation is needed, even if taxes are set to pay the additional costs.
  \item[Envir03] Ecology disadvantages minorities and small businesses.
  \item[Envir04] People and employment are more important than the environment.
  \item[Envir05] I am concerned about global warming.
  \item[Envir06] Actions and decision making are needed to limit greenhouse gas emissions.
  \item[Mobil03] I use the time of my trip in a productive way.
  \item[Mobil05] I reconsider frequently my mode choice.
  \item[Mobil08] I do not feel comfortable when I travel close to people I do not know.
  \item[Mobil09] Taking the bus helps making the city more comfortable and welcoming.
  \item[Mobil10] It is difficult to take the public transport when I travel with my children.
  \item[Mobil12] It is very important to have a beautiful car.
  \item[LifSty01] I always choose the best products regardless of price.
  \item[LifSty07] The pleasure of having something beautiful consists in showing it.
  \item[NbCar] Number of cars in the household.
\end{description}

The specification of the measurement model relies on two sets of
indicators, one for each latent variable. The \emph{car-centric}
attitude is measured using the indicators \texttt{Envir01},
\texttt{Envir02}, \texttt{Envir06}, \texttt{Mobil03}, \texttt{Mobil05},
\texttt{Mobil08}, \texttt{Mobil09}, \texttt{Mobil10}, \texttt{LifSty07},
and \texttt{NbCar}. These indicators capture aspects related to the
perceived convenience, comfort, flexibility, and social meaning of
private car use, as well as practical constraints associated with
alternative modes. The \emph{environmental} attitude is measured using
the indicators \texttt{Envir01}, \texttt{Envir02}, \texttt{Envir03},
\texttt{Envir04}, \texttt{Envir05}, \texttt{Envir06}, \texttt{Mobil12},
\texttt{LifSty01}, and \texttt{NbCar}, which reflect concerns about
environmental protection, sustainability, and the trade-offs between
environmental objectives, economic considerations, and personal
consumption preferences.

The composition of these indicator sets is not imposed \emph{a priori}
but results from a combination of theoretical considerations and an
iterative modeling process. Indicators were selected based on their
conceptual relevance to each latent construct and on empirical
performance during estimation.

The indicator \texttt{NbCar} differs in nature from the other indicators
used in the measurement model. It is not a psychometric indicator based
on attitudinal statements evaluated on a Likert scale, but an observed
household characteristic reporting the number of cars owned by the
household. This variable can take the discrete values $0$, $1$, $2$,
and $3$. Although \texttt{NbCar} does not directly measure attitudes or
perceptions, it provides valuable information about underlying latent
constructs related to mobility preferences and environmental values.
In particular, car ownership reflects long-term mobility decisions and
constraints that are strongly associated with both car-centric and
environmental attitudes. For this reason, \texttt{NbCar} is incorporated
into the model as an indicator.

\subsection{Structural equations}

For the structural equations, we use the linear specification in
Eq.~\eqref{eq:linearStructural}. The set of explanatory variables
included in each structural equation follows the specification used in
the implementation. In particular, the car-centric latent variable
$x_{n,\text{car}}^*$ is specified as a function of
\texttt{high\_education}, \texttt{top\_manager}, \texttt{employees},
\texttt{age\_30\_less}, \texttt{ScaledIncome}, and
\texttt{car\_oriented\_parents}. The environmental latent variable
$x_{n,\text{envir}}^*$ is specified as a function of \texttt{childSuburb},
\texttt{ScaledIncome}, \texttt{city\_center\_as\_kid}, \texttt{artisans},
\texttt{high\_education}, and \texttt{low\_education}. These variables are
constructed from the raw survey information during data preparation
(e.g., \texttt{ScaledIncome} is computed as \texttt{CalculatedIncome}/1000,
\texttt{age\_30\_less} is the indicator \texttt{age} $\le 30$,
\texttt{childSuburb} identifies individuals who lived in suburban areas
as children, and \texttt{car\_oriented\_parents} identifies respondents
reporting very frequent car use by parents). The final specification of
the structural equations results from a combination of behavioral
assumptions and empirical trial-and-error, balancing interpretability,
parameter stability, and overall fit.


\subsection{Measurement equations}

For each individual \( n \) and each indicator \( \ell \) described in
Section~\ref{sec:psycho}, we introduce a latent continuous response
variable, as outlined in Section~\ref{sec:likert}. This latent
response captures the unobserved propensity underlying the observed
ordinal response on a Likert scale.

For the indicators associated with the car-centric attitude, the latent response is modeled as:
\begin{equation}
  I^*_{n\ell} = \lambda_{0\ell} + \lambda_{1\ell} x^*_{n,\text{car}} + \lambda_{2\ell} \upsilon_{n\ell},
\end{equation}
where \( \lambda_{0\ell} \) is an intercept term, \( \lambda_{1\ell} \) is the loading on the latent variable \( x^*_{n,\text{car}} \), \( \lambda_{2\ell} \) scales the stochastic component, and \( \upsilon_{n\ell} \) is a random error term.

The indicator \texttt{Envir01} is selected for the normalization of
the measurement model. Individuals with a stronger car-centric
attitude are expected to be more likely to \emph{disagree} with the
corresponding statement. Accordingly, the loading \( \lambda_{1\ell}
\) is expected to be negative, and fixed to \(-1\) to establish the direction of the latent
construct. The scale parameter \( \lambda_{2\ell} \) is normalized to
1 to ensure identifiability of the model.

Similarly, for the indicators capturing the environmental attitude, we specify:
\begin{equation}
  I^*_{n\ell} = \lambda_{0\ell} + \lambda_{1\ell} x^*_{n,\text{env}} + \lambda_{2\ell} \upsilon_{n\ell},
\end{equation}
with analogous interpretation of the parameters.

The indicator \texttt{Envir02} is selected for the normalization of
this measurement model. Individuals with a stronger urban-preference
attitude are expected to be more likely to \emph{agree} with the
corresponding statement. Accordingly, the loading \( \lambda_{1\ell}
\) is expected to be positive, and fixed to 1 to establish the
direction of the latent construct. The scale parameter \(
\lambda_{2\ell} \) is normalized to 1 to ensure identifiability of the
model.

The threshold specification follows directly from the normalization and
identification principles discussed in Section~\ref{sec:normalization}
and from the ordered probit formulation introduced in
Section~\ref{sec:likert}. In particular, threshold parameterizations are
chosen so as to (i) enforce the ordering constraints, (ii) fix the
location of the ordinal response scale exactly once, and (iii) remain
consistent with the reference-indicator strategy adopted for the latent
variables.

For Likert-type indicators with five response categories, we use a
symmetric threshold parameterization centered at zero:
\[
\begin{aligned}
  \tau_1 &= -(\delta_1 + \delta_2), \\
  \tau_2 &= -\delta_1, \\
  \tau_3 &= \phantom{-}\delta_1, \\
  \tau_4 &= \phantom{-}(\delta_1 + \delta_2),
\end{aligned}
\]
where $\delta_1 > 0$ and $\delta_2 > 0$ are estimated. This
parameterization has three desirable properties. First, it guarantees
strict ordering of the thresholds by construction. Second, it fixes the
location of the threshold system by centering it at zero, thereby
removing the location indeterminacy inherent to ordered probit models.
Third, it reflects the semantic symmetry of standard Likert scales
(e.g., from ``strongly disagree'' to ``strongly agree'') while reducing
the number of free parameters. These thresholds are shared across all
Likert-type indicators, reflecting the modeling assumption that the
response categories have a comparable interpretation across statements.

The indicator \texttt{NbCar} is treated separately, as it is not a
psychometric Likert-scale indicator but an observed household
characteristic reporting the number of cars owned. Although it is used
as an indicator of the latent constructs, its response scale is
inherently asymmetric and quantitative. Since \texttt{NbCar} takes four
ordered values, three thresholds are required. For this indicator, we
adopt a non-symmetric threshold parameterization and fix the first
threshold to zero:
\[
\begin{aligned}
  \tau^{\text{NbCar}}_1 &= 0, \\
  \tau^{\text{NbCar}}_2 &= \tau^{\text{NbCar}}_1 + \delta^{\text{NbCar}}_1, \\
  \tau^{\text{NbCar}}_3 &= \tau^{\text{NbCar}}_2 + \delta^{\text{NbCar}}_2,
\end{aligned}
\]
where $\delta^{\text{NbCar}}_1 > 0$ and $\delta^{\text{NbCar}}_2 > 0$ are
estimated. Fixing $\tau^{\text{NbCar}}_1=0$ provides the required
location normalization for this indicator, while the positive
incremental parameterization ensures ordered thresholds without
imposing symmetry. This choice is fully consistent with the overall
normalization strategy: location is fixed once for the ordinal layer,
and the scale of the latent variables remains anchored through the
reference indicators.

\subsection{Implementation notes}
\label{sec:implementation_notes}

The results reported below are produced using the set of
Python specification files included in the appendix (Sections
\ref{sec:optima.py}--\ref{sec:plot_b06_hybrid_bayes.py}). These files
have been developed and tested with \texttt{Biogeme~3.3.2}. As Biogeme
evolves, minor adaptations of the syntax may be required in future
versions. The goal of the implementation is to keep the various model
variants (choice-only, MIMIC, and hybrid choice; maximum likelihood and
Bayesian estimation) consistent by relying on shared specification
components and a centralized configuration mechanism. This subsection
summarizes the role of each file and the type of specification
information it contains.

\paragraph{Data preparation (\ref{sec:optima.py}).}
The file in Section~\ref{sec:optima.py} is a standard Biogeme data
preparation script. It reads the raw data, applies the sample cleaning
rules (e.g., removal of inconsistent observations), and constructs the
derived variables used throughout the model specification (such as
scaled income, socio-demographic indicators, and other transformed
covariates). The resulting \texttt{Database} object constitutes the
common input for all estimation variants.

\paragraph{Indicator definitions and threshold conventions (\ref{sec:likert_indicators.py}).}
The file in Section~\ref{sec:likert_indicators.py} provides the complete
list of indicators used in the measurement model. For each indicator, it
stores its identifier (name), the corresponding survey statement, and
its type. In the present example, two indicator types are used:
\texttt{likert} for psychometric indicators collected on a Likert scale,
and \texttt{cars} for the discrete indicator \texttt{NbCar} (number of
cars in the household). In addition, the file defines the list of
indicator types and the associated threshold conventions. For each type,
it specifies whether the thresholds are symmetric or not, the list of
admissible response categories, and the ``neutral'' labels (if any) that
are ignored in estimation. These definitions ensure that all measurement
equations and threshold specifications are generated consistently across
the model variants.

\paragraph{Latent variable definitions (\ref{sec:latent_variables.py}).}
The file in Section~\ref{sec:latent_variables.py} defines the latent
variables used in the case study and, for each of them, the list of
explanatory variables entering its structural equation. This file
therefore contains the substantive specification choices for the
structural part of the model: the names of the latent constructs and the
observed covariates assumed to explain them.

\paragraph{Central configuration (\ref{sec:config.py}).}
To ensure that all model variants are generated from the same building
blocks, the implementation relies on a configuration object defined in
Section~\ref{sec:config.py}. This configuration determines which
components are active and how estimation is performed. It contains the
following entries:
\begin{itemize}
  \item \texttt{name}: a string identifier used to label the model run
  and its output files;
  \item \texttt{latent\_variables} (\texttt{"zero"} or \texttt{"two"}):
  whether the specification includes no latent variables or the two
  latent variables of the case study;
  \item \texttt{choice\_model} (\texttt{"yes"} or \texttt{"no"}): whether
  the discrete choice component is included (hybrid/choice-only) or
  omitted (pure MIMIC);
  \item \texttt{estimation} (\texttt{"ml"} or \texttt{"bayes"}): whether
  the model is estimated by maximum likelihood or using Bayesian
  inference;
  \item \texttt{number\_of\_bayesian\_draws\_per\_chain}: the number of
  posterior draws generated per MCMC chain (relevant when
  \texttt{estimation="bayes"});
  \item \texttt{number\_of\_monte\_carlo\_draws}: the number of Monte-Carlo
  draws used to approximate the integrals over the latent variables
  (relevant \texttt{estimation="ml"}).
\end{itemize}
This design avoids duplicating code across variants and makes the
comparison between specifications transparent.

\paragraph{MIMIC component and normalization (\ref{sec:mimic.py}).}
The file in Section~\ref{sec:mimic.py} builds the MIMIC part of the
model (structural and measurement equations) as a function of the
configuration. It is also where the reference-indicator normalization is
declared explicitly. In particular, the reference indicator used to
anchor a latent variable is identified through a normalization object of
the form \texttt{Normalization(indicator='Envir01', coefficient=-1)},
where the \texttt{coefficient} specifies the fixed loading (e.g., $-1$)
used for identification in the corresponding measurement equation. This
file therefore centralizes the measurement-structure assumptions and the
identification choices of the latent-variable system.

\paragraph{Choice model component (\ref{sec:choice_model.py}).}
The file in Section~\ref{sec:choice_model.py} contains the specification
of the discrete choice model. Depending on the configuration, the choice
model is defined either without latent variables (choice-only baseline)
or with latent variables entering the utilities (hybrid choice model).

\paragraph{Estimation control and caching of results (\ref{sec:read_or_estimate.py}).}
The file in Section~\ref{sec:read_or_estimate.py} orchestrates the
execution of the estimation in the requested mode (maximum likelihood or
Bayesian), based on the configuration. For reproducibility and
efficiency, it first checks whether estimation outputs already exist; if
so, results are read from disk rather than recomputed. Otherwise, the
file triggers a new estimation run. This file therefore handles the
``run-or-read'' logic that supports systematic experimentation with
multiple model variants.

\paragraph{Log-likelihood assembly and estimation (\ref{sec:estimate.py}).}
Finally, the file in Section~\ref{sec:estimate.py} assembles the full
model implied by the configuration and generates the corresponding
(log-)likelihood expression. This includes combining the relevant
components (choice probability, measurement likelihood, structural
density) and performing the required integration over latent variables
using Monte-Carlo simulation when appropriate. The same file then calls
the estimation routines corresponding to the selected estimation
paradigm (maximum likelihood or Bayesian). In short, it is the entry
point where the complete hybrid choice model likelihood is constructed
and estimated.

\paragraph{Estimation scripts
(\ref{sec:plot_b01_choice_only_ml.py}--\ref{sec:plot_b06_hybrid_bayes.py}).}
These are six driver scripts, each corresponding to one combination of model
scope (choice-only, MIMIC, or full hybrid choice) and estimation method
(maximum likelihood or Bayesian). Each script defines the appropriate
configuration (Section~\ref{sec:config.py}) and then relies on the
generic estimation workflow (Sections~\ref{sec:read_or_estimate.py} and
\ref{sec:estimate.py}) to either run the estimation or read existing
outputs. 

\begin{itemize}
  \item \textbf{Choice-only, maximum likelihood}
  (Section~\ref{sec:plot_b01_choice_only_ml.py}): the script
  \lstinline$plot_b01_choice_only_ml.py$ estimates the discrete choice
  model without latent variables (\lstinline$latent_variables="zero"$,
  \lstinline$choice_model="yes"$, \lstinline$estimation="ml"$). It
  provides a baseline choice specification used for comparison with
  latent-variable extensions.

  \item \textbf{MIMIC, maximum likelihood}
  (Section~\ref{sec:plot_b02_mimic_ml.py}): the script
  \lstinline$plot_b02_mimic_ml.py$ estimates the latent-variable system
  (structural and measurement equations) without the choice component
  (\lstinline$latent_variables="two"$, \lstinline$choice_model="no"$,
  \lstinline$estimation="ml"$). It focuses on how the latent constructs
  are explained by covariates and reflected in indicators.

  \item \textbf{Hybrid choice, maximum likelihood}
  (Section~\ref{sec:plot_b03_hybrid_ml.py}): the script
  \lstinline$plot_b03_hybrid_ml.py$ estimates the full hybrid choice
  model, combining the choice model with the latent-variable system
  (\lstinline$latent_variables="two"$, \lstinline$choice_model="yes"$,
  \lstinline$estimation="ml"$). The likelihood integrates the choice and
  measurement components over the latent variables.

  \item \textbf{Choice-only, Bayesian}
  (Section~\ref{sec:plot_b04_choice_only_bayes.py}): the script
  \lstinline$plot_b04_choice_only_bayes.py$ estimates the discrete choice
  model without latent variables using Bayesian inference
  (\lstinline$latent_variables="zero"$, \lstinline$choice_model="yes"$,
  \lstinline$estimation="bayes"$). It provides the Bayesian counterpart
  of the maximum-likelihood baseline.

  \item \textbf{MIMIC, Bayesian}
  (Section~\ref{sec:plot_b05_mimic_bayes.py}): the script
  \lstinline$plot_b05_mimic_bayes.py$ estimates the latent-variable
  system without the choice component using Bayesian inference
  (\lstinline$latent_variables="two"$, \lstinline$choice_model="no"$,
  \lstinline$estimation="bayes"$). It delivers posterior inference for
  the structural and measurement parameters.

  \item \textbf{Hybrid choice, Bayesian}
  (Section~\ref{sec:plot_b06_hybrid_bayes.py}): the script
  \lstinline$plot_b06_hybrid_bayes.py$ estimates the complete hybrid
  choice model using Bayesian inference
  (\lstinline$latent_variables="two"$, \lstinline$choice_model="yes"$,
  \lstinline$estimation="bayes"$). It is the most comprehensive variant
  and yields posterior distributions for both the choice parameters and
  the latent-variable system, accounting for the full joint likelihood.
\end{itemize}

All six scripts rely on the same underlying model specification files
(Sections~\ref{sec:latent_variables.py}, \ref{sec:likert_indicators.py},
\ref{sec:mimic.py}, and \ref{sec:choice_model.py}); the differences
between them arise solely from the configuration settings and the chosen
estimation paradigm.
\section{Conclusion}

Choice models with latent variables offer a powerful and flexible
framework for capturing complex behavioral mechanisms underlying
decision-making. By incorporating unobserved psychological constructs
such as attitudes, and perceptions, these models extend
the explanatory power of traditional discrete choice models. They
allow researchers to account for systematic heterogeneity in behavior
that is not directly observed in the data, thereby enhancing both the
behavioral realism and predictive performance of the models.

Despite their potential, these models are inherently more complex to
specify, estimate, and interpret. It is therefore recommended to
proceed incrementally. A practical and effective strategy is to begin
by developing and estimating the choice model and the MIMIC model
independently. This allows the analyst to ensure that both components
are correctly specified and empirically supported.

Once the separate models have been validated, the next step is to
explore their integration through sequential estimation. In this
stage, the latent variables generated from the MIMIC model are
incorporated into the utility specification of the choice model. This
provides valuable insights into how these latent constructs influence
behavior, while still maintaining manageable computational complexity.

Only after the specification has been refined and the results from the
sequential estimation are deemed satisfactory should one proceed to
the simultaneous estimation of all components. This final step --- though
computationally more demanding --- offers the benefit of statistical
efficiency by leveraging all available information jointly. It also
provides a more coherent treatment of the latent variables, since
their estimation is informed not only by the indicators, but also by
the observed choices.




\section{Description of the variables}
\label{sec:variables}

The following table describes the variables involved in the models described in this document.

\begin{longtable}{p{4cm}|p{10.5cm}}
		\hline 
		\textbf{Name} & \textbf{Description}\\
		%\tabularnewline
		\hline 
		TimePT & The duration of the loop performed in public transport (in minutes).\tabularnewline
		\hline 
		WaitingTimePT & The total waiting time in a loop performed in public transports (in minutes).\tabularnewline
		\hline 
		TimeCar & The total duration of a loop made using the car (in minutes).\tabularnewline
		\hline 
		MarginalCostPT & The total cost of a loop performed in public transports, taking into account the ownership of a seasonal ticket by the respondent. If the respondent has a ``GA'' (full Swiss season ticket), a seasonal ticket for the line or the area, this variable takes value zero. If the respondent has a half-fare travelcard, this variable corresponds to half the cost of the trip by public transport..\tabularnewline
		\hline 
		CostCarCHF & The total gas cost of a loop performed with the car in CHF.\tabularnewline
		\hline 
		TripPurpose & The main purpose of the loop: 1 =Work-related trips; 2 =Work- and leisure-related
		trips; 3 =Leisure related trips. -1 represents missing values \tabularnewline
		\hline 
		UrbRur & Binary variable, where: 1 =Rural; 2 =Urban.\tabularnewline
		\hline 
		distance\_km & Total distance performed for the loop.\tabularnewline
		\hline 
		age & Age of the respondent (in years) -1 represents missing values.\tabularnewline
		\hline 
		ResidChild & Main place of residence as a kid ($<18$), 1 is city center (large town), 2 is city center (small town), 3 is suburbs, 4 is suburban town, 5 is country side (village), 6 is countryside (isolated), -1 is for missing data and -2 if respondent didn't answer to any opinion questions. \tabularnewline
		\hline 
		NbCar & Number of cars in the household.-1 for missing value. \tabularnewline
		\hline 
		NbBicy & Number of bikes in the household. -1 for missing value.\tabularnewline
		\hline 
		HouseType & House type, 1 is individual house (or terraced house), 2 is apartment (and other types of multi-family residential), 3 is independent room (subletting). -1 for missing value.\tabularnewline
		\hline 
		Income & Net monthly income of the household in CHF. 1 is less than 2500, 2 is from 2501 to 4000, 3 is from 4001 to 6000, 4 is from 6001 to 8000, 5 is from 8001 to 10'000 and 6 is more than 10'001. -1 for missing value.\tabularnewline
		\hline 
		CalculatedIncome & Net monthly income of the household in CHF, calculated as a continuous variable. The value is the center of the interval of the corresponding incone category. \tabularnewline
		\hline
		FamilSitu & Familiar situation: 1 is single, 2 is in a couple without children, 3 is in a couple with children, 4 is single with your own children, 5 is in a colocation, 6 is with your parents and 7 is for other situations. -1 for missing values.\tabularnewline
		\hline
		SocioProfCat & To which of the following socioprofessional categories do you belong? 1 is for top managers, 2 for intellectual professions, 3 for freelancers, 4 for intermediate professions, 5 for artisans and salespersons, 6 for employees, 7 for workers and 8 for others. -1 for missing values.\tabularnewline
		\hline 
		GenAbST & Is equal to 1 if the respondent has a GA (full Swiss season ticket) and 2 if not.\tabularnewline
		\hline
		Education &Highest education achieved. As mentioned by Wikipedia in English: "The education system in Switzerland is very diverse, because the constitution of Switzerland delegates the authority for the school system mainly to the cantons. The Swiss constitution sets the foundations, namely that primary school is obligatory for every child and is free in public schools and that the confederation can run or support universities." (source: \href{http://en.wikipedia.org/wiki/Education\_in\_Switzerland}{Education in Switzerland (Wikipedia)}, accessed April 16, 2013). It is thus difficult to translate the survey that was originally in French and German. The possible answers in the survey are:
		\begin{enumerate}
			\item Unfinished compulsory education: education is compulsory in Switzerland but pupils may finish it at the legal age without succeeding the final exam.
			\item Compulsory education with diploma.
			\item Vocational education: a three or four-year period of training both in a company and following theoretical courses. Ends with a diploma called "Certificat fédéral de capacité" (i.e., ''professional baccalaureate'') (reference: \href{https://fr.wikipedia.org/wiki/Certificat\_f\%C3\%A9d\%C3\%A9ral\_de\_capacit\%C3\%A9}{Certificat fédéral de capacité (Wikipedia)} - in French).
			\item A 3-year generalist school giving access to teaching school, nursing schools, social work school, universities of applied sciences or vocational education (sometime in less than the normal number of years). It does not give access to universities in Switzerland.
			\item High school: ends with the general baccalaureate exam. The general baccalaureate gives access automatically to universities.
			\item Universities of applied sciences, teaching schools, nursing schools, social work schools: ends with a Bachelor and sometimes a Master, mostly focus on vocational training.
			\item Universities and institutes of technology: ends with an academic Bachelor and in most cases an academic Master.
			\item PhD thesis.
		\end{enumerate}\\
	\hline
		\caption{Description of variables}
		\label{tab:variables}
	\end{longtable}

\clearpage
\section{Complete specification files}

This section presents the Python implementation of the hybrid choice
model used in the case study. The following specification files have
been used for the estimation of the results presented in this chapter.
They have been developed and tested with \texttt{Biogeme~3.3.2}. It is
possible that minor adaptations of the syntax may be required for
future versions of Biogeme.

The files are organized by \emph{role} rather than by estimation
approach: data preparation and configuration, model specification
(latent variables, indicators, MIMIC and choice components), estimation
workflow, and result visualization. This structure mirrors the modeling
logic developed in the previous sections and allows the same core model
to be estimated under different assumptions (choice-only, MIMIC, or
full hybrid model; maximum likelihood or Bayesian estimation).


\subsubsection*{Data preparation and configuration}

These files define the dataset, variable transformations, and global
configuration options shared across all model variants.

\subsection{\lstinline$optima.py$}
\label{sec:optima.py}
\lstinputlisting[style=numbers]{../../docs/source/examples/hybrid_choice/optima.py}

\subsection{\lstinline$config.py$}
\label{sec:config.py}
\lstinputlisting[style=numbers]{../../docs/source/examples/hybrid_choice/config.py}


\subsubsection*{Latent variables and measurement structure}

The following files define the latent variables, the associated
psychometric and non-psychometric indicators, and the MIMIC component
(structural and measurement equations without choices).

\subsection{\lstinline$latent\_variables.py$}
\label{sec:latent_variables.py}
\lstinputlisting[style=numbers]{../../docs/source/examples/hybrid_choice/latent_variables.py}

\subsection{\lstinline$likert\_indicators.py$}
\label{sec:likert_indicators.py}
\lstinputlisting[style=numbers]{../../docs/source/examples/hybrid_choice/likert_indicators.py}

\subsection{\lstinline$mimic.py$}
\label{sec:mimic.py}
\lstinputlisting[style=numbers]{../../docs/source/examples/hybrid_choice/mimic.py}


\subsubsection*{Choice model specification}

This file defines the discrete choice component and its interaction
with the latent variables in the hybrid choice model.

\subsection{\lstinline$choice\_model.py$}
\label{sec:choice_model.py}
\lstinputlisting[style=numbers]{../../docs/source/examples/hybrid_choice/choice_model.py}


\subsubsection*{Estimation workflow}

These scripts orchestrate model estimation, either by reading existing
results or launching new estimation runs, and provide utilities for
batch execution.

\subsection{\lstinline$estimate.py$}
\label{sec:estimate.py}
\lstinputlisting[style=numbers]{../../docs/source/examples/hybrid_choice/estimate.py}

\subsection{\lstinline$read\_or\_estimate.py$}
\label{sec:read_or_estimate.py}
\lstinputlisting[style=numbers]{../../docs/source/examples/hybrid_choice/read_or_estimate.py}


\subsection{\lstinline$plot\_b01\_choice\_only\_ml.py$}
\label{sec:plot_b01_choice_only_ml.py}
\lstinputlisting[style=numbers]{../../docs/source/examples/hybrid_choice/plot_b01_choice_only_ml.py}

\subsection{\lstinline$plot\_b02\_mimic\_ml.py$}
\label{sec:plot_b02_mimic_ml.py}
\lstinputlisting[style=numbers]{../../docs/source/examples/hybrid_choice/plot_b02_mimic_ml.py}

\subsection{\lstinline$plot\_b03\_hybrid\_ml.py$}
\label{sec:plot_b03_hybrid_ml.py}
\lstinputlisting[style=numbers]{../../docs/source/examples/hybrid_choice/plot_b03_hybrid_ml.py}

\subsection{\lstinline$plot\_b04\_choice\_only\_bayes.py$}
\label{sec:plot_b04_choice_only_bayes.py}
\lstinputlisting[style=numbers]{../../docs/source/examples/hybrid_choice/plot_b04_choice_only_bayes.py}

\subsection{\lstinline$plot\_b05\_mimic\_bayes.py$}
\label{sec:plot_b05_mimic_bayes.py}
\lstinputlisting[style=numbers]{../../docs/source/examples/hybrid_choice/plot_b05_mimic_bayes.py}

\subsection{\lstinline$plot\_b06\_hybrid\_bayes.py$}
\label{sec:plot_b06_hybrid_bayes.py}
\lstinputlisting[style=numbers]{../../docs/source/examples/hybrid_choice/plot_b06_hybrid_bayes.py}

\clearpage

\clearpage


\clearpage
\bibliographystyle{dcudoi}
\bibliography{transpor}

\end{document}


