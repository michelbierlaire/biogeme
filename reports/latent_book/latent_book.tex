\documentclass[12pt,a4paper]{article}

\usepackage{michel}
\usepackage[dcucite,abbr]{harvard}
\harvardparenthesis{none}\harvardyearparenthesis{round}
\usepackage{hyperref}
\usepackage{varioref}
\usepackage{longtable}
\usepackage{siunitx}
\sisetup{
  parse-numbers=false,      % Prevents automatic parsing (needed for parentheses & superscripts)
  detect-inline-weight=math,% Ensures proper formatting in tables
  tight-spacing=true        % Keeps spacing consistent
}
% Package to include code
\usepackage{listings}
\usepackage{color}
\lstset{language=Python}
\lstset{numbers=none, basicstyle=\footnotesize,
  numberstyle=\tiny,keywordstyle=\color{blue},stringstyle=\ttfamily,showstringspaces=false}
\lstset{backgroundcolor=\color[rgb]{0.95 0.95 0.95}}
\lstdefinestyle{numbers}{numbers=left, stepnumber=1,
  numberstyle=\tiny,basicstyle=\tiny, numbersep=10pt}
\lstdefinestyle{nonumbers}{numbers=none}
\lstset{
  breaklines=true,
  breakatwhitespace=true,
}

\title{Estimating choice models with latent variables with Biogeme}
\author{Michel Bierlaire \and Moshe Ben-Akiva \and Joan Walker} 
\date{\today}


\begin{document}


\begin{titlepage}
\pagestyle{empty}

\maketitle
\vspace{2cm}

\begin{center}
\small Report TRANSP-OR xxxxxx  \\ Transport and Mobility Laboratory \\ School of Architecture, Civil and Environmental Engineering \\ Ecole Polytechnique F\'ed\'erale de Lausanne \\ \verb+transp-or.epfl.ch+
\begin{center}
\textsc{Series on Biogeme}
\end{center}
\end{center}


\clearpage
\end{titlepage}

\begin{titlepage}
\tableofcontents
\end{titlepage}

The package Biogeme (\texttt{biogeme.epfl.ch}) is designed to estimate
the parameters of various models using maximum likelihood
estimation. It is particularly designed for discrete choice
models.  In this document, we present how to estimate choice
models involving latent variables.

We assume that the reader is already familiar with discrete choice
models, and has successfully installed Biogeme. This document has
been written using Biogeme 3.3.0.



\section{Models and notations}

The literature on discrete choice models with latent variables is vast
(\cite{walker2001extended}, \cite{ashok2002extending},
\cite{greene2003latent}, \cite{ben2002integration}, to cite just a
few). We start this document by a short introduction to the models and
the notations. 

\subsection{Structural equations}
A \emph{latent variable} is a variable that cannot be directly
observed. It is typically modeled using a \textbf{structural
  equation}, which expresses the latent variable as a function of
observed (explanatory) variables and an error term. A general form of
such a structural equation is:
\begin{equation}
\label{eq:structural}
x_{nk}^* = x^*(x_n; \psi_k) + \omega_{nk},
\end{equation}
where $n$ indexes individuals, $x_{nk}^*$ is the $k$th latent variable of interest, $x_n$ is a vector of observed explanatory variables, $\psi_k$ is a vector of parameters to be estimated, and $\omega_{nk}$ is a stochastic error term.

A common specification assumes a linear functional form   with i.i.d. normally distributed  error terms:
\begin{equation}
\label{eq:linearStructural}
 x_{nk}^* = \sum_s \psi_{sk} x_{ns} + \sigma_{\omega k} \omega_{nk},
\end{equation}
where $\omega_{nk} \sim N(0, 1)$, and $\sigma_{\omega k}$ is a scaling parameter for the error term. 

In discrete choice models, for example, the utility $U_{in}$ that
individual $n$ associates with alternative $i$ is a typical example of
a latent variable.

Information about latent variables is obtained indirectly through
\emph{measurements}, which are observable manifestations of the
underlying latent constructs. For example, in discrete choice models,
utility is not directly observed but is inferred from the choices
individuals make. The relationship between a latent variable and its
associated measurements is described by \textbf{measurement
  equations}. The specific form of these equations depends on the
nature of the observed measurements (e.g., continuous, or ordinal).


\subsection{Measurement equations: the continuous case}
\label{sec:continuous}

Since latent variables cannot be directly observed, analysts rely on
indirect measurements to infer their values. A common approach
involves asking respondents to rate the perceived magnitude of the
latent construct on an arbitrary scale. For example: \emph{``How would
you rate the level of pain that you are experiencing, from 0 (no pain)
to 10 (worst pain imaginable)?''}

Each such rating is referred to as an \emph{indicator}, indexed by
$\ell=1, \ldots, L_n$, and is modeled using a \textbf{measurement equation}. This
equation relates the observed indicator to the latent variables and, sometimes,
other explanatory variables:
\begin{equation}
\label{eq:continuousMeasurement}
I_{n\ell} = I_\ell(x_n, x_n^*;\lambda_\ell) + \upsilon_{n\ell}, \; \forall \ell=1, \ldots, L_n, \forall n,
\end{equation}
where $I_{n\ell}$ denotes the response provided by individual $n$ for
indicator $\ell$, $x_n^*$ is the latent variable of interest (e.g., pain
perception), $x_n$ is a vector of observed explanatory variables (such as
socio-demographic characteristics), $\lambda_\ell$ is a vector of
parameters to be estimated, and $\upsilon_{n\ell}$ is the random error term.

A common specification of the measurement function assumes linearity
and i.i.d. normally distributed errors:
\begin{equation}
\label{eq:linearMeasurement}
I_{n\ell} = \lambda_{\ell 0} + \sum_k \lambda_{\ell k}  x_{nk}^* + \sigma_{\upsilon \ell} \upsilon_{n\ell}, \quad \forall \ell,
\end{equation}
where $\lambda_{\ell k}$ are unknown parameters to be estimated, $\sigma_{\upsilon \ell}$ is an indicator-specific scale parameter, and $\upsilon_{n\ell} \sim N(0, 1)$.

If we observe a vector of continuous indicators \( I_n = (I_{n1}, \ldots, I_{nL_n}) \) for individual \( n \), the contribution to the likelihood function, \emph{conditional on the latent variables} \( x^*_{n} \), is given by the product:
\begin{equation}
  \label{eq:conditional_likelihood_continuous}
\prod_{\ell=1}^{L_n} \phi\left( \frac{I_{n\ell} - \lambda_{\ell 0} - \sum_k \lambda_{\ell k} x_{nk}^*}{\sigma_{\upsilon \ell}} \right),
\end{equation}
where \( \phi(\cdot) \) denotes the probability density function (pdf) of the standard normal distribution.

If other types of observations are available for the same individual (such as discrete choices), the corresponding components of the likelihood can be multiplied with the expression above. Once all relevant components are combined, the latent variables must be integrated out, as discussed later.

If the continuous indicators are the only data available for individual \( n \), the contribution to the unconditional likelihood becomes:
\begin{equation}
  \label{eq:likelihood_continuous}
\int_{x_n^*} \left[ \prod_{\ell=1}^{L_n} \phi\left( \frac{I_{n\ell} - \lambda_{\ell 0} - \sum_k \lambda_{\ell k} x_{nk}^*}{\sigma_{\upsilon \ell}} \right) \right] f(x_n^*) \, dx_n^*,
\end{equation}
where \( f(x_n^*) \) is the pdf of the vector of latent variables \( x_n^* \).
As this integral does not have a closed-form expression, it is approximated using Monte Carlo integration (see \cite{Bier19} for a discussion about performing Monte-Carlo integration with Biogeme).




\subsection{Measurement equation: the ordinal case}
\label{sec:likert}

Another type of indicator arises when respondents are asked to
evaluate a statement using an ordinal scale. A typical context for
this type of measurement is the use of a Likert scale
(\cite{likert1932technique}), where individuals express their degree of
agreement or disagreement with a given statement. For example:
\begin{quote}
\emph{``I believe that my own actions have an impact on the planet.''} \\
Response options: strongly agree (2), agree (1), neutral (0), disagree ($-1$), strongly disagree ($-2$).
\end{quote}


To model these types of indicators, we represent the observed
measurement as an \emph{ordered discrete variable} $I_{n\ell}$, which
takes values in a finite, ordered set $\{j_1, j_2, \ldots, j_{M_\ell}\}$. The
measurement equation involves two stages:

\paragraph{Step 1: Latent response formulation.} We first define a continuous response variable, as explained in Section~\ref{sec:continuous}, except that it happens to be unobserved (latent) in this case:
\begin{equation}
I^*_{n\ell} = I^*_\ell(x_n, x_n^*; \lambda_\ell) + \upsilon_{n\ell},
\end{equation}
where $I^*_{n\ell}$ is a continuous latent variable underlying the
reported response, $x_n^*$ is a vector of  relevant latent variables (e.g.,
environmental concern), $x_n$ is a vector of observed explanatory
variables (e.g., age, income), $\lambda$ is a vector of parameters to
be estimated, and $\upsilon_{n\ell}$ is a random error term.

\paragraph{Step 2: Discretization via thresholds.} Since $I^*_{n\ell}$ is not observed, we relate it to the reported discrete measurement $I_{n\ell}$ through a set of threshold parameters:
\begin{equation}
\label{eq:discreteMeas-b}
I_{n\ell} = \left\{
\begin{array}{ll}
j_1 & \text{if } I^*_{n\ell} < \tau_1, \\
j_2 & \text{if } \tau_1 \leq I^*_{n\ell} < \tau_2, \\
\vdots \\
j_m & \text{if } \tau_{m-1} \leq I^*_{n\ell} < \tau_m, \\
\vdots \\
j_M & \text{if } \tau_{M_\ell-1} \leq I^*_{n\ell},
\end{array}
\right.
\end{equation}
where $\tau_1, \ldots, \tau_{M_\ell-1}$ are threshold parameters to be estimated, satisfying the ordering constraint:
\begin{equation}
\label{eq:discreteMeas-c}
\tau_1 \leq \tau_2 \leq \cdots \leq \tau_{M_\ell-1}.
\end{equation}
Note that it is customary to use the same set of parameters for all
individuals $n$ and all indicators $\ell$, which explains the absence
of these indices on the parameter $\tau$.

Defining $\tau_0=-\infty$ and $\tau_{M_\ell}=+\infty$, it simplifies to
\begin{equation}
  \label{eq:prob_indicator}
I_{n\ell} = j_m  \text{ if } \tau_{m-1} \leq I^*_{n\ell} < \tau_m, \; m=1, \ldots, M_\ell. 
\end{equation}

It is often advantageous to impose a symmetric structure on the
definition of the thresholds. In addition, it is more convenient from
an estimation standpoint to parameterize the thresholds in terms of
differences and to constrain these differences to be positive. For
example, when \( M_\ell = 4 \), the thresholds can be defined as
follows:
\[
\begin{aligned}
  \tau_1 &= -\delta_1 -\delta_2, \\
  \tau_2 &= -\delta_1, \\
  \tau_3 &= \phantom{-}\delta_1, \\
  \tau_4 &= \phantom{-}\delta_1 + \delta_2,
\end{aligned}
\]
where \( \delta_1 > 0 \) and \( \delta_2 > 0 \) are the parameters to
be estimated.
This parameterization guarantees that the thresholds are strictly ordered 
and symmetrically centered around zero, which facilitates both 
identification and interpretation. To enforce positivity, we estimate the 
logarithm of the underlying parameters, that is 
$\delta'_i = \ln(\delta_i)$, so that
\[
\begin{aligned}
  \tau_1 &= -\exp(\delta'_1) - \exp(\delta'_2), \\
  \tau_2 &= -\exp(\delta'_1), \\
  \tau_3 &= \phantom{-}\exp(\delta'_1), \\
  \tau_4 &=  \phantom{-}\exp(\delta'_1) + \exp(\delta'_2).
\end{aligned}
\]


If we consider a linear specification,
\begin{equation}
  \label{eq:measurement_latent}
I^*_{n\ell} = \lambda_{\ell 0}  + \sum_k \lambda_{\ell k}  x_{nk}^* + \sigma_{\upsilon \ell} \upsilon_{n\ell}, \quad \forall \ell,
\end{equation}
where the error term $\upsilon_{n\ell} \sim N(0,1)$,
the contribution of each indicator $\ell$ for each observation $n$ to the likelihood function, \emph{conditional on the latent variables}, is defined as follows:
\begin{equation}
  \label{eq:discreteMeas-d}
  \begin{aligned}
    \prob(I_{n\ell} = j_m| x^*_n, x_n ; \lambda_\ell, \Sigma_{\upsilon \ell})
    =& \prob(\tau_{m-1} \leq I^*_{n\ell} \leq \tau_m) \\
    =& \prob(I^*_{n\ell} \leq \tau_m) -  \prob(I^*_{n\ell} \leq \tau_{m-1}),\\
    =& \prob\left( \upsilon_{n\ell} \leq \frac{\tau_m -\lambda_{\ell 0}  - \sum_k \lambda_{\ell k}  x_{nk}^*}{\sigma_{\upsilon \ell}} \right) \\
    &-  \prob \left( \upsilon_{n\ell} \leq \frac{\tau_{m-1}- \lambda_{\ell 0}  - \sum_k \lambda_{\ell k}  x_{nk}^*}{\sigma_{\upsilon \ell}}\right), \\
    =& \Phi\left(  \frac{\tau_m -\lambda_{\ell 0}  - \sum_k \lambda_{\ell k}  x_{nk}^*}{\sigma_{\upsilon \ell}} \right) \\
    &-\Phi\left(\frac{\tau_{m-1}- \lambda_{\ell 0}  - \sum_k \lambda_{\ell k}  x_{nk}^*}{\sigma_{\upsilon \ell}}\right)
  \end{aligned}
\end{equation}
where $j_m$ is the observed category for respondent $n$ and indicator $\ell$.

This specification is known as the \emph{ordered probit model} and is
widely used for modeling ordinal responses that depend on latent
constructs.

If we observe a vector of continuous indicators \( I_n = (I_{n1}, \ldots, I_{nL_n}) \) for individual \( n \), the contribution to the likelihood function, \emph{conditional on the latent variables} \( x^*_{n} \), is given by:
\begin{equation}
\prod_{\ell=1}^{L_n} \prob(I_{n\ell} = j_m| x^*_n, x_n ; \lambda_\ell, \Sigma_{\upsilon \ell}).
\end{equation}

As in the continous case, if other types of observations are available
for the same individual (such as  choices), the corresponding
components of the likelihood can be multiplied with the expression
above. Once all relevant components are combined, the latent variables
must be integrated out, as discussed later.

If the continuous indicators are the only data available for individual \( n \), the contribution to the unconditional likelihood becomes:
\begin{equation}
\int_{x_n^*} \left[ \prod_{\ell=1}^{L_n} \prob(I_{n\ell} = j_m| x^*_n, x_n ; \lambda_\ell, \Sigma_{\upsilon \ell})
\right] f(x_n^*) \, dx_n^*,
\end{equation}
where \( f(x_n^*) \) is the pdf of the vector of latent variables \( x_n^* \).
Again, this integral is approximated using Monte-Carlo integration.


\section{The MIMIC model}

The Multiple Indicators Multiple Causes (MIMIC) model is a structural
equation modeling framework designed to analyze relationships
involving latent variables. In a MIMIC model, the latent variable is
simultaneously influenced by a set of observed explanatory variables
(the ``multiple causes'') and reflected in several observed indicators
(the ``multiple indicators''). This dual structure enables the analyst
to capture both the determinants and the manifestations of latent
constructs, such as attitudes, preferences, or psychological traits.
A seminal introduction to the MIMIC model is provided by
\citeasnoun{Joreskog:1975aa}, who formalized its use within the
broader class of structural equation models.


Under the  specifications in \eqref{eq:linearStructural} and 
\eqref{eq:linearMeasurement}, the latent vector and indicators are jointly normal, 
so the latent variables can be integrated out analytically. 
Let $\boldsymbol{\nu}=(\lambda_{\ell 0})_{\ell}$, 
$\Lambda=[\lambda_{\ell k}]_{\ell,k}$, 
$\Theta=\mathrm{diag}(\sigma_{\upsilon\ell}^2)_{\ell}$, 
and $\Psi=\mathrm{diag}(\sigma_{\omega k}^2)_{k}$. 
Write the structural part as 
$\boldsymbol{x}_n^*=\Gamma \boldsymbol{x}_n+\boldsymbol{\zeta}_n$ 
with $\boldsymbol{\zeta}_n\sim\mathcal N(\mathbf{0},\Psi)$ 
(so $\Gamma_{ks}=\psi_{sk}$). 
Then, conditionally on $\boldsymbol{x}_n^*$, we have 
$\boldsymbol{I}_n\mid \boldsymbol{x}_n^* \sim 
\mathcal N(\boldsymbol{\nu}+\Lambda \boldsymbol{x}_n^*,\,\Theta)$. 
Marginalizing $\boldsymbol{x}_n^*$ yields the multivariate normal
\[
\boldsymbol{I}_n\mid \boldsymbol{x}_n \sim 
\mathcal N\!\big(\;\boldsymbol{\mu}_n,\;\Sigma\;\big),
\qquad
\boldsymbol{\mu}_n=\boldsymbol{\nu}+\Lambda\Gamma \boldsymbol{x}_n,\quad
\Sigma=\Lambda\Psi\Lambda^\top+\Theta.
\]
Therefore, the contribution of individual $n$ to the likelihood is
\begin{equation}
\label{eq:marginal_likelihood_continuous_closed}
L_n \;=\; (2\pi)^{-L_n/2}\,|\Sigma|^{-1/2}\,
\exp\!\left(-\tfrac{1}{2}\big(\boldsymbol{I}_n-\boldsymbol{\mu}_n\big)^{\!\top}
\Sigma^{-1}\big(\boldsymbol{I}_n-\boldsymbol{\mu}_n\big)\right),
\end{equation}
and the sample likelihood is $\prod_n L_n$. 
Hence, no Monte-Carlo integration is required for the continuous-indicator part: 
the conjugate normal--normal structure delivers a closed form. 
(If additional non-Gaussian components---e.g., discrete choices---are included, 
their likelihood terms multiply \eqref{eq:marginal_likelihood_continuous_closed}; 
only those non-Gaussian parts may require numerical integration.)

----
We first investigate the continuous case, that is, the model specification that involves the structural equations \req{eq:linearStructural} and the measurement equations \req{eq:continuousMeasurement}.

In this specific context, the formulation
\eqref{eq:likelihood_continuous} of the likelihood happens to
simplify.  Indeed, both the structural and measurement equations are
linear and the error terms are assumed to be normally distributed.  Because linear
transformations and sums of normal random variables remain normal,
the pair $(x_n^*, I_n)$ is jointly normal. Integrating out the latent
variables $x_n^*$ therefore yields a closed-form multivariate normal
distribution for the indicators $I_n$, with a mean shifted according
to the structural equation and a covariance matrix equal to the sum of
the measurement-error variances and the variance propagated from the
latent variables. In other words, the integral in
\eqref{eq:likelihood_continuous} collapses to the standard
multivariate normal density, without requiring any numerical
approximation.

It can be seen by using \req{eq:linearStructural} into  \req{eq:continuousMeasurement}, to obtain
\begin{equation}
  \begin{aligned}
    I_{n\ell} &= \lambda_{\ell 0}  + \sum_k \lambda_{\ell k}  (\sum_s \psi_{sk} x_{ns} + \sigma_{\omega k} \omega_{nk}) + \sigma_{\upsilon \ell} \upsilon_{n\ell}, \quad \forall \ell, \\
    &= \lambda_{\ell 0} + \sum_s \lambda'_{\ell s} x_{ns} + \sigma'_{\upsilon \ell} \upsilon'_{n\ell}, \quad \forall \ell, 
  \end{aligned}
\end{equation}
where  $\lambda'_{\ell s}=\sum_k \lambda_{\ell k}  \psi_{sk}$, and $\sigma'_{\upsilon \ell} \upsilon'_{n\ell}= \sigma_{\upsilon \ell} \upsilon_{n\ell} + \sum_k \lambda_{\ell k}\sigma_{\omega k} \omega_{nk}$.



In order to normalize the model, we associate each latent variable $k$ with one specific indicator $\ell_k$,  different across latent variables.
The following normalization can be done:
\begin{itemize}
\item As the error terms of the structural and measurement equations are confounded, we set $\sigma_{\omega k}=0$, for each latent variable $k$.
\item As the units of the latent variables are arbitrary, we set the coefficient of latent variable to one\footnote{Or $-1$ if the statement implies a decrease of the latent variable, as illustrated in the case study.}  in the corresponding measurement equation, and the scale parameter to zero: $\lambda_{\ell_k k}=1$, $\sigma_{\upsilon \ell_k}=0$ for each latent variable $k$.
\end{itemize}
Note that the structural equations do not include an intercept, so that there is no need to mormalize the intercept of the corresponding measurement equation. 


In contrast, this simplification does not apply when the indicators
are discrete. In that case, the measurement equations involve
threshold-crossing representations that express the probability
\req{eq:discreteMeas-d} of each observed category as differences of
normal cumulative distribution functions. These nonlinear functions of
the latent variables break the convenient normal–linear structure:
the conditional distribution of the indicators given $x_n^*$ is no
longer Gaussian. As a consequence, the integral over $x_n^*$ in the
likelihood cannot be evaluated in closed form.

\section{Choice Model with Latent Variables}

Consider a discrete choice model. We illustrate the methodology using a logit specification for the choice component, although other models --- such as the nested logit or cross-nested logit --- could also be employed. Under the logit framework, the probability that individual \( n \) chooses alternative \( i \) is given by
\begin{equation}
  \label{eq:logit}
  P(i \mid x_n; \beta) = \frac{e^{\mu V_{in}(x_n; \beta)}}{\sum_{j \in \mathcal{C}_n} e^{\mu V_{jn}(x_n; \beta)}},
\end{equation}
where \( \mu \) is the (unknown) scale parameter, and \( V_{in}(x_n; \beta) \) denotes the systematic utility associated with alternative \( i \) for individual \( n \), as a function of observed explanatory variables \( x_n \) and a parameter vector \( \beta \).

To enhance the behavioral realism of the model, we extend the utility specification by incorporating latent variables. This leads to the following choice probability conditional on the latent variables \( x_n^* \):
\begin{equation}
  \label{eq:choice_latent}
  P(i \mid x_n^*, x_n; \beta) = \frac{e^{\mu V_{in}(x_n^*, x_n; \beta)}}{\sum_{j \in \mathcal{C}_n} e^{\mu V_{jn}(x_n^*, x_n; \beta)}},
\end{equation}
where \( x_n^* \) denotes the vector of latent psychological constructs influencing choice behavior.

From a modeling perspective, the latent variables are included in the utility function in the same way as observed covariates. For this reason, it is useful to first specify the choice model assuming that the latent variables are observed. Once this conditional model is established, the latent nature of these variables can be addressed.

The first approach relies on the structural equations defined in Equation~\eqref{eq:linearStructural}, with parameter values estimated from the MIMIC model:
\begin{equation}
 \widehat{x}_{nk}^* = \widehat{\psi}_{0k} + \sum_s \widehat{\psi}_{sk} x_{ns} + \sigma_{\omega k} \omega_{nk},
\end{equation}
where \( \widehat{\psi} \) is the vector of estimated parameters of the structural equations. Note that the scale parameters \( \sigma_{\omega k} \) cannot be identified in the MIMIC model and are therefore normalized to 1, as they are confounded with the scale of the choice model.

Given this specification, the choice model becomes, conditional on the structural error terms \( \omega_n \):
\begin{equation}
P(i \mid \omega_n, x_n; \beta) = \frac{e^{\mu V_{in}(\omega_n, x_n; \beta)}}{\sum_{j \in \mathcal{C}_n} e^{\mu V_{jn}(\omega_n, x_n; \beta)}},
\end{equation}
where \( \omega_n \) is the vector of normally distributed structural errors.

To obtain the unconditional choice probability, we integrate over the distribution of the latent variables. This results in a mixture of logit models:
\begin{equation}
P(i \mid x_n; \beta) = \int_{\omega_n} \frac{e^{\mu V_{in}(\omega_n, x_n; \beta)}}{\sum_{j \in \mathcal{C}_n} e^{\mu V_{jn}(\omega_n, x_n; \beta)}} \phi(\omega_n) \, d\omega_n,
\end{equation}
where \( \phi(\cdot) \) is the probability density function of the standard normal distribution.
This integral is approximatied using Monte-Carlo integration.

This sequential estimation approach --- first estimating the MIMIC model, then the choice model --- yields consistent parameter estimates. However, it is not statistically efficient. Intuitively, observed choices provide additional information about the latent variables, which is not exploited in a two-step procedure. We now turn to the simultaneous estimation of all components --- structural equations, measurement equations, and the choice model --- based on the full likelihood function. This integrated approach leverages all available information and improves the statistical efficiency of the estimates.

In this context, the likelihood function incorporates the complete set of observations for each individual. This includes not only the observed choice but also responses to the psychometric indicators. Conditional on the latent variables, the contribution of individual \( n \) to the likelihood function is the product of the ordered probit probabilities for each indicator, as specified in Equation~\eqref{eq:discreteMeas-d}, and the choice probability given in Equation~\eqref{eq:choice_latent}:
\[
P(i_n \mid x_n^*, x_n; \beta) \prod_\ell \mathbb{P}(I_{n\ell} = j_m \mid x_n^*, x_n; \lambda_\ell, \Sigma_{\upsilon \ell}).
\]

Substituting the structural equations \eqref{eq:linearStructural}, the expression becomes conditional on \( \omega_n \), and involves now the parameters of the structural equations as well:
\[
P(i_n \mid \omega_n, x_n; \beta, \psi, \sigma) \prod_\ell \mathbb{P}(I_{n\ell} = j_m \mid \omega_n, x_n; \lambda_\ell, \Sigma_{\upsilon \ell}, \psi, \sigma).
\]

To obtain the full contribution of individual \( n \) to the likelihood function, we integrate this expression over the distribution of the latent variables:
\[
\int_{\omega_n} P(i_n \mid x_n^*, x_n; \beta) \prod_\ell \mathbb{P}(I_{n\ell} = j_m \mid x_n^*, x_n; \lambda_\ell, \Sigma_{\upsilon \ell}) \phi(\omega_n) \, d\omega_n,
\]
where \( \phi(\cdot) \) is the standard normal density. As this integral has no closed form, it is approximated via Monte Carlo integration.

\section{A case study}
\label{sec:example}
This example focuses on the estimation of a mode choice model for
residents of Switzerland, using revealed preference data. The data
were collected as part of a research project aimed to assess the market potential
of combined mobility solutions --- particularly in urban agglomerations --- by
identifying the factors that influence individuals in their choice of
transport mode (\cite{OptimaRP2011}).

The survey was conducted between 2009 and 2010 on behalf of CarPostal,
the public transport operator of the Swiss Postal Service. Its primary
objective was to collect data on travel behavior in low-density areas,
which represent the typical service environment of CarPostal. In
addition to revealed preference data, the survey includes several
psychometric indicators, enabling the incorporation of latent
variables into the model specification.

The data file as well as its description is available on the \href{http://biogeme.epfl.ch/#data}{Biogeme webpage}.

We first  estimate a MIMIC model involving two latent variables. The
first one captures a ``car-centric'' attitude. The second one captures
a ``urban preference attitude''. The car-centric attitude captures the
extent to which individuals exhibit a strong preference for private
car use as their primary mode of transportation. This latent construct
reflects values such as independence, flexibility, comfort, and
perceived status associated with driving. Individuals with a high
car-centric attitude are more likely to perceive cars as the most
practical and desirable means of travel, often resisting modal shift
to public transport or active mobility. The urban preference attitude
represents the degree to which individuals value characteristics
associated with dense, mixed-use urban environments. This includes a
positive perception of walkability, access to local services,
efficient public transport, and vibrant public spaces. This attitude
is typically associated with environmental awareness, social
interaction, and a preference for compact urban living.

\subsection{Psychometric indicators}
\label{sec:psycho}
The psychmometric indicators selected to capture the car-centric attitude are:
\begin{description}
\item[Envir01]  Fuel price should be increased to reduce congestion and air pollution.
\item[Envir02]   More public transportation is needed, even if taxes are set to pay the additional costs.
\item[Envir03]  Ecology disadvantages minorities and small businesses.
\item[Envir04]  People and employment are more important than the environment.
\item[Mobil09]  Taking the bus helps making the city more comfortable and welcoming.
\item[Mobil11]  It is difficult to take the public transport when I carry bags or luggage. 
\item[Mobil14]  When I take the car I know I will be on time.
\item[Mobil16]  I do not like changing the mean of transport when I am traveling.
\item[Mobil17]  If I use public transportation I have to cancel certain activities I would have done if I had taken the car. 
\item[LifSty08] For me the car is only a practical way to move.
\end{description}
The psychmometric indicators selected to capture the urban preference attitude are:
\begin{description}
\item[ResidCh01] I like living in a neighborhood where a lot of things happen. 
\item[ResidCh02] The accessibility and mobility conditions are important for the choice of housing.
\item[ResidCh03] Most of my friends live in the same region I live in.
\item[ResidCh05] I would like to live in the city center of a big city.
\item[ResidCh06] I would like to live in a town situated in the outskirts of a city. 
\item[ResidCh07] I would like to live in the countryside.
\item[Mobil07] In general, for my activities, I always have a usual mean of transport.
\item[Mobil24] I have always used public transports all my life.
\end{description}

\subsection{Structural equations}

For the structural equations, we use the linear specification \req{eq:linearStructural} with the following explanatory variables.
For the car-centric attitude $x_{n,\text{car}}^*$, we have:
\begin{itemize}
\item \lstinline $age_65_more = age >= 65$
\item \lstinline $ScaledIncome = CalculatedIncome / 1000$
\item \lstinline $moreThanOneCar = NbCar > 1$
\item \lstinline $moreThanOneBike = NbBicy > 1$
\item \lstinline $individualHouse = HouseType == 1$
\item \lstinline $haveChildren = (FamilSitu == 3) + (FamilSitu == 4) > 0$
\item \lstinline $haveGA = GenAbST == 1$
\item \lstinline $highEducation = Education >= 6$
\end{itemize}
And for the urban preference attitude $x_{n,\text{urban}}^*$, we have:
\begin{itemize}
\item \lstinline $childCenter = ((ResidChild == 1) + (ResidChild == 2)) > 0$
\item \lstinline $childSuburb = ((ResidChild == 3) + (ResidChild == 4)) > 0$
\item \lstinline $highEducation = Education >= 6$
\item \lstinline $artisans = SocioProfCat == 5$
\item \lstinline $employees = SocioProfCat == 6$
\item \lstinline $age_30_less = age <= 30$
\item \lstinline $haveChildren = (FamilSitu == 3) + (FamilSitu == 4) > 0$
\item \lstinline $UrbRur$
\item \lstinline $individualHouse = HouseType == 1$
\end{itemize}

\subsection{Measurement equations}

For each individual \( n \) and each indicator \( \ell \) described in
Section~\ref{sec:psycho}, we introduce a latent continuous response
variable, as outlined in Section~\ref{sec:likert}. This latent
response captures the unobserved propensity underlying the observed
ordinal response on a Likert scale.

For the indicators associated with the car-centric attitude, the latent response is modeled as:
\begin{equation}
  I^*_{n\ell} = \lambda_{0\ell} + \lambda_{1\ell} x^*_{n,\text{car}} + \lambda_{2\ell} \upsilon_{n\ell},
\end{equation}
where \( \lambda_{0\ell} \) is an intercept term, \( \lambda_{1\ell} \) is the loading on the latent variable \( x^*_{n,\text{car}} \), \( \lambda_{2\ell} \) scales the stochastic component, and \( \upsilon_{n\ell} \) is a random error term.

The indicator \texttt{Envir01} is selected for the normalization of
the measurement model. Individuals with a stronger car-centric
attitude are expected to be more likely to \emph{disagree} with the
corresponding statement. Accordingly, the loading \( \lambda_{1\ell}
\) is expected to be negative, and fixed to \(-1\) to establish the direction of the latent
construct. The scale parameter \( \lambda_{2\ell} \) is normalized to
1 to ensure identifiability of the model.

Similarly, for the indicators capturing the urban-preference attitude, we specify:
\begin{equation}
  I^*_{n\ell} = \lambda_{0\ell} + \lambda_{1\ell} x^*_{n,\text{urban}} + \lambda_{2\ell} \upsilon_{n\ell},
\end{equation}
with analogous interpretation of the parameters.

The indicator \texttt{ResidCh01} is selected for the normalization of
this measurement model. Individuals with a stronger urban-preference
attitude are expected to be more likely to \emph{agree} with the
corresponding statement. Accordingly, the loading \( \lambda_{1\ell}
\) is expected to be positive, and fixed to 1 to establish the
direction of the latent construct. The scale parameter \(
\lambda_{2\ell} \) is normalized to 1 to ensure identifiability of the
model.

The thresholds for the ordered probit model are defined as described in Section~\ref{sec:likert}:
\[
\begin{aligned}
  \tau_1 &= -\delta_1 - \delta_2, \\
  \tau_2 &= -\delta_1, \\
  \tau_3 &= \delta_1, \\
  \tau_4 &= \delta_1 + \delta_2,
\end{aligned}
\]
where \( \delta_1 > 0 \) and \( \delta_2 > 0 \) are estimated.

\subsection{Implementation notes}

The Biogeme implementation is structured across three dedicated files:
one defining the variables associated with each latent construct, one
specifying the structural equations, and one containing the
measurement equations. These files are not only used in the estimation
of the MIMIC model, but are also employed in the subsequent estimation
of choice models incorporating latent variables. This design ensures
full consistency in the specification of the structural and
measurement equations across all stages of the analysis.

The file defining the relevant variables is described in
Section~\ref{sec:relevant_data.py}. It specifies two sets,
\lstinline$car_indicators$ and \lstinline$urban_indicators$, which
list the indicators associated with each latent variable. It also
identifies the indicators used for the normalization of the model. In
addition, the file defines two dictionaries, each mapping a latent
variable to the set of explanatory variables used in its corresponding
structural equation. These dictionaries associate variable names with
Biogeme expressions used to compute them.

The file defining the structural equations is described in
Section~\ref{sec:structural_equations.py}. It includes two
functions --- one for each latent variable --- that construct the
corresponding structural equations. These functions operate in two
modes. When called without arguments, they return Biogeme expressions
involving parameters to be estimated. This mode is used, for example,
in the MIMIC model. Alternatively, if a dictionary of parameter values
is provided as input, the functions substitute the given values and
treat them as fixed, enabling evaluation based on previously estimated
parameters. This mode is used for the sequential estimation of the choice model, for example. 


The file containing the measurement equations is described in
Section~\ref{sec:measurement_equations.py}. It begins by organizing
the parameters to be estimated into dictionaries: one for the
intercepts (\lstinline$intercepts$), two for the coefficients
associated with each latent variable (\lstinline$car_coefficients$ and
\lstinline$urban_coefficients$), and one for the scale parameters
(\lstinline$sigma_star$). The required normalizations are also applied
at this stage.

The function \lstinline$generate_model_terms$ is responsible for
constructing the expression representing the contribution of the
latent variables to the measurement equation for a given indicator. It
is designed to accommodate the possibility that an indicator may be
influenced by multiple latent variables, even though this situation
does not arise in the present example.

Finally, the function \lstinline$generate_measurement_equations$
produces the complete set of measurement equations in the form of a
dictionary. This dictionary maps each potential response value of an
indicator to the corresponding expression for its likelihood
contribution.


\subsection{The MIMIC model}

With all generic components defined, we now turn to the script used to
estimate the parameters of the MIMIC model, presented in
Section~\ref{sec:plot_b01_mimic.py}. After retrieving the relevant
elements discussed above, the script constructs the joint log
likelihood function of all observed indicators. This is done by first
creating a dictionary that maps each indicator to its individual log
likelihood contribution, which are then aggregated into a single
expression representing the total log likelihood.

This combined likelihood expression is subsequently linked to the
database using Biogeme, enabling the maximum likelihood estimation of
the model parameters.
The general statistics of the
estimation are reported in Table~\vref{tab:mimic_stats}. The estimated
parameters of the structural equations are reported in
Table~\vref{tab:mimic_params_struct}. The estimated parameters of the
measurement equations are reported in
Table~\vref{tab:mimic_params_car_meas} for the car-centric attitude,
and in Table~\vref{tab:mimic_params_urban_meas} for the
urban-preference attitude. Finally, the estimated values of the
differences between thresholds are reported in
Table~\vref{tab:mimic_thresholds}.
\begin{table}[htb]
  \begin{center}
    \footnotesize
\begin{tabular}{ll}
Number of estimated parameters & 69 \\
Sample size & 1899 \\
Init log likelihood & -109217.9 \\
Final log likelihood & -43773.72 \\
Akaike Information Criterion & 87685.45 \\
Bayesian Information Criterion & 88068.33 \\
\end{tabular}
  \caption{ \label{tab:mimic_stats}Mimic model: general statistics}
  \end{center}

\end{table}

\begin{table}[htb]
    \footnotesize
  \begin{center}
\begin{tabular}{rlr@{.}lr@{.}lr@{.}lr@{.}l}
  &              &   \multicolumn{2}{l}{}         & \multicolumn{2}{l}{Robust}  &  \multicolumn{4}{l}{}  \\
  Parameter & Description & \multicolumn{2}{l}{Coeff.} & \multicolumn{2}{l}{Asympt.} & \multicolumn{2}{l}{$t$-stat} & \multicolumn{2}{l}{$p$-value} \\
  number    &             & \multicolumn{2}{l}{estimate} & \multicolumn{2}{l}{std. error} & \multicolumn{2}{l}{} & \multicolumn{2}{l}{} \\
  \hline
1 & car\_struct\_age\_65\_more & 0&158 & 0&0664 & 2&38 & 0&0174 \\ 
2 & car\_struct\_ScaledIncome & -0&0333 & 0&00707 & -4&71 & 2&53e-06 \\ 
3 & car\_struct\_moreThanOneCar & 0&614 & 0&0584 & 10&5 & 0&0 \\ 
4 & car\_struct\_moreThanOneBike & -0&341 & 0&0584 & -5&84 & 5&33e-09 \\ 
5 & car\_struct\_individualHouse & -0&0939 & 0&0517 & -1&82 & 0&0693 \\ 
6 & car\_struct\_haveChildren & -0&044 & 0&0502 & -0&875 & 0&382 \\ 
7 & car\_struct\_haveGA & -0&599 & 0&0843 & -7&1 & 1&23e-12 \\ 
8 & car\_struct\_highEducation & -0&328 & 0&0585 & -5&61 & 1&99e-08 \\ 
9 & urban\_struct\_childCenter & 0&0979 & 0&0294 & 3&33 & 0&00087 \\ 
10 & urban\_struct\_childSuburb & 0&0892 & 0&0239 & 3&72 & 0&000196 \\ 
11 & urban\_struct\_highEducation & 0&0341 & 0&0153 & 2&24 & 0&0253 \\ 
12 & urban\_struct\_artisans & -0&0986 & 0&035 & -2&81 & 0&00491 \\ 
13 & urban\_struct\_employees & -0&0398 & 0&0174 & -2&28 & 0&0224 \\ 
14 & urban\_struct\_age\_30\_less & 0&162 & 0&0594 & 2&72 & 0&00645 \\ 
15 & urban\_struct\_haveChildren & -0&0246 & 0&0124 & -1&99 & 0&0466 \\ 
16 & urban\_struct\_UrbRur & 0&104 & 0&0333 & 3&13 & 0&00175 \\ 
17 & urban\_struct\_IndividualHouse & 0&0277 & 0&014 & 1&98 & 0&0475 \\ 
\end{tabular}
\caption{MIMIC model: estimated parameters of the structural equations\label{tab:mimic_params_struct}}
  \end{center}
\end{table}

\begin{table}[htb]
    \footnotesize
  \begin{center}
\begin{tabular}{rlr@{.}lr@{.}lr@{.}lr@{.}l}
  &              &   \multicolumn{2}{l}{}         & \multicolumn{2}{l}{Robust}  &  \multicolumn{4}{l}{}  \\
  Parameter & Description & \multicolumn{2}{l}{Coeff.} & \multicolumn{2}{l}{Asympt.} & \multicolumn{2}{l}{$t$-stat} & \multicolumn{2}{l}{$p$-value} \\
  number    &             & \multicolumn{2}{l}{estimate} & \multicolumn{2}{l}{std. error} & \multicolumn{2}{l}{} & \multicolumn{2}{l}{} \\
  \hline
1 & meas\_intercept\_Envir01 & -0&829 & 0&0739 & -11&2 & 0&0 \\ 
2 & meas\_intercept\_Envir02 & 0&42 & 0&0309 & 13&6 & 0&0 \\ 
3 & car\_meas\_b\_Envir02 & -0&45 & 0&0557 & -8&08 & 6&66e-16 \\ 
4 & meas\_sigma\_star\_Envir02 & 0&945 & 0&0228 & 41&4 & 0&0 \\ 
5 & meas\_intercept\_Envir03 & -0&393 & 0&0337 & -11&6 & 0&0 \\ 
6 & car\_meas\_b\_Envir03 & 0&647 & 0&0592 & 10&9 & 0&0 \\ 
7 & meas\_sigma\_star\_Envir03 & 0&879 & 0&0212 & 41&4 & 0&0 \\ 
8 & meas\_intercept\_Envir04 & -0&543 & 0&0313 & -17&3 & 0&0 \\ 
9 & car\_meas\_b\_Envir04 & 0&38 & 0&0601 & 6&32 & 2&67e-10 \\ 
10 & meas\_sigma\_star\_Envir04 & 0&811 & 0&0204 & 39&7 & 0&0 \\ 
11 & meas\_intercept\_Mobil09 & 0&822 & 0&0311 & 26&4 & 0&0 \\ 
12 & car\_meas\_b\_Mobil09 & -0&34 & 0&0511 & -6&65 & 2&85e-11 \\ 
13 & meas\_sigma\_star\_Mobil09 & 0&873 & 0&0241 & 36&2 & 0&0 \\ 
14 & meas\_intercept\_Mobil11 & 0&408 & 0&035 & 11&6 & 0&0 \\ 
15 & car\_meas\_b\_Mobil11 & 0&541 & 0&059 & 9&17 & 0&0 \\ 
16 & meas\_sigma\_star\_Mobil11 & 0&963 & 0&0249 & 38&7 & 0&0 \\ 
17 & meas\_intercept\_Mobil14 & -0&154 & 0&0308 & -5&0 & 5&66e-07 \\ 
18 & car\_meas\_b\_Mobil14 & 0&615 & 0&056 & 11&0 & 0&0 \\ 
19 & meas\_sigma\_star\_Mobil14 & 0&85 & 0&021 & 40&6 & 0&0 \\ 
20 & meas\_intercept\_Mobil16 & 0&167 & 0&0331 & 5&04 & 4&6e-07 \\ 
21 & car\_meas\_b\_Mobil16 & 0&478 & 0&0568 & 8&42 & 0&0 \\ 
22 & meas\_sigma\_star\_Mobil16 & 0&93 & 0&0227 & 41&0 & 0&0 \\ 
23 & meas\_intercept\_Mobil17 & 0&179 & 0&032 & 5&59 & 2&24e-08 \\ 
24 & car\_meas\_b\_Mobil17 & 0&404 & 0&0567 & 7&13 & 1&01e-12 \\ 
25 & meas\_sigma\_star\_Mobil17 & 0&934 & 0&0239 & 39&1 & 0&0 \\ 
26 & meas\_intercept\_LifSty08 & 1&22 & 0&04 & 30&5 & 0&0 \\ 
27 & car\_meas\_b\_LifSty08 & -0&227 & 0&062 & -3&66 & 0&000251 \\ 
28 & meas\_sigma\_star\_LifSty08 & 0&957 & 0&0317 & 30&2 & 0&0 \\ 
\end{tabular}
  \caption{MIMIC model: estimated parameters of the measurement equations for the car-centric attitude\label{tab:mimic_params_car_meas}}
  \end{center}
\end{table}

\begin{table}[htb]
    \footnotesize
  \begin{center}
\begin{tabular}{rlr@{.}lr@{.}lr@{.}lr@{.}l}
  &              &   \multicolumn{2}{l}{}         & \multicolumn{2}{l}{Robust}  &  \multicolumn{4}{l}{}  \\
  Parameter & Description & \multicolumn{2}{l}{Coeff.} & \multicolumn{2}{l}{Asympt.} & \multicolumn{2}{l}{$t$-stat} & \multicolumn{2}{l}{$p$-value} \\
  number    &             & \multicolumn{2}{l}{estimate} & \multicolumn{2}{l}{std. error} & \multicolumn{2}{l}{} & \multicolumn{2}{l}{} \\
  \hline
1 & meas\_intercept\_ResidCh01 & -0&591 & 0&0682 & -8&67 & 0&0 \\ 
2 & meas\_intercept\_ResidCh02 & 1&18 & 0&177 & 6&66 & 2&71e-11 \\ 
3 & urban\_meas\_b\_ResidCh02 & 1&53 & 0&456 & 3&35 & 0&000815 \\ 
4 & meas\_sigma\_star\_ResidCh02 & 0&95 & 0&0245 & 38&8 & 0&0 \\ 
5 & meas\_intercept\_ResidCh03 & -0&284 & 0&224 & -1&27 & 0&205 \\ 
6 & urban\_meas\_b\_ResidCh03 & -1&24 & 0&582 & -2&13 & 0&0328 \\ 
7 & meas\_sigma\_star\_ResidCh03 & 0&959 & 0&0223 & 43&1 & 0&0 \\ 
8 & meas\_intercept\_ResidCh05 & -0&388 & 0&298 & -1&3 & 0&194 \\ 
9 & urban\_meas\_b\_ResidCh05 & 3&02 & 0&775 & 3&89 & 9&88e-05 \\ 
10 & meas\_sigma\_star\_ResidCh05 & 1&15 & 0&0417 & 27&6 & 0&0 \\ 
11 & meas\_intercept\_ResidCh06 & 0&873 & 0&407 & 2&15 & 0&0318 \\ 
12 & urban\_meas\_b\_ResidCh06 & 3&59 & 1&05 & 3&43 & 0&000603 \\ 
13 & meas\_sigma\_star\_ResidCh06 & 1&06 & 0&0267 & 39&8 & 0&0 \\ 
14 & meas\_intercept\_ResidCh07 & -0&156 & 0&338 & -0&461 & 0&645 \\ 
15 & urban\_meas\_b\_ResidCh07 & -3&25 & 0&879 & -3&69 & 0&000223 \\ 
16 & meas\_sigma\_star\_ResidCh07 & 0&929 & 0&0283 & 32&8 & 0&0 \\ 
17 & meas\_intercept\_Mobil07 & 0&633 & 0&117 & 5&41 & 6&41e-08 \\ 
18 & urban\_meas\_b\_Mobil07 & -0&607 & 0&298 & -2&04 & 0&0415 \\ 
19 & meas\_sigma\_star\_Mobil07 & 0&833 & 0&0258 & 32&4 & 0&0 \\ 
20 & meas\_intercept\_Mobil24 & 0&841 & 0&203 & 4&14 & 3&46e-05 \\ 
21 & urban\_meas\_b\_Mobil24 & 1&45 & 0&524 & 2&77 & 0&00566 \\ 
22 & meas\_sigma\_star\_Mobil24 & 1&12 & 0&0288 & 38&8 & 0&0 \\ 
\end{tabular}
  \caption{MIMIC model: estimated parameters of the measurement equations for the urban-preference attitude\label{tab:mimic_params_urban_meas}}
  \end{center}
\end{table}

\begin{table}[htb]
    \footnotesize
  \begin{center}
\begin{tabular}{rlr@{.}lr@{.}lr@{.}lr@{.}l}
  &              &   \multicolumn{2}{l}{}         & \multicolumn{2}{l}{Robust}  &  \multicolumn{4}{l}{}  \\
  Parameter & Description & \multicolumn{2}{l}{Coeff.} & \multicolumn{2}{l}{Asympt.} & \multicolumn{2}{l}{$t$-stat} & \multicolumn{2}{l}{$p$-value} \\
  number    &             & \multicolumn{2}{l}{estimate} & \multicolumn{2}{l}{std. error} & \multicolumn{2}{l}{} & \multicolumn{2}{l}{} \\
  \hline
  1 & $\delta\_1$ & 0&307 & 0&00614 & 49&9 & 0&0 \\ 
2 & $\delta\_2$ & 0&936 & 0&0167 & 55&9 & 0&0 \\ 

\end{tabular}
  \caption{MIMIC model: thresholds of the measurement equations\label{tab:mimic_thresholds}}
  \end{center}
\end{table}

\clearpage
\subsection{The choice model}

The transportation mode choice model developed for the case study presented in Section~\ref{sec:example} considers three alternatives:
\begin{itemize}
\item public transportation,
\item private car,
\item slow modes (e.g., walking, biking).
\end{itemize}
The utility functions are specified as linear combinations of explanatory variables and include alternative-specific constants (intercepts). The full specification is provided in Table~\ref{tab:spec_choice}, where
\begin{itemize}
\item \lstinline $MarginalCostPT_scaled = MarginalCostPT / 10$,
\item \lstinline $CostCarCHF_scaled = CostCarCHF / 10$,
\item \lstinline $PurpHWH = TripPurpose == 1$,
\item \lstinline $PurpOther = TripPurpose == 1$,
\item \lstinline $TimePT_scaled = TimePT / 200$,
\item \lstinline $TimeCar_scaled = TimeCar / 200$,
\item \lstinline $distance_km_scaled = distance_km / 5$.
\end{itemize}


The choice model adopts the logit formulation \req{eq:logit}.
The model is normalized using a moneymetric specification: the coefficient of the cost variable, interacted with the ``home–work–home'' trip purpose, is fixed to \(-1\). This ensures that the utility scale is expressed in monetary units.

\begin{table}[htb]
  \begin{tabular}{cccc}
Parameter    & Pub. transp. & Car & Slow modes \\
    \hline
    \hline
\lstinline$asc_pt$ &    1 &   0 &  0 \\
    \hline
\lstinline$asc_car$ &   0 &   1 &  0 \\
    \hline
\lstinline$beta_cost_hwh$ & \lstinline$MarginalCostPT_scaled$ & \lstinline$CostCarCHF_scaled$ & 0 \\
 & \lstinline$* PurpHWH$ & \lstinline$ * PurpHWH$ & \\
    \hline
\lstinline$beta_cost_other$  & \lstinline$MarginalCostPT_scaled $ & \lstinline$CostCarCHF_scaled$ & 0 \\
& \lstinline$* PurpOther$ & \lstinline$* PurpOther$& 0 \\
    \hline
\lstinline$beta_time_pt$ & \lstinline$TimePT_scaled$ & 0 & 0 \\
    \hline
\lstinline$beta_waiting_time$ & \lstinline$WaitingTimePT$ & 0 & 0 \\
    \hline
\lstinline$beta_time_car$ & 0 & \lstinline$TimeCar_scaled$ & 0\\
    \hline
\lstinline$beta_dist$ & 0 & 0 & \lstinline$distance_km_scaled$ \\
    \hline
  \end{tabular}
  \caption{\label{tab:spec_choice}Specification of the utility functions of the choice model}
\end{table}

The script is described in Section~\ref{sec:plot_b02_choice_only.py} and the results of the estimation are reported in Table~\vref{tab:choice_params}.


\begin{table}[htb]
    \footnotesize
  \begin{center}
\begin{tabular}{rlr@{.}lr@{.}lr@{.}lr@{.}l}
          &              &   \multicolumn{2}{l}{}         & \multicolumn{2}{l}{Robust}  &  \multicolumn{4}{l}{}  \\
Parameter &              &   \multicolumn{2}{l}{Coeff.}   & \multicolumn{2}{l}{Asympt.}       & \multicolumn{4}{l}{}   \\
number    &  Description &   \multicolumn{2}{l}{estimate} & \multicolumn{2}{l}{std. error}    & \multicolumn{2}{l}{$t$-stat}  &  \multicolumn{2}{l}{$p$-value} \\
\hline
1 & choice\_asc\_pt & -0&132 & 0&262 & -0&505 & 0&614 \\ 
2 & choice\_asc\_car & 0&408 & 0&283 & 1&44 & 0&149 \\ 
3 & choice\_beta\_cost\_other & -0&465 & 0&142 & -3&27 & 0&00106 \\ 
4 & choice\_beta\_time\_pt & -1&58 & 0&658 & -2&39 & 0&0167 \\ 
5 & choice\_beta\_time\_car & -5&14 & 1&67 & -3&08 & 0&00209 \\ 
6 & choice\_beta\_waiting\_time & -0&018 & 0&00789 & -2&28 & 0&0228 \\ 
7 & choice\_beta\_dist & -0&983 & 0&317 & -3&1 & 0&00192 \\
8 & scale\_choice\_model & 1&17 & 0&275 & 4&24 & 2&24e-05 \\ 
\end{tabular}
  \end{center}
\begin{tabular}{ll}
Number of estimated parameters & 8 \\
Sample size & 1899 \\
Final log likelihood & -1230.014 \\
Akaike Information Criterion & 2476.028 \\
Bayesian Information Criterion & 2520.421 \\
\end{tabular}
    \caption{Choice model: estimated parameters\label{tab:choice_params}}
\end{table}

\clearpage

\subsection{Sequential estimation}
\label{sec:sequential}
The objective now is to incorporate the latent variables estimated in
the MIMIC model into the choice model. We investigate a specification
in which the latent variables interact with the alternative-specific
constants. This specification, summarized in
Table~\vref{tab:spec_choice_lv}, involves random components --- namely,
the two latent variables --- and thus transforms the choice model into a
\emph{mixture of logit models}.

\begin{table}[htb]
  \begin{tabular}{cccc}
Parameter    & Pub. transp. & Car & Slow modes \\
    \hline
    \hline
\lstinline$asc_pt$ &    1 &   0 &  0 \\
    \hline
\lstinline$asc_car$ &   0 &   1 &  0 \\
    \hline
\lstinline$beta_cost_hwh$ & \lstinline$MarginalCostPT_scaled$ & \lstinline$CostCarCHF_scaled$ & 0 \\
 & $\times$ \lstinline$PurpHWH$ & $\times$\lstinline$  PurpHWH$ & \\
    \hline
\lstinline$beta_cost_other$  & \lstinline$MarginalCostPT_scaled $ & \lstinline$CostCarCHF_scaled$ & 0 \\
& $\times$ \lstinline$PurpOther$ & $\times$ \lstinline$PurpOther$& 0 \\
    \hline
\lstinline$beta_time_pt$ & \lstinline$TimePT_scaled$ & 0 & 0 \\
    \hline
\lstinline$beta_waiting_time$ & \lstinline$WaitingTimePT$ & 0 & 0 \\
    \hline
\lstinline$beta_time_car$ & 0 & \lstinline$TimeCar_scaled$ & 0\\
    \hline
\lstinline$beta_dist$ & 0 & 0 & \lstinline$distance_km_scaled$ \\
\hline
\lstinline$car_centric_pt_cte$ & $x_{n,\text{car}}^*$ & 0 & 0 \\
\lstinline$car_centric_car_cte$ &  0 & $x_{n,\text{car}}^*$ & 0 \\
\lstinline$urban_life_pt_cte$ & $x_{n,\text{urban}}^*$ & 0 & 0  \\
\lstinline$urban_life_car_cte$ & 0 & $x_{n,\text{urban}}^*$ & 0\\
  \end{tabular}
  \caption{\label{tab:spec_choice_lv}Specification of the utility functions of the choice model with latent variables}
\end{table}

The script is presented in
Section~\ref{sec:plot_b03_sequential.py}. The latent variables are
constructed using two components: the deterministic part derived from
the structural equations with parameter values estimated from the
MIMIC model, and a stochastic term capturing unobserved heterogeneity:
\begin{lstlisting}
car_centric_attitude = build_car_centric_attitude(
    estimated_parameters=struct_betas
) + Draws('car_error_term', 'NORMAL_MLHS_ANTI')
\end{lstlisting}

Since the latent variables are integrated out via Monte Carlo methods,
the random term is defined using a draw generator. In this case, the
generator produces draws from a standard normal distribution using
Modified Latin Hypercube Sampling (MLHS), a variance-reduction
technique that ensures a more uniform coverage of the distribution's
support (\cite{Hess:2006aa}). The keyword \lstinline$ANTI$ specifies
the use of antithetic draws, meaning that for each draw \( \omega \),
its symmetric counterpart \( -\omega \) is also included. This
technique improves numerical stability and accelerates convergence by
reducing the variance of the estimator.

Once constructed, the latent variables are incorporated into the
utility functions exactly like any other explanatory variable. The
estimated parameters for this sequential estimation are reported in
Table~\vref{tab:choice_sequential}.


\begin{table}[htb]
  \footnotesize
  \begin{center}
\begin{tabular}{rlr@{.}lr@{.}lr@{.}lr@{.}l}
          &              &   \multicolumn{2}{l}{}         & \multicolumn{2}{l}{Robust}  &  \multicolumn{4}{l}{}  \\
Parameter &              &   \multicolumn{2}{l}{Coeff.}   & \multicolumn{2}{l}{Asympt.}       & \multicolumn{4}{l}{}   \\
number    &  Description &   \multicolumn{2}{l}{estimate} & \multicolumn{2}{l}{std. error}    & \multicolumn{2}{l}{$t$-stat}  &  \multicolumn{2}{l}{$p$-value} \\
\hline
1 & choice\_asc\_pt & 0&436 & 0&417 & 1&05 & 0&295 \\ 
2 & choice\_asc\_car & 1&55 & 0&588 & 2&64 & 0&00824 \\ 
3 & choice\_beta\_cost\_other & -0&416 & 0&148 & -2&81 & 0&00492 \\ 
4 & choice\_beta\_time\_pt & -1&95 & 0&865 & -2&26 & 0&024 \\ 
5 & choice\_beta\_waiting\_time & -0&02 & 0&0101 & -1&98 & 0&0476 \\ 
6 & choice\_beta\_time\_car & -5&58 & 2&03 & -2&75 & 0&00602 \\ 
7 & choice\_beta\_dist & -1&13 & 0&407 & -2&77 & 0&00562 \\ 
8 & choice\_car\_centric\_pt\_cte & 0&652 & 0&265 & 2&46 & 0&014 \\ 
9 & choice\_urban\_life\_pt\_cte & 0&0223 & 0&0826 & 0&27 & 0&787 \\ 
10 & choice\_car\_centric\_car\_cte & 1&62 & 0&52 & 3&11 & 0&00187 \\ 
11 & choice\_urban\_life\_car\_cte & -0&338 & 0&18 & -1&87 & 0&061 \\ 
12 & scale\_choice\_model & 1&11 & 0&298 & 3&72 & 0&0002 \\ 
\end{tabular}
  \end{center}
\begin{flushleft}
\begin{tabular}{ll}
Number of estimated parameters & 12 \\
Sample size & 1899 \\
Final log likelihood & -1191.766 \\
Akaike Information Criterion & 2407.532 \\
Bayesian Information Criterion & 2474.121 \\
Number of draws & 10000 \\
\end{tabular}
\end{flushleft}
  \caption{Choice model with latent variables: sequential estimation\label{tab:choice_sequential}}
\end{table}

\clearpage
\subsection{Simultaneous estimation}

We now proceed with the simultaneous estimation of all components of
the model: the choice model, the structural equations, and the
measurement equations.

The specifications of each component remain consistent with those
previously described. The key difference lies in the treatment of the
scale parameters in the structural equations, which are now estimated
rather than fixed. This modification is motivated by the inclusion of
the latent variables in the utility specification of the choice
model. As a result, the error terms in the structural equations can
now be disentangled from those in the measurement equations.

The script implementing this specification is described in
Section~\ref{sec:plot_b03_simultaneous.py}. As in the sequential case,
the latent variables are constructed from two components: the
deterministic part obtained from the structural equations, and a
stochastic term capturing unobserved heterogeneity:
\begin{lstlisting}
sigma_car_structural = Beta('sigma_car_structural', 0.1, None, None, 0)
car_centric_attitude = build_car_centric_attitude() +
    sigma_car_structural * Draws('car_error_term', 'NORMAL_MLHS_ANTI')
\end{lstlisting}

Compared to the sequential estimation approach, two main differences
arise. First, the scale parameter \lstinline$sigma_car_structural$ is
now explicitly estimated. Second, the structural parameters are no
longer fixed to the values obtained from the MIMIC model but are
jointly estimated within the full model. This ensures internal
consistency across all components.

The conditional likelihood function now combines the contributions of
both the choice model and the measurement model:
\begin{lstlisting}
cond_prob = logit(V, None, Choice) * likelihood_indicator
\end{lstlisting}

Simultaneous estimation presents significant numerical challenges due
to the complexity of the composite likelihood expression. In
particular, the computation of derivatives becomes demanding. For this
reason, additional options are specified when instantiating the
\lstinline$BIOGEME$ object:
\begin{lstlisting}
BIOGEME(
    ...,
    calculating_second_derivatives='never',
    numerically_safe=True,
    max_iterations=5000,
)
\end{lstlisting}

The option \lstinline$calculating_second_derivatives='never'$
instructs Biogeme to skip the calculation of second derivatives, which
often fails due to numerical instability. As a result, statistical
inference is performed using the BHHH approximation
matrix (\cite{BernHallHallHaus74}) instead of the Rao-Cramer bound. The
\lstinline$numerically_safe=True$ flag activates additional safeguards
to avoid numerical issues, especially when computing the logarithm of
expressions approaching zero --- typically the case when multiplying
several small probability terms. Lastly, the number of maximum
iterations is increased to allow the optimization algorithm sufficient
time to converge to the desired level of precision.

Although all parameters have been estimated simultaneously, they are
reported in separate groups to improve clarity and
readability. General estimation statistics are presented in
Table~\vref{tab:simultaneous_stats}. The parameters of the structural
equations are provided in Table~\vref{tab:simultaneous_params_struct},
while those of the measurement equations are reported in
Tables~\ref{tab:simulataneous_params_car_meas}--\ref{tab:simultaneous_thresholds}. Finally,
the parameters of the choice model are shown in
Table~\vref{tab:choice_simultaneous}. Finally, Table~\vref{tab:comparison} compares the parameters of the choice model (i) without latent variables, (ii) with latent variables and sequential estimation, and (iii) with latent variables and simultaneous estimation.



\begin{table}[htb]
  \begin{center}
    \footnotesize
    \begin{tabular}{ll}
Number of estimated parameters & 83 \\
Sample size & 1899 \\
Final log likelihood & -43695.23 \\
Akaike Information Criterion & 87556.45 \\
Bayesian Information Criterion & 88017.03 \\      
Number of draws & 10000 \\
\end{tabular}
  \caption{ \label{tab:simultaneous_stats}Simultaneous model: general statistics}
  \end{center}

\end{table}

\begin{table}[htb]
    \footnotesize
  \begin{center}
\begin{tabular}{rlr@{.}lr@{.}lr@{.}lr@{.}l}
  &              &   \multicolumn{2}{l}{}         & \multicolumn{2}{l}{Robust}  &  \multicolumn{4}{l}{}  \\
  Parameter & Description & \multicolumn{2}{l}{Coeff.} & \multicolumn{2}{l}{Asympt.} & \multicolumn{2}{l}{$t$-stat} & \multicolumn{2}{l}{$p$-value} \\
  number    &             & \multicolumn{2}{l}{estimate} & \multicolumn{2}{l}{std. error} & \multicolumn{2}{l}{} & \multicolumn{2}{l}{} \\
  \hline
1 & car\_struct\_age\_65\_more & 0&175 & 0&104 & 1&69 & 0&0902 \\ 
2 & car\_struct\_ScaledIncome & -0&046 & 0&00962 & -4&78 & 1&73e-06 \\ 
3 & car\_struct\_moreThanOneCar & 0&945 & 0&0802 & 11&8 & 0&0 \\ 
4 & car\_struct\_moreThanOneBike & -0&505 & 0&0872 & -5&79 & 6&96e-09 \\ 
5 & car\_struct\_individualHouse & -0&126 & 0&0775 & -1&62 & 0&104 \\ 
6 & car\_struct\_haveChildren & -0&0314 & 0&0757 & -0&415 & 0&678 \\ 
7 & car\_struct\_haveGA & -0&922 & 0&117 & -7&89 & 3&11e-15 \\ 
8 & car\_struct\_highEducation & -0&451 & 0&0765 & -5&89 & 3&82e-09 \\ 
9 & sigma\_car\_structural & 1&17 & 0&0495 & 23&7 & 0&0 \\ 
10 & urban\_struct\_childCenter & 0&199 & 0&0499 & 3&99 & 6&73e-05 \\ 
11 & urban\_struct\_childSuburb & 0&139 & 0&035 & 3&97 & 7&31e-05 \\ 
12 & urban\_struct\_highEducation & 0&0112 & 0&0342 & 0&327 & 0&743 \\ 
13 & urban\_struct\_artisans & -0&184 & 0&072 & -2&55 & 0&0107 \\ 
14 & urban\_struct\_employees & -0&105 & 0&0309 & -3&39 & 0&000689 \\ 
15 & urban\_struct\_age\_30\_less & 0&386 & 0&0472 & 8&18 & 2&22e-16 \\ 
16 & urban\_struct\_haveChildren & -0&0622 & 0&0301 & -2&07 & 0&0385 \\ 
17 & urban\_struct\_UrbRur & 0&239 & 0&0324 & 7&36 & 1&83e-13 \\ 
18 & urban\_struct\_IndividualHouse & 0&0371 & 0&0307 & 1&21 & 0&227 \\ 
19 & sigma\_urban\_structural & -0&456 & 0&0342 & -13&3 & 0&0 \\ 

  
\end{tabular}
\caption{Simultaneous model: estimated parameters of the structural equations\label{tab:simultaneous_params_struct}}
  \end{center}
\end{table}


\begin{table}[htb]
    \footnotesize
  \begin{center}
\begin{tabular}{rlr@{.}lr@{.}lr@{.}lr@{.}l}
  &              &   \multicolumn{2}{l}{}         & \multicolumn{2}{l}{Robust}  &  \multicolumn{4}{l}{}  \\
  Parameter & Description & \multicolumn{2}{l}{Coeff.} & \multicolumn{2}{l}{Asympt.} & \multicolumn{2}{l}{$t$-stat} & \multicolumn{2}{l}{$p$-value} \\
  number    &             & \multicolumn{2}{l}{estimate} & \multicolumn{2}{l}{std. error} & \multicolumn{2}{l}{} & \multicolumn{2}{l}{} \\
  \hline
1 & meas\_intercept\_Envir01 & -1&13 & 0&104 & -10&9 & 0&0 \\ 
2 & meas\_intercept\_Envir02 & 0&0434 & 0&055 & 0&789 & 0&43 \\ 
3 & car\_meas\_b\_Envir02 & -0&457 & 0&0294 & -15&5 & 0&0 \\ 
4 & meas\_sigma\_star\_Envir02 & 1&11 & 0&0342 & 32&5 & 0&0 \\ 
5 & meas\_intercept\_Envir03 & 0&0879 & 0&058 & 1&52 & 0&13 \\ 
6 & car\_meas\_b\_Envir03 & 0&476 & 0&0296 & 16&0 & 0&0 \\ 
7 & meas\_sigma\_star\_Envir03 & 1&04 & 0&0317 & 32&7 & 0&0 \\ 
8 & meas\_intercept\_Envir04 & -0&332 & 0&0449 & -7&38 & 1&55e-13 \\ 
9 & car\_meas\_b\_Envir04 & 0&33 & 0&0255 & 12&9 & 0&0 \\ 
10 & meas\_sigma\_star\_Envir04 & 0&994 & 0&0293 & 33&9 & 0&0 \\ 
11 & meas\_intercept\_Mobil09 & 0&709 & 0&0488 & 14&5 & 0&0 \\ 
12 & car\_meas\_b\_Mobil09 & -0&346 & 0&0272 & -12&7 & 0&0 \\ 
13 & meas\_sigma\_star\_Mobil09 & 1&07 & 0&0319 & 33&5 & 0&0 \\ 
14 & meas\_intercept\_Mobil11 & 1&11 & 0&0635 & 17&5 & 0&0 \\ 
15 & car\_meas\_b\_Mobil11 & 0&49 & 0&0317 & 15&5 & 0&0 \\ 
16 & meas\_sigma\_star\_Mobil11 & 1&13 & 0&0363 & 31&1 & 0&0 \\ 
17 & meas\_intercept\_Mobil14 & 0&404 & 0&0577 & 6&99 & 2&67e-12 \\ 
18 & car\_meas\_b\_Mobil14 & 0&522 & 0&0274 & 19&1 & 0&0 \\ 
19 & meas\_sigma\_star\_Mobil14 & 0&941 & 0&0297 & 31&6 & 0&0 \\ 
20 & meas\_intercept\_Mobil16 & 0&736 & 0&0574 & 12&8 & 0&0 \\ 
21 & car\_meas\_b\_Mobil16 & 0&455 & 0&027 & 16&8 & 0&0 \\ 
22 & meas\_sigma\_star\_Mobil16 & 1&1 & 0&0345 & 31&8 & 0&0 \\ 
23 & meas\_intercept\_Mobil17 & 0&709 & 0&0586 & 12&1 & 0&0 \\ 
24 & car\_meas\_b\_Mobil17 & 0&439 & 0&0289 & 15&2 & 0&0 \\ 
25 & meas\_sigma\_star\_Mobil17 & 1&1 & 0&036 & 30&7 & 0&0 \\ 
26 & meas\_intercept\_LifSty08 & 1&45 & 0&0533 & 27&2 & 0&0 \\ 
27 & car\_meas\_b\_LifSty08 & -0&0685 & 0&0294 & -2&33 & 0&0198 \\ 
28 & meas\_sigma\_star\_LifSty08 & 1&25 & 0&0391 & 32&0 & 0&0 \\ 
\end{tabular}
  \caption{Simultaneous model: estimated parameters of the measurement equations for the car-centric attitude\label{tab:simulataneous_params_car_meas}}
  \end{center}
\end{table}

\begin{table}[htb]
    \footnotesize
  \begin{center}
\begin{tabular}{rlr@{.}lr@{.}lr@{.}lr@{.}l}
  &              &   \multicolumn{2}{l}{}         & \multicolumn{2}{l}{Robust}  &  \multicolumn{4}{l}{}  \\
  Parameter & Description & \multicolumn{2}{l}{Coeff.} & \multicolumn{2}{l}{Asympt.} & \multicolumn{2}{l}{$t$-stat} & \multicolumn{2}{l}{$p$-value} \\
  number    &             & \multicolumn{2}{l}{estimate} & \multicolumn{2}{l}{std. error} & \multicolumn{2}{l}{} & \multicolumn{2}{l}{} \\
  \hline
1 & meas\_intercept\_ResidCh01 & -0&894 & 0&0669 & -13&4 & 0&0 \\ 
2 & meas\_intercept\_ResidCh02 & 0&425 & 0&063 & 6&75 & 1&52e-11 \\ 
3 & urban\_meas\_b\_ResidCh02 & 0&87 & 0&107 & 8&16 & 4&44e-16 \\ 
4 & meas\_sigma\_star\_ResidCh02 & 1&18 & 0&0407 & 29&1 & 0&0 \\ 
5 & meas\_intercept\_ResidCh03 & 0&251 & 0&044 & 5&71 & 1&15e-08 \\ 
6 & urban\_meas\_b\_ResidCh03 & 0&0226 & 0&0747 & 0&302 & 0&762 \\ 
7 & meas\_sigma\_star\_ResidCh03 & 1&27 & 0&0413 & 30&7 & 0&0 \\ 
8 & meas\_intercept\_ResidCh05 & -3&06 & 0&17 & -18&0 & 0&0 \\ 
9 & urban\_meas\_b\_ResidCh05 & 2&44 & 0&207 & 11&8 & 0&0 \\ 
10 & meas\_sigma\_star\_ResidCh05 & 0&954 & 0&0506 & 18&8 & 0&0 \\ 
11 & meas\_intercept\_ResidCh06 & -1&31 & 0&0972 & -13&5 & 0&0 \\ 
12 & urban\_meas\_b\_ResidCh06 & 1&56 & 0&144 & 10&8 & 0&0 \\ 
13 & meas\_sigma\_star\_ResidCh06 & 1&27 & 0&0416 & 30&4 & 0&0 \\ 
14 & meas\_intercept\_ResidCh07 & 2&25 & 0&125 & 18&0 & 0&0 \\ 
15 & urban\_meas\_b\_ResidCh07 & -1&93 & 0&16 & -12&0 & 0&0 \\ 
16 & meas\_sigma\_star\_ResidCh07 & 0&895 & 0&036 & 24&9 & 0&0 \\ 
17 & meas\_intercept\_Mobil07 & 1&3 & 0&0544 & 23&8 & 0&0 \\ 
18 & urban\_meas\_b\_Mobil07 & -0&378 & 0&0763 & -4&95 & 7&46e-07 \\ 
19 & meas\_sigma\_star\_Mobil07 & 1&08 & 0&0299 & 36&2 & 0&0 \\ 
20 & meas\_intercept\_Mobil24 & 0&209 & 0&0601 & 3&48 & 0&000495 \\ 
21 & urban\_meas\_b\_Mobil24 & 0&395 & 0&101 & 3&9 & 9&56e-05 \\ 
22 & meas\_sigma\_star\_Mobil24 & 1&46 & 0&0473 & 30&9 & 0&0 \\ 
  
\end{tabular}
  \caption{Simultaneous model: estimated parameters of the measurement equations for the urban-preference attitude\label{tab:simulataneous_params_urban_meas}}
  \end{center}
\end{table}

\begin{table}[htb]
    \footnotesize
  \begin{center}
\begin{tabular}{rlr@{.}lr@{.}lr@{.}lr@{.}l}
  &              &   \multicolumn{2}{l}{}         & \multicolumn{2}{l}{Robust}  &  \multicolumn{4}{l}{}  \\
  Parameter & Description & \multicolumn{2}{l}{Coeff.} & \multicolumn{2}{l}{Asympt.} & \multicolumn{2}{l}{$t$-stat} & \multicolumn{2}{l}{$p$-value} \\
  number    &             & \multicolumn{2}{l}{estimate} & \multicolumn{2}{l}{std. error} & \multicolumn{2}{l}{} & \multicolumn{2}{l}{} \\
  \hline
1 & delta\_1 & 0&402 & 0&00842 & 47&8 & 0&0 \\ 
2 & delta\_2 & 1&24 & 0&0243 & 51&0 & 0&0 \\ 
\end{tabular}
  \caption{Simultaneous model: thresholds of the measurement equations\label{tab:simultaneous_thresholds}}
  \end{center}
\end{table}


\begin{table}[htb]
  \footnotesize
  \begin{center}
\begin{tabular}{rlr@{.}lr@{.}lr@{.}lr@{.}l}
          &              &   \multicolumn{2}{l}{}         & \multicolumn{2}{l}{Robust}  &  \multicolumn{4}{l}{}  \\
Parameter &              &   \multicolumn{2}{l}{Coeff.}   & \multicolumn{2}{l}{Asympt.}       & \multicolumn{4}{l}{}   \\
number    &  Description &   \multicolumn{2}{l}{estimate} & \multicolumn{2}{l}{std. error}    & \multicolumn{2}{l}{$t$-stat}  &  \multicolumn{2}{l}{$p$-value} \\
\hline
1 & choice\_asc\_pt & -0&75 & 0&273 & -2&74 & 0&00605 \\ 
2 & choice\_asc\_car & 0&755 & 0&241 & 3&14 & 0&0017 \\ 
3 & choice\_beta\_cost\_other & -0&413 & 0&0802 & -5&15 & 2&6e-07 \\ 
4 & choice\_beta\_time\_pt & -1&8 & 0&371 & -4&86 & 1&18e-06 \\ 
5 & choice\_beta\_waiting\_time & -0&0158 & 0&00632 & -2&5 & 0&0124 \\ 
6 & choice\_beta\_time\_car & -5&26 & 0&645 & -8&16 & 4&44e-16 \\ 
7 & choice\_beta\_dist & -1&29 & 0&144 & -8&92 & 0&0 \\ 
8 & choice\_car\_centric\_pt\_cte & 0&113 & 0&114 & 0&983 & 0&325 \\ 
9 & choice\_urban\_life\_pt\_cte & 0&764 & 0&367 & 2&08 & 0&0372 \\ 
10 & choice\_car\_centric\_car\_cte & 0&678 & 0&13 & 5&21 & 1&86e-07 \\ 
11 & choice\_urban\_life\_car\_cte & -0&28 & 0&326 & -0&86 & 0&39 \\ 
12 & scale\_choice\_model & 1&06 & 0&114 & 9&32 & 0&0 \\ 
\end{tabular}
  \end{center}
  \caption{Choice model with latent variables: simultaneous estimation\label{tab:choice_simultaneous}}
\end{table}

\clearpage

\begin{longtable}{rlSSS}
  \caption{Comparison of the parameters of the choice model \label{tab:comparison}} \\
& & \multicolumn{1}{c}{Choice only} & \multicolumn{1}{c}{Sequential} & \multicolumn{1}{c}{Simultaneous} \\
 & Parameter name &  \multicolumn{1}{c}{Coef./(SE)} &  \multicolumn{1}{c}{Coef./(SE)} &  \multicolumn{1}{c}{Coef./(SE)} \\
  \hline
  \endfirsthead
& & \multicolumn{1}{c}{Choice only} & \multicolumn{1}{c}{Sequential} & \multicolumn{1}{c}{Simultaneous} \\
 & Parameter name &  \multicolumn{1}{c}{Coef./(SE)} &  \multicolumn{1}{c}{Coef./(SE)} &  \multicolumn{1}{c}{Coef./(SE)} \\
  \hline
  \endhead
1 &choice\_asc\_pt& -0.132 & 0.436 & -0.75\textsuperscript{***}  \\
 & &(0.262)&(0.417)&(0.273) \\
2 &choice\_asc\_car& 0.408 & 1.55\textsuperscript{***} & 0.755\textsuperscript{***}  \\
 & &(0.283)&(0.588)&(0.241) \\
3 &choice\_beta\_cost\_other& -0.465\textsuperscript{***} & -0.416\textsuperscript{***} & -0.413\textsuperscript{***}  \\
 & &(0.142)&(0.148)&(0.0802) \\
4 &choice\_beta\_time\_pt& -1.58\textsuperscript{**} & -1.95\textsuperscript{**} & -1.8\textsuperscript{***}  \\
 & &(0.658)&(0.865)&(0.371) \\
5 &choice\_beta\_waiting\_time& -0.018\textsuperscript{**} & -0.02\textsuperscript{**} & -0.0158\textsuperscript{**}  \\
 & &(0.00789)&(0.0101)&(0.00632) \\
6 &choice\_beta\_time\_car& -5.14\textsuperscript{***} & -5.58\textsuperscript{***} & -5.26\textsuperscript{***}  \\
 & &(1.67)&(2.03)&(0.645) \\
7 &choice\_beta\_dist& -0.983\textsuperscript{***} & -1.13\textsuperscript{***} & -1.29\textsuperscript{***}  \\
 & &(0.317)&(0.407)&(0.144) \\
8 &choice\_car\_centric\_pt\_cte & & 0.652\textsuperscript{**} & 0.113  \\
 &  & &(0.265)&(0.114) \\
9 &choice\_urban\_life\_pt\_cte & & 0.0223 & 0.764\textsuperscript{**}  \\
 &  & &(0.0826)&(0.367) \\
10 &choice\_car\_centric\_car\_cte & & 1.62\textsuperscript{***} & 0.678\textsuperscript{***}  \\
 &  & &(0.52)&(0.13) \\
11 &choice\_urban\_life\_car\_cte & & -0.338\textsuperscript{*} & -0.28  \\
 &  & &(0.18)&(0.326) \\
12 &scale\_choice\_model& 1.17\textsuperscript{***} & 1.11\textsuperscript{***} & 1.06\textsuperscript{***}  \\
 & &(0.275)&(0.298)&(0.114) \\
\hline
\multicolumn{2}{l}{Number of observations} &1899 & 1899 & 1899 \\
\multicolumn{2}{l}{Number of parameters} &8 & 12 & 83 \\
\multicolumn{2}{l}{Akaike Information Criterion} &2476.0 & 2407.7 & 87556.5 \\
\multicolumn{2}{l}{Bayesian Information Criterion} &2520.4 & 2474.3 & 88017.0 \\
\hline
\multicolumn{4}{l}{\footnotesize Standard errors: \textsuperscript{***}: $p < 0.01$, \textsuperscript{**}: $p < 0.05$, \textsuperscript{*}: $p < 0.1$}
\end{longtable}
\clearpage

\section{Conclusion}

Choice models with latent variables offer a powerful and flexible
framework for capturing complex behavioral mechanisms underlying
decision-making. By incorporating unobserved psychological constructs
such as attitudes, and perceptions, these models extend
the explanatory power of traditional discrete choice models. They
allow researchers to account for systematic heterogeneity in behavior
that is not directly observed in the data, thereby enhancing both the
behavioral realism and predictive performance of the models.

Despite their potential, these models are inherently more complex to
specify, estimate, and interpret. It is therefore recommended to
proceed incrementally. A practical and effective strategy is to begin
by developing and estimating the choice model and the MIMIC model
independently. This allows the analyst to ensure that both components
are correctly specified and empirically supported.

Once the separate models have been validated, the next step is to
explore their integration through sequential estimation. In this
stage, the latent variables generated from the MIMIC model are
incorporated into the utility specification of the choice model. This
provides valuable insights into how these latent constructs influence
behavior, while still maintaining manageable computational complexity.

Only after the specification has been refined and the results from the
sequential estimation are deemed satisfactory should one proceed to
the simultaneous estimation of all components. This final step --- though
computationally more demanding --- offers the benefit of statistical
efficiency by leveraging all available information jointly. It also
provides a more coherent treatment of the latent variables, since
their estimation is informed not only by the indicators, but also by
the observed choices.



\clearpage
\section{Complete specification files}

The following specification files have been used for the estimation of the presented results. They have been developed for Biogeme 3.3.0. It is possible that the syntax should be slightly adapted for future versions of Biogeme.

\subsection{\lstinline$relevant_data.py$}
\label{sec:relevant_data.py}

\lstinputlisting[style=numbers]{../../docs/source/examples/latent/relevant_data.py}


\subsection{\lstinline$structural_equations.py$}
\label{sec:structural_equations.py}

\lstinputlisting[style=numbers]{../../docs/source/examples/latent/structural_equations.py}

\subsection{\lstinline$measurement_equations.py$}
\label{sec:measurement_equations.py}

\lstinputlisting[style=numbers]{../../docs/source/examples/latent/measurement_equations.py}

\subsection{\lstinline$plot_b01_mimic.py$}
\label{sec:plot_b01_mimic.py}

\lstinputlisting[style=numbers]{../../docs/source/examples/latent/plot_b01_mimic.py}

\subsection{\lstinline$plot_b02_choice_only.py$}
\label{sec:plot_b02_choice_only.py}

\lstinputlisting[style=numbers]{../../docs/source/examples/latent/plot_b02_choice_only.py}

\subsection{\lstinline$plot_b03_sequential.py$}
\label{sec:plot_b03_sequential.py}

\lstinputlisting[style=numbers]{../../docs/source/examples/latent/plot_b03_sequential.py}

\subsection{\lstinline$plot_b03_simultaneous.py$}
\label{sec:plot_b03_simultaneous.py}

\lstinputlisting[style=numbers]{../../docs/source/examples/latent/plot_b03_simultaneous.py}

\clearpage

\section{Description of the variables}

THe following table describes the variables involved in the models described in this document.

\begin{longtable}{p{4cm}|p{10.5cm}}
		\hline 
		\textbf{Name} & \textbf{Description}\\
		%\tabularnewline
		\hline 
		TimePT & The duration of the loop performed in public transport (in minutes).\tabularnewline
		\hline 
		WaitingTimePT & The total waiting time in a loop performed in public transports (in minutes).\tabularnewline
		\hline 
		TimeCar & The total duration of a loop made using the car (in minutes).\tabularnewline
		\hline 
		MarginalCostPT & The total cost of a loop performed in public transports, taking into account the ownership of a seasonal ticket by the respondent. If the respondent has a ``GA'' (full Swiss season ticket), a seasonal ticket for the line or the area, this variable takes value zero. If the respondent has a half-fare travelcard, this variable corresponds to half the cost of the trip by public transport..\tabularnewline
		\hline 
		CostCarCHF & The total gas cost of a loop performed with the car in CHF.\tabularnewline
		\hline 
		TripPurpose & The main purpose of the loop: 1 =Work-related trips; 2 =Work- and leisure-related
		trips; 3 =Leisure related trips. -1 represents missing values \tabularnewline
		\hline 
		UrbRur & Binary variable, where: 1 =Rural; 2 =Urban.\tabularnewline
		\hline 
		distance\_km & Total distance performed for the loop.\tabularnewline
		\hline 
		age & Age of the respondent (in years) -1 represents missing values.\tabularnewline
		\hline 
		ResidChild & Main place of residence as a kid ($<18$), 1 is city center (large town), 2 is city center (small town), 3 is suburbs, 4 is suburban town, 5 is country side (village), 6 is countryside (isolated), -1 is for missing data and -2 if respondent didn't answer to any opinion questions. \tabularnewline
		\hline 
		NbCar & Number of cars in the household.-1 for missing value. \tabularnewline
		\hline 
		NbBicy & Number of bikes in the household. -1 for missing value.\tabularnewline
		\hline 
		HouseType & House type, 1 is individual house (or terraced house), 2 is apartment (and other types of multi-family residential), 3 is independent room (subletting). -1 for missing value.\tabularnewline
		\hline 
		Income & Net monthly income of the household in CHF. 1 is less than 2500, 2 is from 2501 to 4000, 3 is from 4001 to 6000, 4 is from 6001 to 8000, 5 is from 8001 to 10'000 and 6 is more than 10'001. -1 for missing value.\tabularnewline
		\hline 
		CalculatedIncome & Net monthly income of the household in CHF, calculated as a continuous variable. The value is the center of the interval of the corresponding incone category. \tabularnewline
		\hline
		FamilSitu & Familiar situation: 1 is single, 2 is in a couple without children, 3 is in a couple with children, 4 is single with your own children, 5 is in a colocation, 6 is with your parents and 7 is for other situations. -1 for missing values.\tabularnewline
		\hline
		SocioProfCat & To which of the following socioprofessional categories do you belong? 1 is for top managers, 2 for intellectual professions, 3 for freelancers, 4 for intermediate professions, 5 for artisans and salespersons, 6 for employees, 7 for workers and 8 for others. -1 for missing values.\tabularnewline
		\hline 
		GenAbST & Is equal to 1 if the respondent has a GA (full Swiss season ticket) and 2 if not.\tabularnewline
		\hline
		Education &Highest education achieved. As mentioned by Wikipedia in English: "The education system in Switzerland is very diverse, because the constitution of Switzerland delegates the authority for the school system mainly to the cantons. The Swiss constitution sets the foundations, namely that primary school is obligatory for every child and is free in public schools and that the confederation can run or support universities." (source: \href{http://en.wikipedia.org/wiki/Education\_in\_Switzerland}{Education in Switzerland (Wikipedia)}, accessed April 16, 2013). It is thus difficult to translate the survey that was originally in French and German. The possible answers in the survey are:
		\begin{enumerate}
			\item Unfinished compulsory education: education is compulsory in Switzerland but pupils may finish it at the legal age without succeeding the final exam.
			\item Compulsory education with diploma.
			\item Vocational education: a three or four-year period of training both in a company and following theoretical courses. Ends with a diploma called "Certificat fédéral de capacité" (i.e., ''professional baccalaureate'') (reference: \href{https://fr.wikipedia.org/wiki/Certificat\_f\%C3\%A9d\%C3\%A9ral\_de\_capacit\%C3\%A9}{Certificat fédéral de capacité (Wikipedia)} - in French).
			\item A 3-year generalist school giving access to teaching school, nursing schools, social work school, universities of applied sciences or vocational education (sometime in less than the normal number of years). It does not give access to universities in Switzerland.
			\item High school: ends with the general baccalaureate exam. The general baccalaureate gives access automatically to universities.
			\item Universities of applied sciences, teaching schools, nursing schools, social work schools: ends with a Bachelor and sometimes a Master, mostly focus on vocational training.
			\item Universities and institutes of technology: ends with an academic Bachelor and in most cases an academic Master.
			\item PhD thesis.
		\end{enumerate}\\
	\hline
		\caption{Description of variables}
		\label{tab:variables}
	\end{longtable}

\clearpage
\bibliographystyle{dcudoi}
\bibliography{transpor}

\end{document}


