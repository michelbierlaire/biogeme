\documentclass[12pt,a4paper]{article}
\PassOptionsToPackage{hyphens}{url}
\usepackage{michel}
%\usepackage[hyphens]{url}
\usepackage[dcucite,abbr]{harvard}
\harvardparenthesis{none}\harvardyearparenthesis{round}
\usepackage{varioref}
\usepackage{longtable}
\usepackage{siunitx}
\usepackage{pgfplots}
% Fix for matplotlib PGF output: define \mathdefault if missing
\providecommand{\mathdefault}[1]{#1}
\pgfplotsset{compat=newest}
\sisetup{
  parse-numbers=false,      % Prevents automatic parsing (needed for parentheses & superscripts)
  detect-inline-weight=math,% Ensures proper formatting in tables
  tight-spacing=true        % Keeps spacing consistent
}
% Package to include code
\usepackage{listings}
\usepackage{color}
\lstset{language=Python}
\lstset{numbers=none, basicstyle=\footnotesize,
  numberstyle=\tiny,keywordstyle=\color{blue},stringstyle=\ttfamily,showstringspaces=false}
\lstset{backgroundcolor=\color[rgb]{0.95 0.95 0.95}}
\lstdefinestyle{numbers}{numbers=left, stepnumber=1,
  numberstyle=\tiny,basicstyle=\tiny, numbersep=10pt}
\lstdefinestyle{nonumbers}{numbers=none}
\lstset{
  breaklines=true,
  breakatwhitespace=true,
}
\usepackage{geometry}
\geometry{left=2cm, top=1.5cm, right=2cm, bottom=1.5cm}

\title{Bayesian inference with Biogeme}
\author{Michel Bierlaire} 
\date{\today}


\begin{document}


\begin{titlepage}
\pagestyle{empty}

\maketitle
\vspace{2cm}

\begin{center}
\small Report TRANSP-OR xxxxxx  \\ Transport and Mobility Laboratory \\ School of Architecture, Civil and Environmental Engineering \\ Ecole Polytechnique F\'ed\'erale de Lausanne \\ \verb+transp-or.epfl.ch+
\begin{center}
\textsc{Series on Biogeme}
\end{center}
\end{center}


\clearpage
\end{titlepage}

\begin{titlepage}
\tableofcontents
\end{titlepage}

The package Biogeme (\texttt{biogeme.epfl.ch}) is designed to estimate
the parameters of various models. It is particularly designed for
discrete choice models. Originally designed to use maximum likelihood
estimation, it is also possible to use Bayesian inference to estimate
the parameters of the model. It is particularly useful for mixtures
models, as it allows to avoid the calculation of complex Monte-Carlo
integrals.

We assume that the reader is already familiar with discrete choice
models, and has successfully installed Biogeme. This document has
been written using Biogeme 3.3.2.

It is also highly recommended to review foundational concepts such as
simulation methods, Bayesian inference, and Markov chain Monte
Carlo. Although these topics are briefly introduced here, a solid
understanding of them greatly helps in fully appreciating the power
and flexibility of Bayesian estimation for discrete choice models.


\section{Bayesian inference}

Bayesian inference consists of fitting a probabilistic model to observed data
and representing the outcome by a probability distribution over the model
parameters.

Bayesian inference differs fundamentally from frequentist inference in
the way uncertainty about model parameters is represented and
quantified. In the frequentist framework, parameters are treated as
fixed but unknown constants, and uncertainty arises solely from the
randomness of the data-generating process. In contrast, the Bayesian
approach treats parameters as random variables endowed with a prior
distribution, which encodes the information available before observing
the data. This modeling choice does not imply that parameters are
intrinsically random, but rather reflects epistemic uncertainty: the
distribution represents our state of knowledge about plausible
parameter values given the information at hand. After observing data,
Bayes' theorem updates this prior into a posterior distribution, which
synthesizes both prior beliefs and empirical evidence. The posterior
distribution is therefore the central object of Bayesian inference,
providing coherent measures of uncertainty, enabling probabilistic
predictions, and allowing for direct probability statements about
parameters themselves.

Consider a discrete choice model characterized by a vector of
parameters $\boldsymbol{\theta}$ and a likelihood function
$L(\mathcal{D} \mid \boldsymbol{\theta})$, where $\mathcal{D}$ denotes
the observed data: for each individual in the sample, it contains the
values of the explanatory variables as well as the observed choice.
The likelihood function represents the probability that the model,
with parameters $\boldsymbol{\theta}$, reproduces exactly all the
observations in the sample\footnote{Rigorously, this interpretation
holds when the model involves only discrete variables. When the model
includes continuous variables, the likelihood is obtained by
evaluating the joint probability \emph{density} of the data at the
observed values. Although this quantity is not itself a probability,
it plays an analogous role.}.

In the frequentist framework, estimation consists of finding a point estimate 
$\hat{\boldsymbol{\theta}}$ that maximizes the likelihood or the log-likelihood.
In the Bayesian framework, however, the parameters are treated as unknown 
quantities described by a prior density $p(\boldsymbol{\theta})$, reflecting 
the information available before observing the data.

Once the data are observed, inference is performed through Bayes' theorem, 
which combines the prior with the likelihood to obtain the posterior 
distribution of the parameters:
\begin{equation}
  \label{eq:posterior}
p(\boldsymbol{\theta}\mid \mathcal{D})
= 
\frac{L(\mathcal{D}\mid\boldsymbol{\theta} )\, p(\boldsymbol{\theta})}
     {\int L(\mathcal{D}\mid \boldsymbol{\theta}')\, 
            p(\boldsymbol{\theta}')\, d\boldsymbol{\theta}'}.
\end{equation}
The denominator ensures that the 
posterior integrates to one. The bad news is that it is not available in closed form for
choice models. The good news is that it is not needed in practice. 

This distribution is inherently difficult to work with
analytically. For that reason, we rely on simulation to generate
realizations --- referred to as \emph{draws} --- from it. Before explaining how such
draws can be produced in practice, we first provide some intuition for
why simulation is necessary.

\section{Simulation}

The arithmetic of random variables can quickly become intricate. Even in the simple case of two independent random variables $X$ and $Y$, with respective probability density functions (pdf) $f_X$ and $f_Y$, the distribution of their sum is not straightforward. If we define $Z = X + Y$, the pdf of $Z$ is obtained through a transformation known as \emph{convolution}:
\[
f_Z(z) = \int_{-\infty}^{\infty} f_X(x)\, f_Y(z - x)\, dx.
\]
For instance, assume that both $X$  and $Y$ follow a uniform distribution:
 \[
X \sim \mathrm{U}(0,1), 
\qquad
Y \sim \mathrm{U}(0,1).
\]
Then, the calculation of the convolution shows that $Z$ follows a triangular distribution:
\[
f_Z(z) =
\begin{cases}
0, & z < 0, \\[6pt]
z, & 0 \le z \le 1, \\[6pt]
2 - z, & 1 < z \le 2, \\[6pt]
0, & z > 2.
\end{cases}
\]
However, in practice, the convolution integral rarely has a closed form, making it difficult to handle.

The idea of simulation consists in generating concrete numerical
values produced according to the probability law of the random
variables of interest.  Regular arithmetic can then be applied on
those values.

Let $X$ be a random variable with probability density function (pdf)
$f_X$. A \emph{draw from $X$} is a numerical value obtained from a
random mechanism whose outcomes follow exactly the distribution of $X$.

Formally, consider a sequence of independent draws
$X_1, X_2, \dots, X_R$ from $X$. For any fixed $R$, the empirical
distribution of these draws can be represented by a histogram. As
$R$ becomes large, the histogram provides an increasingly accurate
approximation of the true pdf $f_X$.

More precisely, for any interval $[a,b]$,
\begin{equation}
  \label{eq:simulation_convergence}
  \frac{1}{R}\sum_{i=1}^R \mathbf{1}\{X_i \in [a,b]\}
  \;\xrightarrow[R\to\infty]{\text{a.s.}}\;
  \int_a^b f_X(x)\,dx,
\end{equation}
where $\mathbf{1}\{\cdot\}$ denotes the indicator function, and “a.s.” stands
for almost surely, meaning that the convergence holds with probability
1.  This
property demonstrates that the draws reproduce the probability
structure of $X$: the relative frequency with which the draws fall in
any region converges to the probability mass assigned to that region
by the pdf $f_X$.

In this sense, a draw from $X$ is not merely a number, but a realization
generated according to $f_X$, and repeated draws allow us to recover the
shape of the density through their empirical distribution.

This is illustrated in Figure~\ref{fig:triangular}, which displays histograms of 100'000 independent draws from two uniform random variables \(X \sim \mathrm{U}(0,1)\) and \(Y \sim \mathrm{U}(0,1)\), together with the histogram of their sum \(Z = X + Y\). The first two panels show that the empirical distributions of \(X\) and \(Y\) closely match the flat density of the uniform distribution. The third panel presents the resulting distribution of \(Z\), whose histogram approaches the theoretical triangular density obtained by the convolution of the two uniforms. This confirms that, as the number of draws increases, the simulated empirical distributions converge to their corresponding probability density functions.

\begin{figure}
  \centering
  \resizebox{0.7\textwidth}{!}{%
    %% Creator: Matplotlib, PGF backend
%%
%% To include the figure in your LaTeX document, write
%%   \input{<filename>.pgf}
%%
%% Make sure the required packages are loaded in your preamble
%%   \usepackage{pgf}
%%
%% Also ensure that all the required font packages are loaded; for instance,
%% the lmodern package is sometimes necessary when using math font.
%%   \usepackage{lmodern}
%%
%% Figures using additional raster images can only be included by \input if
%% they are in the same directory as the main LaTeX file. For loading figures
%% from other directories you can use the `import` package
%%   \usepackage{import}
%%
%% and then include the figures with
%%   \import{<path to file>}{<filename>.pgf}
%%
%% Matplotlib used the following preamble
%%   \def\mathdefault#1{#1}
%%   \everymath=\expandafter{\the\everymath\displaystyle}
%%   \IfFileExists{scrextend.sty}{
%%     \usepackage[fontsize=10.000000pt]{scrextend}
%%   }{
%%     \renewcommand{\normalsize}{\fontsize{10.000000}{12.000000}\selectfont}
%%     \normalsize
%%   }
%%   
%%   \ifdefined\pdftexversion\else  % non-pdftex case.
%%     \usepackage{fontspec}
%%     \setmainfont{DejaVuSerif.ttf}[Path=\detokenize{/Library/Frameworks/Python.framework/Versions/3.13/lib/python3.13/site-packages/matplotlib/mpl-data/fonts/ttf/}]
%%     \setsansfont{DejaVuSans.ttf}[Path=\detokenize{/Library/Frameworks/Python.framework/Versions/3.13/lib/python3.13/site-packages/matplotlib/mpl-data/fonts/ttf/}]
%%     \setmonofont{DejaVuSansMono.ttf}[Path=\detokenize{/Library/Frameworks/Python.framework/Versions/3.13/lib/python3.13/site-packages/matplotlib/mpl-data/fonts/ttf/}]
%%   \fi
%%   \makeatletter\@ifpackageloaded{underscore}{}{\usepackage[strings]{underscore}}\makeatother
%%
\begingroup%
\makeatletter%
\begin{pgfpicture}%
\pgfpathrectangle{\pgfpointorigin}{\pgfqpoint{8.000000in}{10.000000in}}%
\pgfusepath{use as bounding box, clip}%
\begin{pgfscope}%
\pgfsetbuttcap%
\pgfsetmiterjoin%
\definecolor{currentfill}{rgb}{1.000000,1.000000,1.000000}%
\pgfsetfillcolor{currentfill}%
\pgfsetlinewidth{0.000000pt}%
\definecolor{currentstroke}{rgb}{1.000000,1.000000,1.000000}%
\pgfsetstrokecolor{currentstroke}%
\pgfsetdash{}{0pt}%
\pgfpathmoveto{\pgfqpoint{0.000000in}{0.000000in}}%
\pgfpathlineto{\pgfqpoint{8.000000in}{0.000000in}}%
\pgfpathlineto{\pgfqpoint{8.000000in}{10.000000in}}%
\pgfpathlineto{\pgfqpoint{0.000000in}{10.000000in}}%
\pgfpathlineto{\pgfqpoint{0.000000in}{0.000000in}}%
\pgfpathclose%
\pgfusepath{fill}%
\end{pgfscope}%
\begin{pgfscope}%
\pgfsetbuttcap%
\pgfsetmiterjoin%
\definecolor{currentfill}{rgb}{1.000000,1.000000,1.000000}%
\pgfsetfillcolor{currentfill}%
\pgfsetlinewidth{0.000000pt}%
\definecolor{currentstroke}{rgb}{0.000000,0.000000,0.000000}%
\pgfsetstrokecolor{currentstroke}%
\pgfsetstrokeopacity{0.000000}%
\pgfsetdash{}{0pt}%
\pgfpathmoveto{\pgfqpoint{0.603704in}{7.116358in}}%
\pgfpathlineto{\pgfqpoint{7.850000in}{7.116358in}}%
\pgfpathlineto{\pgfqpoint{7.850000in}{9.641667in}}%
\pgfpathlineto{\pgfqpoint{0.603704in}{9.641667in}}%
\pgfpathlineto{\pgfqpoint{0.603704in}{7.116358in}}%
\pgfpathclose%
\pgfusepath{fill}%
\end{pgfscope}%
\begin{pgfscope}%
\pgfpathrectangle{\pgfqpoint{0.603704in}{7.116358in}}{\pgfqpoint{7.246296in}{2.525309in}}%
\pgfusepath{clip}%
\pgfsetbuttcap%
\pgfsetmiterjoin%
\definecolor{currentfill}{rgb}{0.121569,0.466667,0.705882}%
\pgfsetfillcolor{currentfill}%
\pgfsetfillopacity{0.700000}%
\pgfsetlinewidth{0.000000pt}%
\definecolor{currentstroke}{rgb}{0.000000,0.000000,0.000000}%
\pgfsetstrokecolor{currentstroke}%
\pgfsetstrokeopacity{0.700000}%
\pgfsetdash{}{0pt}%
\pgfpathmoveto{\pgfqpoint{0.933081in}{7.116358in}}%
\pgfpathlineto{\pgfqpoint{1.064832in}{7.116358in}}%
\pgfpathlineto{\pgfqpoint{1.064832in}{9.384015in}}%
\pgfpathlineto{\pgfqpoint{0.933081in}{9.384015in}}%
\pgfpathlineto{\pgfqpoint{0.933081in}{7.116358in}}%
\pgfpathclose%
\pgfusepath{fill}%
\end{pgfscope}%
\begin{pgfscope}%
\pgfpathrectangle{\pgfqpoint{0.603704in}{7.116358in}}{\pgfqpoint{7.246296in}{2.525309in}}%
\pgfusepath{clip}%
\pgfsetbuttcap%
\pgfsetmiterjoin%
\definecolor{currentfill}{rgb}{0.121569,0.466667,0.705882}%
\pgfsetfillcolor{currentfill}%
\pgfsetfillopacity{0.700000}%
\pgfsetlinewidth{0.000000pt}%
\definecolor{currentstroke}{rgb}{0.000000,0.000000,0.000000}%
\pgfsetstrokecolor{currentstroke}%
\pgfsetstrokeopacity{0.700000}%
\pgfsetdash{}{0pt}%
\pgfpathmoveto{\pgfqpoint{1.064832in}{7.116358in}}%
\pgfpathlineto{\pgfqpoint{1.196583in}{7.116358in}}%
\pgfpathlineto{\pgfqpoint{1.196583in}{9.468302in}}%
\pgfpathlineto{\pgfqpoint{1.064832in}{9.468302in}}%
\pgfpathlineto{\pgfqpoint{1.064832in}{7.116358in}}%
\pgfpathclose%
\pgfusepath{fill}%
\end{pgfscope}%
\begin{pgfscope}%
\pgfpathrectangle{\pgfqpoint{0.603704in}{7.116358in}}{\pgfqpoint{7.246296in}{2.525309in}}%
\pgfusepath{clip}%
\pgfsetbuttcap%
\pgfsetmiterjoin%
\definecolor{currentfill}{rgb}{0.121569,0.466667,0.705882}%
\pgfsetfillcolor{currentfill}%
\pgfsetfillopacity{0.700000}%
\pgfsetlinewidth{0.000000pt}%
\definecolor{currentstroke}{rgb}{0.000000,0.000000,0.000000}%
\pgfsetstrokecolor{currentstroke}%
\pgfsetstrokeopacity{0.700000}%
\pgfsetdash{}{0pt}%
\pgfpathmoveto{\pgfqpoint{1.196583in}{7.116358in}}%
\pgfpathlineto{\pgfqpoint{1.328334in}{7.116358in}}%
\pgfpathlineto{\pgfqpoint{1.328334in}{9.432509in}}%
\pgfpathlineto{\pgfqpoint{1.196583in}{9.432509in}}%
\pgfpathlineto{\pgfqpoint{1.196583in}{7.116358in}}%
\pgfpathclose%
\pgfusepath{fill}%
\end{pgfscope}%
\begin{pgfscope}%
\pgfpathrectangle{\pgfqpoint{0.603704in}{7.116358in}}{\pgfqpoint{7.246296in}{2.525309in}}%
\pgfusepath{clip}%
\pgfsetbuttcap%
\pgfsetmiterjoin%
\definecolor{currentfill}{rgb}{0.121569,0.466667,0.705882}%
\pgfsetfillcolor{currentfill}%
\pgfsetfillopacity{0.700000}%
\pgfsetlinewidth{0.000000pt}%
\definecolor{currentstroke}{rgb}{0.000000,0.000000,0.000000}%
\pgfsetstrokecolor{currentstroke}%
\pgfsetstrokeopacity{0.700000}%
\pgfsetdash{}{0pt}%
\pgfpathmoveto{\pgfqpoint{1.328334in}{7.116358in}}%
\pgfpathlineto{\pgfqpoint{1.460085in}{7.116358in}}%
\pgfpathlineto{\pgfqpoint{1.460085in}{9.402489in}}%
\pgfpathlineto{\pgfqpoint{1.328334in}{9.402489in}}%
\pgfpathlineto{\pgfqpoint{1.328334in}{7.116358in}}%
\pgfpathclose%
\pgfusepath{fill}%
\end{pgfscope}%
\begin{pgfscope}%
\pgfpathrectangle{\pgfqpoint{0.603704in}{7.116358in}}{\pgfqpoint{7.246296in}{2.525309in}}%
\pgfusepath{clip}%
\pgfsetbuttcap%
\pgfsetmiterjoin%
\definecolor{currentfill}{rgb}{0.121569,0.466667,0.705882}%
\pgfsetfillcolor{currentfill}%
\pgfsetfillopacity{0.700000}%
\pgfsetlinewidth{0.000000pt}%
\definecolor{currentstroke}{rgb}{0.000000,0.000000,0.000000}%
\pgfsetstrokecolor{currentstroke}%
\pgfsetstrokeopacity{0.700000}%
\pgfsetdash{}{0pt}%
\pgfpathmoveto{\pgfqpoint{1.460085in}{7.116358in}}%
\pgfpathlineto{\pgfqpoint{1.591835in}{7.116358in}}%
\pgfpathlineto{\pgfqpoint{1.591835in}{9.419808in}}%
\pgfpathlineto{\pgfqpoint{1.460085in}{9.419808in}}%
\pgfpathlineto{\pgfqpoint{1.460085in}{7.116358in}}%
\pgfpathclose%
\pgfusepath{fill}%
\end{pgfscope}%
\begin{pgfscope}%
\pgfpathrectangle{\pgfqpoint{0.603704in}{7.116358in}}{\pgfqpoint{7.246296in}{2.525309in}}%
\pgfusepath{clip}%
\pgfsetbuttcap%
\pgfsetmiterjoin%
\definecolor{currentfill}{rgb}{0.121569,0.466667,0.705882}%
\pgfsetfillcolor{currentfill}%
\pgfsetfillopacity{0.700000}%
\pgfsetlinewidth{0.000000pt}%
\definecolor{currentstroke}{rgb}{0.000000,0.000000,0.000000}%
\pgfsetstrokecolor{currentstroke}%
\pgfsetstrokeopacity{0.700000}%
\pgfsetdash{}{0pt}%
\pgfpathmoveto{\pgfqpoint{1.591835in}{7.116358in}}%
\pgfpathlineto{\pgfqpoint{1.723586in}{7.116358in}}%
\pgfpathlineto{\pgfqpoint{1.723586in}{9.403643in}}%
\pgfpathlineto{\pgfqpoint{1.591835in}{9.403643in}}%
\pgfpathlineto{\pgfqpoint{1.591835in}{7.116358in}}%
\pgfpathclose%
\pgfusepath{fill}%
\end{pgfscope}%
\begin{pgfscope}%
\pgfpathrectangle{\pgfqpoint{0.603704in}{7.116358in}}{\pgfqpoint{7.246296in}{2.525309in}}%
\pgfusepath{clip}%
\pgfsetbuttcap%
\pgfsetmiterjoin%
\definecolor{currentfill}{rgb}{0.121569,0.466667,0.705882}%
\pgfsetfillcolor{currentfill}%
\pgfsetfillopacity{0.700000}%
\pgfsetlinewidth{0.000000pt}%
\definecolor{currentstroke}{rgb}{0.000000,0.000000,0.000000}%
\pgfsetstrokecolor{currentstroke}%
\pgfsetstrokeopacity{0.700000}%
\pgfsetdash{}{0pt}%
\pgfpathmoveto{\pgfqpoint{1.723586in}{7.116358in}}%
\pgfpathlineto{\pgfqpoint{1.855337in}{7.116358in}}%
\pgfpathlineto{\pgfqpoint{1.855337in}{9.438282in}}%
\pgfpathlineto{\pgfqpoint{1.723586in}{9.438282in}}%
\pgfpathlineto{\pgfqpoint{1.723586in}{7.116358in}}%
\pgfpathclose%
\pgfusepath{fill}%
\end{pgfscope}%
\begin{pgfscope}%
\pgfpathrectangle{\pgfqpoint{0.603704in}{7.116358in}}{\pgfqpoint{7.246296in}{2.525309in}}%
\pgfusepath{clip}%
\pgfsetbuttcap%
\pgfsetmiterjoin%
\definecolor{currentfill}{rgb}{0.121569,0.466667,0.705882}%
\pgfsetfillcolor{currentfill}%
\pgfsetfillopacity{0.700000}%
\pgfsetlinewidth{0.000000pt}%
\definecolor{currentstroke}{rgb}{0.000000,0.000000,0.000000}%
\pgfsetstrokecolor{currentstroke}%
\pgfsetstrokeopacity{0.700000}%
\pgfsetdash{}{0pt}%
\pgfpathmoveto{\pgfqpoint{1.855337in}{7.116358in}}%
\pgfpathlineto{\pgfqpoint{1.987088in}{7.116358in}}%
\pgfpathlineto{\pgfqpoint{1.987088in}{9.409417in}}%
\pgfpathlineto{\pgfqpoint{1.855337in}{9.409417in}}%
\pgfpathlineto{\pgfqpoint{1.855337in}{7.116358in}}%
\pgfpathclose%
\pgfusepath{fill}%
\end{pgfscope}%
\begin{pgfscope}%
\pgfpathrectangle{\pgfqpoint{0.603704in}{7.116358in}}{\pgfqpoint{7.246296in}{2.525309in}}%
\pgfusepath{clip}%
\pgfsetbuttcap%
\pgfsetmiterjoin%
\definecolor{currentfill}{rgb}{0.121569,0.466667,0.705882}%
\pgfsetfillcolor{currentfill}%
\pgfsetfillopacity{0.700000}%
\pgfsetlinewidth{0.000000pt}%
\definecolor{currentstroke}{rgb}{0.000000,0.000000,0.000000}%
\pgfsetstrokecolor{currentstroke}%
\pgfsetstrokeopacity{0.700000}%
\pgfsetdash{}{0pt}%
\pgfpathmoveto{\pgfqpoint{1.987088in}{7.116358in}}%
\pgfpathlineto{\pgfqpoint{2.118839in}{7.116358in}}%
\pgfpathlineto{\pgfqpoint{2.118839in}{9.415190in}}%
\pgfpathlineto{\pgfqpoint{1.987088in}{9.415190in}}%
\pgfpathlineto{\pgfqpoint{1.987088in}{7.116358in}}%
\pgfpathclose%
\pgfusepath{fill}%
\end{pgfscope}%
\begin{pgfscope}%
\pgfpathrectangle{\pgfqpoint{0.603704in}{7.116358in}}{\pgfqpoint{7.246296in}{2.525309in}}%
\pgfusepath{clip}%
\pgfsetbuttcap%
\pgfsetmiterjoin%
\definecolor{currentfill}{rgb}{0.121569,0.466667,0.705882}%
\pgfsetfillcolor{currentfill}%
\pgfsetfillopacity{0.700000}%
\pgfsetlinewidth{0.000000pt}%
\definecolor{currentstroke}{rgb}{0.000000,0.000000,0.000000}%
\pgfsetstrokecolor{currentstroke}%
\pgfsetstrokeopacity{0.700000}%
\pgfsetdash{}{0pt}%
\pgfpathmoveto{\pgfqpoint{2.118839in}{7.116358in}}%
\pgfpathlineto{\pgfqpoint{2.250590in}{7.116358in}}%
\pgfpathlineto{\pgfqpoint{2.250590in}{9.440591in}}%
\pgfpathlineto{\pgfqpoint{2.118839in}{9.440591in}}%
\pgfpathlineto{\pgfqpoint{2.118839in}{7.116358in}}%
\pgfpathclose%
\pgfusepath{fill}%
\end{pgfscope}%
\begin{pgfscope}%
\pgfpathrectangle{\pgfqpoint{0.603704in}{7.116358in}}{\pgfqpoint{7.246296in}{2.525309in}}%
\pgfusepath{clip}%
\pgfsetbuttcap%
\pgfsetmiterjoin%
\definecolor{currentfill}{rgb}{0.121569,0.466667,0.705882}%
\pgfsetfillcolor{currentfill}%
\pgfsetfillopacity{0.700000}%
\pgfsetlinewidth{0.000000pt}%
\definecolor{currentstroke}{rgb}{0.000000,0.000000,0.000000}%
\pgfsetstrokecolor{currentstroke}%
\pgfsetstrokeopacity{0.700000}%
\pgfsetdash{}{0pt}%
\pgfpathmoveto{\pgfqpoint{2.250590in}{7.116358in}}%
\pgfpathlineto{\pgfqpoint{2.382340in}{7.116358in}}%
\pgfpathlineto{\pgfqpoint{2.382340in}{9.330903in}}%
\pgfpathlineto{\pgfqpoint{2.250590in}{9.330903in}}%
\pgfpathlineto{\pgfqpoint{2.250590in}{7.116358in}}%
\pgfpathclose%
\pgfusepath{fill}%
\end{pgfscope}%
\begin{pgfscope}%
\pgfpathrectangle{\pgfqpoint{0.603704in}{7.116358in}}{\pgfqpoint{7.246296in}{2.525309in}}%
\pgfusepath{clip}%
\pgfsetbuttcap%
\pgfsetmiterjoin%
\definecolor{currentfill}{rgb}{0.121569,0.466667,0.705882}%
\pgfsetfillcolor{currentfill}%
\pgfsetfillopacity{0.700000}%
\pgfsetlinewidth{0.000000pt}%
\definecolor{currentstroke}{rgb}{0.000000,0.000000,0.000000}%
\pgfsetstrokecolor{currentstroke}%
\pgfsetstrokeopacity{0.700000}%
\pgfsetdash{}{0pt}%
\pgfpathmoveto{\pgfqpoint{2.382340in}{7.116358in}}%
\pgfpathlineto{\pgfqpoint{2.514091in}{7.116358in}}%
\pgfpathlineto{\pgfqpoint{2.514091in}{9.521414in}}%
\pgfpathlineto{\pgfqpoint{2.382340in}{9.521414in}}%
\pgfpathlineto{\pgfqpoint{2.382340in}{7.116358in}}%
\pgfpathclose%
\pgfusepath{fill}%
\end{pgfscope}%
\begin{pgfscope}%
\pgfpathrectangle{\pgfqpoint{0.603704in}{7.116358in}}{\pgfqpoint{7.246296in}{2.525309in}}%
\pgfusepath{clip}%
\pgfsetbuttcap%
\pgfsetmiterjoin%
\definecolor{currentfill}{rgb}{0.121569,0.466667,0.705882}%
\pgfsetfillcolor{currentfill}%
\pgfsetfillopacity{0.700000}%
\pgfsetlinewidth{0.000000pt}%
\definecolor{currentstroke}{rgb}{0.000000,0.000000,0.000000}%
\pgfsetstrokecolor{currentstroke}%
\pgfsetstrokeopacity{0.700000}%
\pgfsetdash{}{0pt}%
\pgfpathmoveto{\pgfqpoint{2.514091in}{7.116358in}}%
\pgfpathlineto{\pgfqpoint{2.645842in}{7.116358in}}%
\pgfpathlineto{\pgfqpoint{2.645842in}{9.333212in}}%
\pgfpathlineto{\pgfqpoint{2.514091in}{9.333212in}}%
\pgfpathlineto{\pgfqpoint{2.514091in}{7.116358in}}%
\pgfpathclose%
\pgfusepath{fill}%
\end{pgfscope}%
\begin{pgfscope}%
\pgfpathrectangle{\pgfqpoint{0.603704in}{7.116358in}}{\pgfqpoint{7.246296in}{2.525309in}}%
\pgfusepath{clip}%
\pgfsetbuttcap%
\pgfsetmiterjoin%
\definecolor{currentfill}{rgb}{0.121569,0.466667,0.705882}%
\pgfsetfillcolor{currentfill}%
\pgfsetfillopacity{0.700000}%
\pgfsetlinewidth{0.000000pt}%
\definecolor{currentstroke}{rgb}{0.000000,0.000000,0.000000}%
\pgfsetstrokecolor{currentstroke}%
\pgfsetstrokeopacity{0.700000}%
\pgfsetdash{}{0pt}%
\pgfpathmoveto{\pgfqpoint{2.645842in}{7.116358in}}%
\pgfpathlineto{\pgfqpoint{2.777593in}{7.116358in}}%
\pgfpathlineto{\pgfqpoint{2.777593in}{9.442900in}}%
\pgfpathlineto{\pgfqpoint{2.645842in}{9.442900in}}%
\pgfpathlineto{\pgfqpoint{2.645842in}{7.116358in}}%
\pgfpathclose%
\pgfusepath{fill}%
\end{pgfscope}%
\begin{pgfscope}%
\pgfpathrectangle{\pgfqpoint{0.603704in}{7.116358in}}{\pgfqpoint{7.246296in}{2.525309in}}%
\pgfusepath{clip}%
\pgfsetbuttcap%
\pgfsetmiterjoin%
\definecolor{currentfill}{rgb}{0.121569,0.466667,0.705882}%
\pgfsetfillcolor{currentfill}%
\pgfsetfillopacity{0.700000}%
\pgfsetlinewidth{0.000000pt}%
\definecolor{currentstroke}{rgb}{0.000000,0.000000,0.000000}%
\pgfsetstrokecolor{currentstroke}%
\pgfsetstrokeopacity{0.700000}%
\pgfsetdash{}{0pt}%
\pgfpathmoveto{\pgfqpoint{2.777593in}{7.116358in}}%
\pgfpathlineto{\pgfqpoint{2.909344in}{7.116358in}}%
\pgfpathlineto{\pgfqpoint{2.909344in}{9.505249in}}%
\pgfpathlineto{\pgfqpoint{2.777593in}{9.505249in}}%
\pgfpathlineto{\pgfqpoint{2.777593in}{7.116358in}}%
\pgfpathclose%
\pgfusepath{fill}%
\end{pgfscope}%
\begin{pgfscope}%
\pgfpathrectangle{\pgfqpoint{0.603704in}{7.116358in}}{\pgfqpoint{7.246296in}{2.525309in}}%
\pgfusepath{clip}%
\pgfsetbuttcap%
\pgfsetmiterjoin%
\definecolor{currentfill}{rgb}{0.121569,0.466667,0.705882}%
\pgfsetfillcolor{currentfill}%
\pgfsetfillopacity{0.700000}%
\pgfsetlinewidth{0.000000pt}%
\definecolor{currentstroke}{rgb}{0.000000,0.000000,0.000000}%
\pgfsetstrokecolor{currentstroke}%
\pgfsetstrokeopacity{0.700000}%
\pgfsetdash{}{0pt}%
\pgfpathmoveto{\pgfqpoint{2.909344in}{7.116358in}}%
\pgfpathlineto{\pgfqpoint{3.041095in}{7.116358in}}%
\pgfpathlineto{\pgfqpoint{3.041095in}{9.462529in}}%
\pgfpathlineto{\pgfqpoint{2.909344in}{9.462529in}}%
\pgfpathlineto{\pgfqpoint{2.909344in}{7.116358in}}%
\pgfpathclose%
\pgfusepath{fill}%
\end{pgfscope}%
\begin{pgfscope}%
\pgfpathrectangle{\pgfqpoint{0.603704in}{7.116358in}}{\pgfqpoint{7.246296in}{2.525309in}}%
\pgfusepath{clip}%
\pgfsetbuttcap%
\pgfsetmiterjoin%
\definecolor{currentfill}{rgb}{0.121569,0.466667,0.705882}%
\pgfsetfillcolor{currentfill}%
\pgfsetfillopacity{0.700000}%
\pgfsetlinewidth{0.000000pt}%
\definecolor{currentstroke}{rgb}{0.000000,0.000000,0.000000}%
\pgfsetstrokecolor{currentstroke}%
\pgfsetstrokeopacity{0.700000}%
\pgfsetdash{}{0pt}%
\pgfpathmoveto{\pgfqpoint{3.041095in}{7.116358in}}%
\pgfpathlineto{\pgfqpoint{3.172845in}{7.116358in}}%
\pgfpathlineto{\pgfqpoint{3.172845in}{9.485621in}}%
\pgfpathlineto{\pgfqpoint{3.041095in}{9.485621in}}%
\pgfpathlineto{\pgfqpoint{3.041095in}{7.116358in}}%
\pgfpathclose%
\pgfusepath{fill}%
\end{pgfscope}%
\begin{pgfscope}%
\pgfpathrectangle{\pgfqpoint{0.603704in}{7.116358in}}{\pgfqpoint{7.246296in}{2.525309in}}%
\pgfusepath{clip}%
\pgfsetbuttcap%
\pgfsetmiterjoin%
\definecolor{currentfill}{rgb}{0.121569,0.466667,0.705882}%
\pgfsetfillcolor{currentfill}%
\pgfsetfillopacity{0.700000}%
\pgfsetlinewidth{0.000000pt}%
\definecolor{currentstroke}{rgb}{0.000000,0.000000,0.000000}%
\pgfsetstrokecolor{currentstroke}%
\pgfsetstrokeopacity{0.700000}%
\pgfsetdash{}{0pt}%
\pgfpathmoveto{\pgfqpoint{3.172845in}{7.116358in}}%
\pgfpathlineto{\pgfqpoint{3.304596in}{7.116358in}}%
\pgfpathlineto{\pgfqpoint{3.304596in}{9.387479in}}%
\pgfpathlineto{\pgfqpoint{3.172845in}{9.387479in}}%
\pgfpathlineto{\pgfqpoint{3.172845in}{7.116358in}}%
\pgfpathclose%
\pgfusepath{fill}%
\end{pgfscope}%
\begin{pgfscope}%
\pgfpathrectangle{\pgfqpoint{0.603704in}{7.116358in}}{\pgfqpoint{7.246296in}{2.525309in}}%
\pgfusepath{clip}%
\pgfsetbuttcap%
\pgfsetmiterjoin%
\definecolor{currentfill}{rgb}{0.121569,0.466667,0.705882}%
\pgfsetfillcolor{currentfill}%
\pgfsetfillopacity{0.700000}%
\pgfsetlinewidth{0.000000pt}%
\definecolor{currentstroke}{rgb}{0.000000,0.000000,0.000000}%
\pgfsetstrokecolor{currentstroke}%
\pgfsetstrokeopacity{0.700000}%
\pgfsetdash{}{0pt}%
\pgfpathmoveto{\pgfqpoint{3.304596in}{7.116358in}}%
\pgfpathlineto{\pgfqpoint{3.436347in}{7.116358in}}%
\pgfpathlineto{\pgfqpoint{3.436347in}{9.381706in}}%
\pgfpathlineto{\pgfqpoint{3.304596in}{9.381706in}}%
\pgfpathlineto{\pgfqpoint{3.304596in}{7.116358in}}%
\pgfpathclose%
\pgfusepath{fill}%
\end{pgfscope}%
\begin{pgfscope}%
\pgfpathrectangle{\pgfqpoint{0.603704in}{7.116358in}}{\pgfqpoint{7.246296in}{2.525309in}}%
\pgfusepath{clip}%
\pgfsetbuttcap%
\pgfsetmiterjoin%
\definecolor{currentfill}{rgb}{0.121569,0.466667,0.705882}%
\pgfsetfillcolor{currentfill}%
\pgfsetfillopacity{0.700000}%
\pgfsetlinewidth{0.000000pt}%
\definecolor{currentstroke}{rgb}{0.000000,0.000000,0.000000}%
\pgfsetstrokecolor{currentstroke}%
\pgfsetstrokeopacity{0.700000}%
\pgfsetdash{}{0pt}%
\pgfpathmoveto{\pgfqpoint{3.436347in}{7.116358in}}%
\pgfpathlineto{\pgfqpoint{3.568098in}{7.116358in}}%
\pgfpathlineto{\pgfqpoint{3.568098in}{9.362077in}}%
\pgfpathlineto{\pgfqpoint{3.436347in}{9.362077in}}%
\pgfpathlineto{\pgfqpoint{3.436347in}{7.116358in}}%
\pgfpathclose%
\pgfusepath{fill}%
\end{pgfscope}%
\begin{pgfscope}%
\pgfpathrectangle{\pgfqpoint{0.603704in}{7.116358in}}{\pgfqpoint{7.246296in}{2.525309in}}%
\pgfusepath{clip}%
\pgfsetbuttcap%
\pgfsetmiterjoin%
\definecolor{currentfill}{rgb}{0.121569,0.466667,0.705882}%
\pgfsetfillcolor{currentfill}%
\pgfsetfillopacity{0.700000}%
\pgfsetlinewidth{0.000000pt}%
\definecolor{currentstroke}{rgb}{0.000000,0.000000,0.000000}%
\pgfsetstrokecolor{currentstroke}%
\pgfsetstrokeopacity{0.700000}%
\pgfsetdash{}{0pt}%
\pgfpathmoveto{\pgfqpoint{3.568098in}{7.116358in}}%
\pgfpathlineto{\pgfqpoint{3.699849in}{7.116358in}}%
\pgfpathlineto{\pgfqpoint{3.699849in}{9.424426in}}%
\pgfpathlineto{\pgfqpoint{3.568098in}{9.424426in}}%
\pgfpathlineto{\pgfqpoint{3.568098in}{7.116358in}}%
\pgfpathclose%
\pgfusepath{fill}%
\end{pgfscope}%
\begin{pgfscope}%
\pgfpathrectangle{\pgfqpoint{0.603704in}{7.116358in}}{\pgfqpoint{7.246296in}{2.525309in}}%
\pgfusepath{clip}%
\pgfsetbuttcap%
\pgfsetmiterjoin%
\definecolor{currentfill}{rgb}{0.121569,0.466667,0.705882}%
\pgfsetfillcolor{currentfill}%
\pgfsetfillopacity{0.700000}%
\pgfsetlinewidth{0.000000pt}%
\definecolor{currentstroke}{rgb}{0.000000,0.000000,0.000000}%
\pgfsetstrokecolor{currentstroke}%
\pgfsetstrokeopacity{0.700000}%
\pgfsetdash{}{0pt}%
\pgfpathmoveto{\pgfqpoint{3.699849in}{7.116358in}}%
\pgfpathlineto{\pgfqpoint{3.831600in}{7.116358in}}%
\pgfpathlineto{\pgfqpoint{3.831600in}{9.401334in}}%
\pgfpathlineto{\pgfqpoint{3.699849in}{9.401334in}}%
\pgfpathlineto{\pgfqpoint{3.699849in}{7.116358in}}%
\pgfpathclose%
\pgfusepath{fill}%
\end{pgfscope}%
\begin{pgfscope}%
\pgfpathrectangle{\pgfqpoint{0.603704in}{7.116358in}}{\pgfqpoint{7.246296in}{2.525309in}}%
\pgfusepath{clip}%
\pgfsetbuttcap%
\pgfsetmiterjoin%
\definecolor{currentfill}{rgb}{0.121569,0.466667,0.705882}%
\pgfsetfillcolor{currentfill}%
\pgfsetfillopacity{0.700000}%
\pgfsetlinewidth{0.000000pt}%
\definecolor{currentstroke}{rgb}{0.000000,0.000000,0.000000}%
\pgfsetstrokecolor{currentstroke}%
\pgfsetstrokeopacity{0.700000}%
\pgfsetdash{}{0pt}%
\pgfpathmoveto{\pgfqpoint{3.831600in}{7.116358in}}%
\pgfpathlineto{\pgfqpoint{3.963350in}{7.116358in}}%
\pgfpathlineto{\pgfqpoint{3.963350in}{9.429045in}}%
\pgfpathlineto{\pgfqpoint{3.831600in}{9.429045in}}%
\pgfpathlineto{\pgfqpoint{3.831600in}{7.116358in}}%
\pgfpathclose%
\pgfusepath{fill}%
\end{pgfscope}%
\begin{pgfscope}%
\pgfpathrectangle{\pgfqpoint{0.603704in}{7.116358in}}{\pgfqpoint{7.246296in}{2.525309in}}%
\pgfusepath{clip}%
\pgfsetbuttcap%
\pgfsetmiterjoin%
\definecolor{currentfill}{rgb}{0.121569,0.466667,0.705882}%
\pgfsetfillcolor{currentfill}%
\pgfsetfillopacity{0.700000}%
\pgfsetlinewidth{0.000000pt}%
\definecolor{currentstroke}{rgb}{0.000000,0.000000,0.000000}%
\pgfsetstrokecolor{currentstroke}%
\pgfsetstrokeopacity{0.700000}%
\pgfsetdash{}{0pt}%
\pgfpathmoveto{\pgfqpoint{3.963350in}{7.116358in}}%
\pgfpathlineto{\pgfqpoint{4.095101in}{7.116358in}}%
\pgfpathlineto{\pgfqpoint{4.095101in}{9.435973in}}%
\pgfpathlineto{\pgfqpoint{3.963350in}{9.435973in}}%
\pgfpathlineto{\pgfqpoint{3.963350in}{7.116358in}}%
\pgfpathclose%
\pgfusepath{fill}%
\end{pgfscope}%
\begin{pgfscope}%
\pgfpathrectangle{\pgfqpoint{0.603704in}{7.116358in}}{\pgfqpoint{7.246296in}{2.525309in}}%
\pgfusepath{clip}%
\pgfsetbuttcap%
\pgfsetmiterjoin%
\definecolor{currentfill}{rgb}{0.121569,0.466667,0.705882}%
\pgfsetfillcolor{currentfill}%
\pgfsetfillopacity{0.700000}%
\pgfsetlinewidth{0.000000pt}%
\definecolor{currentstroke}{rgb}{0.000000,0.000000,0.000000}%
\pgfsetstrokecolor{currentstroke}%
\pgfsetstrokeopacity{0.700000}%
\pgfsetdash{}{0pt}%
\pgfpathmoveto{\pgfqpoint{4.095101in}{7.116358in}}%
\pgfpathlineto{\pgfqpoint{4.226852in}{7.116358in}}%
\pgfpathlineto{\pgfqpoint{4.226852in}{9.504095in}}%
\pgfpathlineto{\pgfqpoint{4.095101in}{9.504095in}}%
\pgfpathlineto{\pgfqpoint{4.095101in}{7.116358in}}%
\pgfpathclose%
\pgfusepath{fill}%
\end{pgfscope}%
\begin{pgfscope}%
\pgfpathrectangle{\pgfqpoint{0.603704in}{7.116358in}}{\pgfqpoint{7.246296in}{2.525309in}}%
\pgfusepath{clip}%
\pgfsetbuttcap%
\pgfsetmiterjoin%
\definecolor{currentfill}{rgb}{0.121569,0.466667,0.705882}%
\pgfsetfillcolor{currentfill}%
\pgfsetfillopacity{0.700000}%
\pgfsetlinewidth{0.000000pt}%
\definecolor{currentstroke}{rgb}{0.000000,0.000000,0.000000}%
\pgfsetstrokecolor{currentstroke}%
\pgfsetstrokeopacity{0.700000}%
\pgfsetdash{}{0pt}%
\pgfpathmoveto{\pgfqpoint{4.226852in}{7.116358in}}%
\pgfpathlineto{\pgfqpoint{4.358603in}{7.116358in}}%
\pgfpathlineto{\pgfqpoint{4.358603in}{9.382860in}}%
\pgfpathlineto{\pgfqpoint{4.226852in}{9.382860in}}%
\pgfpathlineto{\pgfqpoint{4.226852in}{7.116358in}}%
\pgfpathclose%
\pgfusepath{fill}%
\end{pgfscope}%
\begin{pgfscope}%
\pgfpathrectangle{\pgfqpoint{0.603704in}{7.116358in}}{\pgfqpoint{7.246296in}{2.525309in}}%
\pgfusepath{clip}%
\pgfsetbuttcap%
\pgfsetmiterjoin%
\definecolor{currentfill}{rgb}{0.121569,0.466667,0.705882}%
\pgfsetfillcolor{currentfill}%
\pgfsetfillopacity{0.700000}%
\pgfsetlinewidth{0.000000pt}%
\definecolor{currentstroke}{rgb}{0.000000,0.000000,0.000000}%
\pgfsetstrokecolor{currentstroke}%
\pgfsetstrokeopacity{0.700000}%
\pgfsetdash{}{0pt}%
\pgfpathmoveto{\pgfqpoint{4.358603in}{7.116358in}}%
\pgfpathlineto{\pgfqpoint{4.490354in}{7.116358in}}%
\pgfpathlineto{\pgfqpoint{4.490354in}{9.409417in}}%
\pgfpathlineto{\pgfqpoint{4.358603in}{9.409417in}}%
\pgfpathlineto{\pgfqpoint{4.358603in}{7.116358in}}%
\pgfpathclose%
\pgfusepath{fill}%
\end{pgfscope}%
\begin{pgfscope}%
\pgfpathrectangle{\pgfqpoint{0.603704in}{7.116358in}}{\pgfqpoint{7.246296in}{2.525309in}}%
\pgfusepath{clip}%
\pgfsetbuttcap%
\pgfsetmiterjoin%
\definecolor{currentfill}{rgb}{0.121569,0.466667,0.705882}%
\pgfsetfillcolor{currentfill}%
\pgfsetfillopacity{0.700000}%
\pgfsetlinewidth{0.000000pt}%
\definecolor{currentstroke}{rgb}{0.000000,0.000000,0.000000}%
\pgfsetstrokecolor{currentstroke}%
\pgfsetstrokeopacity{0.700000}%
\pgfsetdash{}{0pt}%
\pgfpathmoveto{\pgfqpoint{4.490354in}{7.116358in}}%
\pgfpathlineto{\pgfqpoint{4.622105in}{7.116358in}}%
\pgfpathlineto{\pgfqpoint{4.622105in}{9.405953in}}%
\pgfpathlineto{\pgfqpoint{4.490354in}{9.405953in}}%
\pgfpathlineto{\pgfqpoint{4.490354in}{7.116358in}}%
\pgfpathclose%
\pgfusepath{fill}%
\end{pgfscope}%
\begin{pgfscope}%
\pgfpathrectangle{\pgfqpoint{0.603704in}{7.116358in}}{\pgfqpoint{7.246296in}{2.525309in}}%
\pgfusepath{clip}%
\pgfsetbuttcap%
\pgfsetmiterjoin%
\definecolor{currentfill}{rgb}{0.121569,0.466667,0.705882}%
\pgfsetfillcolor{currentfill}%
\pgfsetfillopacity{0.700000}%
\pgfsetlinewidth{0.000000pt}%
\definecolor{currentstroke}{rgb}{0.000000,0.000000,0.000000}%
\pgfsetstrokecolor{currentstroke}%
\pgfsetstrokeopacity{0.700000}%
\pgfsetdash{}{0pt}%
\pgfpathmoveto{\pgfqpoint{4.622105in}{7.116358in}}%
\pgfpathlineto{\pgfqpoint{4.753855in}{7.116358in}}%
\pgfpathlineto{\pgfqpoint{4.753855in}{9.404798in}}%
\pgfpathlineto{\pgfqpoint{4.622105in}{9.404798in}}%
\pgfpathlineto{\pgfqpoint{4.622105in}{7.116358in}}%
\pgfpathclose%
\pgfusepath{fill}%
\end{pgfscope}%
\begin{pgfscope}%
\pgfpathrectangle{\pgfqpoint{0.603704in}{7.116358in}}{\pgfqpoint{7.246296in}{2.525309in}}%
\pgfusepath{clip}%
\pgfsetbuttcap%
\pgfsetmiterjoin%
\definecolor{currentfill}{rgb}{0.121569,0.466667,0.705882}%
\pgfsetfillcolor{currentfill}%
\pgfsetfillopacity{0.700000}%
\pgfsetlinewidth{0.000000pt}%
\definecolor{currentstroke}{rgb}{0.000000,0.000000,0.000000}%
\pgfsetstrokecolor{currentstroke}%
\pgfsetstrokeopacity{0.700000}%
\pgfsetdash{}{0pt}%
\pgfpathmoveto{\pgfqpoint{4.753855in}{7.116358in}}%
\pgfpathlineto{\pgfqpoint{4.885606in}{7.116358in}}%
\pgfpathlineto{\pgfqpoint{4.885606in}{9.496012in}}%
\pgfpathlineto{\pgfqpoint{4.753855in}{9.496012in}}%
\pgfpathlineto{\pgfqpoint{4.753855in}{7.116358in}}%
\pgfpathclose%
\pgfusepath{fill}%
\end{pgfscope}%
\begin{pgfscope}%
\pgfpathrectangle{\pgfqpoint{0.603704in}{7.116358in}}{\pgfqpoint{7.246296in}{2.525309in}}%
\pgfusepath{clip}%
\pgfsetbuttcap%
\pgfsetmiterjoin%
\definecolor{currentfill}{rgb}{0.121569,0.466667,0.705882}%
\pgfsetfillcolor{currentfill}%
\pgfsetfillopacity{0.700000}%
\pgfsetlinewidth{0.000000pt}%
\definecolor{currentstroke}{rgb}{0.000000,0.000000,0.000000}%
\pgfsetstrokecolor{currentstroke}%
\pgfsetstrokeopacity{0.700000}%
\pgfsetdash{}{0pt}%
\pgfpathmoveto{\pgfqpoint{4.885606in}{7.116358in}}%
\pgfpathlineto{\pgfqpoint{5.017357in}{7.116358in}}%
\pgfpathlineto{\pgfqpoint{5.017357in}{9.347068in}}%
\pgfpathlineto{\pgfqpoint{4.885606in}{9.347068in}}%
\pgfpathlineto{\pgfqpoint{4.885606in}{7.116358in}}%
\pgfpathclose%
\pgfusepath{fill}%
\end{pgfscope}%
\begin{pgfscope}%
\pgfpathrectangle{\pgfqpoint{0.603704in}{7.116358in}}{\pgfqpoint{7.246296in}{2.525309in}}%
\pgfusepath{clip}%
\pgfsetbuttcap%
\pgfsetmiterjoin%
\definecolor{currentfill}{rgb}{0.121569,0.466667,0.705882}%
\pgfsetfillcolor{currentfill}%
\pgfsetfillopacity{0.700000}%
\pgfsetlinewidth{0.000000pt}%
\definecolor{currentstroke}{rgb}{0.000000,0.000000,0.000000}%
\pgfsetstrokecolor{currentstroke}%
\pgfsetstrokeopacity{0.700000}%
\pgfsetdash{}{0pt}%
\pgfpathmoveto{\pgfqpoint{5.017357in}{7.116358in}}%
\pgfpathlineto{\pgfqpoint{5.149108in}{7.116358in}}%
\pgfpathlineto{\pgfqpoint{5.149108in}{9.470611in}}%
\pgfpathlineto{\pgfqpoint{5.017357in}{9.470611in}}%
\pgfpathlineto{\pgfqpoint{5.017357in}{7.116358in}}%
\pgfpathclose%
\pgfusepath{fill}%
\end{pgfscope}%
\begin{pgfscope}%
\pgfpathrectangle{\pgfqpoint{0.603704in}{7.116358in}}{\pgfqpoint{7.246296in}{2.525309in}}%
\pgfusepath{clip}%
\pgfsetbuttcap%
\pgfsetmiterjoin%
\definecolor{currentfill}{rgb}{0.121569,0.466667,0.705882}%
\pgfsetfillcolor{currentfill}%
\pgfsetfillopacity{0.700000}%
\pgfsetlinewidth{0.000000pt}%
\definecolor{currentstroke}{rgb}{0.000000,0.000000,0.000000}%
\pgfsetstrokecolor{currentstroke}%
\pgfsetstrokeopacity{0.700000}%
\pgfsetdash{}{0pt}%
\pgfpathmoveto{\pgfqpoint{5.149108in}{7.116358in}}%
\pgfpathlineto{\pgfqpoint{5.280859in}{7.116358in}}%
\pgfpathlineto{\pgfqpoint{5.280859in}{9.409417in}}%
\pgfpathlineto{\pgfqpoint{5.149108in}{9.409417in}}%
\pgfpathlineto{\pgfqpoint{5.149108in}{7.116358in}}%
\pgfpathclose%
\pgfusepath{fill}%
\end{pgfscope}%
\begin{pgfscope}%
\pgfpathrectangle{\pgfqpoint{0.603704in}{7.116358in}}{\pgfqpoint{7.246296in}{2.525309in}}%
\pgfusepath{clip}%
\pgfsetbuttcap%
\pgfsetmiterjoin%
\definecolor{currentfill}{rgb}{0.121569,0.466667,0.705882}%
\pgfsetfillcolor{currentfill}%
\pgfsetfillopacity{0.700000}%
\pgfsetlinewidth{0.000000pt}%
\definecolor{currentstroke}{rgb}{0.000000,0.000000,0.000000}%
\pgfsetstrokecolor{currentstroke}%
\pgfsetstrokeopacity{0.700000}%
\pgfsetdash{}{0pt}%
\pgfpathmoveto{\pgfqpoint{5.280859in}{7.116358in}}%
\pgfpathlineto{\pgfqpoint{5.412610in}{7.116358in}}%
\pgfpathlineto{\pgfqpoint{5.412610in}{9.493703in}}%
\pgfpathlineto{\pgfqpoint{5.280859in}{9.493703in}}%
\pgfpathlineto{\pgfqpoint{5.280859in}{7.116358in}}%
\pgfpathclose%
\pgfusepath{fill}%
\end{pgfscope}%
\begin{pgfscope}%
\pgfpathrectangle{\pgfqpoint{0.603704in}{7.116358in}}{\pgfqpoint{7.246296in}{2.525309in}}%
\pgfusepath{clip}%
\pgfsetbuttcap%
\pgfsetmiterjoin%
\definecolor{currentfill}{rgb}{0.121569,0.466667,0.705882}%
\pgfsetfillcolor{currentfill}%
\pgfsetfillopacity{0.700000}%
\pgfsetlinewidth{0.000000pt}%
\definecolor{currentstroke}{rgb}{0.000000,0.000000,0.000000}%
\pgfsetstrokecolor{currentstroke}%
\pgfsetstrokeopacity{0.700000}%
\pgfsetdash{}{0pt}%
\pgfpathmoveto{\pgfqpoint{5.412610in}{7.116358in}}%
\pgfpathlineto{\pgfqpoint{5.544360in}{7.116358in}}%
\pgfpathlineto{\pgfqpoint{5.544360in}{9.450983in}}%
\pgfpathlineto{\pgfqpoint{5.412610in}{9.450983in}}%
\pgfpathlineto{\pgfqpoint{5.412610in}{7.116358in}}%
\pgfpathclose%
\pgfusepath{fill}%
\end{pgfscope}%
\begin{pgfscope}%
\pgfpathrectangle{\pgfqpoint{0.603704in}{7.116358in}}{\pgfqpoint{7.246296in}{2.525309in}}%
\pgfusepath{clip}%
\pgfsetbuttcap%
\pgfsetmiterjoin%
\definecolor{currentfill}{rgb}{0.121569,0.466667,0.705882}%
\pgfsetfillcolor{currentfill}%
\pgfsetfillopacity{0.700000}%
\pgfsetlinewidth{0.000000pt}%
\definecolor{currentstroke}{rgb}{0.000000,0.000000,0.000000}%
\pgfsetstrokecolor{currentstroke}%
\pgfsetstrokeopacity{0.700000}%
\pgfsetdash{}{0pt}%
\pgfpathmoveto{\pgfqpoint{5.544360in}{7.116358in}}%
\pgfpathlineto{\pgfqpoint{5.676111in}{7.116358in}}%
\pgfpathlineto{\pgfqpoint{5.676111in}{9.454446in}}%
\pgfpathlineto{\pgfqpoint{5.544360in}{9.454446in}}%
\pgfpathlineto{\pgfqpoint{5.544360in}{7.116358in}}%
\pgfpathclose%
\pgfusepath{fill}%
\end{pgfscope}%
\begin{pgfscope}%
\pgfpathrectangle{\pgfqpoint{0.603704in}{7.116358in}}{\pgfqpoint{7.246296in}{2.525309in}}%
\pgfusepath{clip}%
\pgfsetbuttcap%
\pgfsetmiterjoin%
\definecolor{currentfill}{rgb}{0.121569,0.466667,0.705882}%
\pgfsetfillcolor{currentfill}%
\pgfsetfillopacity{0.700000}%
\pgfsetlinewidth{0.000000pt}%
\definecolor{currentstroke}{rgb}{0.000000,0.000000,0.000000}%
\pgfsetstrokecolor{currentstroke}%
\pgfsetstrokeopacity{0.700000}%
\pgfsetdash{}{0pt}%
\pgfpathmoveto{\pgfqpoint{5.676111in}{7.116358in}}%
\pgfpathlineto{\pgfqpoint{5.807862in}{7.116358in}}%
\pgfpathlineto{\pgfqpoint{5.807862in}{9.375933in}}%
\pgfpathlineto{\pgfqpoint{5.676111in}{9.375933in}}%
\pgfpathlineto{\pgfqpoint{5.676111in}{7.116358in}}%
\pgfpathclose%
\pgfusepath{fill}%
\end{pgfscope}%
\begin{pgfscope}%
\pgfpathrectangle{\pgfqpoint{0.603704in}{7.116358in}}{\pgfqpoint{7.246296in}{2.525309in}}%
\pgfusepath{clip}%
\pgfsetbuttcap%
\pgfsetmiterjoin%
\definecolor{currentfill}{rgb}{0.121569,0.466667,0.705882}%
\pgfsetfillcolor{currentfill}%
\pgfsetfillopacity{0.700000}%
\pgfsetlinewidth{0.000000pt}%
\definecolor{currentstroke}{rgb}{0.000000,0.000000,0.000000}%
\pgfsetstrokecolor{currentstroke}%
\pgfsetstrokeopacity{0.700000}%
\pgfsetdash{}{0pt}%
\pgfpathmoveto{\pgfqpoint{5.807862in}{7.116358in}}%
\pgfpathlineto{\pgfqpoint{5.939613in}{7.116358in}}%
\pgfpathlineto{\pgfqpoint{5.939613in}{9.469456in}}%
\pgfpathlineto{\pgfqpoint{5.807862in}{9.469456in}}%
\pgfpathlineto{\pgfqpoint{5.807862in}{7.116358in}}%
\pgfpathclose%
\pgfusepath{fill}%
\end{pgfscope}%
\begin{pgfscope}%
\pgfpathrectangle{\pgfqpoint{0.603704in}{7.116358in}}{\pgfqpoint{7.246296in}{2.525309in}}%
\pgfusepath{clip}%
\pgfsetbuttcap%
\pgfsetmiterjoin%
\definecolor{currentfill}{rgb}{0.121569,0.466667,0.705882}%
\pgfsetfillcolor{currentfill}%
\pgfsetfillopacity{0.700000}%
\pgfsetlinewidth{0.000000pt}%
\definecolor{currentstroke}{rgb}{0.000000,0.000000,0.000000}%
\pgfsetstrokecolor{currentstroke}%
\pgfsetstrokeopacity{0.700000}%
\pgfsetdash{}{0pt}%
\pgfpathmoveto{\pgfqpoint{5.939613in}{7.116358in}}%
\pgfpathlineto{\pgfqpoint{6.071364in}{7.116358in}}%
\pgfpathlineto{\pgfqpoint{6.071364in}{9.384015in}}%
\pgfpathlineto{\pgfqpoint{5.939613in}{9.384015in}}%
\pgfpathlineto{\pgfqpoint{5.939613in}{7.116358in}}%
\pgfpathclose%
\pgfusepath{fill}%
\end{pgfscope}%
\begin{pgfscope}%
\pgfpathrectangle{\pgfqpoint{0.603704in}{7.116358in}}{\pgfqpoint{7.246296in}{2.525309in}}%
\pgfusepath{clip}%
\pgfsetbuttcap%
\pgfsetmiterjoin%
\definecolor{currentfill}{rgb}{0.121569,0.466667,0.705882}%
\pgfsetfillcolor{currentfill}%
\pgfsetfillopacity{0.700000}%
\pgfsetlinewidth{0.000000pt}%
\definecolor{currentstroke}{rgb}{0.000000,0.000000,0.000000}%
\pgfsetstrokecolor{currentstroke}%
\pgfsetstrokeopacity{0.700000}%
\pgfsetdash{}{0pt}%
\pgfpathmoveto{\pgfqpoint{6.071364in}{7.116358in}}%
\pgfpathlineto{\pgfqpoint{6.203115in}{7.116358in}}%
\pgfpathlineto{\pgfqpoint{6.203115in}{9.377087in}}%
\pgfpathlineto{\pgfqpoint{6.071364in}{9.377087in}}%
\pgfpathlineto{\pgfqpoint{6.071364in}{7.116358in}}%
\pgfpathclose%
\pgfusepath{fill}%
\end{pgfscope}%
\begin{pgfscope}%
\pgfpathrectangle{\pgfqpoint{0.603704in}{7.116358in}}{\pgfqpoint{7.246296in}{2.525309in}}%
\pgfusepath{clip}%
\pgfsetbuttcap%
\pgfsetmiterjoin%
\definecolor{currentfill}{rgb}{0.121569,0.466667,0.705882}%
\pgfsetfillcolor{currentfill}%
\pgfsetfillopacity{0.700000}%
\pgfsetlinewidth{0.000000pt}%
\definecolor{currentstroke}{rgb}{0.000000,0.000000,0.000000}%
\pgfsetstrokecolor{currentstroke}%
\pgfsetstrokeopacity{0.700000}%
\pgfsetdash{}{0pt}%
\pgfpathmoveto{\pgfqpoint{6.203115in}{7.116358in}}%
\pgfpathlineto{\pgfqpoint{6.334865in}{7.116358in}}%
\pgfpathlineto{\pgfqpoint{6.334865in}{9.483312in}}%
\pgfpathlineto{\pgfqpoint{6.203115in}{9.483312in}}%
\pgfpathlineto{\pgfqpoint{6.203115in}{7.116358in}}%
\pgfpathclose%
\pgfusepath{fill}%
\end{pgfscope}%
\begin{pgfscope}%
\pgfpathrectangle{\pgfqpoint{0.603704in}{7.116358in}}{\pgfqpoint{7.246296in}{2.525309in}}%
\pgfusepath{clip}%
\pgfsetbuttcap%
\pgfsetmiterjoin%
\definecolor{currentfill}{rgb}{0.121569,0.466667,0.705882}%
\pgfsetfillcolor{currentfill}%
\pgfsetfillopacity{0.700000}%
\pgfsetlinewidth{0.000000pt}%
\definecolor{currentstroke}{rgb}{0.000000,0.000000,0.000000}%
\pgfsetstrokecolor{currentstroke}%
\pgfsetstrokeopacity{0.700000}%
\pgfsetdash{}{0pt}%
\pgfpathmoveto{\pgfqpoint{6.334865in}{7.116358in}}%
\pgfpathlineto{\pgfqpoint{6.466616in}{7.116358in}}%
\pgfpathlineto{\pgfqpoint{6.466616in}{9.516795in}}%
\pgfpathlineto{\pgfqpoint{6.334865in}{9.516795in}}%
\pgfpathlineto{\pgfqpoint{6.334865in}{7.116358in}}%
\pgfpathclose%
\pgfusepath{fill}%
\end{pgfscope}%
\begin{pgfscope}%
\pgfpathrectangle{\pgfqpoint{0.603704in}{7.116358in}}{\pgfqpoint{7.246296in}{2.525309in}}%
\pgfusepath{clip}%
\pgfsetbuttcap%
\pgfsetmiterjoin%
\definecolor{currentfill}{rgb}{0.121569,0.466667,0.705882}%
\pgfsetfillcolor{currentfill}%
\pgfsetfillopacity{0.700000}%
\pgfsetlinewidth{0.000000pt}%
\definecolor{currentstroke}{rgb}{0.000000,0.000000,0.000000}%
\pgfsetstrokecolor{currentstroke}%
\pgfsetstrokeopacity{0.700000}%
\pgfsetdash{}{0pt}%
\pgfpathmoveto{\pgfqpoint{6.466616in}{7.116358in}}%
\pgfpathlineto{\pgfqpoint{6.598367in}{7.116358in}}%
\pgfpathlineto{\pgfqpoint{6.598367in}{9.392097in}}%
\pgfpathlineto{\pgfqpoint{6.466616in}{9.392097in}}%
\pgfpathlineto{\pgfqpoint{6.466616in}{7.116358in}}%
\pgfpathclose%
\pgfusepath{fill}%
\end{pgfscope}%
\begin{pgfscope}%
\pgfpathrectangle{\pgfqpoint{0.603704in}{7.116358in}}{\pgfqpoint{7.246296in}{2.525309in}}%
\pgfusepath{clip}%
\pgfsetbuttcap%
\pgfsetmiterjoin%
\definecolor{currentfill}{rgb}{0.121569,0.466667,0.705882}%
\pgfsetfillcolor{currentfill}%
\pgfsetfillopacity{0.700000}%
\pgfsetlinewidth{0.000000pt}%
\definecolor{currentstroke}{rgb}{0.000000,0.000000,0.000000}%
\pgfsetstrokecolor{currentstroke}%
\pgfsetstrokeopacity{0.700000}%
\pgfsetdash{}{0pt}%
\pgfpathmoveto{\pgfqpoint{6.598367in}{7.116358in}}%
\pgfpathlineto{\pgfqpoint{6.730118in}{7.116358in}}%
\pgfpathlineto{\pgfqpoint{6.730118in}{9.483312in}}%
\pgfpathlineto{\pgfqpoint{6.598367in}{9.483312in}}%
\pgfpathlineto{\pgfqpoint{6.598367in}{7.116358in}}%
\pgfpathclose%
\pgfusepath{fill}%
\end{pgfscope}%
\begin{pgfscope}%
\pgfpathrectangle{\pgfqpoint{0.603704in}{7.116358in}}{\pgfqpoint{7.246296in}{2.525309in}}%
\pgfusepath{clip}%
\pgfsetbuttcap%
\pgfsetmiterjoin%
\definecolor{currentfill}{rgb}{0.121569,0.466667,0.705882}%
\pgfsetfillcolor{currentfill}%
\pgfsetfillopacity{0.700000}%
\pgfsetlinewidth{0.000000pt}%
\definecolor{currentstroke}{rgb}{0.000000,0.000000,0.000000}%
\pgfsetstrokecolor{currentstroke}%
\pgfsetstrokeopacity{0.700000}%
\pgfsetdash{}{0pt}%
\pgfpathmoveto{\pgfqpoint{6.730118in}{7.116358in}}%
\pgfpathlineto{\pgfqpoint{6.861869in}{7.116358in}}%
\pgfpathlineto{\pgfqpoint{6.861869in}{9.374778in}}%
\pgfpathlineto{\pgfqpoint{6.730118in}{9.374778in}}%
\pgfpathlineto{\pgfqpoint{6.730118in}{7.116358in}}%
\pgfpathclose%
\pgfusepath{fill}%
\end{pgfscope}%
\begin{pgfscope}%
\pgfpathrectangle{\pgfqpoint{0.603704in}{7.116358in}}{\pgfqpoint{7.246296in}{2.525309in}}%
\pgfusepath{clip}%
\pgfsetbuttcap%
\pgfsetmiterjoin%
\definecolor{currentfill}{rgb}{0.121569,0.466667,0.705882}%
\pgfsetfillcolor{currentfill}%
\pgfsetfillopacity{0.700000}%
\pgfsetlinewidth{0.000000pt}%
\definecolor{currentstroke}{rgb}{0.000000,0.000000,0.000000}%
\pgfsetstrokecolor{currentstroke}%
\pgfsetstrokeopacity{0.700000}%
\pgfsetdash{}{0pt}%
\pgfpathmoveto{\pgfqpoint{6.861869in}{7.116358in}}%
\pgfpathlineto{\pgfqpoint{6.993620in}{7.116358in}}%
\pgfpathlineto{\pgfqpoint{6.993620in}{9.323975in}}%
\pgfpathlineto{\pgfqpoint{6.861869in}{9.323975in}}%
\pgfpathlineto{\pgfqpoint{6.861869in}{7.116358in}}%
\pgfpathclose%
\pgfusepath{fill}%
\end{pgfscope}%
\begin{pgfscope}%
\pgfpathrectangle{\pgfqpoint{0.603704in}{7.116358in}}{\pgfqpoint{7.246296in}{2.525309in}}%
\pgfusepath{clip}%
\pgfsetbuttcap%
\pgfsetmiterjoin%
\definecolor{currentfill}{rgb}{0.121569,0.466667,0.705882}%
\pgfsetfillcolor{currentfill}%
\pgfsetfillopacity{0.700000}%
\pgfsetlinewidth{0.000000pt}%
\definecolor{currentstroke}{rgb}{0.000000,0.000000,0.000000}%
\pgfsetstrokecolor{currentstroke}%
\pgfsetstrokeopacity{0.700000}%
\pgfsetdash{}{0pt}%
\pgfpathmoveto{\pgfqpoint{6.993620in}{7.116358in}}%
\pgfpathlineto{\pgfqpoint{7.125370in}{7.116358in}}%
\pgfpathlineto{\pgfqpoint{7.125370in}{9.373624in}}%
\pgfpathlineto{\pgfqpoint{6.993620in}{9.373624in}}%
\pgfpathlineto{\pgfqpoint{6.993620in}{7.116358in}}%
\pgfpathclose%
\pgfusepath{fill}%
\end{pgfscope}%
\begin{pgfscope}%
\pgfpathrectangle{\pgfqpoint{0.603704in}{7.116358in}}{\pgfqpoint{7.246296in}{2.525309in}}%
\pgfusepath{clip}%
\pgfsetbuttcap%
\pgfsetmiterjoin%
\definecolor{currentfill}{rgb}{0.121569,0.466667,0.705882}%
\pgfsetfillcolor{currentfill}%
\pgfsetfillopacity{0.700000}%
\pgfsetlinewidth{0.000000pt}%
\definecolor{currentstroke}{rgb}{0.000000,0.000000,0.000000}%
\pgfsetstrokecolor{currentstroke}%
\pgfsetstrokeopacity{0.700000}%
\pgfsetdash{}{0pt}%
\pgfpathmoveto{\pgfqpoint{7.125370in}{7.116358in}}%
\pgfpathlineto{\pgfqpoint{7.257121in}{7.116358in}}%
\pgfpathlineto{\pgfqpoint{7.257121in}{9.469456in}}%
\pgfpathlineto{\pgfqpoint{7.125370in}{9.469456in}}%
\pgfpathlineto{\pgfqpoint{7.125370in}{7.116358in}}%
\pgfpathclose%
\pgfusepath{fill}%
\end{pgfscope}%
\begin{pgfscope}%
\pgfpathrectangle{\pgfqpoint{0.603704in}{7.116358in}}{\pgfqpoint{7.246296in}{2.525309in}}%
\pgfusepath{clip}%
\pgfsetbuttcap%
\pgfsetmiterjoin%
\definecolor{currentfill}{rgb}{0.121569,0.466667,0.705882}%
\pgfsetfillcolor{currentfill}%
\pgfsetfillopacity{0.700000}%
\pgfsetlinewidth{0.000000pt}%
\definecolor{currentstroke}{rgb}{0.000000,0.000000,0.000000}%
\pgfsetstrokecolor{currentstroke}%
\pgfsetstrokeopacity{0.700000}%
\pgfsetdash{}{0pt}%
\pgfpathmoveto{\pgfqpoint{7.257121in}{7.116358in}}%
\pgfpathlineto{\pgfqpoint{7.388872in}{7.116358in}}%
\pgfpathlineto{\pgfqpoint{7.388872in}{9.404798in}}%
\pgfpathlineto{\pgfqpoint{7.257121in}{9.404798in}}%
\pgfpathlineto{\pgfqpoint{7.257121in}{7.116358in}}%
\pgfpathclose%
\pgfusepath{fill}%
\end{pgfscope}%
\begin{pgfscope}%
\pgfpathrectangle{\pgfqpoint{0.603704in}{7.116358in}}{\pgfqpoint{7.246296in}{2.525309in}}%
\pgfusepath{clip}%
\pgfsetbuttcap%
\pgfsetmiterjoin%
\definecolor{currentfill}{rgb}{0.121569,0.466667,0.705882}%
\pgfsetfillcolor{currentfill}%
\pgfsetfillopacity{0.700000}%
\pgfsetlinewidth{0.000000pt}%
\definecolor{currentstroke}{rgb}{0.000000,0.000000,0.000000}%
\pgfsetstrokecolor{currentstroke}%
\pgfsetstrokeopacity{0.700000}%
\pgfsetdash{}{0pt}%
\pgfpathmoveto{\pgfqpoint{7.388872in}{7.116358in}}%
\pgfpathlineto{\pgfqpoint{7.520623in}{7.116358in}}%
\pgfpathlineto{\pgfqpoint{7.520623in}{9.502940in}}%
\pgfpathlineto{\pgfqpoint{7.388872in}{9.502940in}}%
\pgfpathlineto{\pgfqpoint{7.388872in}{7.116358in}}%
\pgfpathclose%
\pgfusepath{fill}%
\end{pgfscope}%
\begin{pgfscope}%
\pgfsetbuttcap%
\pgfsetroundjoin%
\definecolor{currentfill}{rgb}{0.000000,0.000000,0.000000}%
\pgfsetfillcolor{currentfill}%
\pgfsetlinewidth{0.803000pt}%
\definecolor{currentstroke}{rgb}{0.000000,0.000000,0.000000}%
\pgfsetstrokecolor{currentstroke}%
\pgfsetdash{}{0pt}%
\pgfsys@defobject{currentmarker}{\pgfqpoint{0.000000in}{-0.048611in}}{\pgfqpoint{0.000000in}{0.000000in}}{%
\pgfpathmoveto{\pgfqpoint{0.000000in}{0.000000in}}%
\pgfpathlineto{\pgfqpoint{0.000000in}{-0.048611in}}%
\pgfusepath{stroke,fill}%
}%
\begin{pgfscope}%
\pgfsys@transformshift{0.933033in}{7.116358in}%
\pgfsys@useobject{currentmarker}{}%
\end{pgfscope}%
\end{pgfscope}%
\begin{pgfscope}%
\definecolor{textcolor}{rgb}{0.000000,0.000000,0.000000}%
\pgfsetstrokecolor{textcolor}%
\pgfsetfillcolor{textcolor}%
\pgftext[x=0.933033in,y=7.019136in,,top]{\color{textcolor}{\rmfamily\fontsize{10.000000}{12.000000}\selectfont\catcode`\^=\active\def^{\ifmmode\sp\else\^{}\fi}\catcode`\%=\active\def%{\%}$\mathdefault{0.0}$}}%
\end{pgfscope}%
\begin{pgfscope}%
\pgfsetbuttcap%
\pgfsetroundjoin%
\definecolor{currentfill}{rgb}{0.000000,0.000000,0.000000}%
\pgfsetfillcolor{currentfill}%
\pgfsetlinewidth{0.803000pt}%
\definecolor{currentstroke}{rgb}{0.000000,0.000000,0.000000}%
\pgfsetstrokecolor{currentstroke}%
\pgfsetdash{}{0pt}%
\pgfsys@defobject{currentmarker}{\pgfqpoint{0.000000in}{-0.048611in}}{\pgfqpoint{0.000000in}{0.000000in}}{%
\pgfpathmoveto{\pgfqpoint{0.000000in}{0.000000in}}%
\pgfpathlineto{\pgfqpoint{0.000000in}{-0.048611in}}%
\pgfusepath{stroke,fill}%
}%
\begin{pgfscope}%
\pgfsys@transformshift{2.250587in}{7.116358in}%
\pgfsys@useobject{currentmarker}{}%
\end{pgfscope}%
\end{pgfscope}%
\begin{pgfscope}%
\definecolor{textcolor}{rgb}{0.000000,0.000000,0.000000}%
\pgfsetstrokecolor{textcolor}%
\pgfsetfillcolor{textcolor}%
\pgftext[x=2.250587in,y=7.019136in,,top]{\color{textcolor}{\rmfamily\fontsize{10.000000}{12.000000}\selectfont\catcode`\^=\active\def^{\ifmmode\sp\else\^{}\fi}\catcode`\%=\active\def%{\%}$\mathdefault{0.2}$}}%
\end{pgfscope}%
\begin{pgfscope}%
\pgfsetbuttcap%
\pgfsetroundjoin%
\definecolor{currentfill}{rgb}{0.000000,0.000000,0.000000}%
\pgfsetfillcolor{currentfill}%
\pgfsetlinewidth{0.803000pt}%
\definecolor{currentstroke}{rgb}{0.000000,0.000000,0.000000}%
\pgfsetstrokecolor{currentstroke}%
\pgfsetdash{}{0pt}%
\pgfsys@defobject{currentmarker}{\pgfqpoint{0.000000in}{-0.048611in}}{\pgfqpoint{0.000000in}{0.000000in}}{%
\pgfpathmoveto{\pgfqpoint{0.000000in}{0.000000in}}%
\pgfpathlineto{\pgfqpoint{0.000000in}{-0.048611in}}%
\pgfusepath{stroke,fill}%
}%
\begin{pgfscope}%
\pgfsys@transformshift{3.568141in}{7.116358in}%
\pgfsys@useobject{currentmarker}{}%
\end{pgfscope}%
\end{pgfscope}%
\begin{pgfscope}%
\definecolor{textcolor}{rgb}{0.000000,0.000000,0.000000}%
\pgfsetstrokecolor{textcolor}%
\pgfsetfillcolor{textcolor}%
\pgftext[x=3.568141in,y=7.019136in,,top]{\color{textcolor}{\rmfamily\fontsize{10.000000}{12.000000}\selectfont\catcode`\^=\active\def^{\ifmmode\sp\else\^{}\fi}\catcode`\%=\active\def%{\%}$\mathdefault{0.4}$}}%
\end{pgfscope}%
\begin{pgfscope}%
\pgfsetbuttcap%
\pgfsetroundjoin%
\definecolor{currentfill}{rgb}{0.000000,0.000000,0.000000}%
\pgfsetfillcolor{currentfill}%
\pgfsetlinewidth{0.803000pt}%
\definecolor{currentstroke}{rgb}{0.000000,0.000000,0.000000}%
\pgfsetstrokecolor{currentstroke}%
\pgfsetdash{}{0pt}%
\pgfsys@defobject{currentmarker}{\pgfqpoint{0.000000in}{-0.048611in}}{\pgfqpoint{0.000000in}{0.000000in}}{%
\pgfpathmoveto{\pgfqpoint{0.000000in}{0.000000in}}%
\pgfpathlineto{\pgfqpoint{0.000000in}{-0.048611in}}%
\pgfusepath{stroke,fill}%
}%
\begin{pgfscope}%
\pgfsys@transformshift{4.885695in}{7.116358in}%
\pgfsys@useobject{currentmarker}{}%
\end{pgfscope}%
\end{pgfscope}%
\begin{pgfscope}%
\definecolor{textcolor}{rgb}{0.000000,0.000000,0.000000}%
\pgfsetstrokecolor{textcolor}%
\pgfsetfillcolor{textcolor}%
\pgftext[x=4.885695in,y=7.019136in,,top]{\color{textcolor}{\rmfamily\fontsize{10.000000}{12.000000}\selectfont\catcode`\^=\active\def^{\ifmmode\sp\else\^{}\fi}\catcode`\%=\active\def%{\%}$\mathdefault{0.6}$}}%
\end{pgfscope}%
\begin{pgfscope}%
\pgfsetbuttcap%
\pgfsetroundjoin%
\definecolor{currentfill}{rgb}{0.000000,0.000000,0.000000}%
\pgfsetfillcolor{currentfill}%
\pgfsetlinewidth{0.803000pt}%
\definecolor{currentstroke}{rgb}{0.000000,0.000000,0.000000}%
\pgfsetstrokecolor{currentstroke}%
\pgfsetdash{}{0pt}%
\pgfsys@defobject{currentmarker}{\pgfqpoint{0.000000in}{-0.048611in}}{\pgfqpoint{0.000000in}{0.000000in}}{%
\pgfpathmoveto{\pgfqpoint{0.000000in}{0.000000in}}%
\pgfpathlineto{\pgfqpoint{0.000000in}{-0.048611in}}%
\pgfusepath{stroke,fill}%
}%
\begin{pgfscope}%
\pgfsys@transformshift{6.203250in}{7.116358in}%
\pgfsys@useobject{currentmarker}{}%
\end{pgfscope}%
\end{pgfscope}%
\begin{pgfscope}%
\definecolor{textcolor}{rgb}{0.000000,0.000000,0.000000}%
\pgfsetstrokecolor{textcolor}%
\pgfsetfillcolor{textcolor}%
\pgftext[x=6.203250in,y=7.019136in,,top]{\color{textcolor}{\rmfamily\fontsize{10.000000}{12.000000}\selectfont\catcode`\^=\active\def^{\ifmmode\sp\else\^{}\fi}\catcode`\%=\active\def%{\%}$\mathdefault{0.8}$}}%
\end{pgfscope}%
\begin{pgfscope}%
\pgfsetbuttcap%
\pgfsetroundjoin%
\definecolor{currentfill}{rgb}{0.000000,0.000000,0.000000}%
\pgfsetfillcolor{currentfill}%
\pgfsetlinewidth{0.803000pt}%
\definecolor{currentstroke}{rgb}{0.000000,0.000000,0.000000}%
\pgfsetstrokecolor{currentstroke}%
\pgfsetdash{}{0pt}%
\pgfsys@defobject{currentmarker}{\pgfqpoint{0.000000in}{-0.048611in}}{\pgfqpoint{0.000000in}{0.000000in}}{%
\pgfpathmoveto{\pgfqpoint{0.000000in}{0.000000in}}%
\pgfpathlineto{\pgfqpoint{0.000000in}{-0.048611in}}%
\pgfusepath{stroke,fill}%
}%
\begin{pgfscope}%
\pgfsys@transformshift{7.520804in}{7.116358in}%
\pgfsys@useobject{currentmarker}{}%
\end{pgfscope}%
\end{pgfscope}%
\begin{pgfscope}%
\definecolor{textcolor}{rgb}{0.000000,0.000000,0.000000}%
\pgfsetstrokecolor{textcolor}%
\pgfsetfillcolor{textcolor}%
\pgftext[x=7.520804in,y=7.019136in,,top]{\color{textcolor}{\rmfamily\fontsize{10.000000}{12.000000}\selectfont\catcode`\^=\active\def^{\ifmmode\sp\else\^{}\fi}\catcode`\%=\active\def%{\%}$\mathdefault{1.0}$}}%
\end{pgfscope}%
\begin{pgfscope}%
\definecolor{textcolor}{rgb}{0.000000,0.000000,0.000000}%
\pgfsetstrokecolor{textcolor}%
\pgfsetfillcolor{textcolor}%
\pgftext[x=4.226852in,y=6.840123in,,top]{\color{textcolor}{\rmfamily\fontsize{10.000000}{12.000000}\selectfont\catcode`\^=\active\def^{\ifmmode\sp\else\^{}\fi}\catcode`\%=\active\def%{\%}$x$}}%
\end{pgfscope}%
\begin{pgfscope}%
\pgfsetbuttcap%
\pgfsetroundjoin%
\definecolor{currentfill}{rgb}{0.000000,0.000000,0.000000}%
\pgfsetfillcolor{currentfill}%
\pgfsetlinewidth{0.803000pt}%
\definecolor{currentstroke}{rgb}{0.000000,0.000000,0.000000}%
\pgfsetstrokecolor{currentstroke}%
\pgfsetdash{}{0pt}%
\pgfsys@defobject{currentmarker}{\pgfqpoint{-0.048611in}{0.000000in}}{\pgfqpoint{-0.000000in}{0.000000in}}{%
\pgfpathmoveto{\pgfqpoint{-0.000000in}{0.000000in}}%
\pgfpathlineto{\pgfqpoint{-0.048611in}{0.000000in}}%
\pgfusepath{stroke,fill}%
}%
\begin{pgfscope}%
\pgfsys@transformshift{0.603704in}{7.116358in}%
\pgfsys@useobject{currentmarker}{}%
\end{pgfscope}%
\end{pgfscope}%
\begin{pgfscope}%
\definecolor{textcolor}{rgb}{0.000000,0.000000,0.000000}%
\pgfsetstrokecolor{textcolor}%
\pgfsetfillcolor{textcolor}%
\pgftext[x=0.329012in, y=7.068133in, left, base]{\color{textcolor}{\rmfamily\fontsize{10.000000}{12.000000}\selectfont\catcode`\^=\active\def^{\ifmmode\sp\else\^{}\fi}\catcode`\%=\active\def%{\%}$\mathdefault{0.0}$}}%
\end{pgfscope}%
\begin{pgfscope}%
\pgfsetbuttcap%
\pgfsetroundjoin%
\definecolor{currentfill}{rgb}{0.000000,0.000000,0.000000}%
\pgfsetfillcolor{currentfill}%
\pgfsetlinewidth{0.803000pt}%
\definecolor{currentstroke}{rgb}{0.000000,0.000000,0.000000}%
\pgfsetstrokecolor{currentstroke}%
\pgfsetdash{}{0pt}%
\pgfsys@defobject{currentmarker}{\pgfqpoint{-0.048611in}{0.000000in}}{\pgfqpoint{-0.000000in}{0.000000in}}{%
\pgfpathmoveto{\pgfqpoint{-0.000000in}{0.000000in}}%
\pgfpathlineto{\pgfqpoint{-0.048611in}{0.000000in}}%
\pgfusepath{stroke,fill}%
}%
\begin{pgfscope}%
\pgfsys@transformshift{0.603704in}{7.578186in}%
\pgfsys@useobject{currentmarker}{}%
\end{pgfscope}%
\end{pgfscope}%
\begin{pgfscope}%
\definecolor{textcolor}{rgb}{0.000000,0.000000,0.000000}%
\pgfsetstrokecolor{textcolor}%
\pgfsetfillcolor{textcolor}%
\pgftext[x=0.329012in, y=7.529961in, left, base]{\color{textcolor}{\rmfamily\fontsize{10.000000}{12.000000}\selectfont\catcode`\^=\active\def^{\ifmmode\sp\else\^{}\fi}\catcode`\%=\active\def%{\%}$\mathdefault{0.2}$}}%
\end{pgfscope}%
\begin{pgfscope}%
\pgfsetbuttcap%
\pgfsetroundjoin%
\definecolor{currentfill}{rgb}{0.000000,0.000000,0.000000}%
\pgfsetfillcolor{currentfill}%
\pgfsetlinewidth{0.803000pt}%
\definecolor{currentstroke}{rgb}{0.000000,0.000000,0.000000}%
\pgfsetstrokecolor{currentstroke}%
\pgfsetdash{}{0pt}%
\pgfsys@defobject{currentmarker}{\pgfqpoint{-0.048611in}{0.000000in}}{\pgfqpoint{-0.000000in}{0.000000in}}{%
\pgfpathmoveto{\pgfqpoint{-0.000000in}{0.000000in}}%
\pgfpathlineto{\pgfqpoint{-0.048611in}{0.000000in}}%
\pgfusepath{stroke,fill}%
}%
\begin{pgfscope}%
\pgfsys@transformshift{0.603704in}{8.040015in}%
\pgfsys@useobject{currentmarker}{}%
\end{pgfscope}%
\end{pgfscope}%
\begin{pgfscope}%
\definecolor{textcolor}{rgb}{0.000000,0.000000,0.000000}%
\pgfsetstrokecolor{textcolor}%
\pgfsetfillcolor{textcolor}%
\pgftext[x=0.329012in, y=7.991790in, left, base]{\color{textcolor}{\rmfamily\fontsize{10.000000}{12.000000}\selectfont\catcode`\^=\active\def^{\ifmmode\sp\else\^{}\fi}\catcode`\%=\active\def%{\%}$\mathdefault{0.4}$}}%
\end{pgfscope}%
\begin{pgfscope}%
\pgfsetbuttcap%
\pgfsetroundjoin%
\definecolor{currentfill}{rgb}{0.000000,0.000000,0.000000}%
\pgfsetfillcolor{currentfill}%
\pgfsetlinewidth{0.803000pt}%
\definecolor{currentstroke}{rgb}{0.000000,0.000000,0.000000}%
\pgfsetstrokecolor{currentstroke}%
\pgfsetdash{}{0pt}%
\pgfsys@defobject{currentmarker}{\pgfqpoint{-0.048611in}{0.000000in}}{\pgfqpoint{-0.000000in}{0.000000in}}{%
\pgfpathmoveto{\pgfqpoint{-0.000000in}{0.000000in}}%
\pgfpathlineto{\pgfqpoint{-0.048611in}{0.000000in}}%
\pgfusepath{stroke,fill}%
}%
\begin{pgfscope}%
\pgfsys@transformshift{0.603704in}{8.501843in}%
\pgfsys@useobject{currentmarker}{}%
\end{pgfscope}%
\end{pgfscope}%
\begin{pgfscope}%
\definecolor{textcolor}{rgb}{0.000000,0.000000,0.000000}%
\pgfsetstrokecolor{textcolor}%
\pgfsetfillcolor{textcolor}%
\pgftext[x=0.329012in, y=8.453618in, left, base]{\color{textcolor}{\rmfamily\fontsize{10.000000}{12.000000}\selectfont\catcode`\^=\active\def^{\ifmmode\sp\else\^{}\fi}\catcode`\%=\active\def%{\%}$\mathdefault{0.6}$}}%
\end{pgfscope}%
\begin{pgfscope}%
\pgfsetbuttcap%
\pgfsetroundjoin%
\definecolor{currentfill}{rgb}{0.000000,0.000000,0.000000}%
\pgfsetfillcolor{currentfill}%
\pgfsetlinewidth{0.803000pt}%
\definecolor{currentstroke}{rgb}{0.000000,0.000000,0.000000}%
\pgfsetstrokecolor{currentstroke}%
\pgfsetdash{}{0pt}%
\pgfsys@defobject{currentmarker}{\pgfqpoint{-0.048611in}{0.000000in}}{\pgfqpoint{-0.000000in}{0.000000in}}{%
\pgfpathmoveto{\pgfqpoint{-0.000000in}{0.000000in}}%
\pgfpathlineto{\pgfqpoint{-0.048611in}{0.000000in}}%
\pgfusepath{stroke,fill}%
}%
\begin{pgfscope}%
\pgfsys@transformshift{0.603704in}{8.963672in}%
\pgfsys@useobject{currentmarker}{}%
\end{pgfscope}%
\end{pgfscope}%
\begin{pgfscope}%
\definecolor{textcolor}{rgb}{0.000000,0.000000,0.000000}%
\pgfsetstrokecolor{textcolor}%
\pgfsetfillcolor{textcolor}%
\pgftext[x=0.329012in, y=8.915447in, left, base]{\color{textcolor}{\rmfamily\fontsize{10.000000}{12.000000}\selectfont\catcode`\^=\active\def^{\ifmmode\sp\else\^{}\fi}\catcode`\%=\active\def%{\%}$\mathdefault{0.8}$}}%
\end{pgfscope}%
\begin{pgfscope}%
\pgfsetbuttcap%
\pgfsetroundjoin%
\definecolor{currentfill}{rgb}{0.000000,0.000000,0.000000}%
\pgfsetfillcolor{currentfill}%
\pgfsetlinewidth{0.803000pt}%
\definecolor{currentstroke}{rgb}{0.000000,0.000000,0.000000}%
\pgfsetstrokecolor{currentstroke}%
\pgfsetdash{}{0pt}%
\pgfsys@defobject{currentmarker}{\pgfqpoint{-0.048611in}{0.000000in}}{\pgfqpoint{-0.000000in}{0.000000in}}{%
\pgfpathmoveto{\pgfqpoint{-0.000000in}{0.000000in}}%
\pgfpathlineto{\pgfqpoint{-0.048611in}{0.000000in}}%
\pgfusepath{stroke,fill}%
}%
\begin{pgfscope}%
\pgfsys@transformshift{0.603704in}{9.425501in}%
\pgfsys@useobject{currentmarker}{}%
\end{pgfscope}%
\end{pgfscope}%
\begin{pgfscope}%
\definecolor{textcolor}{rgb}{0.000000,0.000000,0.000000}%
\pgfsetstrokecolor{textcolor}%
\pgfsetfillcolor{textcolor}%
\pgftext[x=0.329012in, y=9.377275in, left, base]{\color{textcolor}{\rmfamily\fontsize{10.000000}{12.000000}\selectfont\catcode`\^=\active\def^{\ifmmode\sp\else\^{}\fi}\catcode`\%=\active\def%{\%}$\mathdefault{1.0}$}}%
\end{pgfscope}%
\begin{pgfscope}%
\definecolor{textcolor}{rgb}{0.000000,0.000000,0.000000}%
\pgfsetstrokecolor{textcolor}%
\pgfsetfillcolor{textcolor}%
\pgftext[x=0.273457in,y=8.379012in,,bottom,rotate=90.000000]{\color{textcolor}{\rmfamily\fontsize{10.000000}{12.000000}\selectfont\catcode`\^=\active\def^{\ifmmode\sp\else\^{}\fi}\catcode`\%=\active\def%{\%}density}}%
\end{pgfscope}%
\begin{pgfscope}%
\pgfsetrectcap%
\pgfsetmiterjoin%
\pgfsetlinewidth{0.803000pt}%
\definecolor{currentstroke}{rgb}{0.000000,0.000000,0.000000}%
\pgfsetstrokecolor{currentstroke}%
\pgfsetdash{}{0pt}%
\pgfpathmoveto{\pgfqpoint{0.603704in}{7.116358in}}%
\pgfpathlineto{\pgfqpoint{0.603704in}{9.641667in}}%
\pgfusepath{stroke}%
\end{pgfscope}%
\begin{pgfscope}%
\pgfsetrectcap%
\pgfsetmiterjoin%
\pgfsetlinewidth{0.803000pt}%
\definecolor{currentstroke}{rgb}{0.000000,0.000000,0.000000}%
\pgfsetstrokecolor{currentstroke}%
\pgfsetdash{}{0pt}%
\pgfpathmoveto{\pgfqpoint{7.850000in}{7.116358in}}%
\pgfpathlineto{\pgfqpoint{7.850000in}{9.641667in}}%
\pgfusepath{stroke}%
\end{pgfscope}%
\begin{pgfscope}%
\pgfsetrectcap%
\pgfsetmiterjoin%
\pgfsetlinewidth{0.803000pt}%
\definecolor{currentstroke}{rgb}{0.000000,0.000000,0.000000}%
\pgfsetstrokecolor{currentstroke}%
\pgfsetdash{}{0pt}%
\pgfpathmoveto{\pgfqpoint{0.603704in}{7.116358in}}%
\pgfpathlineto{\pgfqpoint{7.850000in}{7.116358in}}%
\pgfusepath{stroke}%
\end{pgfscope}%
\begin{pgfscope}%
\pgfsetrectcap%
\pgfsetmiterjoin%
\pgfsetlinewidth{0.803000pt}%
\definecolor{currentstroke}{rgb}{0.000000,0.000000,0.000000}%
\pgfsetstrokecolor{currentstroke}%
\pgfsetdash{}{0pt}%
\pgfpathmoveto{\pgfqpoint{0.603704in}{9.641667in}}%
\pgfpathlineto{\pgfqpoint{7.850000in}{9.641667in}}%
\pgfusepath{stroke}%
\end{pgfscope}%
\begin{pgfscope}%
\definecolor{textcolor}{rgb}{0.000000,0.000000,0.000000}%
\pgfsetstrokecolor{textcolor}%
\pgfsetfillcolor{textcolor}%
\pgftext[x=4.226852in,y=9.725000in,,base]{\color{textcolor}{\rmfamily\fontsize{12.000000}{14.400000}\selectfont\catcode`\^=\active\def^{\ifmmode\sp\else\^{}\fi}\catcode`\%=\active\def%{\%}Histogram of $X \sim \mathcal{U}(0,1)$}}%
\end{pgfscope}%
\begin{pgfscope}%
\pgfsetbuttcap%
\pgfsetmiterjoin%
\definecolor{currentfill}{rgb}{1.000000,1.000000,1.000000}%
\pgfsetfillcolor{currentfill}%
\pgfsetlinewidth{0.000000pt}%
\definecolor{currentstroke}{rgb}{0.000000,0.000000,0.000000}%
\pgfsetstrokecolor{currentstroke}%
\pgfsetstrokeopacity{0.000000}%
\pgfsetdash{}{0pt}%
\pgfpathmoveto{\pgfqpoint{0.603704in}{3.833024in}}%
\pgfpathlineto{\pgfqpoint{7.850000in}{3.833024in}}%
\pgfpathlineto{\pgfqpoint{7.850000in}{6.358333in}}%
\pgfpathlineto{\pgfqpoint{0.603704in}{6.358333in}}%
\pgfpathlineto{\pgfqpoint{0.603704in}{3.833024in}}%
\pgfpathclose%
\pgfusepath{fill}%
\end{pgfscope}%
\begin{pgfscope}%
\pgfpathrectangle{\pgfqpoint{0.603704in}{3.833024in}}{\pgfqpoint{7.246296in}{2.525309in}}%
\pgfusepath{clip}%
\pgfsetbuttcap%
\pgfsetmiterjoin%
\definecolor{currentfill}{rgb}{0.121569,0.466667,0.705882}%
\pgfsetfillcolor{currentfill}%
\pgfsetfillopacity{0.700000}%
\pgfsetlinewidth{0.000000pt}%
\definecolor{currentstroke}{rgb}{0.000000,0.000000,0.000000}%
\pgfsetstrokecolor{currentstroke}%
\pgfsetstrokeopacity{0.700000}%
\pgfsetdash{}{0pt}%
\pgfpathmoveto{\pgfqpoint{0.933081in}{3.833024in}}%
\pgfpathlineto{\pgfqpoint{1.064832in}{3.833024in}}%
\pgfpathlineto{\pgfqpoint{1.064832in}{6.051059in}}%
\pgfpathlineto{\pgfqpoint{0.933081in}{6.051059in}}%
\pgfpathlineto{\pgfqpoint{0.933081in}{3.833024in}}%
\pgfpathclose%
\pgfusepath{fill}%
\end{pgfscope}%
\begin{pgfscope}%
\pgfpathrectangle{\pgfqpoint{0.603704in}{3.833024in}}{\pgfqpoint{7.246296in}{2.525309in}}%
\pgfusepath{clip}%
\pgfsetbuttcap%
\pgfsetmiterjoin%
\definecolor{currentfill}{rgb}{0.121569,0.466667,0.705882}%
\pgfsetfillcolor{currentfill}%
\pgfsetfillopacity{0.700000}%
\pgfsetlinewidth{0.000000pt}%
\definecolor{currentstroke}{rgb}{0.000000,0.000000,0.000000}%
\pgfsetstrokecolor{currentstroke}%
\pgfsetstrokeopacity{0.700000}%
\pgfsetdash{}{0pt}%
\pgfpathmoveto{\pgfqpoint{1.064832in}{3.833024in}}%
\pgfpathlineto{\pgfqpoint{1.196583in}{3.833024in}}%
\pgfpathlineto{\pgfqpoint{1.196583in}{6.203869in}}%
\pgfpathlineto{\pgfqpoint{1.064832in}{6.203869in}}%
\pgfpathlineto{\pgfqpoint{1.064832in}{3.833024in}}%
\pgfpathclose%
\pgfusepath{fill}%
\end{pgfscope}%
\begin{pgfscope}%
\pgfpathrectangle{\pgfqpoint{0.603704in}{3.833024in}}{\pgfqpoint{7.246296in}{2.525309in}}%
\pgfusepath{clip}%
\pgfsetbuttcap%
\pgfsetmiterjoin%
\definecolor{currentfill}{rgb}{0.121569,0.466667,0.705882}%
\pgfsetfillcolor{currentfill}%
\pgfsetfillopacity{0.700000}%
\pgfsetlinewidth{0.000000pt}%
\definecolor{currentstroke}{rgb}{0.000000,0.000000,0.000000}%
\pgfsetstrokecolor{currentstroke}%
\pgfsetstrokeopacity{0.700000}%
\pgfsetdash{}{0pt}%
\pgfpathmoveto{\pgfqpoint{1.196583in}{3.833024in}}%
\pgfpathlineto{\pgfqpoint{1.328334in}{3.833024in}}%
\pgfpathlineto{\pgfqpoint{1.328334in}{6.128604in}}%
\pgfpathlineto{\pgfqpoint{1.196583in}{6.128604in}}%
\pgfpathlineto{\pgfqpoint{1.196583in}{3.833024in}}%
\pgfpathclose%
\pgfusepath{fill}%
\end{pgfscope}%
\begin{pgfscope}%
\pgfpathrectangle{\pgfqpoint{0.603704in}{3.833024in}}{\pgfqpoint{7.246296in}{2.525309in}}%
\pgfusepath{clip}%
\pgfsetbuttcap%
\pgfsetmiterjoin%
\definecolor{currentfill}{rgb}{0.121569,0.466667,0.705882}%
\pgfsetfillcolor{currentfill}%
\pgfsetfillopacity{0.700000}%
\pgfsetlinewidth{0.000000pt}%
\definecolor{currentstroke}{rgb}{0.000000,0.000000,0.000000}%
\pgfsetstrokecolor{currentstroke}%
\pgfsetstrokeopacity{0.700000}%
\pgfsetdash{}{0pt}%
\pgfpathmoveto{\pgfqpoint{1.328334in}{3.833024in}}%
\pgfpathlineto{\pgfqpoint{1.460085in}{3.833024in}}%
\pgfpathlineto{\pgfqpoint{1.460085in}{6.106937in}}%
\pgfpathlineto{\pgfqpoint{1.328334in}{6.106937in}}%
\pgfpathlineto{\pgfqpoint{1.328334in}{3.833024in}}%
\pgfpathclose%
\pgfusepath{fill}%
\end{pgfscope}%
\begin{pgfscope}%
\pgfpathrectangle{\pgfqpoint{0.603704in}{3.833024in}}{\pgfqpoint{7.246296in}{2.525309in}}%
\pgfusepath{clip}%
\pgfsetbuttcap%
\pgfsetmiterjoin%
\definecolor{currentfill}{rgb}{0.121569,0.466667,0.705882}%
\pgfsetfillcolor{currentfill}%
\pgfsetfillopacity{0.700000}%
\pgfsetlinewidth{0.000000pt}%
\definecolor{currentstroke}{rgb}{0.000000,0.000000,0.000000}%
\pgfsetstrokecolor{currentstroke}%
\pgfsetstrokeopacity{0.700000}%
\pgfsetdash{}{0pt}%
\pgfpathmoveto{\pgfqpoint{1.460085in}{3.833024in}}%
\pgfpathlineto{\pgfqpoint{1.591835in}{3.833024in}}%
\pgfpathlineto{\pgfqpoint{1.591835in}{6.166237in}}%
\pgfpathlineto{\pgfqpoint{1.460085in}{6.166237in}}%
\pgfpathlineto{\pgfqpoint{1.460085in}{3.833024in}}%
\pgfpathclose%
\pgfusepath{fill}%
\end{pgfscope}%
\begin{pgfscope}%
\pgfpathrectangle{\pgfqpoint{0.603704in}{3.833024in}}{\pgfqpoint{7.246296in}{2.525309in}}%
\pgfusepath{clip}%
\pgfsetbuttcap%
\pgfsetmiterjoin%
\definecolor{currentfill}{rgb}{0.121569,0.466667,0.705882}%
\pgfsetfillcolor{currentfill}%
\pgfsetfillopacity{0.700000}%
\pgfsetlinewidth{0.000000pt}%
\definecolor{currentstroke}{rgb}{0.000000,0.000000,0.000000}%
\pgfsetstrokecolor{currentstroke}%
\pgfsetstrokeopacity{0.700000}%
\pgfsetdash{}{0pt}%
\pgfpathmoveto{\pgfqpoint{1.591835in}{3.833024in}}%
\pgfpathlineto{\pgfqpoint{1.723586in}{3.833024in}}%
\pgfpathlineto{\pgfqpoint{1.723586in}{6.130885in}}%
\pgfpathlineto{\pgfqpoint{1.591835in}{6.130885in}}%
\pgfpathlineto{\pgfqpoint{1.591835in}{3.833024in}}%
\pgfpathclose%
\pgfusepath{fill}%
\end{pgfscope}%
\begin{pgfscope}%
\pgfpathrectangle{\pgfqpoint{0.603704in}{3.833024in}}{\pgfqpoint{7.246296in}{2.525309in}}%
\pgfusepath{clip}%
\pgfsetbuttcap%
\pgfsetmiterjoin%
\definecolor{currentfill}{rgb}{0.121569,0.466667,0.705882}%
\pgfsetfillcolor{currentfill}%
\pgfsetfillopacity{0.700000}%
\pgfsetlinewidth{0.000000pt}%
\definecolor{currentstroke}{rgb}{0.000000,0.000000,0.000000}%
\pgfsetstrokecolor{currentstroke}%
\pgfsetstrokeopacity{0.700000}%
\pgfsetdash{}{0pt}%
\pgfpathmoveto{\pgfqpoint{1.723586in}{3.833024in}}%
\pgfpathlineto{\pgfqpoint{1.855337in}{3.833024in}}%
\pgfpathlineto{\pgfqpoint{1.855337in}{6.121762in}}%
\pgfpathlineto{\pgfqpoint{1.723586in}{6.121762in}}%
\pgfpathlineto{\pgfqpoint{1.723586in}{3.833024in}}%
\pgfpathclose%
\pgfusepath{fill}%
\end{pgfscope}%
\begin{pgfscope}%
\pgfpathrectangle{\pgfqpoint{0.603704in}{3.833024in}}{\pgfqpoint{7.246296in}{2.525309in}}%
\pgfusepath{clip}%
\pgfsetbuttcap%
\pgfsetmiterjoin%
\definecolor{currentfill}{rgb}{0.121569,0.466667,0.705882}%
\pgfsetfillcolor{currentfill}%
\pgfsetfillopacity{0.700000}%
\pgfsetlinewidth{0.000000pt}%
\definecolor{currentstroke}{rgb}{0.000000,0.000000,0.000000}%
\pgfsetstrokecolor{currentstroke}%
\pgfsetstrokeopacity{0.700000}%
\pgfsetdash{}{0pt}%
\pgfpathmoveto{\pgfqpoint{1.855337in}{3.833024in}}%
\pgfpathlineto{\pgfqpoint{1.987088in}{3.833024in}}%
\pgfpathlineto{\pgfqpoint{1.987088in}{6.122902in}}%
\pgfpathlineto{\pgfqpoint{1.855337in}{6.122902in}}%
\pgfpathlineto{\pgfqpoint{1.855337in}{3.833024in}}%
\pgfpathclose%
\pgfusepath{fill}%
\end{pgfscope}%
\begin{pgfscope}%
\pgfpathrectangle{\pgfqpoint{0.603704in}{3.833024in}}{\pgfqpoint{7.246296in}{2.525309in}}%
\pgfusepath{clip}%
\pgfsetbuttcap%
\pgfsetmiterjoin%
\definecolor{currentfill}{rgb}{0.121569,0.466667,0.705882}%
\pgfsetfillcolor{currentfill}%
\pgfsetfillopacity{0.700000}%
\pgfsetlinewidth{0.000000pt}%
\definecolor{currentstroke}{rgb}{0.000000,0.000000,0.000000}%
\pgfsetstrokecolor{currentstroke}%
\pgfsetstrokeopacity{0.700000}%
\pgfsetdash{}{0pt}%
\pgfpathmoveto{\pgfqpoint{1.987088in}{3.833024in}}%
\pgfpathlineto{\pgfqpoint{2.118839in}{3.833024in}}%
\pgfpathlineto{\pgfqpoint{2.118839in}{6.189044in}}%
\pgfpathlineto{\pgfqpoint{1.987088in}{6.189044in}}%
\pgfpathlineto{\pgfqpoint{1.987088in}{3.833024in}}%
\pgfpathclose%
\pgfusepath{fill}%
\end{pgfscope}%
\begin{pgfscope}%
\pgfpathrectangle{\pgfqpoint{0.603704in}{3.833024in}}{\pgfqpoint{7.246296in}{2.525309in}}%
\pgfusepath{clip}%
\pgfsetbuttcap%
\pgfsetmiterjoin%
\definecolor{currentfill}{rgb}{0.121569,0.466667,0.705882}%
\pgfsetfillcolor{currentfill}%
\pgfsetfillopacity{0.700000}%
\pgfsetlinewidth{0.000000pt}%
\definecolor{currentstroke}{rgb}{0.000000,0.000000,0.000000}%
\pgfsetstrokecolor{currentstroke}%
\pgfsetstrokeopacity{0.700000}%
\pgfsetdash{}{0pt}%
\pgfpathmoveto{\pgfqpoint{2.118839in}{3.833024in}}%
\pgfpathlineto{\pgfqpoint{2.250590in}{3.833024in}}%
\pgfpathlineto{\pgfqpoint{2.250590in}{6.136587in}}%
\pgfpathlineto{\pgfqpoint{2.118839in}{6.136587in}}%
\pgfpathlineto{\pgfqpoint{2.118839in}{3.833024in}}%
\pgfpathclose%
\pgfusepath{fill}%
\end{pgfscope}%
\begin{pgfscope}%
\pgfpathrectangle{\pgfqpoint{0.603704in}{3.833024in}}{\pgfqpoint{7.246296in}{2.525309in}}%
\pgfusepath{clip}%
\pgfsetbuttcap%
\pgfsetmiterjoin%
\definecolor{currentfill}{rgb}{0.121569,0.466667,0.705882}%
\pgfsetfillcolor{currentfill}%
\pgfsetfillopacity{0.700000}%
\pgfsetlinewidth{0.000000pt}%
\definecolor{currentstroke}{rgb}{0.000000,0.000000,0.000000}%
\pgfsetstrokecolor{currentstroke}%
\pgfsetstrokeopacity{0.700000}%
\pgfsetdash{}{0pt}%
\pgfpathmoveto{\pgfqpoint{2.250590in}{3.833024in}}%
\pgfpathlineto{\pgfqpoint{2.382340in}{3.833024in}}%
\pgfpathlineto{\pgfqpoint{2.382340in}{6.038514in}}%
\pgfpathlineto{\pgfqpoint{2.250590in}{6.038514in}}%
\pgfpathlineto{\pgfqpoint{2.250590in}{3.833024in}}%
\pgfpathclose%
\pgfusepath{fill}%
\end{pgfscope}%
\begin{pgfscope}%
\pgfpathrectangle{\pgfqpoint{0.603704in}{3.833024in}}{\pgfqpoint{7.246296in}{2.525309in}}%
\pgfusepath{clip}%
\pgfsetbuttcap%
\pgfsetmiterjoin%
\definecolor{currentfill}{rgb}{0.121569,0.466667,0.705882}%
\pgfsetfillcolor{currentfill}%
\pgfsetfillopacity{0.700000}%
\pgfsetlinewidth{0.000000pt}%
\definecolor{currentstroke}{rgb}{0.000000,0.000000,0.000000}%
\pgfsetstrokecolor{currentstroke}%
\pgfsetstrokeopacity{0.700000}%
\pgfsetdash{}{0pt}%
\pgfpathmoveto{\pgfqpoint{2.382340in}{3.833024in}}%
\pgfpathlineto{\pgfqpoint{2.514091in}{3.833024in}}%
\pgfpathlineto{\pgfqpoint{2.514091in}{6.024830in}}%
\pgfpathlineto{\pgfqpoint{2.382340in}{6.024830in}}%
\pgfpathlineto{\pgfqpoint{2.382340in}{3.833024in}}%
\pgfpathclose%
\pgfusepath{fill}%
\end{pgfscope}%
\begin{pgfscope}%
\pgfpathrectangle{\pgfqpoint{0.603704in}{3.833024in}}{\pgfqpoint{7.246296in}{2.525309in}}%
\pgfusepath{clip}%
\pgfsetbuttcap%
\pgfsetmiterjoin%
\definecolor{currentfill}{rgb}{0.121569,0.466667,0.705882}%
\pgfsetfillcolor{currentfill}%
\pgfsetfillopacity{0.700000}%
\pgfsetlinewidth{0.000000pt}%
\definecolor{currentstroke}{rgb}{0.000000,0.000000,0.000000}%
\pgfsetstrokecolor{currentstroke}%
\pgfsetstrokeopacity{0.700000}%
\pgfsetdash{}{0pt}%
\pgfpathmoveto{\pgfqpoint{2.514091in}{3.833024in}}%
\pgfpathlineto{\pgfqpoint{2.645842in}{3.833024in}}%
\pgfpathlineto{\pgfqpoint{2.645842in}{6.143429in}}%
\pgfpathlineto{\pgfqpoint{2.514091in}{6.143429in}}%
\pgfpathlineto{\pgfqpoint{2.514091in}{3.833024in}}%
\pgfpathclose%
\pgfusepath{fill}%
\end{pgfscope}%
\begin{pgfscope}%
\pgfpathrectangle{\pgfqpoint{0.603704in}{3.833024in}}{\pgfqpoint{7.246296in}{2.525309in}}%
\pgfusepath{clip}%
\pgfsetbuttcap%
\pgfsetmiterjoin%
\definecolor{currentfill}{rgb}{0.121569,0.466667,0.705882}%
\pgfsetfillcolor{currentfill}%
\pgfsetfillopacity{0.700000}%
\pgfsetlinewidth{0.000000pt}%
\definecolor{currentstroke}{rgb}{0.000000,0.000000,0.000000}%
\pgfsetstrokecolor{currentstroke}%
\pgfsetstrokeopacity{0.700000}%
\pgfsetdash{}{0pt}%
\pgfpathmoveto{\pgfqpoint{2.645842in}{3.833024in}}%
\pgfpathlineto{\pgfqpoint{2.777593in}{3.833024in}}%
\pgfpathlineto{\pgfqpoint{2.777593in}{6.088691in}}%
\pgfpathlineto{\pgfqpoint{2.645842in}{6.088691in}}%
\pgfpathlineto{\pgfqpoint{2.645842in}{3.833024in}}%
\pgfpathclose%
\pgfusepath{fill}%
\end{pgfscope}%
\begin{pgfscope}%
\pgfpathrectangle{\pgfqpoint{0.603704in}{3.833024in}}{\pgfqpoint{7.246296in}{2.525309in}}%
\pgfusepath{clip}%
\pgfsetbuttcap%
\pgfsetmiterjoin%
\definecolor{currentfill}{rgb}{0.121569,0.466667,0.705882}%
\pgfsetfillcolor{currentfill}%
\pgfsetfillopacity{0.700000}%
\pgfsetlinewidth{0.000000pt}%
\definecolor{currentstroke}{rgb}{0.000000,0.000000,0.000000}%
\pgfsetstrokecolor{currentstroke}%
\pgfsetstrokeopacity{0.700000}%
\pgfsetdash{}{0pt}%
\pgfpathmoveto{\pgfqpoint{2.777593in}{3.833024in}}%
\pgfpathlineto{\pgfqpoint{2.909344in}{3.833024in}}%
\pgfpathlineto{\pgfqpoint{2.909344in}{6.111499in}}%
\pgfpathlineto{\pgfqpoint{2.777593in}{6.111499in}}%
\pgfpathlineto{\pgfqpoint{2.777593in}{3.833024in}}%
\pgfpathclose%
\pgfusepath{fill}%
\end{pgfscope}%
\begin{pgfscope}%
\pgfpathrectangle{\pgfqpoint{0.603704in}{3.833024in}}{\pgfqpoint{7.246296in}{2.525309in}}%
\pgfusepath{clip}%
\pgfsetbuttcap%
\pgfsetmiterjoin%
\definecolor{currentfill}{rgb}{0.121569,0.466667,0.705882}%
\pgfsetfillcolor{currentfill}%
\pgfsetfillopacity{0.700000}%
\pgfsetlinewidth{0.000000pt}%
\definecolor{currentstroke}{rgb}{0.000000,0.000000,0.000000}%
\pgfsetstrokecolor{currentstroke}%
\pgfsetstrokeopacity{0.700000}%
\pgfsetdash{}{0pt}%
\pgfpathmoveto{\pgfqpoint{2.909344in}{3.833024in}}%
\pgfpathlineto{\pgfqpoint{3.041095in}{3.833024in}}%
\pgfpathlineto{\pgfqpoint{3.041095in}{6.087551in}}%
\pgfpathlineto{\pgfqpoint{2.909344in}{6.087551in}}%
\pgfpathlineto{\pgfqpoint{2.909344in}{3.833024in}}%
\pgfpathclose%
\pgfusepath{fill}%
\end{pgfscope}%
\begin{pgfscope}%
\pgfpathrectangle{\pgfqpoint{0.603704in}{3.833024in}}{\pgfqpoint{7.246296in}{2.525309in}}%
\pgfusepath{clip}%
\pgfsetbuttcap%
\pgfsetmiterjoin%
\definecolor{currentfill}{rgb}{0.121569,0.466667,0.705882}%
\pgfsetfillcolor{currentfill}%
\pgfsetfillopacity{0.700000}%
\pgfsetlinewidth{0.000000pt}%
\definecolor{currentstroke}{rgb}{0.000000,0.000000,0.000000}%
\pgfsetstrokecolor{currentstroke}%
\pgfsetstrokeopacity{0.700000}%
\pgfsetdash{}{0pt}%
\pgfpathmoveto{\pgfqpoint{3.041095in}{3.833024in}}%
\pgfpathlineto{\pgfqpoint{3.172845in}{3.833024in}}%
\pgfpathlineto{\pgfqpoint{3.172845in}{6.167377in}}%
\pgfpathlineto{\pgfqpoint{3.041095in}{6.167377in}}%
\pgfpathlineto{\pgfqpoint{3.041095in}{3.833024in}}%
\pgfpathclose%
\pgfusepath{fill}%
\end{pgfscope}%
\begin{pgfscope}%
\pgfpathrectangle{\pgfqpoint{0.603704in}{3.833024in}}{\pgfqpoint{7.246296in}{2.525309in}}%
\pgfusepath{clip}%
\pgfsetbuttcap%
\pgfsetmiterjoin%
\definecolor{currentfill}{rgb}{0.121569,0.466667,0.705882}%
\pgfsetfillcolor{currentfill}%
\pgfsetfillopacity{0.700000}%
\pgfsetlinewidth{0.000000pt}%
\definecolor{currentstroke}{rgb}{0.000000,0.000000,0.000000}%
\pgfsetstrokecolor{currentstroke}%
\pgfsetstrokeopacity{0.700000}%
\pgfsetdash{}{0pt}%
\pgfpathmoveto{\pgfqpoint{3.172845in}{3.833024in}}%
\pgfpathlineto{\pgfqpoint{3.304596in}{3.833024in}}%
\pgfpathlineto{\pgfqpoint{3.304596in}{6.089831in}}%
\pgfpathlineto{\pgfqpoint{3.172845in}{6.089831in}}%
\pgfpathlineto{\pgfqpoint{3.172845in}{3.833024in}}%
\pgfpathclose%
\pgfusepath{fill}%
\end{pgfscope}%
\begin{pgfscope}%
\pgfpathrectangle{\pgfqpoint{0.603704in}{3.833024in}}{\pgfqpoint{7.246296in}{2.525309in}}%
\pgfusepath{clip}%
\pgfsetbuttcap%
\pgfsetmiterjoin%
\definecolor{currentfill}{rgb}{0.121569,0.466667,0.705882}%
\pgfsetfillcolor{currentfill}%
\pgfsetfillopacity{0.700000}%
\pgfsetlinewidth{0.000000pt}%
\definecolor{currentstroke}{rgb}{0.000000,0.000000,0.000000}%
\pgfsetstrokecolor{currentstroke}%
\pgfsetstrokeopacity{0.700000}%
\pgfsetdash{}{0pt}%
\pgfpathmoveto{\pgfqpoint{3.304596in}{3.833024in}}%
\pgfpathlineto{\pgfqpoint{3.436347in}{3.833024in}}%
\pgfpathlineto{\pgfqpoint{3.436347in}{6.106937in}}%
\pgfpathlineto{\pgfqpoint{3.304596in}{6.106937in}}%
\pgfpathlineto{\pgfqpoint{3.304596in}{3.833024in}}%
\pgfpathclose%
\pgfusepath{fill}%
\end{pgfscope}%
\begin{pgfscope}%
\pgfpathrectangle{\pgfqpoint{0.603704in}{3.833024in}}{\pgfqpoint{7.246296in}{2.525309in}}%
\pgfusepath{clip}%
\pgfsetbuttcap%
\pgfsetmiterjoin%
\definecolor{currentfill}{rgb}{0.121569,0.466667,0.705882}%
\pgfsetfillcolor{currentfill}%
\pgfsetfillopacity{0.700000}%
\pgfsetlinewidth{0.000000pt}%
\definecolor{currentstroke}{rgb}{0.000000,0.000000,0.000000}%
\pgfsetstrokecolor{currentstroke}%
\pgfsetstrokeopacity{0.700000}%
\pgfsetdash{}{0pt}%
\pgfpathmoveto{\pgfqpoint{3.436347in}{3.833024in}}%
\pgfpathlineto{\pgfqpoint{3.568098in}{3.833024in}}%
\pgfpathlineto{\pgfqpoint{3.568098in}{6.162816in}}%
\pgfpathlineto{\pgfqpoint{3.436347in}{6.162816in}}%
\pgfpathlineto{\pgfqpoint{3.436347in}{3.833024in}}%
\pgfpathclose%
\pgfusepath{fill}%
\end{pgfscope}%
\begin{pgfscope}%
\pgfpathrectangle{\pgfqpoint{0.603704in}{3.833024in}}{\pgfqpoint{7.246296in}{2.525309in}}%
\pgfusepath{clip}%
\pgfsetbuttcap%
\pgfsetmiterjoin%
\definecolor{currentfill}{rgb}{0.121569,0.466667,0.705882}%
\pgfsetfillcolor{currentfill}%
\pgfsetfillopacity{0.700000}%
\pgfsetlinewidth{0.000000pt}%
\definecolor{currentstroke}{rgb}{0.000000,0.000000,0.000000}%
\pgfsetstrokecolor{currentstroke}%
\pgfsetstrokeopacity{0.700000}%
\pgfsetdash{}{0pt}%
\pgfpathmoveto{\pgfqpoint{3.568098in}{3.833024in}}%
\pgfpathlineto{\pgfqpoint{3.699849in}{3.833024in}}%
\pgfpathlineto{\pgfqpoint{3.699849in}{6.230098in}}%
\pgfpathlineto{\pgfqpoint{3.568098in}{6.230098in}}%
\pgfpathlineto{\pgfqpoint{3.568098in}{3.833024in}}%
\pgfpathclose%
\pgfusepath{fill}%
\end{pgfscope}%
\begin{pgfscope}%
\pgfpathrectangle{\pgfqpoint{0.603704in}{3.833024in}}{\pgfqpoint{7.246296in}{2.525309in}}%
\pgfusepath{clip}%
\pgfsetbuttcap%
\pgfsetmiterjoin%
\definecolor{currentfill}{rgb}{0.121569,0.466667,0.705882}%
\pgfsetfillcolor{currentfill}%
\pgfsetfillopacity{0.700000}%
\pgfsetlinewidth{0.000000pt}%
\definecolor{currentstroke}{rgb}{0.000000,0.000000,0.000000}%
\pgfsetstrokecolor{currentstroke}%
\pgfsetstrokeopacity{0.700000}%
\pgfsetdash{}{0pt}%
\pgfpathmoveto{\pgfqpoint{3.699849in}{3.833024in}}%
\pgfpathlineto{\pgfqpoint{3.831600in}{3.833024in}}%
\pgfpathlineto{\pgfqpoint{3.831600in}{6.087551in}}%
\pgfpathlineto{\pgfqpoint{3.699849in}{6.087551in}}%
\pgfpathlineto{\pgfqpoint{3.699849in}{3.833024in}}%
\pgfpathclose%
\pgfusepath{fill}%
\end{pgfscope}%
\begin{pgfscope}%
\pgfpathrectangle{\pgfqpoint{0.603704in}{3.833024in}}{\pgfqpoint{7.246296in}{2.525309in}}%
\pgfusepath{clip}%
\pgfsetbuttcap%
\pgfsetmiterjoin%
\definecolor{currentfill}{rgb}{0.121569,0.466667,0.705882}%
\pgfsetfillcolor{currentfill}%
\pgfsetfillopacity{0.700000}%
\pgfsetlinewidth{0.000000pt}%
\definecolor{currentstroke}{rgb}{0.000000,0.000000,0.000000}%
\pgfsetstrokecolor{currentstroke}%
\pgfsetstrokeopacity{0.700000}%
\pgfsetdash{}{0pt}%
\pgfpathmoveto{\pgfqpoint{3.831600in}{3.833024in}}%
\pgfpathlineto{\pgfqpoint{3.963350in}{3.833024in}}%
\pgfpathlineto{\pgfqpoint{3.963350in}{6.047637in}}%
\pgfpathlineto{\pgfqpoint{3.831600in}{6.047637in}}%
\pgfpathlineto{\pgfqpoint{3.831600in}{3.833024in}}%
\pgfpathclose%
\pgfusepath{fill}%
\end{pgfscope}%
\begin{pgfscope}%
\pgfpathrectangle{\pgfqpoint{0.603704in}{3.833024in}}{\pgfqpoint{7.246296in}{2.525309in}}%
\pgfusepath{clip}%
\pgfsetbuttcap%
\pgfsetmiterjoin%
\definecolor{currentfill}{rgb}{0.121569,0.466667,0.705882}%
\pgfsetfillcolor{currentfill}%
\pgfsetfillopacity{0.700000}%
\pgfsetlinewidth{0.000000pt}%
\definecolor{currentstroke}{rgb}{0.000000,0.000000,0.000000}%
\pgfsetstrokecolor{currentstroke}%
\pgfsetstrokeopacity{0.700000}%
\pgfsetdash{}{0pt}%
\pgfpathmoveto{\pgfqpoint{3.963350in}{3.833024in}}%
\pgfpathlineto{\pgfqpoint{4.095101in}{3.833024in}}%
\pgfpathlineto{\pgfqpoint{4.095101in}{6.081849in}}%
\pgfpathlineto{\pgfqpoint{3.963350in}{6.081849in}}%
\pgfpathlineto{\pgfqpoint{3.963350in}{3.833024in}}%
\pgfpathclose%
\pgfusepath{fill}%
\end{pgfscope}%
\begin{pgfscope}%
\pgfpathrectangle{\pgfqpoint{0.603704in}{3.833024in}}{\pgfqpoint{7.246296in}{2.525309in}}%
\pgfusepath{clip}%
\pgfsetbuttcap%
\pgfsetmiterjoin%
\definecolor{currentfill}{rgb}{0.121569,0.466667,0.705882}%
\pgfsetfillcolor{currentfill}%
\pgfsetfillopacity{0.700000}%
\pgfsetlinewidth{0.000000pt}%
\definecolor{currentstroke}{rgb}{0.000000,0.000000,0.000000}%
\pgfsetstrokecolor{currentstroke}%
\pgfsetstrokeopacity{0.700000}%
\pgfsetdash{}{0pt}%
\pgfpathmoveto{\pgfqpoint{4.095101in}{3.833024in}}%
\pgfpathlineto{\pgfqpoint{4.226852in}{3.833024in}}%
\pgfpathlineto{\pgfqpoint{4.226852in}{6.069305in}}%
\pgfpathlineto{\pgfqpoint{4.095101in}{6.069305in}}%
\pgfpathlineto{\pgfqpoint{4.095101in}{3.833024in}}%
\pgfpathclose%
\pgfusepath{fill}%
\end{pgfscope}%
\begin{pgfscope}%
\pgfpathrectangle{\pgfqpoint{0.603704in}{3.833024in}}{\pgfqpoint{7.246296in}{2.525309in}}%
\pgfusepath{clip}%
\pgfsetbuttcap%
\pgfsetmiterjoin%
\definecolor{currentfill}{rgb}{0.121569,0.466667,0.705882}%
\pgfsetfillcolor{currentfill}%
\pgfsetfillopacity{0.700000}%
\pgfsetlinewidth{0.000000pt}%
\definecolor{currentstroke}{rgb}{0.000000,0.000000,0.000000}%
\pgfsetstrokecolor{currentstroke}%
\pgfsetstrokeopacity{0.700000}%
\pgfsetdash{}{0pt}%
\pgfpathmoveto{\pgfqpoint{4.226852in}{3.833024in}}%
\pgfpathlineto{\pgfqpoint{4.358603in}{3.833024in}}%
\pgfpathlineto{\pgfqpoint{4.358603in}{6.102376in}}%
\pgfpathlineto{\pgfqpoint{4.226852in}{6.102376in}}%
\pgfpathlineto{\pgfqpoint{4.226852in}{3.833024in}}%
\pgfpathclose%
\pgfusepath{fill}%
\end{pgfscope}%
\begin{pgfscope}%
\pgfpathrectangle{\pgfqpoint{0.603704in}{3.833024in}}{\pgfqpoint{7.246296in}{2.525309in}}%
\pgfusepath{clip}%
\pgfsetbuttcap%
\pgfsetmiterjoin%
\definecolor{currentfill}{rgb}{0.121569,0.466667,0.705882}%
\pgfsetfillcolor{currentfill}%
\pgfsetfillopacity{0.700000}%
\pgfsetlinewidth{0.000000pt}%
\definecolor{currentstroke}{rgb}{0.000000,0.000000,0.000000}%
\pgfsetstrokecolor{currentstroke}%
\pgfsetstrokeopacity{0.700000}%
\pgfsetdash{}{0pt}%
\pgfpathmoveto{\pgfqpoint{4.358603in}{3.833024in}}%
\pgfpathlineto{\pgfqpoint{4.490354in}{3.833024in}}%
\pgfpathlineto{\pgfqpoint{4.490354in}{6.059041in}}%
\pgfpathlineto{\pgfqpoint{4.358603in}{6.059041in}}%
\pgfpathlineto{\pgfqpoint{4.358603in}{3.833024in}}%
\pgfpathclose%
\pgfusepath{fill}%
\end{pgfscope}%
\begin{pgfscope}%
\pgfpathrectangle{\pgfqpoint{0.603704in}{3.833024in}}{\pgfqpoint{7.246296in}{2.525309in}}%
\pgfusepath{clip}%
\pgfsetbuttcap%
\pgfsetmiterjoin%
\definecolor{currentfill}{rgb}{0.121569,0.466667,0.705882}%
\pgfsetfillcolor{currentfill}%
\pgfsetfillopacity{0.700000}%
\pgfsetlinewidth{0.000000pt}%
\definecolor{currentstroke}{rgb}{0.000000,0.000000,0.000000}%
\pgfsetstrokecolor{currentstroke}%
\pgfsetstrokeopacity{0.700000}%
\pgfsetdash{}{0pt}%
\pgfpathmoveto{\pgfqpoint{4.490354in}{3.833024in}}%
\pgfpathlineto{\pgfqpoint{4.622105in}{3.833024in}}%
\pgfpathlineto{\pgfqpoint{4.622105in}{6.218694in}}%
\pgfpathlineto{\pgfqpoint{4.490354in}{6.218694in}}%
\pgfpathlineto{\pgfqpoint{4.490354in}{3.833024in}}%
\pgfpathclose%
\pgfusepath{fill}%
\end{pgfscope}%
\begin{pgfscope}%
\pgfpathrectangle{\pgfqpoint{0.603704in}{3.833024in}}{\pgfqpoint{7.246296in}{2.525309in}}%
\pgfusepath{clip}%
\pgfsetbuttcap%
\pgfsetmiterjoin%
\definecolor{currentfill}{rgb}{0.121569,0.466667,0.705882}%
\pgfsetfillcolor{currentfill}%
\pgfsetfillopacity{0.700000}%
\pgfsetlinewidth{0.000000pt}%
\definecolor{currentstroke}{rgb}{0.000000,0.000000,0.000000}%
\pgfsetstrokecolor{currentstroke}%
\pgfsetstrokeopacity{0.700000}%
\pgfsetdash{}{0pt}%
\pgfpathmoveto{\pgfqpoint{4.622105in}{3.833024in}}%
\pgfpathlineto{\pgfqpoint{4.753855in}{3.833024in}}%
\pgfpathlineto{\pgfqpoint{4.753855in}{6.101235in}}%
\pgfpathlineto{\pgfqpoint{4.622105in}{6.101235in}}%
\pgfpathlineto{\pgfqpoint{4.622105in}{3.833024in}}%
\pgfpathclose%
\pgfusepath{fill}%
\end{pgfscope}%
\begin{pgfscope}%
\pgfpathrectangle{\pgfqpoint{0.603704in}{3.833024in}}{\pgfqpoint{7.246296in}{2.525309in}}%
\pgfusepath{clip}%
\pgfsetbuttcap%
\pgfsetmiterjoin%
\definecolor{currentfill}{rgb}{0.121569,0.466667,0.705882}%
\pgfsetfillcolor{currentfill}%
\pgfsetfillopacity{0.700000}%
\pgfsetlinewidth{0.000000pt}%
\definecolor{currentstroke}{rgb}{0.000000,0.000000,0.000000}%
\pgfsetstrokecolor{currentstroke}%
\pgfsetstrokeopacity{0.700000}%
\pgfsetdash{}{0pt}%
\pgfpathmoveto{\pgfqpoint{4.753855in}{3.833024in}}%
\pgfpathlineto{\pgfqpoint{4.885606in}{3.833024in}}%
\pgfpathlineto{\pgfqpoint{4.885606in}{6.081849in}}%
\pgfpathlineto{\pgfqpoint{4.753855in}{6.081849in}}%
\pgfpathlineto{\pgfqpoint{4.753855in}{3.833024in}}%
\pgfpathclose%
\pgfusepath{fill}%
\end{pgfscope}%
\begin{pgfscope}%
\pgfpathrectangle{\pgfqpoint{0.603704in}{3.833024in}}{\pgfqpoint{7.246296in}{2.525309in}}%
\pgfusepath{clip}%
\pgfsetbuttcap%
\pgfsetmiterjoin%
\definecolor{currentfill}{rgb}{0.121569,0.466667,0.705882}%
\pgfsetfillcolor{currentfill}%
\pgfsetfillopacity{0.700000}%
\pgfsetlinewidth{0.000000pt}%
\definecolor{currentstroke}{rgb}{0.000000,0.000000,0.000000}%
\pgfsetstrokecolor{currentstroke}%
\pgfsetstrokeopacity{0.700000}%
\pgfsetdash{}{0pt}%
\pgfpathmoveto{\pgfqpoint{4.885606in}{3.833024in}}%
\pgfpathlineto{\pgfqpoint{5.017357in}{3.833024in}}%
\pgfpathlineto{\pgfqpoint{5.017357in}{6.238081in}}%
\pgfpathlineto{\pgfqpoint{4.885606in}{6.238081in}}%
\pgfpathlineto{\pgfqpoint{4.885606in}{3.833024in}}%
\pgfpathclose%
\pgfusepath{fill}%
\end{pgfscope}%
\begin{pgfscope}%
\pgfpathrectangle{\pgfqpoint{0.603704in}{3.833024in}}{\pgfqpoint{7.246296in}{2.525309in}}%
\pgfusepath{clip}%
\pgfsetbuttcap%
\pgfsetmiterjoin%
\definecolor{currentfill}{rgb}{0.121569,0.466667,0.705882}%
\pgfsetfillcolor{currentfill}%
\pgfsetfillopacity{0.700000}%
\pgfsetlinewidth{0.000000pt}%
\definecolor{currentstroke}{rgb}{0.000000,0.000000,0.000000}%
\pgfsetstrokecolor{currentstroke}%
\pgfsetstrokeopacity{0.700000}%
\pgfsetdash{}{0pt}%
\pgfpathmoveto{\pgfqpoint{5.017357in}{3.833024in}}%
\pgfpathlineto{\pgfqpoint{5.149108in}{3.833024in}}%
\pgfpathlineto{\pgfqpoint{5.149108in}{6.067024in}}%
\pgfpathlineto{\pgfqpoint{5.017357in}{6.067024in}}%
\pgfpathlineto{\pgfqpoint{5.017357in}{3.833024in}}%
\pgfpathclose%
\pgfusepath{fill}%
\end{pgfscope}%
\begin{pgfscope}%
\pgfpathrectangle{\pgfqpoint{0.603704in}{3.833024in}}{\pgfqpoint{7.246296in}{2.525309in}}%
\pgfusepath{clip}%
\pgfsetbuttcap%
\pgfsetmiterjoin%
\definecolor{currentfill}{rgb}{0.121569,0.466667,0.705882}%
\pgfsetfillcolor{currentfill}%
\pgfsetfillopacity{0.700000}%
\pgfsetlinewidth{0.000000pt}%
\definecolor{currentstroke}{rgb}{0.000000,0.000000,0.000000}%
\pgfsetstrokecolor{currentstroke}%
\pgfsetstrokeopacity{0.700000}%
\pgfsetdash{}{0pt}%
\pgfpathmoveto{\pgfqpoint{5.149108in}{3.833024in}}%
\pgfpathlineto{\pgfqpoint{5.280859in}{3.833024in}}%
\pgfpathlineto{\pgfqpoint{5.280859in}{6.160535in}}%
\pgfpathlineto{\pgfqpoint{5.149108in}{6.160535in}}%
\pgfpathlineto{\pgfqpoint{5.149108in}{3.833024in}}%
\pgfpathclose%
\pgfusepath{fill}%
\end{pgfscope}%
\begin{pgfscope}%
\pgfpathrectangle{\pgfqpoint{0.603704in}{3.833024in}}{\pgfqpoint{7.246296in}{2.525309in}}%
\pgfusepath{clip}%
\pgfsetbuttcap%
\pgfsetmiterjoin%
\definecolor{currentfill}{rgb}{0.121569,0.466667,0.705882}%
\pgfsetfillcolor{currentfill}%
\pgfsetfillopacity{0.700000}%
\pgfsetlinewidth{0.000000pt}%
\definecolor{currentstroke}{rgb}{0.000000,0.000000,0.000000}%
\pgfsetstrokecolor{currentstroke}%
\pgfsetstrokeopacity{0.700000}%
\pgfsetdash{}{0pt}%
\pgfpathmoveto{\pgfqpoint{5.280859in}{3.833024in}}%
\pgfpathlineto{\pgfqpoint{5.412610in}{3.833024in}}%
\pgfpathlineto{\pgfqpoint{5.412610in}{6.125183in}}%
\pgfpathlineto{\pgfqpoint{5.280859in}{6.125183in}}%
\pgfpathlineto{\pgfqpoint{5.280859in}{3.833024in}}%
\pgfpathclose%
\pgfusepath{fill}%
\end{pgfscope}%
\begin{pgfscope}%
\pgfpathrectangle{\pgfqpoint{0.603704in}{3.833024in}}{\pgfqpoint{7.246296in}{2.525309in}}%
\pgfusepath{clip}%
\pgfsetbuttcap%
\pgfsetmiterjoin%
\definecolor{currentfill}{rgb}{0.121569,0.466667,0.705882}%
\pgfsetfillcolor{currentfill}%
\pgfsetfillopacity{0.700000}%
\pgfsetlinewidth{0.000000pt}%
\definecolor{currentstroke}{rgb}{0.000000,0.000000,0.000000}%
\pgfsetstrokecolor{currentstroke}%
\pgfsetstrokeopacity{0.700000}%
\pgfsetdash{}{0pt}%
\pgfpathmoveto{\pgfqpoint{5.412610in}{3.833024in}}%
\pgfpathlineto{\pgfqpoint{5.544360in}{3.833024in}}%
\pgfpathlineto{\pgfqpoint{5.544360in}{6.186764in}}%
\pgfpathlineto{\pgfqpoint{5.412610in}{6.186764in}}%
\pgfpathlineto{\pgfqpoint{5.412610in}{3.833024in}}%
\pgfpathclose%
\pgfusepath{fill}%
\end{pgfscope}%
\begin{pgfscope}%
\pgfpathrectangle{\pgfqpoint{0.603704in}{3.833024in}}{\pgfqpoint{7.246296in}{2.525309in}}%
\pgfusepath{clip}%
\pgfsetbuttcap%
\pgfsetmiterjoin%
\definecolor{currentfill}{rgb}{0.121569,0.466667,0.705882}%
\pgfsetfillcolor{currentfill}%
\pgfsetfillopacity{0.700000}%
\pgfsetlinewidth{0.000000pt}%
\definecolor{currentstroke}{rgb}{0.000000,0.000000,0.000000}%
\pgfsetstrokecolor{currentstroke}%
\pgfsetstrokeopacity{0.700000}%
\pgfsetdash{}{0pt}%
\pgfpathmoveto{\pgfqpoint{5.544360in}{3.833024in}}%
\pgfpathlineto{\pgfqpoint{5.676111in}{3.833024in}}%
\pgfpathlineto{\pgfqpoint{5.676111in}{6.038514in}}%
\pgfpathlineto{\pgfqpoint{5.544360in}{6.038514in}}%
\pgfpathlineto{\pgfqpoint{5.544360in}{3.833024in}}%
\pgfpathclose%
\pgfusepath{fill}%
\end{pgfscope}%
\begin{pgfscope}%
\pgfpathrectangle{\pgfqpoint{0.603704in}{3.833024in}}{\pgfqpoint{7.246296in}{2.525309in}}%
\pgfusepath{clip}%
\pgfsetbuttcap%
\pgfsetmiterjoin%
\definecolor{currentfill}{rgb}{0.121569,0.466667,0.705882}%
\pgfsetfillcolor{currentfill}%
\pgfsetfillopacity{0.700000}%
\pgfsetlinewidth{0.000000pt}%
\definecolor{currentstroke}{rgb}{0.000000,0.000000,0.000000}%
\pgfsetstrokecolor{currentstroke}%
\pgfsetstrokeopacity{0.700000}%
\pgfsetdash{}{0pt}%
\pgfpathmoveto{\pgfqpoint{5.676111in}{3.833024in}}%
\pgfpathlineto{\pgfqpoint{5.807862in}{3.833024in}}%
\pgfpathlineto{\pgfqpoint{5.807862in}{6.096674in}}%
\pgfpathlineto{\pgfqpoint{5.676111in}{6.096674in}}%
\pgfpathlineto{\pgfqpoint{5.676111in}{3.833024in}}%
\pgfpathclose%
\pgfusepath{fill}%
\end{pgfscope}%
\begin{pgfscope}%
\pgfpathrectangle{\pgfqpoint{0.603704in}{3.833024in}}{\pgfqpoint{7.246296in}{2.525309in}}%
\pgfusepath{clip}%
\pgfsetbuttcap%
\pgfsetmiterjoin%
\definecolor{currentfill}{rgb}{0.121569,0.466667,0.705882}%
\pgfsetfillcolor{currentfill}%
\pgfsetfillopacity{0.700000}%
\pgfsetlinewidth{0.000000pt}%
\definecolor{currentstroke}{rgb}{0.000000,0.000000,0.000000}%
\pgfsetstrokecolor{currentstroke}%
\pgfsetstrokeopacity{0.700000}%
\pgfsetdash{}{0pt}%
\pgfpathmoveto{\pgfqpoint{5.807862in}{3.833024in}}%
\pgfpathlineto{\pgfqpoint{5.939613in}{3.833024in}}%
\pgfpathlineto{\pgfqpoint{5.939613in}{6.047637in}}%
\pgfpathlineto{\pgfqpoint{5.807862in}{6.047637in}}%
\pgfpathlineto{\pgfqpoint{5.807862in}{3.833024in}}%
\pgfpathclose%
\pgfusepath{fill}%
\end{pgfscope}%
\begin{pgfscope}%
\pgfpathrectangle{\pgfqpoint{0.603704in}{3.833024in}}{\pgfqpoint{7.246296in}{2.525309in}}%
\pgfusepath{clip}%
\pgfsetbuttcap%
\pgfsetmiterjoin%
\definecolor{currentfill}{rgb}{0.121569,0.466667,0.705882}%
\pgfsetfillcolor{currentfill}%
\pgfsetfillopacity{0.700000}%
\pgfsetlinewidth{0.000000pt}%
\definecolor{currentstroke}{rgb}{0.000000,0.000000,0.000000}%
\pgfsetstrokecolor{currentstroke}%
\pgfsetstrokeopacity{0.700000}%
\pgfsetdash{}{0pt}%
\pgfpathmoveto{\pgfqpoint{5.939613in}{3.833024in}}%
\pgfpathlineto{\pgfqpoint{6.071364in}{3.833024in}}%
\pgfpathlineto{\pgfqpoint{6.071364in}{6.140008in}}%
\pgfpathlineto{\pgfqpoint{5.939613in}{6.140008in}}%
\pgfpathlineto{\pgfqpoint{5.939613in}{3.833024in}}%
\pgfpathclose%
\pgfusepath{fill}%
\end{pgfscope}%
\begin{pgfscope}%
\pgfpathrectangle{\pgfqpoint{0.603704in}{3.833024in}}{\pgfqpoint{7.246296in}{2.525309in}}%
\pgfusepath{clip}%
\pgfsetbuttcap%
\pgfsetmiterjoin%
\definecolor{currentfill}{rgb}{0.121569,0.466667,0.705882}%
\pgfsetfillcolor{currentfill}%
\pgfsetfillopacity{0.700000}%
\pgfsetlinewidth{0.000000pt}%
\definecolor{currentstroke}{rgb}{0.000000,0.000000,0.000000}%
\pgfsetstrokecolor{currentstroke}%
\pgfsetstrokeopacity{0.700000}%
\pgfsetdash{}{0pt}%
\pgfpathmoveto{\pgfqpoint{6.071364in}{3.833024in}}%
\pgfpathlineto{\pgfqpoint{6.203115in}{3.833024in}}%
\pgfpathlineto{\pgfqpoint{6.203115in}{6.150271in}}%
\pgfpathlineto{\pgfqpoint{6.071364in}{6.150271in}}%
\pgfpathlineto{\pgfqpoint{6.071364in}{3.833024in}}%
\pgfpathclose%
\pgfusepath{fill}%
\end{pgfscope}%
\begin{pgfscope}%
\pgfpathrectangle{\pgfqpoint{0.603704in}{3.833024in}}{\pgfqpoint{7.246296in}{2.525309in}}%
\pgfusepath{clip}%
\pgfsetbuttcap%
\pgfsetmiterjoin%
\definecolor{currentfill}{rgb}{0.121569,0.466667,0.705882}%
\pgfsetfillcolor{currentfill}%
\pgfsetfillopacity{0.700000}%
\pgfsetlinewidth{0.000000pt}%
\definecolor{currentstroke}{rgb}{0.000000,0.000000,0.000000}%
\pgfsetstrokecolor{currentstroke}%
\pgfsetstrokeopacity{0.700000}%
\pgfsetdash{}{0pt}%
\pgfpathmoveto{\pgfqpoint{6.203115in}{3.833024in}}%
\pgfpathlineto{\pgfqpoint{6.334865in}{3.833024in}}%
\pgfpathlineto{\pgfqpoint{6.334865in}{6.048778in}}%
\pgfpathlineto{\pgfqpoint{6.203115in}{6.048778in}}%
\pgfpathlineto{\pgfqpoint{6.203115in}{3.833024in}}%
\pgfpathclose%
\pgfusepath{fill}%
\end{pgfscope}%
\begin{pgfscope}%
\pgfpathrectangle{\pgfqpoint{0.603704in}{3.833024in}}{\pgfqpoint{7.246296in}{2.525309in}}%
\pgfusepath{clip}%
\pgfsetbuttcap%
\pgfsetmiterjoin%
\definecolor{currentfill}{rgb}{0.121569,0.466667,0.705882}%
\pgfsetfillcolor{currentfill}%
\pgfsetfillopacity{0.700000}%
\pgfsetlinewidth{0.000000pt}%
\definecolor{currentstroke}{rgb}{0.000000,0.000000,0.000000}%
\pgfsetstrokecolor{currentstroke}%
\pgfsetstrokeopacity{0.700000}%
\pgfsetdash{}{0pt}%
\pgfpathmoveto{\pgfqpoint{6.334865in}{3.833024in}}%
\pgfpathlineto{\pgfqpoint{6.466616in}{3.833024in}}%
\pgfpathlineto{\pgfqpoint{6.466616in}{6.045357in}}%
\pgfpathlineto{\pgfqpoint{6.334865in}{6.045357in}}%
\pgfpathlineto{\pgfqpoint{6.334865in}{3.833024in}}%
\pgfpathclose%
\pgfusepath{fill}%
\end{pgfscope}%
\begin{pgfscope}%
\pgfpathrectangle{\pgfqpoint{0.603704in}{3.833024in}}{\pgfqpoint{7.246296in}{2.525309in}}%
\pgfusepath{clip}%
\pgfsetbuttcap%
\pgfsetmiterjoin%
\definecolor{currentfill}{rgb}{0.121569,0.466667,0.705882}%
\pgfsetfillcolor{currentfill}%
\pgfsetfillopacity{0.700000}%
\pgfsetlinewidth{0.000000pt}%
\definecolor{currentstroke}{rgb}{0.000000,0.000000,0.000000}%
\pgfsetstrokecolor{currentstroke}%
\pgfsetstrokeopacity{0.700000}%
\pgfsetdash{}{0pt}%
\pgfpathmoveto{\pgfqpoint{6.466616in}{3.833024in}}%
\pgfpathlineto{\pgfqpoint{6.598367in}{3.833024in}}%
\pgfpathlineto{\pgfqpoint{6.598367in}{6.119481in}}%
\pgfpathlineto{\pgfqpoint{6.466616in}{6.119481in}}%
\pgfpathlineto{\pgfqpoint{6.466616in}{3.833024in}}%
\pgfpathclose%
\pgfusepath{fill}%
\end{pgfscope}%
\begin{pgfscope}%
\pgfpathrectangle{\pgfqpoint{0.603704in}{3.833024in}}{\pgfqpoint{7.246296in}{2.525309in}}%
\pgfusepath{clip}%
\pgfsetbuttcap%
\pgfsetmiterjoin%
\definecolor{currentfill}{rgb}{0.121569,0.466667,0.705882}%
\pgfsetfillcolor{currentfill}%
\pgfsetfillopacity{0.700000}%
\pgfsetlinewidth{0.000000pt}%
\definecolor{currentstroke}{rgb}{0.000000,0.000000,0.000000}%
\pgfsetstrokecolor{currentstroke}%
\pgfsetstrokeopacity{0.700000}%
\pgfsetdash{}{0pt}%
\pgfpathmoveto{\pgfqpoint{6.598367in}{3.833024in}}%
\pgfpathlineto{\pgfqpoint{6.730118in}{3.833024in}}%
\pgfpathlineto{\pgfqpoint{6.730118in}{6.081849in}}%
\pgfpathlineto{\pgfqpoint{6.598367in}{6.081849in}}%
\pgfpathlineto{\pgfqpoint{6.598367in}{3.833024in}}%
\pgfpathclose%
\pgfusepath{fill}%
\end{pgfscope}%
\begin{pgfscope}%
\pgfpathrectangle{\pgfqpoint{0.603704in}{3.833024in}}{\pgfqpoint{7.246296in}{2.525309in}}%
\pgfusepath{clip}%
\pgfsetbuttcap%
\pgfsetmiterjoin%
\definecolor{currentfill}{rgb}{0.121569,0.466667,0.705882}%
\pgfsetfillcolor{currentfill}%
\pgfsetfillopacity{0.700000}%
\pgfsetlinewidth{0.000000pt}%
\definecolor{currentstroke}{rgb}{0.000000,0.000000,0.000000}%
\pgfsetstrokecolor{currentstroke}%
\pgfsetstrokeopacity{0.700000}%
\pgfsetdash{}{0pt}%
\pgfpathmoveto{\pgfqpoint{6.730118in}{3.833024in}}%
\pgfpathlineto{\pgfqpoint{6.861869in}{3.833024in}}%
\pgfpathlineto{\pgfqpoint{6.861869in}{6.036234in}}%
\pgfpathlineto{\pgfqpoint{6.730118in}{6.036234in}}%
\pgfpathlineto{\pgfqpoint{6.730118in}{3.833024in}}%
\pgfpathclose%
\pgfusepath{fill}%
\end{pgfscope}%
\begin{pgfscope}%
\pgfpathrectangle{\pgfqpoint{0.603704in}{3.833024in}}{\pgfqpoint{7.246296in}{2.525309in}}%
\pgfusepath{clip}%
\pgfsetbuttcap%
\pgfsetmiterjoin%
\definecolor{currentfill}{rgb}{0.121569,0.466667,0.705882}%
\pgfsetfillcolor{currentfill}%
\pgfsetfillopacity{0.700000}%
\pgfsetlinewidth{0.000000pt}%
\definecolor{currentstroke}{rgb}{0.000000,0.000000,0.000000}%
\pgfsetstrokecolor{currentstroke}%
\pgfsetstrokeopacity{0.700000}%
\pgfsetdash{}{0pt}%
\pgfpathmoveto{\pgfqpoint{6.861869in}{3.833024in}}%
\pgfpathlineto{\pgfqpoint{6.993620in}{3.833024in}}%
\pgfpathlineto{\pgfqpoint{6.993620in}{6.053339in}}%
\pgfpathlineto{\pgfqpoint{6.861869in}{6.053339in}}%
\pgfpathlineto{\pgfqpoint{6.861869in}{3.833024in}}%
\pgfpathclose%
\pgfusepath{fill}%
\end{pgfscope}%
\begin{pgfscope}%
\pgfpathrectangle{\pgfqpoint{0.603704in}{3.833024in}}{\pgfqpoint{7.246296in}{2.525309in}}%
\pgfusepath{clip}%
\pgfsetbuttcap%
\pgfsetmiterjoin%
\definecolor{currentfill}{rgb}{0.121569,0.466667,0.705882}%
\pgfsetfillcolor{currentfill}%
\pgfsetfillopacity{0.700000}%
\pgfsetlinewidth{0.000000pt}%
\definecolor{currentstroke}{rgb}{0.000000,0.000000,0.000000}%
\pgfsetstrokecolor{currentstroke}%
\pgfsetstrokeopacity{0.700000}%
\pgfsetdash{}{0pt}%
\pgfpathmoveto{\pgfqpoint{6.993620in}{3.833024in}}%
\pgfpathlineto{\pgfqpoint{7.125370in}{3.833024in}}%
\pgfpathlineto{\pgfqpoint{7.125370in}{6.159394in}}%
\pgfpathlineto{\pgfqpoint{6.993620in}{6.159394in}}%
\pgfpathlineto{\pgfqpoint{6.993620in}{3.833024in}}%
\pgfpathclose%
\pgfusepath{fill}%
\end{pgfscope}%
\begin{pgfscope}%
\pgfpathrectangle{\pgfqpoint{0.603704in}{3.833024in}}{\pgfqpoint{7.246296in}{2.525309in}}%
\pgfusepath{clip}%
\pgfsetbuttcap%
\pgfsetmiterjoin%
\definecolor{currentfill}{rgb}{0.121569,0.466667,0.705882}%
\pgfsetfillcolor{currentfill}%
\pgfsetfillopacity{0.700000}%
\pgfsetlinewidth{0.000000pt}%
\definecolor{currentstroke}{rgb}{0.000000,0.000000,0.000000}%
\pgfsetstrokecolor{currentstroke}%
\pgfsetstrokeopacity{0.700000}%
\pgfsetdash{}{0pt}%
\pgfpathmoveto{\pgfqpoint{7.125370in}{3.833024in}}%
\pgfpathlineto{\pgfqpoint{7.257121in}{3.833024in}}%
\pgfpathlineto{\pgfqpoint{7.257121in}{6.152552in}}%
\pgfpathlineto{\pgfqpoint{7.125370in}{6.152552in}}%
\pgfpathlineto{\pgfqpoint{7.125370in}{3.833024in}}%
\pgfpathclose%
\pgfusepath{fill}%
\end{pgfscope}%
\begin{pgfscope}%
\pgfpathrectangle{\pgfqpoint{0.603704in}{3.833024in}}{\pgfqpoint{7.246296in}{2.525309in}}%
\pgfusepath{clip}%
\pgfsetbuttcap%
\pgfsetmiterjoin%
\definecolor{currentfill}{rgb}{0.121569,0.466667,0.705882}%
\pgfsetfillcolor{currentfill}%
\pgfsetfillopacity{0.700000}%
\pgfsetlinewidth{0.000000pt}%
\definecolor{currentstroke}{rgb}{0.000000,0.000000,0.000000}%
\pgfsetstrokecolor{currentstroke}%
\pgfsetstrokeopacity{0.700000}%
\pgfsetdash{}{0pt}%
\pgfpathmoveto{\pgfqpoint{7.257121in}{3.833024in}}%
\pgfpathlineto{\pgfqpoint{7.388872in}{3.833024in}}%
\pgfpathlineto{\pgfqpoint{7.388872in}{6.154833in}}%
\pgfpathlineto{\pgfqpoint{7.257121in}{6.154833in}}%
\pgfpathlineto{\pgfqpoint{7.257121in}{3.833024in}}%
\pgfpathclose%
\pgfusepath{fill}%
\end{pgfscope}%
\begin{pgfscope}%
\pgfpathrectangle{\pgfqpoint{0.603704in}{3.833024in}}{\pgfqpoint{7.246296in}{2.525309in}}%
\pgfusepath{clip}%
\pgfsetbuttcap%
\pgfsetmiterjoin%
\definecolor{currentfill}{rgb}{0.121569,0.466667,0.705882}%
\pgfsetfillcolor{currentfill}%
\pgfsetfillopacity{0.700000}%
\pgfsetlinewidth{0.000000pt}%
\definecolor{currentstroke}{rgb}{0.000000,0.000000,0.000000}%
\pgfsetstrokecolor{currentstroke}%
\pgfsetstrokeopacity{0.700000}%
\pgfsetdash{}{0pt}%
\pgfpathmoveto{\pgfqpoint{7.388872in}{3.833024in}}%
\pgfpathlineto{\pgfqpoint{7.520623in}{3.833024in}}%
\pgfpathlineto{\pgfqpoint{7.520623in}{6.127464in}}%
\pgfpathlineto{\pgfqpoint{7.388872in}{6.127464in}}%
\pgfpathlineto{\pgfqpoint{7.388872in}{3.833024in}}%
\pgfpathclose%
\pgfusepath{fill}%
\end{pgfscope}%
\begin{pgfscope}%
\pgfsetbuttcap%
\pgfsetroundjoin%
\definecolor{currentfill}{rgb}{0.000000,0.000000,0.000000}%
\pgfsetfillcolor{currentfill}%
\pgfsetlinewidth{0.803000pt}%
\definecolor{currentstroke}{rgb}{0.000000,0.000000,0.000000}%
\pgfsetstrokecolor{currentstroke}%
\pgfsetdash{}{0pt}%
\pgfsys@defobject{currentmarker}{\pgfqpoint{0.000000in}{-0.048611in}}{\pgfqpoint{0.000000in}{0.000000in}}{%
\pgfpathmoveto{\pgfqpoint{0.000000in}{0.000000in}}%
\pgfpathlineto{\pgfqpoint{0.000000in}{-0.048611in}}%
\pgfusepath{stroke,fill}%
}%
\begin{pgfscope}%
\pgfsys@transformshift{0.932997in}{3.833024in}%
\pgfsys@useobject{currentmarker}{}%
\end{pgfscope}%
\end{pgfscope}%
\begin{pgfscope}%
\definecolor{textcolor}{rgb}{0.000000,0.000000,0.000000}%
\pgfsetstrokecolor{textcolor}%
\pgfsetfillcolor{textcolor}%
\pgftext[x=0.932997in,y=3.735802in,,top]{\color{textcolor}{\rmfamily\fontsize{10.000000}{12.000000}\selectfont\catcode`\^=\active\def^{\ifmmode\sp\else\^{}\fi}\catcode`\%=\active\def%{\%}$\mathdefault{0.0}$}}%
\end{pgfscope}%
\begin{pgfscope}%
\pgfsetbuttcap%
\pgfsetroundjoin%
\definecolor{currentfill}{rgb}{0.000000,0.000000,0.000000}%
\pgfsetfillcolor{currentfill}%
\pgfsetlinewidth{0.803000pt}%
\definecolor{currentstroke}{rgb}{0.000000,0.000000,0.000000}%
\pgfsetstrokecolor{currentstroke}%
\pgfsetdash{}{0pt}%
\pgfsys@defobject{currentmarker}{\pgfqpoint{0.000000in}{-0.048611in}}{\pgfqpoint{0.000000in}{0.000000in}}{%
\pgfpathmoveto{\pgfqpoint{0.000000in}{0.000000in}}%
\pgfpathlineto{\pgfqpoint{0.000000in}{-0.048611in}}%
\pgfusepath{stroke,fill}%
}%
\begin{pgfscope}%
\pgfsys@transformshift{2.250539in}{3.833024in}%
\pgfsys@useobject{currentmarker}{}%
\end{pgfscope}%
\end{pgfscope}%
\begin{pgfscope}%
\definecolor{textcolor}{rgb}{0.000000,0.000000,0.000000}%
\pgfsetstrokecolor{textcolor}%
\pgfsetfillcolor{textcolor}%
\pgftext[x=2.250539in,y=3.735802in,,top]{\color{textcolor}{\rmfamily\fontsize{10.000000}{12.000000}\selectfont\catcode`\^=\active\def^{\ifmmode\sp\else\^{}\fi}\catcode`\%=\active\def%{\%}$\mathdefault{0.2}$}}%
\end{pgfscope}%
\begin{pgfscope}%
\pgfsetbuttcap%
\pgfsetroundjoin%
\definecolor{currentfill}{rgb}{0.000000,0.000000,0.000000}%
\pgfsetfillcolor{currentfill}%
\pgfsetlinewidth{0.803000pt}%
\definecolor{currentstroke}{rgb}{0.000000,0.000000,0.000000}%
\pgfsetstrokecolor{currentstroke}%
\pgfsetdash{}{0pt}%
\pgfsys@defobject{currentmarker}{\pgfqpoint{0.000000in}{-0.048611in}}{\pgfqpoint{0.000000in}{0.000000in}}{%
\pgfpathmoveto{\pgfqpoint{0.000000in}{0.000000in}}%
\pgfpathlineto{\pgfqpoint{0.000000in}{-0.048611in}}%
\pgfusepath{stroke,fill}%
}%
\begin{pgfscope}%
\pgfsys@transformshift{3.568081in}{3.833024in}%
\pgfsys@useobject{currentmarker}{}%
\end{pgfscope}%
\end{pgfscope}%
\begin{pgfscope}%
\definecolor{textcolor}{rgb}{0.000000,0.000000,0.000000}%
\pgfsetstrokecolor{textcolor}%
\pgfsetfillcolor{textcolor}%
\pgftext[x=3.568081in,y=3.735802in,,top]{\color{textcolor}{\rmfamily\fontsize{10.000000}{12.000000}\selectfont\catcode`\^=\active\def^{\ifmmode\sp\else\^{}\fi}\catcode`\%=\active\def%{\%}$\mathdefault{0.4}$}}%
\end{pgfscope}%
\begin{pgfscope}%
\pgfsetbuttcap%
\pgfsetroundjoin%
\definecolor{currentfill}{rgb}{0.000000,0.000000,0.000000}%
\pgfsetfillcolor{currentfill}%
\pgfsetlinewidth{0.803000pt}%
\definecolor{currentstroke}{rgb}{0.000000,0.000000,0.000000}%
\pgfsetstrokecolor{currentstroke}%
\pgfsetdash{}{0pt}%
\pgfsys@defobject{currentmarker}{\pgfqpoint{0.000000in}{-0.048611in}}{\pgfqpoint{0.000000in}{0.000000in}}{%
\pgfpathmoveto{\pgfqpoint{0.000000in}{0.000000in}}%
\pgfpathlineto{\pgfqpoint{0.000000in}{-0.048611in}}%
\pgfusepath{stroke,fill}%
}%
\begin{pgfscope}%
\pgfsys@transformshift{4.885622in}{3.833024in}%
\pgfsys@useobject{currentmarker}{}%
\end{pgfscope}%
\end{pgfscope}%
\begin{pgfscope}%
\definecolor{textcolor}{rgb}{0.000000,0.000000,0.000000}%
\pgfsetstrokecolor{textcolor}%
\pgfsetfillcolor{textcolor}%
\pgftext[x=4.885622in,y=3.735802in,,top]{\color{textcolor}{\rmfamily\fontsize{10.000000}{12.000000}\selectfont\catcode`\^=\active\def^{\ifmmode\sp\else\^{}\fi}\catcode`\%=\active\def%{\%}$\mathdefault{0.6}$}}%
\end{pgfscope}%
\begin{pgfscope}%
\pgfsetbuttcap%
\pgfsetroundjoin%
\definecolor{currentfill}{rgb}{0.000000,0.000000,0.000000}%
\pgfsetfillcolor{currentfill}%
\pgfsetlinewidth{0.803000pt}%
\definecolor{currentstroke}{rgb}{0.000000,0.000000,0.000000}%
\pgfsetstrokecolor{currentstroke}%
\pgfsetdash{}{0pt}%
\pgfsys@defobject{currentmarker}{\pgfqpoint{0.000000in}{-0.048611in}}{\pgfqpoint{0.000000in}{0.000000in}}{%
\pgfpathmoveto{\pgfqpoint{0.000000in}{0.000000in}}%
\pgfpathlineto{\pgfqpoint{0.000000in}{-0.048611in}}%
\pgfusepath{stroke,fill}%
}%
\begin{pgfscope}%
\pgfsys@transformshift{6.203164in}{3.833024in}%
\pgfsys@useobject{currentmarker}{}%
\end{pgfscope}%
\end{pgfscope}%
\begin{pgfscope}%
\definecolor{textcolor}{rgb}{0.000000,0.000000,0.000000}%
\pgfsetstrokecolor{textcolor}%
\pgfsetfillcolor{textcolor}%
\pgftext[x=6.203164in,y=3.735802in,,top]{\color{textcolor}{\rmfamily\fontsize{10.000000}{12.000000}\selectfont\catcode`\^=\active\def^{\ifmmode\sp\else\^{}\fi}\catcode`\%=\active\def%{\%}$\mathdefault{0.8}$}}%
\end{pgfscope}%
\begin{pgfscope}%
\pgfsetbuttcap%
\pgfsetroundjoin%
\definecolor{currentfill}{rgb}{0.000000,0.000000,0.000000}%
\pgfsetfillcolor{currentfill}%
\pgfsetlinewidth{0.803000pt}%
\definecolor{currentstroke}{rgb}{0.000000,0.000000,0.000000}%
\pgfsetstrokecolor{currentstroke}%
\pgfsetdash{}{0pt}%
\pgfsys@defobject{currentmarker}{\pgfqpoint{0.000000in}{-0.048611in}}{\pgfqpoint{0.000000in}{0.000000in}}{%
\pgfpathmoveto{\pgfqpoint{0.000000in}{0.000000in}}%
\pgfpathlineto{\pgfqpoint{0.000000in}{-0.048611in}}%
\pgfusepath{stroke,fill}%
}%
\begin{pgfscope}%
\pgfsys@transformshift{7.520706in}{3.833024in}%
\pgfsys@useobject{currentmarker}{}%
\end{pgfscope}%
\end{pgfscope}%
\begin{pgfscope}%
\definecolor{textcolor}{rgb}{0.000000,0.000000,0.000000}%
\pgfsetstrokecolor{textcolor}%
\pgfsetfillcolor{textcolor}%
\pgftext[x=7.520706in,y=3.735802in,,top]{\color{textcolor}{\rmfamily\fontsize{10.000000}{12.000000}\selectfont\catcode`\^=\active\def^{\ifmmode\sp\else\^{}\fi}\catcode`\%=\active\def%{\%}$\mathdefault{1.0}$}}%
\end{pgfscope}%
\begin{pgfscope}%
\definecolor{textcolor}{rgb}{0.000000,0.000000,0.000000}%
\pgfsetstrokecolor{textcolor}%
\pgfsetfillcolor{textcolor}%
\pgftext[x=4.226852in,y=3.556790in,,top]{\color{textcolor}{\rmfamily\fontsize{10.000000}{12.000000}\selectfont\catcode`\^=\active\def^{\ifmmode\sp\else\^{}\fi}\catcode`\%=\active\def%{\%}$y$}}%
\end{pgfscope}%
\begin{pgfscope}%
\pgfsetbuttcap%
\pgfsetroundjoin%
\definecolor{currentfill}{rgb}{0.000000,0.000000,0.000000}%
\pgfsetfillcolor{currentfill}%
\pgfsetlinewidth{0.803000pt}%
\definecolor{currentstroke}{rgb}{0.000000,0.000000,0.000000}%
\pgfsetstrokecolor{currentstroke}%
\pgfsetdash{}{0pt}%
\pgfsys@defobject{currentmarker}{\pgfqpoint{-0.048611in}{0.000000in}}{\pgfqpoint{-0.000000in}{0.000000in}}{%
\pgfpathmoveto{\pgfqpoint{-0.000000in}{0.000000in}}%
\pgfpathlineto{\pgfqpoint{-0.048611in}{0.000000in}}%
\pgfusepath{stroke,fill}%
}%
\begin{pgfscope}%
\pgfsys@transformshift{0.603704in}{3.833024in}%
\pgfsys@useobject{currentmarker}{}%
\end{pgfscope}%
\end{pgfscope}%
\begin{pgfscope}%
\definecolor{textcolor}{rgb}{0.000000,0.000000,0.000000}%
\pgfsetstrokecolor{textcolor}%
\pgfsetfillcolor{textcolor}%
\pgftext[x=0.329012in, y=3.784799in, left, base]{\color{textcolor}{\rmfamily\fontsize{10.000000}{12.000000}\selectfont\catcode`\^=\active\def^{\ifmmode\sp\else\^{}\fi}\catcode`\%=\active\def%{\%}$\mathdefault{0.0}$}}%
\end{pgfscope}%
\begin{pgfscope}%
\pgfsetbuttcap%
\pgfsetroundjoin%
\definecolor{currentfill}{rgb}{0.000000,0.000000,0.000000}%
\pgfsetfillcolor{currentfill}%
\pgfsetlinewidth{0.803000pt}%
\definecolor{currentstroke}{rgb}{0.000000,0.000000,0.000000}%
\pgfsetstrokecolor{currentstroke}%
\pgfsetdash{}{0pt}%
\pgfsys@defobject{currentmarker}{\pgfqpoint{-0.048611in}{0.000000in}}{\pgfqpoint{-0.000000in}{0.000000in}}{%
\pgfpathmoveto{\pgfqpoint{-0.000000in}{0.000000in}}%
\pgfpathlineto{\pgfqpoint{-0.048611in}{0.000000in}}%
\pgfusepath{stroke,fill}%
}%
\begin{pgfscope}%
\pgfsys@transformshift{0.603704in}{4.289164in}%
\pgfsys@useobject{currentmarker}{}%
\end{pgfscope}%
\end{pgfscope}%
\begin{pgfscope}%
\definecolor{textcolor}{rgb}{0.000000,0.000000,0.000000}%
\pgfsetstrokecolor{textcolor}%
\pgfsetfillcolor{textcolor}%
\pgftext[x=0.329012in, y=4.240939in, left, base]{\color{textcolor}{\rmfamily\fontsize{10.000000}{12.000000}\selectfont\catcode`\^=\active\def^{\ifmmode\sp\else\^{}\fi}\catcode`\%=\active\def%{\%}$\mathdefault{0.2}$}}%
\end{pgfscope}%
\begin{pgfscope}%
\pgfsetbuttcap%
\pgfsetroundjoin%
\definecolor{currentfill}{rgb}{0.000000,0.000000,0.000000}%
\pgfsetfillcolor{currentfill}%
\pgfsetlinewidth{0.803000pt}%
\definecolor{currentstroke}{rgb}{0.000000,0.000000,0.000000}%
\pgfsetstrokecolor{currentstroke}%
\pgfsetdash{}{0pt}%
\pgfsys@defobject{currentmarker}{\pgfqpoint{-0.048611in}{0.000000in}}{\pgfqpoint{-0.000000in}{0.000000in}}{%
\pgfpathmoveto{\pgfqpoint{-0.000000in}{0.000000in}}%
\pgfpathlineto{\pgfqpoint{-0.048611in}{0.000000in}}%
\pgfusepath{stroke,fill}%
}%
\begin{pgfscope}%
\pgfsys@transformshift{0.603704in}{4.745303in}%
\pgfsys@useobject{currentmarker}{}%
\end{pgfscope}%
\end{pgfscope}%
\begin{pgfscope}%
\definecolor{textcolor}{rgb}{0.000000,0.000000,0.000000}%
\pgfsetstrokecolor{textcolor}%
\pgfsetfillcolor{textcolor}%
\pgftext[x=0.329012in, y=4.697078in, left, base]{\color{textcolor}{\rmfamily\fontsize{10.000000}{12.000000}\selectfont\catcode`\^=\active\def^{\ifmmode\sp\else\^{}\fi}\catcode`\%=\active\def%{\%}$\mathdefault{0.4}$}}%
\end{pgfscope}%
\begin{pgfscope}%
\pgfsetbuttcap%
\pgfsetroundjoin%
\definecolor{currentfill}{rgb}{0.000000,0.000000,0.000000}%
\pgfsetfillcolor{currentfill}%
\pgfsetlinewidth{0.803000pt}%
\definecolor{currentstroke}{rgb}{0.000000,0.000000,0.000000}%
\pgfsetstrokecolor{currentstroke}%
\pgfsetdash{}{0pt}%
\pgfsys@defobject{currentmarker}{\pgfqpoint{-0.048611in}{0.000000in}}{\pgfqpoint{-0.000000in}{0.000000in}}{%
\pgfpathmoveto{\pgfqpoint{-0.000000in}{0.000000in}}%
\pgfpathlineto{\pgfqpoint{-0.048611in}{0.000000in}}%
\pgfusepath{stroke,fill}%
}%
\begin{pgfscope}%
\pgfsys@transformshift{0.603704in}{5.201443in}%
\pgfsys@useobject{currentmarker}{}%
\end{pgfscope}%
\end{pgfscope}%
\begin{pgfscope}%
\definecolor{textcolor}{rgb}{0.000000,0.000000,0.000000}%
\pgfsetstrokecolor{textcolor}%
\pgfsetfillcolor{textcolor}%
\pgftext[x=0.329012in, y=5.153217in, left, base]{\color{textcolor}{\rmfamily\fontsize{10.000000}{12.000000}\selectfont\catcode`\^=\active\def^{\ifmmode\sp\else\^{}\fi}\catcode`\%=\active\def%{\%}$\mathdefault{0.6}$}}%
\end{pgfscope}%
\begin{pgfscope}%
\pgfsetbuttcap%
\pgfsetroundjoin%
\definecolor{currentfill}{rgb}{0.000000,0.000000,0.000000}%
\pgfsetfillcolor{currentfill}%
\pgfsetlinewidth{0.803000pt}%
\definecolor{currentstroke}{rgb}{0.000000,0.000000,0.000000}%
\pgfsetstrokecolor{currentstroke}%
\pgfsetdash{}{0pt}%
\pgfsys@defobject{currentmarker}{\pgfqpoint{-0.048611in}{0.000000in}}{\pgfqpoint{-0.000000in}{0.000000in}}{%
\pgfpathmoveto{\pgfqpoint{-0.000000in}{0.000000in}}%
\pgfpathlineto{\pgfqpoint{-0.048611in}{0.000000in}}%
\pgfusepath{stroke,fill}%
}%
\begin{pgfscope}%
\pgfsys@transformshift{0.603704in}{5.657582in}%
\pgfsys@useobject{currentmarker}{}%
\end{pgfscope}%
\end{pgfscope}%
\begin{pgfscope}%
\definecolor{textcolor}{rgb}{0.000000,0.000000,0.000000}%
\pgfsetstrokecolor{textcolor}%
\pgfsetfillcolor{textcolor}%
\pgftext[x=0.329012in, y=5.609357in, left, base]{\color{textcolor}{\rmfamily\fontsize{10.000000}{12.000000}\selectfont\catcode`\^=\active\def^{\ifmmode\sp\else\^{}\fi}\catcode`\%=\active\def%{\%}$\mathdefault{0.8}$}}%
\end{pgfscope}%
\begin{pgfscope}%
\pgfsetbuttcap%
\pgfsetroundjoin%
\definecolor{currentfill}{rgb}{0.000000,0.000000,0.000000}%
\pgfsetfillcolor{currentfill}%
\pgfsetlinewidth{0.803000pt}%
\definecolor{currentstroke}{rgb}{0.000000,0.000000,0.000000}%
\pgfsetstrokecolor{currentstroke}%
\pgfsetdash{}{0pt}%
\pgfsys@defobject{currentmarker}{\pgfqpoint{-0.048611in}{0.000000in}}{\pgfqpoint{-0.000000in}{0.000000in}}{%
\pgfpathmoveto{\pgfqpoint{-0.000000in}{0.000000in}}%
\pgfpathlineto{\pgfqpoint{-0.048611in}{0.000000in}}%
\pgfusepath{stroke,fill}%
}%
\begin{pgfscope}%
\pgfsys@transformshift{0.603704in}{6.113722in}%
\pgfsys@useobject{currentmarker}{}%
\end{pgfscope}%
\end{pgfscope}%
\begin{pgfscope}%
\definecolor{textcolor}{rgb}{0.000000,0.000000,0.000000}%
\pgfsetstrokecolor{textcolor}%
\pgfsetfillcolor{textcolor}%
\pgftext[x=0.329012in, y=6.065496in, left, base]{\color{textcolor}{\rmfamily\fontsize{10.000000}{12.000000}\selectfont\catcode`\^=\active\def^{\ifmmode\sp\else\^{}\fi}\catcode`\%=\active\def%{\%}$\mathdefault{1.0}$}}%
\end{pgfscope}%
\begin{pgfscope}%
\definecolor{textcolor}{rgb}{0.000000,0.000000,0.000000}%
\pgfsetstrokecolor{textcolor}%
\pgfsetfillcolor{textcolor}%
\pgftext[x=0.273457in,y=5.095679in,,bottom,rotate=90.000000]{\color{textcolor}{\rmfamily\fontsize{10.000000}{12.000000}\selectfont\catcode`\^=\active\def^{\ifmmode\sp\else\^{}\fi}\catcode`\%=\active\def%{\%}density}}%
\end{pgfscope}%
\begin{pgfscope}%
\pgfsetrectcap%
\pgfsetmiterjoin%
\pgfsetlinewidth{0.803000pt}%
\definecolor{currentstroke}{rgb}{0.000000,0.000000,0.000000}%
\pgfsetstrokecolor{currentstroke}%
\pgfsetdash{}{0pt}%
\pgfpathmoveto{\pgfqpoint{0.603704in}{3.833024in}}%
\pgfpathlineto{\pgfqpoint{0.603704in}{6.358333in}}%
\pgfusepath{stroke}%
\end{pgfscope}%
\begin{pgfscope}%
\pgfsetrectcap%
\pgfsetmiterjoin%
\pgfsetlinewidth{0.803000pt}%
\definecolor{currentstroke}{rgb}{0.000000,0.000000,0.000000}%
\pgfsetstrokecolor{currentstroke}%
\pgfsetdash{}{0pt}%
\pgfpathmoveto{\pgfqpoint{7.850000in}{3.833024in}}%
\pgfpathlineto{\pgfqpoint{7.850000in}{6.358333in}}%
\pgfusepath{stroke}%
\end{pgfscope}%
\begin{pgfscope}%
\pgfsetrectcap%
\pgfsetmiterjoin%
\pgfsetlinewidth{0.803000pt}%
\definecolor{currentstroke}{rgb}{0.000000,0.000000,0.000000}%
\pgfsetstrokecolor{currentstroke}%
\pgfsetdash{}{0pt}%
\pgfpathmoveto{\pgfqpoint{0.603704in}{3.833024in}}%
\pgfpathlineto{\pgfqpoint{7.850000in}{3.833024in}}%
\pgfusepath{stroke}%
\end{pgfscope}%
\begin{pgfscope}%
\pgfsetrectcap%
\pgfsetmiterjoin%
\pgfsetlinewidth{0.803000pt}%
\definecolor{currentstroke}{rgb}{0.000000,0.000000,0.000000}%
\pgfsetstrokecolor{currentstroke}%
\pgfsetdash{}{0pt}%
\pgfpathmoveto{\pgfqpoint{0.603704in}{6.358333in}}%
\pgfpathlineto{\pgfqpoint{7.850000in}{6.358333in}}%
\pgfusepath{stroke}%
\end{pgfscope}%
\begin{pgfscope}%
\definecolor{textcolor}{rgb}{0.000000,0.000000,0.000000}%
\pgfsetstrokecolor{textcolor}%
\pgfsetfillcolor{textcolor}%
\pgftext[x=4.226852in,y=6.441667in,,base]{\color{textcolor}{\rmfamily\fontsize{12.000000}{14.400000}\selectfont\catcode`\^=\active\def^{\ifmmode\sp\else\^{}\fi}\catcode`\%=\active\def%{\%}Histogram of $Y \sim \mathcal{U}(0,1)$}}%
\end{pgfscope}%
\begin{pgfscope}%
\pgfsetbuttcap%
\pgfsetmiterjoin%
\definecolor{currentfill}{rgb}{1.000000,1.000000,1.000000}%
\pgfsetfillcolor{currentfill}%
\pgfsetlinewidth{0.000000pt}%
\definecolor{currentstroke}{rgb}{0.000000,0.000000,0.000000}%
\pgfsetstrokecolor{currentstroke}%
\pgfsetstrokeopacity{0.000000}%
\pgfsetdash{}{0pt}%
\pgfpathmoveto{\pgfqpoint{0.603704in}{0.549691in}}%
\pgfpathlineto{\pgfqpoint{7.850000in}{0.549691in}}%
\pgfpathlineto{\pgfqpoint{7.850000in}{3.075000in}}%
\pgfpathlineto{\pgfqpoint{0.603704in}{3.075000in}}%
\pgfpathlineto{\pgfqpoint{0.603704in}{0.549691in}}%
\pgfpathclose%
\pgfusepath{fill}%
\end{pgfscope}%
\begin{pgfscope}%
\pgfpathrectangle{\pgfqpoint{0.603704in}{0.549691in}}{\pgfqpoint{7.246296in}{2.525309in}}%
\pgfusepath{clip}%
\pgfsetbuttcap%
\pgfsetmiterjoin%
\definecolor{currentfill}{rgb}{0.121569,0.466667,0.705882}%
\pgfsetfillcolor{currentfill}%
\pgfsetfillopacity{0.700000}%
\pgfsetlinewidth{0.000000pt}%
\definecolor{currentstroke}{rgb}{0.000000,0.000000,0.000000}%
\pgfsetstrokecolor{currentstroke}%
\pgfsetstrokeopacity{0.700000}%
\pgfsetdash{}{0pt}%
\pgfpathmoveto{\pgfqpoint{0.938421in}{0.549691in}}%
\pgfpathlineto{\pgfqpoint{1.069915in}{0.549691in}}%
\pgfpathlineto{\pgfqpoint{1.069915in}{0.599190in}}%
\pgfpathlineto{\pgfqpoint{0.938421in}{0.599190in}}%
\pgfpathlineto{\pgfqpoint{0.938421in}{0.549691in}}%
\pgfpathclose%
\pgfusepath{fill}%
\end{pgfscope}%
\begin{pgfscope}%
\pgfpathrectangle{\pgfqpoint{0.603704in}{0.549691in}}{\pgfqpoint{7.246296in}{2.525309in}}%
\pgfusepath{clip}%
\pgfsetbuttcap%
\pgfsetmiterjoin%
\definecolor{currentfill}{rgb}{0.121569,0.466667,0.705882}%
\pgfsetfillcolor{currentfill}%
\pgfsetfillopacity{0.700000}%
\pgfsetlinewidth{0.000000pt}%
\definecolor{currentstroke}{rgb}{0.000000,0.000000,0.000000}%
\pgfsetstrokecolor{currentstroke}%
\pgfsetstrokeopacity{0.700000}%
\pgfsetdash{}{0pt}%
\pgfpathmoveto{\pgfqpoint{1.069915in}{0.549691in}}%
\pgfpathlineto{\pgfqpoint{1.201409in}{0.549691in}}%
\pgfpathlineto{\pgfqpoint{1.201409in}{0.685512in}}%
\pgfpathlineto{\pgfqpoint{1.069915in}{0.685512in}}%
\pgfpathlineto{\pgfqpoint{1.069915in}{0.549691in}}%
\pgfpathclose%
\pgfusepath{fill}%
\end{pgfscope}%
\begin{pgfscope}%
\pgfpathrectangle{\pgfqpoint{0.603704in}{0.549691in}}{\pgfqpoint{7.246296in}{2.525309in}}%
\pgfusepath{clip}%
\pgfsetbuttcap%
\pgfsetmiterjoin%
\definecolor{currentfill}{rgb}{0.121569,0.466667,0.705882}%
\pgfsetfillcolor{currentfill}%
\pgfsetfillopacity{0.700000}%
\pgfsetlinewidth{0.000000pt}%
\definecolor{currentstroke}{rgb}{0.000000,0.000000,0.000000}%
\pgfsetstrokecolor{currentstroke}%
\pgfsetstrokeopacity{0.700000}%
\pgfsetdash{}{0pt}%
\pgfpathmoveto{\pgfqpoint{1.201409in}{0.549691in}}%
\pgfpathlineto{\pgfqpoint{1.332903in}{0.549691in}}%
\pgfpathlineto{\pgfqpoint{1.332903in}{0.786925in}}%
\pgfpathlineto{\pgfqpoint{1.201409in}{0.786925in}}%
\pgfpathlineto{\pgfqpoint{1.201409in}{0.549691in}}%
\pgfpathclose%
\pgfusepath{fill}%
\end{pgfscope}%
\begin{pgfscope}%
\pgfpathrectangle{\pgfqpoint{0.603704in}{0.549691in}}{\pgfqpoint{7.246296in}{2.525309in}}%
\pgfusepath{clip}%
\pgfsetbuttcap%
\pgfsetmiterjoin%
\definecolor{currentfill}{rgb}{0.121569,0.466667,0.705882}%
\pgfsetfillcolor{currentfill}%
\pgfsetfillopacity{0.700000}%
\pgfsetlinewidth{0.000000pt}%
\definecolor{currentstroke}{rgb}{0.000000,0.000000,0.000000}%
\pgfsetstrokecolor{currentstroke}%
\pgfsetstrokeopacity{0.700000}%
\pgfsetdash{}{0pt}%
\pgfpathmoveto{\pgfqpoint{1.332903in}{0.549691in}}%
\pgfpathlineto{\pgfqpoint{1.464397in}{0.549691in}}%
\pgfpathlineto{\pgfqpoint{1.464397in}{0.882301in}}%
\pgfpathlineto{\pgfqpoint{1.332903in}{0.882301in}}%
\pgfpathlineto{\pgfqpoint{1.332903in}{0.549691in}}%
\pgfpathclose%
\pgfusepath{fill}%
\end{pgfscope}%
\begin{pgfscope}%
\pgfpathrectangle{\pgfqpoint{0.603704in}{0.549691in}}{\pgfqpoint{7.246296in}{2.525309in}}%
\pgfusepath{clip}%
\pgfsetbuttcap%
\pgfsetmiterjoin%
\definecolor{currentfill}{rgb}{0.121569,0.466667,0.705882}%
\pgfsetfillcolor{currentfill}%
\pgfsetfillopacity{0.700000}%
\pgfsetlinewidth{0.000000pt}%
\definecolor{currentstroke}{rgb}{0.000000,0.000000,0.000000}%
\pgfsetstrokecolor{currentstroke}%
\pgfsetstrokeopacity{0.700000}%
\pgfsetdash{}{0pt}%
\pgfpathmoveto{\pgfqpoint{1.464397in}{0.549691in}}%
\pgfpathlineto{\pgfqpoint{1.595891in}{0.549691in}}%
\pgfpathlineto{\pgfqpoint{1.595891in}{0.997598in}}%
\pgfpathlineto{\pgfqpoint{1.464397in}{0.997598in}}%
\pgfpathlineto{\pgfqpoint{1.464397in}{0.549691in}}%
\pgfpathclose%
\pgfusepath{fill}%
\end{pgfscope}%
\begin{pgfscope}%
\pgfpathrectangle{\pgfqpoint{0.603704in}{0.549691in}}{\pgfqpoint{7.246296in}{2.525309in}}%
\pgfusepath{clip}%
\pgfsetbuttcap%
\pgfsetmiterjoin%
\definecolor{currentfill}{rgb}{0.121569,0.466667,0.705882}%
\pgfsetfillcolor{currentfill}%
\pgfsetfillopacity{0.700000}%
\pgfsetlinewidth{0.000000pt}%
\definecolor{currentstroke}{rgb}{0.000000,0.000000,0.000000}%
\pgfsetstrokecolor{currentstroke}%
\pgfsetstrokeopacity{0.700000}%
\pgfsetdash{}{0pt}%
\pgfpathmoveto{\pgfqpoint{1.595891in}{0.549691in}}%
\pgfpathlineto{\pgfqpoint{1.727385in}{0.549691in}}%
\pgfpathlineto{\pgfqpoint{1.727385in}{1.060378in}}%
\pgfpathlineto{\pgfqpoint{1.595891in}{1.060378in}}%
\pgfpathlineto{\pgfqpoint{1.595891in}{0.549691in}}%
\pgfpathclose%
\pgfusepath{fill}%
\end{pgfscope}%
\begin{pgfscope}%
\pgfpathrectangle{\pgfqpoint{0.603704in}{0.549691in}}{\pgfqpoint{7.246296in}{2.525309in}}%
\pgfusepath{clip}%
\pgfsetbuttcap%
\pgfsetmiterjoin%
\definecolor{currentfill}{rgb}{0.121569,0.466667,0.705882}%
\pgfsetfillcolor{currentfill}%
\pgfsetfillopacity{0.700000}%
\pgfsetlinewidth{0.000000pt}%
\definecolor{currentstroke}{rgb}{0.000000,0.000000,0.000000}%
\pgfsetstrokecolor{currentstroke}%
\pgfsetstrokeopacity{0.700000}%
\pgfsetdash{}{0pt}%
\pgfpathmoveto{\pgfqpoint{1.727385in}{0.549691in}}%
\pgfpathlineto{\pgfqpoint{1.858879in}{0.549691in}}%
\pgfpathlineto{\pgfqpoint{1.858879in}{1.212497in}}%
\pgfpathlineto{\pgfqpoint{1.727385in}{1.212497in}}%
\pgfpathlineto{\pgfqpoint{1.727385in}{0.549691in}}%
\pgfpathclose%
\pgfusepath{fill}%
\end{pgfscope}%
\begin{pgfscope}%
\pgfpathrectangle{\pgfqpoint{0.603704in}{0.549691in}}{\pgfqpoint{7.246296in}{2.525309in}}%
\pgfusepath{clip}%
\pgfsetbuttcap%
\pgfsetmiterjoin%
\definecolor{currentfill}{rgb}{0.121569,0.466667,0.705882}%
\pgfsetfillcolor{currentfill}%
\pgfsetfillopacity{0.700000}%
\pgfsetlinewidth{0.000000pt}%
\definecolor{currentstroke}{rgb}{0.000000,0.000000,0.000000}%
\pgfsetstrokecolor{currentstroke}%
\pgfsetstrokeopacity{0.700000}%
\pgfsetdash{}{0pt}%
\pgfpathmoveto{\pgfqpoint{1.858879in}{0.549691in}}%
\pgfpathlineto{\pgfqpoint{1.990373in}{0.549691in}}%
\pgfpathlineto{\pgfqpoint{1.990373in}{1.292178in}}%
\pgfpathlineto{\pgfqpoint{1.858879in}{1.292178in}}%
\pgfpathlineto{\pgfqpoint{1.858879in}{0.549691in}}%
\pgfpathclose%
\pgfusepath{fill}%
\end{pgfscope}%
\begin{pgfscope}%
\pgfpathrectangle{\pgfqpoint{0.603704in}{0.549691in}}{\pgfqpoint{7.246296in}{2.525309in}}%
\pgfusepath{clip}%
\pgfsetbuttcap%
\pgfsetmiterjoin%
\definecolor{currentfill}{rgb}{0.121569,0.466667,0.705882}%
\pgfsetfillcolor{currentfill}%
\pgfsetfillopacity{0.700000}%
\pgfsetlinewidth{0.000000pt}%
\definecolor{currentstroke}{rgb}{0.000000,0.000000,0.000000}%
\pgfsetstrokecolor{currentstroke}%
\pgfsetstrokeopacity{0.700000}%
\pgfsetdash{}{0pt}%
\pgfpathmoveto{\pgfqpoint{1.990373in}{0.549691in}}%
\pgfpathlineto{\pgfqpoint{2.121867in}{0.549691in}}%
\pgfpathlineto{\pgfqpoint{2.121867in}{1.371256in}}%
\pgfpathlineto{\pgfqpoint{1.990373in}{1.371256in}}%
\pgfpathlineto{\pgfqpoint{1.990373in}{0.549691in}}%
\pgfpathclose%
\pgfusepath{fill}%
\end{pgfscope}%
\begin{pgfscope}%
\pgfpathrectangle{\pgfqpoint{0.603704in}{0.549691in}}{\pgfqpoint{7.246296in}{2.525309in}}%
\pgfusepath{clip}%
\pgfsetbuttcap%
\pgfsetmiterjoin%
\definecolor{currentfill}{rgb}{0.121569,0.466667,0.705882}%
\pgfsetfillcolor{currentfill}%
\pgfsetfillopacity{0.700000}%
\pgfsetlinewidth{0.000000pt}%
\definecolor{currentstroke}{rgb}{0.000000,0.000000,0.000000}%
\pgfsetstrokecolor{currentstroke}%
\pgfsetstrokeopacity{0.700000}%
\pgfsetdash{}{0pt}%
\pgfpathmoveto{\pgfqpoint{2.121867in}{0.549691in}}%
\pgfpathlineto{\pgfqpoint{2.253361in}{0.549691in}}%
\pgfpathlineto{\pgfqpoint{2.253361in}{1.442487in}}%
\pgfpathlineto{\pgfqpoint{2.121867in}{1.442487in}}%
\pgfpathlineto{\pgfqpoint{2.121867in}{0.549691in}}%
\pgfpathclose%
\pgfusepath{fill}%
\end{pgfscope}%
\begin{pgfscope}%
\pgfpathrectangle{\pgfqpoint{0.603704in}{0.549691in}}{\pgfqpoint{7.246296in}{2.525309in}}%
\pgfusepath{clip}%
\pgfsetbuttcap%
\pgfsetmiterjoin%
\definecolor{currentfill}{rgb}{0.121569,0.466667,0.705882}%
\pgfsetfillcolor{currentfill}%
\pgfsetfillopacity{0.700000}%
\pgfsetlinewidth{0.000000pt}%
\definecolor{currentstroke}{rgb}{0.000000,0.000000,0.000000}%
\pgfsetstrokecolor{currentstroke}%
\pgfsetstrokeopacity{0.700000}%
\pgfsetdash{}{0pt}%
\pgfpathmoveto{\pgfqpoint{2.253361in}{0.549691in}}%
\pgfpathlineto{\pgfqpoint{2.384855in}{0.549691in}}%
\pgfpathlineto{\pgfqpoint{2.384855in}{1.588570in}}%
\pgfpathlineto{\pgfqpoint{2.253361in}{1.588570in}}%
\pgfpathlineto{\pgfqpoint{2.253361in}{0.549691in}}%
\pgfpathclose%
\pgfusepath{fill}%
\end{pgfscope}%
\begin{pgfscope}%
\pgfpathrectangle{\pgfqpoint{0.603704in}{0.549691in}}{\pgfqpoint{7.246296in}{2.525309in}}%
\pgfusepath{clip}%
\pgfsetbuttcap%
\pgfsetmiterjoin%
\definecolor{currentfill}{rgb}{0.121569,0.466667,0.705882}%
\pgfsetfillcolor{currentfill}%
\pgfsetfillopacity{0.700000}%
\pgfsetlinewidth{0.000000pt}%
\definecolor{currentstroke}{rgb}{0.000000,0.000000,0.000000}%
\pgfsetstrokecolor{currentstroke}%
\pgfsetstrokeopacity{0.700000}%
\pgfsetdash{}{0pt}%
\pgfpathmoveto{\pgfqpoint{2.384855in}{0.549691in}}%
\pgfpathlineto{\pgfqpoint{2.516349in}{0.549691in}}%
\pgfpathlineto{\pgfqpoint{2.516349in}{1.655575in}}%
\pgfpathlineto{\pgfqpoint{2.384855in}{1.655575in}}%
\pgfpathlineto{\pgfqpoint{2.384855in}{0.549691in}}%
\pgfpathclose%
\pgfusepath{fill}%
\end{pgfscope}%
\begin{pgfscope}%
\pgfpathrectangle{\pgfqpoint{0.603704in}{0.549691in}}{\pgfqpoint{7.246296in}{2.525309in}}%
\pgfusepath{clip}%
\pgfsetbuttcap%
\pgfsetmiterjoin%
\definecolor{currentfill}{rgb}{0.121569,0.466667,0.705882}%
\pgfsetfillcolor{currentfill}%
\pgfsetfillopacity{0.700000}%
\pgfsetlinewidth{0.000000pt}%
\definecolor{currentstroke}{rgb}{0.000000,0.000000,0.000000}%
\pgfsetstrokecolor{currentstroke}%
\pgfsetstrokeopacity{0.700000}%
\pgfsetdash{}{0pt}%
\pgfpathmoveto{\pgfqpoint{2.516349in}{0.549691in}}%
\pgfpathlineto{\pgfqpoint{2.647843in}{0.549691in}}%
\pgfpathlineto{\pgfqpoint{2.647843in}{1.744915in}}%
\pgfpathlineto{\pgfqpoint{2.516349in}{1.744915in}}%
\pgfpathlineto{\pgfqpoint{2.516349in}{0.549691in}}%
\pgfpathclose%
\pgfusepath{fill}%
\end{pgfscope}%
\begin{pgfscope}%
\pgfpathrectangle{\pgfqpoint{0.603704in}{0.549691in}}{\pgfqpoint{7.246296in}{2.525309in}}%
\pgfusepath{clip}%
\pgfsetbuttcap%
\pgfsetmiterjoin%
\definecolor{currentfill}{rgb}{0.121569,0.466667,0.705882}%
\pgfsetfillcolor{currentfill}%
\pgfsetfillopacity{0.700000}%
\pgfsetlinewidth{0.000000pt}%
\definecolor{currentstroke}{rgb}{0.000000,0.000000,0.000000}%
\pgfsetstrokecolor{currentstroke}%
\pgfsetstrokeopacity{0.700000}%
\pgfsetdash{}{0pt}%
\pgfpathmoveto{\pgfqpoint{2.647843in}{0.549691in}}%
\pgfpathlineto{\pgfqpoint{2.779337in}{0.549691in}}%
\pgfpathlineto{\pgfqpoint{2.779337in}{1.872888in}}%
\pgfpathlineto{\pgfqpoint{2.647843in}{1.872888in}}%
\pgfpathlineto{\pgfqpoint{2.647843in}{0.549691in}}%
\pgfpathclose%
\pgfusepath{fill}%
\end{pgfscope}%
\begin{pgfscope}%
\pgfpathrectangle{\pgfqpoint{0.603704in}{0.549691in}}{\pgfqpoint{7.246296in}{2.525309in}}%
\pgfusepath{clip}%
\pgfsetbuttcap%
\pgfsetmiterjoin%
\definecolor{currentfill}{rgb}{0.121569,0.466667,0.705882}%
\pgfsetfillcolor{currentfill}%
\pgfsetfillopacity{0.700000}%
\pgfsetlinewidth{0.000000pt}%
\definecolor{currentstroke}{rgb}{0.000000,0.000000,0.000000}%
\pgfsetstrokecolor{currentstroke}%
\pgfsetstrokeopacity{0.700000}%
\pgfsetdash{}{0pt}%
\pgfpathmoveto{\pgfqpoint{2.779337in}{0.549691in}}%
\pgfpathlineto{\pgfqpoint{2.910831in}{0.549691in}}%
\pgfpathlineto{\pgfqpoint{2.910831in}{1.929027in}}%
\pgfpathlineto{\pgfqpoint{2.779337in}{1.929027in}}%
\pgfpathlineto{\pgfqpoint{2.779337in}{0.549691in}}%
\pgfpathclose%
\pgfusepath{fill}%
\end{pgfscope}%
\begin{pgfscope}%
\pgfpathrectangle{\pgfqpoint{0.603704in}{0.549691in}}{\pgfqpoint{7.246296in}{2.525309in}}%
\pgfusepath{clip}%
\pgfsetbuttcap%
\pgfsetmiterjoin%
\definecolor{currentfill}{rgb}{0.121569,0.466667,0.705882}%
\pgfsetfillcolor{currentfill}%
\pgfsetfillopacity{0.700000}%
\pgfsetlinewidth{0.000000pt}%
\definecolor{currentstroke}{rgb}{0.000000,0.000000,0.000000}%
\pgfsetstrokecolor{currentstroke}%
\pgfsetstrokeopacity{0.700000}%
\pgfsetdash{}{0pt}%
\pgfpathmoveto{\pgfqpoint{2.910831in}{0.549691in}}%
\pgfpathlineto{\pgfqpoint{3.042325in}{0.549691in}}%
\pgfpathlineto{\pgfqpoint{3.042325in}{2.060623in}}%
\pgfpathlineto{\pgfqpoint{2.910831in}{2.060623in}}%
\pgfpathlineto{\pgfqpoint{2.910831in}{0.549691in}}%
\pgfpathclose%
\pgfusepath{fill}%
\end{pgfscope}%
\begin{pgfscope}%
\pgfpathrectangle{\pgfqpoint{0.603704in}{0.549691in}}{\pgfqpoint{7.246296in}{2.525309in}}%
\pgfusepath{clip}%
\pgfsetbuttcap%
\pgfsetmiterjoin%
\definecolor{currentfill}{rgb}{0.121569,0.466667,0.705882}%
\pgfsetfillcolor{currentfill}%
\pgfsetfillopacity{0.700000}%
\pgfsetlinewidth{0.000000pt}%
\definecolor{currentstroke}{rgb}{0.000000,0.000000,0.000000}%
\pgfsetstrokecolor{currentstroke}%
\pgfsetstrokeopacity{0.700000}%
\pgfsetdash{}{0pt}%
\pgfpathmoveto{\pgfqpoint{3.042325in}{0.549691in}}%
\pgfpathlineto{\pgfqpoint{3.173819in}{0.549691in}}%
\pgfpathlineto{\pgfqpoint{3.173819in}{2.124609in}}%
\pgfpathlineto{\pgfqpoint{3.042325in}{2.124609in}}%
\pgfpathlineto{\pgfqpoint{3.042325in}{0.549691in}}%
\pgfpathclose%
\pgfusepath{fill}%
\end{pgfscope}%
\begin{pgfscope}%
\pgfpathrectangle{\pgfqpoint{0.603704in}{0.549691in}}{\pgfqpoint{7.246296in}{2.525309in}}%
\pgfusepath{clip}%
\pgfsetbuttcap%
\pgfsetmiterjoin%
\definecolor{currentfill}{rgb}{0.121569,0.466667,0.705882}%
\pgfsetfillcolor{currentfill}%
\pgfsetfillopacity{0.700000}%
\pgfsetlinewidth{0.000000pt}%
\definecolor{currentstroke}{rgb}{0.000000,0.000000,0.000000}%
\pgfsetstrokecolor{currentstroke}%
\pgfsetstrokeopacity{0.700000}%
\pgfsetdash{}{0pt}%
\pgfpathmoveto{\pgfqpoint{3.173819in}{0.549691in}}%
\pgfpathlineto{\pgfqpoint{3.305313in}{0.549691in}}%
\pgfpathlineto{\pgfqpoint{3.305313in}{2.291820in}}%
\pgfpathlineto{\pgfqpoint{3.173819in}{2.291820in}}%
\pgfpathlineto{\pgfqpoint{3.173819in}{0.549691in}}%
\pgfpathclose%
\pgfusepath{fill}%
\end{pgfscope}%
\begin{pgfscope}%
\pgfpathrectangle{\pgfqpoint{0.603704in}{0.549691in}}{\pgfqpoint{7.246296in}{2.525309in}}%
\pgfusepath{clip}%
\pgfsetbuttcap%
\pgfsetmiterjoin%
\definecolor{currentfill}{rgb}{0.121569,0.466667,0.705882}%
\pgfsetfillcolor{currentfill}%
\pgfsetfillopacity{0.700000}%
\pgfsetlinewidth{0.000000pt}%
\definecolor{currentstroke}{rgb}{0.000000,0.000000,0.000000}%
\pgfsetstrokecolor{currentstroke}%
\pgfsetstrokeopacity{0.700000}%
\pgfsetdash{}{0pt}%
\pgfpathmoveto{\pgfqpoint{3.305313in}{0.549691in}}%
\pgfpathlineto{\pgfqpoint{3.436807in}{0.549691in}}%
\pgfpathlineto{\pgfqpoint{3.436807in}{2.303289in}}%
\pgfpathlineto{\pgfqpoint{3.305313in}{2.303289in}}%
\pgfpathlineto{\pgfqpoint{3.305313in}{0.549691in}}%
\pgfpathclose%
\pgfusepath{fill}%
\end{pgfscope}%
\begin{pgfscope}%
\pgfpathrectangle{\pgfqpoint{0.603704in}{0.549691in}}{\pgfqpoint{7.246296in}{2.525309in}}%
\pgfusepath{clip}%
\pgfsetbuttcap%
\pgfsetmiterjoin%
\definecolor{currentfill}{rgb}{0.121569,0.466667,0.705882}%
\pgfsetfillcolor{currentfill}%
\pgfsetfillopacity{0.700000}%
\pgfsetlinewidth{0.000000pt}%
\definecolor{currentstroke}{rgb}{0.000000,0.000000,0.000000}%
\pgfsetstrokecolor{currentstroke}%
\pgfsetstrokeopacity{0.700000}%
\pgfsetdash{}{0pt}%
\pgfpathmoveto{\pgfqpoint{3.436807in}{0.549691in}}%
\pgfpathlineto{\pgfqpoint{3.568301in}{0.549691in}}%
\pgfpathlineto{\pgfqpoint{3.568301in}{2.431263in}}%
\pgfpathlineto{\pgfqpoint{3.436807in}{2.431263in}}%
\pgfpathlineto{\pgfqpoint{3.436807in}{0.549691in}}%
\pgfpathclose%
\pgfusepath{fill}%
\end{pgfscope}%
\begin{pgfscope}%
\pgfpathrectangle{\pgfqpoint{0.603704in}{0.549691in}}{\pgfqpoint{7.246296in}{2.525309in}}%
\pgfusepath{clip}%
\pgfsetbuttcap%
\pgfsetmiterjoin%
\definecolor{currentfill}{rgb}{0.121569,0.466667,0.705882}%
\pgfsetfillcolor{currentfill}%
\pgfsetfillopacity{0.700000}%
\pgfsetlinewidth{0.000000pt}%
\definecolor{currentstroke}{rgb}{0.000000,0.000000,0.000000}%
\pgfsetstrokecolor{currentstroke}%
\pgfsetstrokeopacity{0.700000}%
\pgfsetdash{}{0pt}%
\pgfpathmoveto{\pgfqpoint{3.568301in}{0.549691in}}%
\pgfpathlineto{\pgfqpoint{3.699795in}{0.549691in}}%
\pgfpathlineto{\pgfqpoint{3.699795in}{2.541127in}}%
\pgfpathlineto{\pgfqpoint{3.568301in}{2.541127in}}%
\pgfpathlineto{\pgfqpoint{3.568301in}{0.549691in}}%
\pgfpathclose%
\pgfusepath{fill}%
\end{pgfscope}%
\begin{pgfscope}%
\pgfpathrectangle{\pgfqpoint{0.603704in}{0.549691in}}{\pgfqpoint{7.246296in}{2.525309in}}%
\pgfusepath{clip}%
\pgfsetbuttcap%
\pgfsetmiterjoin%
\definecolor{currentfill}{rgb}{0.121569,0.466667,0.705882}%
\pgfsetfillcolor{currentfill}%
\pgfsetfillopacity{0.700000}%
\pgfsetlinewidth{0.000000pt}%
\definecolor{currentstroke}{rgb}{0.000000,0.000000,0.000000}%
\pgfsetstrokecolor{currentstroke}%
\pgfsetstrokeopacity{0.700000}%
\pgfsetdash{}{0pt}%
\pgfpathmoveto{\pgfqpoint{3.699795in}{0.549691in}}%
\pgfpathlineto{\pgfqpoint{3.831289in}{0.549691in}}%
\pgfpathlineto{\pgfqpoint{3.831289in}{2.622016in}}%
\pgfpathlineto{\pgfqpoint{3.699795in}{2.622016in}}%
\pgfpathlineto{\pgfqpoint{3.699795in}{0.549691in}}%
\pgfpathclose%
\pgfusepath{fill}%
\end{pgfscope}%
\begin{pgfscope}%
\pgfpathrectangle{\pgfqpoint{0.603704in}{0.549691in}}{\pgfqpoint{7.246296in}{2.525309in}}%
\pgfusepath{clip}%
\pgfsetbuttcap%
\pgfsetmiterjoin%
\definecolor{currentfill}{rgb}{0.121569,0.466667,0.705882}%
\pgfsetfillcolor{currentfill}%
\pgfsetfillopacity{0.700000}%
\pgfsetlinewidth{0.000000pt}%
\definecolor{currentstroke}{rgb}{0.000000,0.000000,0.000000}%
\pgfsetstrokecolor{currentstroke}%
\pgfsetstrokeopacity{0.700000}%
\pgfsetdash{}{0pt}%
\pgfpathmoveto{\pgfqpoint{3.831289in}{0.549691in}}%
\pgfpathlineto{\pgfqpoint{3.962783in}{0.549691in}}%
\pgfpathlineto{\pgfqpoint{3.962783in}{2.681777in}}%
\pgfpathlineto{\pgfqpoint{3.831289in}{2.681777in}}%
\pgfpathlineto{\pgfqpoint{3.831289in}{0.549691in}}%
\pgfpathclose%
\pgfusepath{fill}%
\end{pgfscope}%
\begin{pgfscope}%
\pgfpathrectangle{\pgfqpoint{0.603704in}{0.549691in}}{\pgfqpoint{7.246296in}{2.525309in}}%
\pgfusepath{clip}%
\pgfsetbuttcap%
\pgfsetmiterjoin%
\definecolor{currentfill}{rgb}{0.121569,0.466667,0.705882}%
\pgfsetfillcolor{currentfill}%
\pgfsetfillopacity{0.700000}%
\pgfsetlinewidth{0.000000pt}%
\definecolor{currentstroke}{rgb}{0.000000,0.000000,0.000000}%
\pgfsetstrokecolor{currentstroke}%
\pgfsetstrokeopacity{0.700000}%
\pgfsetdash{}{0pt}%
\pgfpathmoveto{\pgfqpoint{3.962783in}{0.549691in}}%
\pgfpathlineto{\pgfqpoint{4.094277in}{0.549691in}}%
\pgfpathlineto{\pgfqpoint{4.094277in}{2.818805in}}%
\pgfpathlineto{\pgfqpoint{3.962783in}{2.818805in}}%
\pgfpathlineto{\pgfqpoint{3.962783in}{0.549691in}}%
\pgfpathclose%
\pgfusepath{fill}%
\end{pgfscope}%
\begin{pgfscope}%
\pgfpathrectangle{\pgfqpoint{0.603704in}{0.549691in}}{\pgfqpoint{7.246296in}{2.525309in}}%
\pgfusepath{clip}%
\pgfsetbuttcap%
\pgfsetmiterjoin%
\definecolor{currentfill}{rgb}{0.121569,0.466667,0.705882}%
\pgfsetfillcolor{currentfill}%
\pgfsetfillopacity{0.700000}%
\pgfsetlinewidth{0.000000pt}%
\definecolor{currentstroke}{rgb}{0.000000,0.000000,0.000000}%
\pgfsetstrokecolor{currentstroke}%
\pgfsetstrokeopacity{0.700000}%
\pgfsetdash{}{0pt}%
\pgfpathmoveto{\pgfqpoint{4.094277in}{0.549691in}}%
\pgfpathlineto{\pgfqpoint{4.225771in}{0.549691in}}%
\pgfpathlineto{\pgfqpoint{4.225771in}{2.924443in}}%
\pgfpathlineto{\pgfqpoint{4.094277in}{2.924443in}}%
\pgfpathlineto{\pgfqpoint{4.094277in}{0.549691in}}%
\pgfpathclose%
\pgfusepath{fill}%
\end{pgfscope}%
\begin{pgfscope}%
\pgfpathrectangle{\pgfqpoint{0.603704in}{0.549691in}}{\pgfqpoint{7.246296in}{2.525309in}}%
\pgfusepath{clip}%
\pgfsetbuttcap%
\pgfsetmiterjoin%
\definecolor{currentfill}{rgb}{0.121569,0.466667,0.705882}%
\pgfsetfillcolor{currentfill}%
\pgfsetfillopacity{0.700000}%
\pgfsetlinewidth{0.000000pt}%
\definecolor{currentstroke}{rgb}{0.000000,0.000000,0.000000}%
\pgfsetstrokecolor{currentstroke}%
\pgfsetstrokeopacity{0.700000}%
\pgfsetdash{}{0pt}%
\pgfpathmoveto{\pgfqpoint{4.225771in}{0.549691in}}%
\pgfpathlineto{\pgfqpoint{4.357265in}{0.549691in}}%
\pgfpathlineto{\pgfqpoint{4.357265in}{2.932894in}}%
\pgfpathlineto{\pgfqpoint{4.225771in}{2.932894in}}%
\pgfpathlineto{\pgfqpoint{4.225771in}{0.549691in}}%
\pgfpathclose%
\pgfusepath{fill}%
\end{pgfscope}%
\begin{pgfscope}%
\pgfpathrectangle{\pgfqpoint{0.603704in}{0.549691in}}{\pgfqpoint{7.246296in}{2.525309in}}%
\pgfusepath{clip}%
\pgfsetbuttcap%
\pgfsetmiterjoin%
\definecolor{currentfill}{rgb}{0.121569,0.466667,0.705882}%
\pgfsetfillcolor{currentfill}%
\pgfsetfillopacity{0.700000}%
\pgfsetlinewidth{0.000000pt}%
\definecolor{currentstroke}{rgb}{0.000000,0.000000,0.000000}%
\pgfsetstrokecolor{currentstroke}%
\pgfsetstrokeopacity{0.700000}%
\pgfsetdash{}{0pt}%
\pgfpathmoveto{\pgfqpoint{4.357265in}{0.549691in}}%
\pgfpathlineto{\pgfqpoint{4.488759in}{0.549691in}}%
\pgfpathlineto{\pgfqpoint{4.488759in}{2.812768in}}%
\pgfpathlineto{\pgfqpoint{4.357265in}{2.812768in}}%
\pgfpathlineto{\pgfqpoint{4.357265in}{0.549691in}}%
\pgfpathclose%
\pgfusepath{fill}%
\end{pgfscope}%
\begin{pgfscope}%
\pgfpathrectangle{\pgfqpoint{0.603704in}{0.549691in}}{\pgfqpoint{7.246296in}{2.525309in}}%
\pgfusepath{clip}%
\pgfsetbuttcap%
\pgfsetmiterjoin%
\definecolor{currentfill}{rgb}{0.121569,0.466667,0.705882}%
\pgfsetfillcolor{currentfill}%
\pgfsetfillopacity{0.700000}%
\pgfsetlinewidth{0.000000pt}%
\definecolor{currentstroke}{rgb}{0.000000,0.000000,0.000000}%
\pgfsetstrokecolor{currentstroke}%
\pgfsetstrokeopacity{0.700000}%
\pgfsetdash{}{0pt}%
\pgfpathmoveto{\pgfqpoint{4.488759in}{0.549691in}}%
\pgfpathlineto{\pgfqpoint{4.620253in}{0.549691in}}%
\pgfpathlineto{\pgfqpoint{4.620253in}{2.756629in}}%
\pgfpathlineto{\pgfqpoint{4.488759in}{2.756629in}}%
\pgfpathlineto{\pgfqpoint{4.488759in}{0.549691in}}%
\pgfpathclose%
\pgfusepath{fill}%
\end{pgfscope}%
\begin{pgfscope}%
\pgfpathrectangle{\pgfqpoint{0.603704in}{0.549691in}}{\pgfqpoint{7.246296in}{2.525309in}}%
\pgfusepath{clip}%
\pgfsetbuttcap%
\pgfsetmiterjoin%
\definecolor{currentfill}{rgb}{0.121569,0.466667,0.705882}%
\pgfsetfillcolor{currentfill}%
\pgfsetfillopacity{0.700000}%
\pgfsetlinewidth{0.000000pt}%
\definecolor{currentstroke}{rgb}{0.000000,0.000000,0.000000}%
\pgfsetstrokecolor{currentstroke}%
\pgfsetstrokeopacity{0.700000}%
\pgfsetdash{}{0pt}%
\pgfpathmoveto{\pgfqpoint{4.620253in}{0.549691in}}%
\pgfpathlineto{\pgfqpoint{4.751746in}{0.549691in}}%
\pgfpathlineto{\pgfqpoint{4.751746in}{2.609339in}}%
\pgfpathlineto{\pgfqpoint{4.620253in}{2.609339in}}%
\pgfpathlineto{\pgfqpoint{4.620253in}{0.549691in}}%
\pgfpathclose%
\pgfusepath{fill}%
\end{pgfscope}%
\begin{pgfscope}%
\pgfpathrectangle{\pgfqpoint{0.603704in}{0.549691in}}{\pgfqpoint{7.246296in}{2.525309in}}%
\pgfusepath{clip}%
\pgfsetbuttcap%
\pgfsetmiterjoin%
\definecolor{currentfill}{rgb}{0.121569,0.466667,0.705882}%
\pgfsetfillcolor{currentfill}%
\pgfsetfillopacity{0.700000}%
\pgfsetlinewidth{0.000000pt}%
\definecolor{currentstroke}{rgb}{0.000000,0.000000,0.000000}%
\pgfsetstrokecolor{currentstroke}%
\pgfsetstrokeopacity{0.700000}%
\pgfsetdash{}{0pt}%
\pgfpathmoveto{\pgfqpoint{4.751746in}{0.549691in}}%
\pgfpathlineto{\pgfqpoint{4.883240in}{0.549691in}}%
\pgfpathlineto{\pgfqpoint{4.883240in}{2.534487in}}%
\pgfpathlineto{\pgfqpoint{4.751746in}{2.534487in}}%
\pgfpathlineto{\pgfqpoint{4.751746in}{0.549691in}}%
\pgfpathclose%
\pgfusepath{fill}%
\end{pgfscope}%
\begin{pgfscope}%
\pgfpathrectangle{\pgfqpoint{0.603704in}{0.549691in}}{\pgfqpoint{7.246296in}{2.525309in}}%
\pgfusepath{clip}%
\pgfsetbuttcap%
\pgfsetmiterjoin%
\definecolor{currentfill}{rgb}{0.121569,0.466667,0.705882}%
\pgfsetfillcolor{currentfill}%
\pgfsetfillopacity{0.700000}%
\pgfsetlinewidth{0.000000pt}%
\definecolor{currentstroke}{rgb}{0.000000,0.000000,0.000000}%
\pgfsetstrokecolor{currentstroke}%
\pgfsetstrokeopacity{0.700000}%
\pgfsetdash{}{0pt}%
\pgfpathmoveto{\pgfqpoint{4.883240in}{0.549691in}}%
\pgfpathlineto{\pgfqpoint{5.014734in}{0.549691in}}%
\pgfpathlineto{\pgfqpoint{5.014734in}{2.418586in}}%
\pgfpathlineto{\pgfqpoint{4.883240in}{2.418586in}}%
\pgfpathlineto{\pgfqpoint{4.883240in}{0.549691in}}%
\pgfpathclose%
\pgfusepath{fill}%
\end{pgfscope}%
\begin{pgfscope}%
\pgfpathrectangle{\pgfqpoint{0.603704in}{0.549691in}}{\pgfqpoint{7.246296in}{2.525309in}}%
\pgfusepath{clip}%
\pgfsetbuttcap%
\pgfsetmiterjoin%
\definecolor{currentfill}{rgb}{0.121569,0.466667,0.705882}%
\pgfsetfillcolor{currentfill}%
\pgfsetfillopacity{0.700000}%
\pgfsetlinewidth{0.000000pt}%
\definecolor{currentstroke}{rgb}{0.000000,0.000000,0.000000}%
\pgfsetstrokecolor{currentstroke}%
\pgfsetstrokeopacity{0.700000}%
\pgfsetdash{}{0pt}%
\pgfpathmoveto{\pgfqpoint{5.014734in}{0.549691in}}%
\pgfpathlineto{\pgfqpoint{5.146228in}{0.549691in}}%
\pgfpathlineto{\pgfqpoint{5.146228in}{2.350374in}}%
\pgfpathlineto{\pgfqpoint{5.014734in}{2.350374in}}%
\pgfpathlineto{\pgfqpoint{5.014734in}{0.549691in}}%
\pgfpathclose%
\pgfusepath{fill}%
\end{pgfscope}%
\begin{pgfscope}%
\pgfpathrectangle{\pgfqpoint{0.603704in}{0.549691in}}{\pgfqpoint{7.246296in}{2.525309in}}%
\pgfusepath{clip}%
\pgfsetbuttcap%
\pgfsetmiterjoin%
\definecolor{currentfill}{rgb}{0.121569,0.466667,0.705882}%
\pgfsetfillcolor{currentfill}%
\pgfsetfillopacity{0.700000}%
\pgfsetlinewidth{0.000000pt}%
\definecolor{currentstroke}{rgb}{0.000000,0.000000,0.000000}%
\pgfsetstrokecolor{currentstroke}%
\pgfsetstrokeopacity{0.700000}%
\pgfsetdash{}{0pt}%
\pgfpathmoveto{\pgfqpoint{5.146228in}{0.549691in}}%
\pgfpathlineto{\pgfqpoint{5.277722in}{0.549691in}}%
\pgfpathlineto{\pgfqpoint{5.277722in}{2.209120in}}%
\pgfpathlineto{\pgfqpoint{5.146228in}{2.209120in}}%
\pgfpathlineto{\pgfqpoint{5.146228in}{0.549691in}}%
\pgfpathclose%
\pgfusepath{fill}%
\end{pgfscope}%
\begin{pgfscope}%
\pgfpathrectangle{\pgfqpoint{0.603704in}{0.549691in}}{\pgfqpoint{7.246296in}{2.525309in}}%
\pgfusepath{clip}%
\pgfsetbuttcap%
\pgfsetmiterjoin%
\definecolor{currentfill}{rgb}{0.121569,0.466667,0.705882}%
\pgfsetfillcolor{currentfill}%
\pgfsetfillopacity{0.700000}%
\pgfsetlinewidth{0.000000pt}%
\definecolor{currentstroke}{rgb}{0.000000,0.000000,0.000000}%
\pgfsetstrokecolor{currentstroke}%
\pgfsetstrokeopacity{0.700000}%
\pgfsetdash{}{0pt}%
\pgfpathmoveto{\pgfqpoint{5.277722in}{0.549691in}}%
\pgfpathlineto{\pgfqpoint{5.409216in}{0.549691in}}%
\pgfpathlineto{\pgfqpoint{5.409216in}{2.162639in}}%
\pgfpathlineto{\pgfqpoint{5.277722in}{2.162639in}}%
\pgfpathlineto{\pgfqpoint{5.277722in}{0.549691in}}%
\pgfpathclose%
\pgfusepath{fill}%
\end{pgfscope}%
\begin{pgfscope}%
\pgfpathrectangle{\pgfqpoint{0.603704in}{0.549691in}}{\pgfqpoint{7.246296in}{2.525309in}}%
\pgfusepath{clip}%
\pgfsetbuttcap%
\pgfsetmiterjoin%
\definecolor{currentfill}{rgb}{0.121569,0.466667,0.705882}%
\pgfsetfillcolor{currentfill}%
\pgfsetfillopacity{0.700000}%
\pgfsetlinewidth{0.000000pt}%
\definecolor{currentstroke}{rgb}{0.000000,0.000000,0.000000}%
\pgfsetstrokecolor{currentstroke}%
\pgfsetstrokeopacity{0.700000}%
\pgfsetdash{}{0pt}%
\pgfpathmoveto{\pgfqpoint{5.409216in}{0.549691in}}%
\pgfpathlineto{\pgfqpoint{5.540710in}{0.549691in}}%
\pgfpathlineto{\pgfqpoint{5.540710in}{2.051568in}}%
\pgfpathlineto{\pgfqpoint{5.409216in}{2.051568in}}%
\pgfpathlineto{\pgfqpoint{5.409216in}{0.549691in}}%
\pgfpathclose%
\pgfusepath{fill}%
\end{pgfscope}%
\begin{pgfscope}%
\pgfpathrectangle{\pgfqpoint{0.603704in}{0.549691in}}{\pgfqpoint{7.246296in}{2.525309in}}%
\pgfusepath{clip}%
\pgfsetbuttcap%
\pgfsetmiterjoin%
\definecolor{currentfill}{rgb}{0.121569,0.466667,0.705882}%
\pgfsetfillcolor{currentfill}%
\pgfsetfillopacity{0.700000}%
\pgfsetlinewidth{0.000000pt}%
\definecolor{currentstroke}{rgb}{0.000000,0.000000,0.000000}%
\pgfsetstrokecolor{currentstroke}%
\pgfsetstrokeopacity{0.700000}%
\pgfsetdash{}{0pt}%
\pgfpathmoveto{\pgfqpoint{5.540710in}{0.549691in}}%
\pgfpathlineto{\pgfqpoint{5.672204in}{0.549691in}}%
\pgfpathlineto{\pgfqpoint{5.672204in}{1.927216in}}%
\pgfpathlineto{\pgfqpoint{5.540710in}{1.927216in}}%
\pgfpathlineto{\pgfqpoint{5.540710in}{0.549691in}}%
\pgfpathclose%
\pgfusepath{fill}%
\end{pgfscope}%
\begin{pgfscope}%
\pgfpathrectangle{\pgfqpoint{0.603704in}{0.549691in}}{\pgfqpoint{7.246296in}{2.525309in}}%
\pgfusepath{clip}%
\pgfsetbuttcap%
\pgfsetmiterjoin%
\definecolor{currentfill}{rgb}{0.121569,0.466667,0.705882}%
\pgfsetfillcolor{currentfill}%
\pgfsetfillopacity{0.700000}%
\pgfsetlinewidth{0.000000pt}%
\definecolor{currentstroke}{rgb}{0.000000,0.000000,0.000000}%
\pgfsetstrokecolor{currentstroke}%
\pgfsetstrokeopacity{0.700000}%
\pgfsetdash{}{0pt}%
\pgfpathmoveto{\pgfqpoint{5.672204in}{0.549691in}}%
\pgfpathlineto{\pgfqpoint{5.803698in}{0.549691in}}%
\pgfpathlineto{\pgfqpoint{5.803698in}{1.854779in}}%
\pgfpathlineto{\pgfqpoint{5.672204in}{1.854779in}}%
\pgfpathlineto{\pgfqpoint{5.672204in}{0.549691in}}%
\pgfpathclose%
\pgfusepath{fill}%
\end{pgfscope}%
\begin{pgfscope}%
\pgfpathrectangle{\pgfqpoint{0.603704in}{0.549691in}}{\pgfqpoint{7.246296in}{2.525309in}}%
\pgfusepath{clip}%
\pgfsetbuttcap%
\pgfsetmiterjoin%
\definecolor{currentfill}{rgb}{0.121569,0.466667,0.705882}%
\pgfsetfillcolor{currentfill}%
\pgfsetfillopacity{0.700000}%
\pgfsetlinewidth{0.000000pt}%
\definecolor{currentstroke}{rgb}{0.000000,0.000000,0.000000}%
\pgfsetstrokecolor{currentstroke}%
\pgfsetstrokeopacity{0.700000}%
\pgfsetdash{}{0pt}%
\pgfpathmoveto{\pgfqpoint{5.803698in}{0.549691in}}%
\pgfpathlineto{\pgfqpoint{5.935192in}{0.549691in}}%
\pgfpathlineto{\pgfqpoint{5.935192in}{1.744311in}}%
\pgfpathlineto{\pgfqpoint{5.803698in}{1.744311in}}%
\pgfpathlineto{\pgfqpoint{5.803698in}{0.549691in}}%
\pgfpathclose%
\pgfusepath{fill}%
\end{pgfscope}%
\begin{pgfscope}%
\pgfpathrectangle{\pgfqpoint{0.603704in}{0.549691in}}{\pgfqpoint{7.246296in}{2.525309in}}%
\pgfusepath{clip}%
\pgfsetbuttcap%
\pgfsetmiterjoin%
\definecolor{currentfill}{rgb}{0.121569,0.466667,0.705882}%
\pgfsetfillcolor{currentfill}%
\pgfsetfillopacity{0.700000}%
\pgfsetlinewidth{0.000000pt}%
\definecolor{currentstroke}{rgb}{0.000000,0.000000,0.000000}%
\pgfsetstrokecolor{currentstroke}%
\pgfsetstrokeopacity{0.700000}%
\pgfsetdash{}{0pt}%
\pgfpathmoveto{\pgfqpoint{5.935192in}{0.549691in}}%
\pgfpathlineto{\pgfqpoint{6.066686in}{0.549691in}}%
\pgfpathlineto{\pgfqpoint{6.066686in}{1.635654in}}%
\pgfpathlineto{\pgfqpoint{5.935192in}{1.635654in}}%
\pgfpathlineto{\pgfqpoint{5.935192in}{0.549691in}}%
\pgfpathclose%
\pgfusepath{fill}%
\end{pgfscope}%
\begin{pgfscope}%
\pgfpathrectangle{\pgfqpoint{0.603704in}{0.549691in}}{\pgfqpoint{7.246296in}{2.525309in}}%
\pgfusepath{clip}%
\pgfsetbuttcap%
\pgfsetmiterjoin%
\definecolor{currentfill}{rgb}{0.121569,0.466667,0.705882}%
\pgfsetfillcolor{currentfill}%
\pgfsetfillopacity{0.700000}%
\pgfsetlinewidth{0.000000pt}%
\definecolor{currentstroke}{rgb}{0.000000,0.000000,0.000000}%
\pgfsetstrokecolor{currentstroke}%
\pgfsetstrokeopacity{0.700000}%
\pgfsetdash{}{0pt}%
\pgfpathmoveto{\pgfqpoint{6.066686in}{0.549691in}}%
\pgfpathlineto{\pgfqpoint{6.198180in}{0.549691in}}%
\pgfpathlineto{\pgfqpoint{6.198180in}{1.561406in}}%
\pgfpathlineto{\pgfqpoint{6.066686in}{1.561406in}}%
\pgfpathlineto{\pgfqpoint{6.066686in}{0.549691in}}%
\pgfpathclose%
\pgfusepath{fill}%
\end{pgfscope}%
\begin{pgfscope}%
\pgfpathrectangle{\pgfqpoint{0.603704in}{0.549691in}}{\pgfqpoint{7.246296in}{2.525309in}}%
\pgfusepath{clip}%
\pgfsetbuttcap%
\pgfsetmiterjoin%
\definecolor{currentfill}{rgb}{0.121569,0.466667,0.705882}%
\pgfsetfillcolor{currentfill}%
\pgfsetfillopacity{0.700000}%
\pgfsetlinewidth{0.000000pt}%
\definecolor{currentstroke}{rgb}{0.000000,0.000000,0.000000}%
\pgfsetstrokecolor{currentstroke}%
\pgfsetstrokeopacity{0.700000}%
\pgfsetdash{}{0pt}%
\pgfpathmoveto{\pgfqpoint{6.198180in}{0.549691in}}%
\pgfpathlineto{\pgfqpoint{6.329674in}{0.549691in}}%
\pgfpathlineto{\pgfqpoint{6.329674in}{1.461200in}}%
\pgfpathlineto{\pgfqpoint{6.198180in}{1.461200in}}%
\pgfpathlineto{\pgfqpoint{6.198180in}{0.549691in}}%
\pgfpathclose%
\pgfusepath{fill}%
\end{pgfscope}%
\begin{pgfscope}%
\pgfpathrectangle{\pgfqpoint{0.603704in}{0.549691in}}{\pgfqpoint{7.246296in}{2.525309in}}%
\pgfusepath{clip}%
\pgfsetbuttcap%
\pgfsetmiterjoin%
\definecolor{currentfill}{rgb}{0.121569,0.466667,0.705882}%
\pgfsetfillcolor{currentfill}%
\pgfsetfillopacity{0.700000}%
\pgfsetlinewidth{0.000000pt}%
\definecolor{currentstroke}{rgb}{0.000000,0.000000,0.000000}%
\pgfsetstrokecolor{currentstroke}%
\pgfsetstrokeopacity{0.700000}%
\pgfsetdash{}{0pt}%
\pgfpathmoveto{\pgfqpoint{6.329674in}{0.549691in}}%
\pgfpathlineto{\pgfqpoint{6.461168in}{0.549691in}}%
\pgfpathlineto{\pgfqpoint{6.461168in}{1.377293in}}%
\pgfpathlineto{\pgfqpoint{6.329674in}{1.377293in}}%
\pgfpathlineto{\pgfqpoint{6.329674in}{0.549691in}}%
\pgfpathclose%
\pgfusepath{fill}%
\end{pgfscope}%
\begin{pgfscope}%
\pgfpathrectangle{\pgfqpoint{0.603704in}{0.549691in}}{\pgfqpoint{7.246296in}{2.525309in}}%
\pgfusepath{clip}%
\pgfsetbuttcap%
\pgfsetmiterjoin%
\definecolor{currentfill}{rgb}{0.121569,0.466667,0.705882}%
\pgfsetfillcolor{currentfill}%
\pgfsetfillopacity{0.700000}%
\pgfsetlinewidth{0.000000pt}%
\definecolor{currentstroke}{rgb}{0.000000,0.000000,0.000000}%
\pgfsetstrokecolor{currentstroke}%
\pgfsetstrokeopacity{0.700000}%
\pgfsetdash{}{0pt}%
\pgfpathmoveto{\pgfqpoint{6.461168in}{0.549691in}}%
\pgfpathlineto{\pgfqpoint{6.592662in}{0.549691in}}%
\pgfpathlineto{\pgfqpoint{6.592662in}{1.250527in}}%
\pgfpathlineto{\pgfqpoint{6.461168in}{1.250527in}}%
\pgfpathlineto{\pgfqpoint{6.461168in}{0.549691in}}%
\pgfpathclose%
\pgfusepath{fill}%
\end{pgfscope}%
\begin{pgfscope}%
\pgfpathrectangle{\pgfqpoint{0.603704in}{0.549691in}}{\pgfqpoint{7.246296in}{2.525309in}}%
\pgfusepath{clip}%
\pgfsetbuttcap%
\pgfsetmiterjoin%
\definecolor{currentfill}{rgb}{0.121569,0.466667,0.705882}%
\pgfsetfillcolor{currentfill}%
\pgfsetfillopacity{0.700000}%
\pgfsetlinewidth{0.000000pt}%
\definecolor{currentstroke}{rgb}{0.000000,0.000000,0.000000}%
\pgfsetstrokecolor{currentstroke}%
\pgfsetstrokeopacity{0.700000}%
\pgfsetdash{}{0pt}%
\pgfpathmoveto{\pgfqpoint{6.592662in}{0.549691in}}%
\pgfpathlineto{\pgfqpoint{6.724156in}{0.549691in}}%
\pgfpathlineto{\pgfqpoint{6.724156in}{1.169638in}}%
\pgfpathlineto{\pgfqpoint{6.592662in}{1.169638in}}%
\pgfpathlineto{\pgfqpoint{6.592662in}{0.549691in}}%
\pgfpathclose%
\pgfusepath{fill}%
\end{pgfscope}%
\begin{pgfscope}%
\pgfpathrectangle{\pgfqpoint{0.603704in}{0.549691in}}{\pgfqpoint{7.246296in}{2.525309in}}%
\pgfusepath{clip}%
\pgfsetbuttcap%
\pgfsetmiterjoin%
\definecolor{currentfill}{rgb}{0.121569,0.466667,0.705882}%
\pgfsetfillcolor{currentfill}%
\pgfsetfillopacity{0.700000}%
\pgfsetlinewidth{0.000000pt}%
\definecolor{currentstroke}{rgb}{0.000000,0.000000,0.000000}%
\pgfsetstrokecolor{currentstroke}%
\pgfsetstrokeopacity{0.700000}%
\pgfsetdash{}{0pt}%
\pgfpathmoveto{\pgfqpoint{6.724156in}{0.549691in}}%
\pgfpathlineto{\pgfqpoint{6.855650in}{0.549691in}}%
\pgfpathlineto{\pgfqpoint{6.855650in}{1.111084in}}%
\pgfpathlineto{\pgfqpoint{6.724156in}{1.111084in}}%
\pgfpathlineto{\pgfqpoint{6.724156in}{0.549691in}}%
\pgfpathclose%
\pgfusepath{fill}%
\end{pgfscope}%
\begin{pgfscope}%
\pgfpathrectangle{\pgfqpoint{0.603704in}{0.549691in}}{\pgfqpoint{7.246296in}{2.525309in}}%
\pgfusepath{clip}%
\pgfsetbuttcap%
\pgfsetmiterjoin%
\definecolor{currentfill}{rgb}{0.121569,0.466667,0.705882}%
\pgfsetfillcolor{currentfill}%
\pgfsetfillopacity{0.700000}%
\pgfsetlinewidth{0.000000pt}%
\definecolor{currentstroke}{rgb}{0.000000,0.000000,0.000000}%
\pgfsetstrokecolor{currentstroke}%
\pgfsetstrokeopacity{0.700000}%
\pgfsetdash{}{0pt}%
\pgfpathmoveto{\pgfqpoint{6.855650in}{0.549691in}}%
\pgfpathlineto{\pgfqpoint{6.987144in}{0.549691in}}%
\pgfpathlineto{\pgfqpoint{6.987144in}{1.004842in}}%
\pgfpathlineto{\pgfqpoint{6.855650in}{1.004842in}}%
\pgfpathlineto{\pgfqpoint{6.855650in}{0.549691in}}%
\pgfpathclose%
\pgfusepath{fill}%
\end{pgfscope}%
\begin{pgfscope}%
\pgfpathrectangle{\pgfqpoint{0.603704in}{0.549691in}}{\pgfqpoint{7.246296in}{2.525309in}}%
\pgfusepath{clip}%
\pgfsetbuttcap%
\pgfsetmiterjoin%
\definecolor{currentfill}{rgb}{0.121569,0.466667,0.705882}%
\pgfsetfillcolor{currentfill}%
\pgfsetfillopacity{0.700000}%
\pgfsetlinewidth{0.000000pt}%
\definecolor{currentstroke}{rgb}{0.000000,0.000000,0.000000}%
\pgfsetstrokecolor{currentstroke}%
\pgfsetstrokeopacity{0.700000}%
\pgfsetdash{}{0pt}%
\pgfpathmoveto{\pgfqpoint{6.987144in}{0.549691in}}%
\pgfpathlineto{\pgfqpoint{7.118638in}{0.549691in}}%
\pgfpathlineto{\pgfqpoint{7.118638in}{0.888338in}}%
\pgfpathlineto{\pgfqpoint{6.987144in}{0.888338in}}%
\pgfpathlineto{\pgfqpoint{6.987144in}{0.549691in}}%
\pgfpathclose%
\pgfusepath{fill}%
\end{pgfscope}%
\begin{pgfscope}%
\pgfpathrectangle{\pgfqpoint{0.603704in}{0.549691in}}{\pgfqpoint{7.246296in}{2.525309in}}%
\pgfusepath{clip}%
\pgfsetbuttcap%
\pgfsetmiterjoin%
\definecolor{currentfill}{rgb}{0.121569,0.466667,0.705882}%
\pgfsetfillcolor{currentfill}%
\pgfsetfillopacity{0.700000}%
\pgfsetlinewidth{0.000000pt}%
\definecolor{currentstroke}{rgb}{0.000000,0.000000,0.000000}%
\pgfsetstrokecolor{currentstroke}%
\pgfsetstrokeopacity{0.700000}%
\pgfsetdash{}{0pt}%
\pgfpathmoveto{\pgfqpoint{7.118638in}{0.549691in}}%
\pgfpathlineto{\pgfqpoint{7.250132in}{0.549691in}}%
\pgfpathlineto{\pgfqpoint{7.250132in}{0.817107in}}%
\pgfpathlineto{\pgfqpoint{7.118638in}{0.817107in}}%
\pgfpathlineto{\pgfqpoint{7.118638in}{0.549691in}}%
\pgfpathclose%
\pgfusepath{fill}%
\end{pgfscope}%
\begin{pgfscope}%
\pgfpathrectangle{\pgfqpoint{0.603704in}{0.549691in}}{\pgfqpoint{7.246296in}{2.525309in}}%
\pgfusepath{clip}%
\pgfsetbuttcap%
\pgfsetmiterjoin%
\definecolor{currentfill}{rgb}{0.121569,0.466667,0.705882}%
\pgfsetfillcolor{currentfill}%
\pgfsetfillopacity{0.700000}%
\pgfsetlinewidth{0.000000pt}%
\definecolor{currentstroke}{rgb}{0.000000,0.000000,0.000000}%
\pgfsetstrokecolor{currentstroke}%
\pgfsetstrokeopacity{0.700000}%
\pgfsetdash{}{0pt}%
\pgfpathmoveto{\pgfqpoint{7.250132in}{0.549691in}}%
\pgfpathlineto{\pgfqpoint{7.381626in}{0.549691in}}%
\pgfpathlineto{\pgfqpoint{7.381626in}{0.691548in}}%
\pgfpathlineto{\pgfqpoint{7.250132in}{0.691548in}}%
\pgfpathlineto{\pgfqpoint{7.250132in}{0.549691in}}%
\pgfpathclose%
\pgfusepath{fill}%
\end{pgfscope}%
\begin{pgfscope}%
\pgfpathrectangle{\pgfqpoint{0.603704in}{0.549691in}}{\pgfqpoint{7.246296in}{2.525309in}}%
\pgfusepath{clip}%
\pgfsetbuttcap%
\pgfsetmiterjoin%
\definecolor{currentfill}{rgb}{0.121569,0.466667,0.705882}%
\pgfsetfillcolor{currentfill}%
\pgfsetfillopacity{0.700000}%
\pgfsetlinewidth{0.000000pt}%
\definecolor{currentstroke}{rgb}{0.000000,0.000000,0.000000}%
\pgfsetstrokecolor{currentstroke}%
\pgfsetstrokeopacity{0.700000}%
\pgfsetdash{}{0pt}%
\pgfpathmoveto{\pgfqpoint{7.381626in}{0.549691in}}%
\pgfpathlineto{\pgfqpoint{7.513120in}{0.549691in}}%
\pgfpathlineto{\pgfqpoint{7.513120in}{0.594965in}}%
\pgfpathlineto{\pgfqpoint{7.381626in}{0.594965in}}%
\pgfpathlineto{\pgfqpoint{7.381626in}{0.549691in}}%
\pgfpathclose%
\pgfusepath{fill}%
\end{pgfscope}%
\begin{pgfscope}%
\pgfsetbuttcap%
\pgfsetroundjoin%
\definecolor{currentfill}{rgb}{0.000000,0.000000,0.000000}%
\pgfsetfillcolor{currentfill}%
\pgfsetlinewidth{0.803000pt}%
\definecolor{currentstroke}{rgb}{0.000000,0.000000,0.000000}%
\pgfsetstrokecolor{currentstroke}%
\pgfsetdash{}{0pt}%
\pgfsys@defobject{currentmarker}{\pgfqpoint{0.000000in}{-0.048611in}}{\pgfqpoint{0.000000in}{0.000000in}}{%
\pgfpathmoveto{\pgfqpoint{0.000000in}{0.000000in}}%
\pgfpathlineto{\pgfqpoint{0.000000in}{-0.048611in}}%
\pgfusepath{stroke,fill}%
}%
\begin{pgfscope}%
\pgfsys@transformshift{0.933081in}{0.549691in}%
\pgfsys@useobject{currentmarker}{}%
\end{pgfscope}%
\end{pgfscope}%
\begin{pgfscope}%
\definecolor{textcolor}{rgb}{0.000000,0.000000,0.000000}%
\pgfsetstrokecolor{textcolor}%
\pgfsetfillcolor{textcolor}%
\pgftext[x=0.933081in,y=0.452469in,,top]{\color{textcolor}{\rmfamily\fontsize{10.000000}{12.000000}\selectfont\catcode`\^=\active\def^{\ifmmode\sp\else\^{}\fi}\catcode`\%=\active\def%{\%}$\mathdefault{0.00}$}}%
\end{pgfscope}%
\begin{pgfscope}%
\pgfsetbuttcap%
\pgfsetroundjoin%
\definecolor{currentfill}{rgb}{0.000000,0.000000,0.000000}%
\pgfsetfillcolor{currentfill}%
\pgfsetlinewidth{0.803000pt}%
\definecolor{currentstroke}{rgb}{0.000000,0.000000,0.000000}%
\pgfsetstrokecolor{currentstroke}%
\pgfsetdash{}{0pt}%
\pgfsys@defobject{currentmarker}{\pgfqpoint{0.000000in}{-0.048611in}}{\pgfqpoint{0.000000in}{0.000000in}}{%
\pgfpathmoveto{\pgfqpoint{0.000000in}{0.000000in}}%
\pgfpathlineto{\pgfqpoint{0.000000in}{-0.048611in}}%
\pgfusepath{stroke,fill}%
}%
\begin{pgfscope}%
\pgfsys@transformshift{1.756524in}{0.549691in}%
\pgfsys@useobject{currentmarker}{}%
\end{pgfscope}%
\end{pgfscope}%
\begin{pgfscope}%
\definecolor{textcolor}{rgb}{0.000000,0.000000,0.000000}%
\pgfsetstrokecolor{textcolor}%
\pgfsetfillcolor{textcolor}%
\pgftext[x=1.756524in,y=0.452469in,,top]{\color{textcolor}{\rmfamily\fontsize{10.000000}{12.000000}\selectfont\catcode`\^=\active\def^{\ifmmode\sp\else\^{}\fi}\catcode`\%=\active\def%{\%}$\mathdefault{0.25}$}}%
\end{pgfscope}%
\begin{pgfscope}%
\pgfsetbuttcap%
\pgfsetroundjoin%
\definecolor{currentfill}{rgb}{0.000000,0.000000,0.000000}%
\pgfsetfillcolor{currentfill}%
\pgfsetlinewidth{0.803000pt}%
\definecolor{currentstroke}{rgb}{0.000000,0.000000,0.000000}%
\pgfsetstrokecolor{currentstroke}%
\pgfsetdash{}{0pt}%
\pgfsys@defobject{currentmarker}{\pgfqpoint{0.000000in}{-0.048611in}}{\pgfqpoint{0.000000in}{0.000000in}}{%
\pgfpathmoveto{\pgfqpoint{0.000000in}{0.000000in}}%
\pgfpathlineto{\pgfqpoint{0.000000in}{-0.048611in}}%
\pgfusepath{stroke,fill}%
}%
\begin{pgfscope}%
\pgfsys@transformshift{2.579967in}{0.549691in}%
\pgfsys@useobject{currentmarker}{}%
\end{pgfscope}%
\end{pgfscope}%
\begin{pgfscope}%
\definecolor{textcolor}{rgb}{0.000000,0.000000,0.000000}%
\pgfsetstrokecolor{textcolor}%
\pgfsetfillcolor{textcolor}%
\pgftext[x=2.579967in,y=0.452469in,,top]{\color{textcolor}{\rmfamily\fontsize{10.000000}{12.000000}\selectfont\catcode`\^=\active\def^{\ifmmode\sp\else\^{}\fi}\catcode`\%=\active\def%{\%}$\mathdefault{0.50}$}}%
\end{pgfscope}%
\begin{pgfscope}%
\pgfsetbuttcap%
\pgfsetroundjoin%
\definecolor{currentfill}{rgb}{0.000000,0.000000,0.000000}%
\pgfsetfillcolor{currentfill}%
\pgfsetlinewidth{0.803000pt}%
\definecolor{currentstroke}{rgb}{0.000000,0.000000,0.000000}%
\pgfsetstrokecolor{currentstroke}%
\pgfsetdash{}{0pt}%
\pgfsys@defobject{currentmarker}{\pgfqpoint{0.000000in}{-0.048611in}}{\pgfqpoint{0.000000in}{0.000000in}}{%
\pgfpathmoveto{\pgfqpoint{0.000000in}{0.000000in}}%
\pgfpathlineto{\pgfqpoint{0.000000in}{-0.048611in}}%
\pgfusepath{stroke,fill}%
}%
\begin{pgfscope}%
\pgfsys@transformshift{3.403409in}{0.549691in}%
\pgfsys@useobject{currentmarker}{}%
\end{pgfscope}%
\end{pgfscope}%
\begin{pgfscope}%
\definecolor{textcolor}{rgb}{0.000000,0.000000,0.000000}%
\pgfsetstrokecolor{textcolor}%
\pgfsetfillcolor{textcolor}%
\pgftext[x=3.403409in,y=0.452469in,,top]{\color{textcolor}{\rmfamily\fontsize{10.000000}{12.000000}\selectfont\catcode`\^=\active\def^{\ifmmode\sp\else\^{}\fi}\catcode`\%=\active\def%{\%}$\mathdefault{0.75}$}}%
\end{pgfscope}%
\begin{pgfscope}%
\pgfsetbuttcap%
\pgfsetroundjoin%
\definecolor{currentfill}{rgb}{0.000000,0.000000,0.000000}%
\pgfsetfillcolor{currentfill}%
\pgfsetlinewidth{0.803000pt}%
\definecolor{currentstroke}{rgb}{0.000000,0.000000,0.000000}%
\pgfsetstrokecolor{currentstroke}%
\pgfsetdash{}{0pt}%
\pgfsys@defobject{currentmarker}{\pgfqpoint{0.000000in}{-0.048611in}}{\pgfqpoint{0.000000in}{0.000000in}}{%
\pgfpathmoveto{\pgfqpoint{0.000000in}{0.000000in}}%
\pgfpathlineto{\pgfqpoint{0.000000in}{-0.048611in}}%
\pgfusepath{stroke,fill}%
}%
\begin{pgfscope}%
\pgfsys@transformshift{4.226852in}{0.549691in}%
\pgfsys@useobject{currentmarker}{}%
\end{pgfscope}%
\end{pgfscope}%
\begin{pgfscope}%
\definecolor{textcolor}{rgb}{0.000000,0.000000,0.000000}%
\pgfsetstrokecolor{textcolor}%
\pgfsetfillcolor{textcolor}%
\pgftext[x=4.226852in,y=0.452469in,,top]{\color{textcolor}{\rmfamily\fontsize{10.000000}{12.000000}\selectfont\catcode`\^=\active\def^{\ifmmode\sp\else\^{}\fi}\catcode`\%=\active\def%{\%}$\mathdefault{1.00}$}}%
\end{pgfscope}%
\begin{pgfscope}%
\pgfsetbuttcap%
\pgfsetroundjoin%
\definecolor{currentfill}{rgb}{0.000000,0.000000,0.000000}%
\pgfsetfillcolor{currentfill}%
\pgfsetlinewidth{0.803000pt}%
\definecolor{currentstroke}{rgb}{0.000000,0.000000,0.000000}%
\pgfsetstrokecolor{currentstroke}%
\pgfsetdash{}{0pt}%
\pgfsys@defobject{currentmarker}{\pgfqpoint{0.000000in}{-0.048611in}}{\pgfqpoint{0.000000in}{0.000000in}}{%
\pgfpathmoveto{\pgfqpoint{0.000000in}{0.000000in}}%
\pgfpathlineto{\pgfqpoint{0.000000in}{-0.048611in}}%
\pgfusepath{stroke,fill}%
}%
\begin{pgfscope}%
\pgfsys@transformshift{5.050295in}{0.549691in}%
\pgfsys@useobject{currentmarker}{}%
\end{pgfscope}%
\end{pgfscope}%
\begin{pgfscope}%
\definecolor{textcolor}{rgb}{0.000000,0.000000,0.000000}%
\pgfsetstrokecolor{textcolor}%
\pgfsetfillcolor{textcolor}%
\pgftext[x=5.050295in,y=0.452469in,,top]{\color{textcolor}{\rmfamily\fontsize{10.000000}{12.000000}\selectfont\catcode`\^=\active\def^{\ifmmode\sp\else\^{}\fi}\catcode`\%=\active\def%{\%}$\mathdefault{1.25}$}}%
\end{pgfscope}%
\begin{pgfscope}%
\pgfsetbuttcap%
\pgfsetroundjoin%
\definecolor{currentfill}{rgb}{0.000000,0.000000,0.000000}%
\pgfsetfillcolor{currentfill}%
\pgfsetlinewidth{0.803000pt}%
\definecolor{currentstroke}{rgb}{0.000000,0.000000,0.000000}%
\pgfsetstrokecolor{currentstroke}%
\pgfsetdash{}{0pt}%
\pgfsys@defobject{currentmarker}{\pgfqpoint{0.000000in}{-0.048611in}}{\pgfqpoint{0.000000in}{0.000000in}}{%
\pgfpathmoveto{\pgfqpoint{0.000000in}{0.000000in}}%
\pgfpathlineto{\pgfqpoint{0.000000in}{-0.048611in}}%
\pgfusepath{stroke,fill}%
}%
\begin{pgfscope}%
\pgfsys@transformshift{5.873738in}{0.549691in}%
\pgfsys@useobject{currentmarker}{}%
\end{pgfscope}%
\end{pgfscope}%
\begin{pgfscope}%
\definecolor{textcolor}{rgb}{0.000000,0.000000,0.000000}%
\pgfsetstrokecolor{textcolor}%
\pgfsetfillcolor{textcolor}%
\pgftext[x=5.873738in,y=0.452469in,,top]{\color{textcolor}{\rmfamily\fontsize{10.000000}{12.000000}\selectfont\catcode`\^=\active\def^{\ifmmode\sp\else\^{}\fi}\catcode`\%=\active\def%{\%}$\mathdefault{1.50}$}}%
\end{pgfscope}%
\begin{pgfscope}%
\pgfsetbuttcap%
\pgfsetroundjoin%
\definecolor{currentfill}{rgb}{0.000000,0.000000,0.000000}%
\pgfsetfillcolor{currentfill}%
\pgfsetlinewidth{0.803000pt}%
\definecolor{currentstroke}{rgb}{0.000000,0.000000,0.000000}%
\pgfsetstrokecolor{currentstroke}%
\pgfsetdash{}{0pt}%
\pgfsys@defobject{currentmarker}{\pgfqpoint{0.000000in}{-0.048611in}}{\pgfqpoint{0.000000in}{0.000000in}}{%
\pgfpathmoveto{\pgfqpoint{0.000000in}{0.000000in}}%
\pgfpathlineto{\pgfqpoint{0.000000in}{-0.048611in}}%
\pgfusepath{stroke,fill}%
}%
\begin{pgfscope}%
\pgfsys@transformshift{6.697180in}{0.549691in}%
\pgfsys@useobject{currentmarker}{}%
\end{pgfscope}%
\end{pgfscope}%
\begin{pgfscope}%
\definecolor{textcolor}{rgb}{0.000000,0.000000,0.000000}%
\pgfsetstrokecolor{textcolor}%
\pgfsetfillcolor{textcolor}%
\pgftext[x=6.697180in,y=0.452469in,,top]{\color{textcolor}{\rmfamily\fontsize{10.000000}{12.000000}\selectfont\catcode`\^=\active\def^{\ifmmode\sp\else\^{}\fi}\catcode`\%=\active\def%{\%}$\mathdefault{1.75}$}}%
\end{pgfscope}%
\begin{pgfscope}%
\pgfsetbuttcap%
\pgfsetroundjoin%
\definecolor{currentfill}{rgb}{0.000000,0.000000,0.000000}%
\pgfsetfillcolor{currentfill}%
\pgfsetlinewidth{0.803000pt}%
\definecolor{currentstroke}{rgb}{0.000000,0.000000,0.000000}%
\pgfsetstrokecolor{currentstroke}%
\pgfsetdash{}{0pt}%
\pgfsys@defobject{currentmarker}{\pgfqpoint{0.000000in}{-0.048611in}}{\pgfqpoint{0.000000in}{0.000000in}}{%
\pgfpathmoveto{\pgfqpoint{0.000000in}{0.000000in}}%
\pgfpathlineto{\pgfqpoint{0.000000in}{-0.048611in}}%
\pgfusepath{stroke,fill}%
}%
\begin{pgfscope}%
\pgfsys@transformshift{7.520623in}{0.549691in}%
\pgfsys@useobject{currentmarker}{}%
\end{pgfscope}%
\end{pgfscope}%
\begin{pgfscope}%
\definecolor{textcolor}{rgb}{0.000000,0.000000,0.000000}%
\pgfsetstrokecolor{textcolor}%
\pgfsetfillcolor{textcolor}%
\pgftext[x=7.520623in,y=0.452469in,,top]{\color{textcolor}{\rmfamily\fontsize{10.000000}{12.000000}\selectfont\catcode`\^=\active\def^{\ifmmode\sp\else\^{}\fi}\catcode`\%=\active\def%{\%}$\mathdefault{2.00}$}}%
\end{pgfscope}%
\begin{pgfscope}%
\definecolor{textcolor}{rgb}{0.000000,0.000000,0.000000}%
\pgfsetstrokecolor{textcolor}%
\pgfsetfillcolor{textcolor}%
\pgftext[x=4.226852in,y=0.273457in,,top]{\color{textcolor}{\rmfamily\fontsize{10.000000}{12.000000}\selectfont\catcode`\^=\active\def^{\ifmmode\sp\else\^{}\fi}\catcode`\%=\active\def%{\%}$z$}}%
\end{pgfscope}%
\begin{pgfscope}%
\pgfsetbuttcap%
\pgfsetroundjoin%
\definecolor{currentfill}{rgb}{0.000000,0.000000,0.000000}%
\pgfsetfillcolor{currentfill}%
\pgfsetlinewidth{0.803000pt}%
\definecolor{currentstroke}{rgb}{0.000000,0.000000,0.000000}%
\pgfsetstrokecolor{currentstroke}%
\pgfsetdash{}{0pt}%
\pgfsys@defobject{currentmarker}{\pgfqpoint{-0.048611in}{0.000000in}}{\pgfqpoint{-0.000000in}{0.000000in}}{%
\pgfpathmoveto{\pgfqpoint{-0.000000in}{0.000000in}}%
\pgfpathlineto{\pgfqpoint{-0.048611in}{0.000000in}}%
\pgfusepath{stroke,fill}%
}%
\begin{pgfscope}%
\pgfsys@transformshift{0.603704in}{0.549691in}%
\pgfsys@useobject{currentmarker}{}%
\end{pgfscope}%
\end{pgfscope}%
\begin{pgfscope}%
\definecolor{textcolor}{rgb}{0.000000,0.000000,0.000000}%
\pgfsetstrokecolor{textcolor}%
\pgfsetfillcolor{textcolor}%
\pgftext[x=0.329012in, y=0.501466in, left, base]{\color{textcolor}{\rmfamily\fontsize{10.000000}{12.000000}\selectfont\catcode`\^=\active\def^{\ifmmode\sp\else\^{}\fi}\catcode`\%=\active\def%{\%}$\mathdefault{0.0}$}}%
\end{pgfscope}%
\begin{pgfscope}%
\pgfsetbuttcap%
\pgfsetroundjoin%
\definecolor{currentfill}{rgb}{0.000000,0.000000,0.000000}%
\pgfsetfillcolor{currentfill}%
\pgfsetlinewidth{0.803000pt}%
\definecolor{currentstroke}{rgb}{0.000000,0.000000,0.000000}%
\pgfsetstrokecolor{currentstroke}%
\pgfsetdash{}{0pt}%
\pgfsys@defobject{currentmarker}{\pgfqpoint{-0.048611in}{0.000000in}}{\pgfqpoint{-0.000000in}{0.000000in}}{%
\pgfpathmoveto{\pgfqpoint{-0.000000in}{0.000000in}}%
\pgfpathlineto{\pgfqpoint{-0.048611in}{0.000000in}}%
\pgfusepath{stroke,fill}%
}%
\begin{pgfscope}%
\pgfsys@transformshift{0.603704in}{1.031668in}%
\pgfsys@useobject{currentmarker}{}%
\end{pgfscope}%
\end{pgfscope}%
\begin{pgfscope}%
\definecolor{textcolor}{rgb}{0.000000,0.000000,0.000000}%
\pgfsetstrokecolor{textcolor}%
\pgfsetfillcolor{textcolor}%
\pgftext[x=0.329012in, y=0.983443in, left, base]{\color{textcolor}{\rmfamily\fontsize{10.000000}{12.000000}\selectfont\catcode`\^=\active\def^{\ifmmode\sp\else\^{}\fi}\catcode`\%=\active\def%{\%}$\mathdefault{0.2}$}}%
\end{pgfscope}%
\begin{pgfscope}%
\pgfsetbuttcap%
\pgfsetroundjoin%
\definecolor{currentfill}{rgb}{0.000000,0.000000,0.000000}%
\pgfsetfillcolor{currentfill}%
\pgfsetlinewidth{0.803000pt}%
\definecolor{currentstroke}{rgb}{0.000000,0.000000,0.000000}%
\pgfsetstrokecolor{currentstroke}%
\pgfsetdash{}{0pt}%
\pgfsys@defobject{currentmarker}{\pgfqpoint{-0.048611in}{0.000000in}}{\pgfqpoint{-0.000000in}{0.000000in}}{%
\pgfpathmoveto{\pgfqpoint{-0.000000in}{0.000000in}}%
\pgfpathlineto{\pgfqpoint{-0.048611in}{0.000000in}}%
\pgfusepath{stroke,fill}%
}%
\begin{pgfscope}%
\pgfsys@transformshift{0.603704in}{1.513645in}%
\pgfsys@useobject{currentmarker}{}%
\end{pgfscope}%
\end{pgfscope}%
\begin{pgfscope}%
\definecolor{textcolor}{rgb}{0.000000,0.000000,0.000000}%
\pgfsetstrokecolor{textcolor}%
\pgfsetfillcolor{textcolor}%
\pgftext[x=0.329012in, y=1.465420in, left, base]{\color{textcolor}{\rmfamily\fontsize{10.000000}{12.000000}\selectfont\catcode`\^=\active\def^{\ifmmode\sp\else\^{}\fi}\catcode`\%=\active\def%{\%}$\mathdefault{0.4}$}}%
\end{pgfscope}%
\begin{pgfscope}%
\pgfsetbuttcap%
\pgfsetroundjoin%
\definecolor{currentfill}{rgb}{0.000000,0.000000,0.000000}%
\pgfsetfillcolor{currentfill}%
\pgfsetlinewidth{0.803000pt}%
\definecolor{currentstroke}{rgb}{0.000000,0.000000,0.000000}%
\pgfsetstrokecolor{currentstroke}%
\pgfsetdash{}{0pt}%
\pgfsys@defobject{currentmarker}{\pgfqpoint{-0.048611in}{0.000000in}}{\pgfqpoint{-0.000000in}{0.000000in}}{%
\pgfpathmoveto{\pgfqpoint{-0.000000in}{0.000000in}}%
\pgfpathlineto{\pgfqpoint{-0.048611in}{0.000000in}}%
\pgfusepath{stroke,fill}%
}%
\begin{pgfscope}%
\pgfsys@transformshift{0.603704in}{1.995622in}%
\pgfsys@useobject{currentmarker}{}%
\end{pgfscope}%
\end{pgfscope}%
\begin{pgfscope}%
\definecolor{textcolor}{rgb}{0.000000,0.000000,0.000000}%
\pgfsetstrokecolor{textcolor}%
\pgfsetfillcolor{textcolor}%
\pgftext[x=0.329012in, y=1.947397in, left, base]{\color{textcolor}{\rmfamily\fontsize{10.000000}{12.000000}\selectfont\catcode`\^=\active\def^{\ifmmode\sp\else\^{}\fi}\catcode`\%=\active\def%{\%}$\mathdefault{0.6}$}}%
\end{pgfscope}%
\begin{pgfscope}%
\pgfsetbuttcap%
\pgfsetroundjoin%
\definecolor{currentfill}{rgb}{0.000000,0.000000,0.000000}%
\pgfsetfillcolor{currentfill}%
\pgfsetlinewidth{0.803000pt}%
\definecolor{currentstroke}{rgb}{0.000000,0.000000,0.000000}%
\pgfsetstrokecolor{currentstroke}%
\pgfsetdash{}{0pt}%
\pgfsys@defobject{currentmarker}{\pgfqpoint{-0.048611in}{0.000000in}}{\pgfqpoint{-0.000000in}{0.000000in}}{%
\pgfpathmoveto{\pgfqpoint{-0.000000in}{0.000000in}}%
\pgfpathlineto{\pgfqpoint{-0.048611in}{0.000000in}}%
\pgfusepath{stroke,fill}%
}%
\begin{pgfscope}%
\pgfsys@transformshift{0.603704in}{2.477600in}%
\pgfsys@useobject{currentmarker}{}%
\end{pgfscope}%
\end{pgfscope}%
\begin{pgfscope}%
\definecolor{textcolor}{rgb}{0.000000,0.000000,0.000000}%
\pgfsetstrokecolor{textcolor}%
\pgfsetfillcolor{textcolor}%
\pgftext[x=0.329012in, y=2.429374in, left, base]{\color{textcolor}{\rmfamily\fontsize{10.000000}{12.000000}\selectfont\catcode`\^=\active\def^{\ifmmode\sp\else\^{}\fi}\catcode`\%=\active\def%{\%}$\mathdefault{0.8}$}}%
\end{pgfscope}%
\begin{pgfscope}%
\pgfsetbuttcap%
\pgfsetroundjoin%
\definecolor{currentfill}{rgb}{0.000000,0.000000,0.000000}%
\pgfsetfillcolor{currentfill}%
\pgfsetlinewidth{0.803000pt}%
\definecolor{currentstroke}{rgb}{0.000000,0.000000,0.000000}%
\pgfsetstrokecolor{currentstroke}%
\pgfsetdash{}{0pt}%
\pgfsys@defobject{currentmarker}{\pgfqpoint{-0.048611in}{0.000000in}}{\pgfqpoint{-0.000000in}{0.000000in}}{%
\pgfpathmoveto{\pgfqpoint{-0.000000in}{0.000000in}}%
\pgfpathlineto{\pgfqpoint{-0.048611in}{0.000000in}}%
\pgfusepath{stroke,fill}%
}%
\begin{pgfscope}%
\pgfsys@transformshift{0.603704in}{2.959577in}%
\pgfsys@useobject{currentmarker}{}%
\end{pgfscope}%
\end{pgfscope}%
\begin{pgfscope}%
\definecolor{textcolor}{rgb}{0.000000,0.000000,0.000000}%
\pgfsetstrokecolor{textcolor}%
\pgfsetfillcolor{textcolor}%
\pgftext[x=0.329012in, y=2.911351in, left, base]{\color{textcolor}{\rmfamily\fontsize{10.000000}{12.000000}\selectfont\catcode`\^=\active\def^{\ifmmode\sp\else\^{}\fi}\catcode`\%=\active\def%{\%}$\mathdefault{1.0}$}}%
\end{pgfscope}%
\begin{pgfscope}%
\definecolor{textcolor}{rgb}{0.000000,0.000000,0.000000}%
\pgfsetstrokecolor{textcolor}%
\pgfsetfillcolor{textcolor}%
\pgftext[x=0.273457in,y=1.812346in,,bottom,rotate=90.000000]{\color{textcolor}{\rmfamily\fontsize{10.000000}{12.000000}\selectfont\catcode`\^=\active\def^{\ifmmode\sp\else\^{}\fi}\catcode`\%=\active\def%{\%}density}}%
\end{pgfscope}%
\begin{pgfscope}%
\pgfpathrectangle{\pgfqpoint{0.603704in}{0.549691in}}{\pgfqpoint{7.246296in}{2.525309in}}%
\pgfusepath{clip}%
\pgfsetrectcap%
\pgfsetroundjoin%
\pgfsetlinewidth{2.007500pt}%
\definecolor{currentstroke}{rgb}{1.000000,0.000000,0.000000}%
\pgfsetstrokecolor{currentstroke}%
\pgfsetdash{}{0pt}%
\pgfpathmoveto{\pgfqpoint{0.933081in}{0.549691in}}%
\pgfpathlineto{\pgfqpoint{4.220251in}{2.954747in}}%
\pgfpathlineto{\pgfqpoint{4.233453in}{2.954747in}}%
\pgfpathlineto{\pgfqpoint{7.520623in}{0.549691in}}%
\pgfpathlineto{\pgfqpoint{7.520623in}{0.549691in}}%
\pgfusepath{stroke}%
\end{pgfscope}%
\begin{pgfscope}%
\pgfsetrectcap%
\pgfsetmiterjoin%
\pgfsetlinewidth{0.803000pt}%
\definecolor{currentstroke}{rgb}{0.000000,0.000000,0.000000}%
\pgfsetstrokecolor{currentstroke}%
\pgfsetdash{}{0pt}%
\pgfpathmoveto{\pgfqpoint{0.603704in}{0.549691in}}%
\pgfpathlineto{\pgfqpoint{0.603704in}{3.075000in}}%
\pgfusepath{stroke}%
\end{pgfscope}%
\begin{pgfscope}%
\pgfsetrectcap%
\pgfsetmiterjoin%
\pgfsetlinewidth{0.803000pt}%
\definecolor{currentstroke}{rgb}{0.000000,0.000000,0.000000}%
\pgfsetstrokecolor{currentstroke}%
\pgfsetdash{}{0pt}%
\pgfpathmoveto{\pgfqpoint{7.850000in}{0.549691in}}%
\pgfpathlineto{\pgfqpoint{7.850000in}{3.075000in}}%
\pgfusepath{stroke}%
\end{pgfscope}%
\begin{pgfscope}%
\pgfsetrectcap%
\pgfsetmiterjoin%
\pgfsetlinewidth{0.803000pt}%
\definecolor{currentstroke}{rgb}{0.000000,0.000000,0.000000}%
\pgfsetstrokecolor{currentstroke}%
\pgfsetdash{}{0pt}%
\pgfpathmoveto{\pgfqpoint{0.603704in}{0.549691in}}%
\pgfpathlineto{\pgfqpoint{7.850000in}{0.549691in}}%
\pgfusepath{stroke}%
\end{pgfscope}%
\begin{pgfscope}%
\pgfsetrectcap%
\pgfsetmiterjoin%
\pgfsetlinewidth{0.803000pt}%
\definecolor{currentstroke}{rgb}{0.000000,0.000000,0.000000}%
\pgfsetstrokecolor{currentstroke}%
\pgfsetdash{}{0pt}%
\pgfpathmoveto{\pgfqpoint{0.603704in}{3.075000in}}%
\pgfpathlineto{\pgfqpoint{7.850000in}{3.075000in}}%
\pgfusepath{stroke}%
\end{pgfscope}%
\begin{pgfscope}%
\definecolor{textcolor}{rgb}{0.000000,0.000000,0.000000}%
\pgfsetstrokecolor{textcolor}%
\pgfsetfillcolor{textcolor}%
\pgftext[x=4.226852in,y=3.158333in,,base]{\color{textcolor}{\rmfamily\fontsize{12.000000}{14.400000}\selectfont\catcode`\^=\active\def^{\ifmmode\sp\else\^{}\fi}\catcode`\%=\active\def%{\%}Histogram of $Z = X + Y$}}%
\end{pgfscope}%
\begin{pgfscope}%
\pgfsetbuttcap%
\pgfsetmiterjoin%
\definecolor{currentfill}{rgb}{1.000000,1.000000,1.000000}%
\pgfsetfillcolor{currentfill}%
\pgfsetfillopacity{0.800000}%
\pgfsetlinewidth{1.003750pt}%
\definecolor{currentstroke}{rgb}{0.800000,0.800000,0.800000}%
\pgfsetstrokecolor{currentstroke}%
\pgfsetstrokeopacity{0.800000}%
\pgfsetdash{}{0pt}%
\pgfpathmoveto{\pgfqpoint{6.547144in}{2.576543in}}%
\pgfpathlineto{\pgfqpoint{7.752778in}{2.576543in}}%
\pgfpathquadraticcurveto{\pgfqpoint{7.780556in}{2.576543in}}{\pgfqpoint{7.780556in}{2.604321in}}%
\pgfpathlineto{\pgfqpoint{7.780556in}{2.977778in}}%
\pgfpathquadraticcurveto{\pgfqpoint{7.780556in}{3.005556in}}{\pgfqpoint{7.752778in}{3.005556in}}%
\pgfpathlineto{\pgfqpoint{6.547144in}{3.005556in}}%
\pgfpathquadraticcurveto{\pgfqpoint{6.519366in}{3.005556in}}{\pgfqpoint{6.519366in}{2.977778in}}%
\pgfpathlineto{\pgfqpoint{6.519366in}{2.604321in}}%
\pgfpathquadraticcurveto{\pgfqpoint{6.519366in}{2.576543in}}{\pgfqpoint{6.547144in}{2.576543in}}%
\pgfpathlineto{\pgfqpoint{6.547144in}{2.576543in}}%
\pgfpathclose%
\pgfusepath{stroke,fill}%
\end{pgfscope}%
\begin{pgfscope}%
\pgfsetbuttcap%
\pgfsetmiterjoin%
\definecolor{currentfill}{rgb}{0.121569,0.466667,0.705882}%
\pgfsetfillcolor{currentfill}%
\pgfsetfillopacity{0.700000}%
\pgfsetlinewidth{0.000000pt}%
\definecolor{currentstroke}{rgb}{0.000000,0.000000,0.000000}%
\pgfsetstrokecolor{currentstroke}%
\pgfsetstrokeopacity{0.700000}%
\pgfsetdash{}{0pt}%
\pgfpathmoveto{\pgfqpoint{6.574922in}{2.852778in}}%
\pgfpathlineto{\pgfqpoint{6.852700in}{2.852778in}}%
\pgfpathlineto{\pgfqpoint{6.852700in}{2.950000in}}%
\pgfpathlineto{\pgfqpoint{6.574922in}{2.950000in}}%
\pgfpathlineto{\pgfqpoint{6.574922in}{2.852778in}}%
\pgfpathclose%
\pgfusepath{fill}%
\end{pgfscope}%
\begin{pgfscope}%
\definecolor{textcolor}{rgb}{0.000000,0.000000,0.000000}%
\pgfsetstrokecolor{textcolor}%
\pgfsetfillcolor{textcolor}%
\pgftext[x=6.963811in,y=2.852778in,left,base]{\color{textcolor}{\rmfamily\fontsize{10.000000}{12.000000}\selectfont\catcode`\^=\active\def^{\ifmmode\sp\else\^{}\fi}\catcode`\%=\active\def%{\%}Simulated $Z$}}%
\end{pgfscope}%
\begin{pgfscope}%
\pgfsetrectcap%
\pgfsetroundjoin%
\pgfsetlinewidth{2.007500pt}%
\definecolor{currentstroke}{rgb}{1.000000,0.000000,0.000000}%
\pgfsetstrokecolor{currentstroke}%
\pgfsetdash{}{0pt}%
\pgfpathmoveto{\pgfqpoint{6.574922in}{2.707716in}}%
\pgfpathlineto{\pgfqpoint{6.713811in}{2.707716in}}%
\pgfpathlineto{\pgfqpoint{6.852700in}{2.707716in}}%
\pgfusepath{stroke}%
\end{pgfscope}%
\begin{pgfscope}%
\definecolor{textcolor}{rgb}{0.000000,0.000000,0.000000}%
\pgfsetstrokecolor{textcolor}%
\pgfsetfillcolor{textcolor}%
\pgftext[x=6.963811in,y=2.659105in,left,base]{\color{textcolor}{\rmfamily\fontsize{10.000000}{12.000000}\selectfont\catcode`\^=\active\def^{\ifmmode\sp\else\^{}\fi}\catcode`\%=\active\def%{\%}True density}}%
\end{pgfscope}%
\end{pgfpicture}%
\makeatother%
\endgroup%

  }
  \caption{\label{fig:triangular}Histograms of $X$, $Y$, and $Z=X+Y$ with theoretical density of $Z$.}
\end{figure}

Most programming languages and numerical libraries provide a built-in
function for generating draws from the uniform distribution \(\mathrm
U(0,1)\). Although the sequence returned by such a function is
deterministic (a pseudo-random number generator), it nonetheless
exhibits all the statistical properties of a truly random sequence ---
in particular the convergence property
\req{eq:simulation_convergence}.  Once we have uniform draws, a
variety of simple algorithms exist for transforming them into draws from
other distributions (normal, extreme value, gamma, etc.) For a
comprehensive treatment of these methods, the reader is referred to
the standard text by \citeasnoun{Ross12}.

Unfortunately, sampling from the posterior distribution
\req{eq:posterior} of the parameters of a choice model cannot be
achieved through simple transformations of uniform draws. Instead, it
requires more advanced simulation techniques, known as Markov chain
Monte--Carlo (MCMC) methods. As those methods are the core of Bayesian
inference, we provide a brief introduction in the next section. We
invite the interested reader to consult the literature for a more
comprehensive description (\cite{Wang_2022}, \cite{Gelman:2013aa}).

\section{Markov Chain Monte--Carlo methods}

The term \emph{Monte-Carlo} refers to the city of Monte-Carlo in the
Principality of Monaco, famous for its casino. In mathematics and
statistics, the expression ``Monte-Carlo'' is used whenever randomness is
used as a computational tool, typically to approximate integrals,
expectations, or probability distributions.

A \emph{Markov chain} is a stochastic process, that is, a sequence of
random variables, with specific mathematical properties that make their
long-run behavior analytically tractable. Under appropriate conditions
(irreducibility, aperiodicity, and [WHAT IS IT???]positive recurrence), a Markov chain
converges to a distribution called its \emph{stationary distribution}.
Intuitively, \emph{stationarity} means that, once the chain has run long
enough, the distribution of its state no longer changes over time.
Rigorously, if $X_t$ denotes the state at iteration $t$, then
stationarity means that there exists a distribution $\pi$ such that
\[
  \prob(X_{t+1} = j) = \prob(X_t = j) = \pi_j
  \qquad \text{for all states } j \text{ and all } t \text{ large enough}.
\]
Equivalently, if the chain is initiated with $X_0 \sim \pi$, then all
future states $X_t$ also follow the same distribution.

The idea behind \emph{Markov Chain Monte-Carlo} (MCMC) methods is to
construct a Markov chain whose stationary distribution is precisely the
distribution from which we wish to draw samples (for example, the
posterior distribution of model parameters). By simulating the chain for
a sufficiently large number of iterations, the generated sequence
approximates draws from the target distribution.

Formally, a Markov chain $(X_t)_{t \ge 0}$ is defined on a state space
(which may be discrete or continuous) together with a
\emph{transition probability}. For simplicity, we introduce the concept
in the discrete case; the continuous version is entirely analogous, with
probability density functions replacing probabilities, and integrals
replacing sums.

For each pair of states $i$ and $j$, the transition probability is
\begin{equation}
  \label{eq:transition}
  P_{ij} = \prob(X_{t+1} = j \mid X_t = i).
\end{equation}
A key property of Markov chains is that the transition probabilities do
not depend on the iteration index $t$. Moreover, for each state $i$,
\[
  \sum_{j} P_{ij} = 1,
\]
so that $P_{ij}$ defines a proper probability distribution over the next
state.

A stationary distribution is a vector $\pi = (\pi_j)_j$ satisfying the
system
\begin{equation}
  \label{eq:stationary-1}
  \pi_j = \sum_i \pi_i P_{ij} \qquad \text{for all states } j,
\end{equation}
with the normalization condition
\begin{equation}
  \label{eq:stationary-2}
  \sum_j \pi_j = 1.
\end{equation}
Equation~\eqref{eq:stationary-1} states that if $X_t$ has distribution
$\pi$, then $X_{t+1}$ also has distribution $\pi$. Thus the chain is in
equilibrium.

In many MCMC algorithms, the Markov chains used to generate samples
satisfy an additional property known as \emph{time reversibility}. A
chain is time reversible with respect to a distribution $\pi$ if
\begin{equation}
  \label{eq:time_reversible}
  \pi_i P_{ij} = \pi_j P_{ji} \qquad \text{for all } i \neq j.
\end{equation}
This condition is also known as \emph{detailed balance}, and implies
that $\pi$ is stationary. Indeed, summing~\eqref{eq:time_reversible}
over all $i$ directly yields \eqref{eq:stationary-1}. Many classical
MCMC algorithms, such as the Metropolis--Hastings method, are
explicitly designed to satisfy detailed balance with respect to the
target distribution.

We illustrate the notion of stationary and time-reversible Markov chains
with a simple example involving customer engagement on an online service
(e.g., a subscription-based platform).

We consider a single user observed once per day. On any given day,
the user is in exactly one of the following three engagement states:
\begin{itemize}
  \item State~1: low engagement (rarely logs in, uses very few features),
  \item State~2: medium engagement (uses the service somewhat regularly),
  \item State~3: high engagement (uses the service intensively and frequently).
\end{itemize}
We assume that the evolution of the user's engagement from day to day
can be modeled as a homogeneous Markov chain $(X_t)_{t \geq 0}$ taking
values in $\{1,2,3\}$, with the following transition matrix:
\[
P
=
\begin{pmatrix}
0.7 & 0.2 & 0.1 \\
0.2 & 0.5 & 0.3 \\
0.1 & 0.3 & 0.6
\end{pmatrix}.
\]
Each entry $P_{ij}$ denotes the probability that the user moves from
state $i$ on day $t$ to state $j$ on day $t+1$.
The entries of $P$ can be read as follows:
\begin{itemize}
  \item From low engagement (state~1):
    \begin{itemize}
      \item the user stays low-engagement the next day with probability $0.7$,
      \item moves up to medium engagement with probability $0.2$,
      \item jumps directly to high engagement with probability $0.1$.
    \end{itemize}
  \item From medium engagement (state~2):
    \begin{itemize}
      \item the user drops to low engagement with probability $0.2$,
      \item remains at medium engagement with probability $0.5$,
      \item increases to high engagement with probability $0.3$.
    \end{itemize}
  \item From high engagement (state~3):
    \begin{itemize}
      \item the user cools down to medium engagement with probability $0.3$,
      \item remains highly engaged with probability $0.6$,
      \item drops directly to low engagement with probability $0.1$.
    \end{itemize}
\end{itemize}

It is easy to verify that the Markov chain admits the uniform stationary distribution
\[
\pi = \bigl(\pi_1, \pi_2, \pi_3\bigr)
= \left( \tfrac{1}{3}, \tfrac{1}{3}, \tfrac{1}{3} \right).
\]
The Markov chain is also time-reversible with respect to
$\pi$. This follows from the fact that $\pi_i = \pi_j = 1/3$ for all $i,j$, and that $P$ is symmetric

The behavior of the Markov chain introduced above is illustrated in
Figure~\ref{fig:markov}.  The figure displays the empirical frequency
of each state as the simulation evolves over time. At the beginning of
the run, these empirical frequencies fluctuate widely and do not yet
reflect the target distribution.  As the number of iterations
increases, however, the proportions stabilize and gradually approach
the theoretical stationary distribution $\pi = (1/3,\,1/3,\,1/3)$
derived earlier.  A crucial practical implication is that the draws
generated during the early iterations---before the chain has
approached stationarity---should not be used as representative samples
from the target distribution.  Only after the chain has ``settled''
near equilibrium do the simulated states behave as valid draws from
the desired stationary distribution.  Typically, in this example, we
would simply discard all the 1000 draws displayed in
Figure~\ref{fig:markov}, and start using the chain to generate more
draws (Figure~\ref{fig:markov_2} illustrates the chain from step 1000 to step 2000).

\begin{figure}
  \centering
  \resizebox{\textwidth}{!}{%
    %% Creator: Matplotlib, PGF backend
%%
%% To include the figure in your LaTeX document, write
%%   \input{<filename>.pgf}
%%
%% Make sure the required packages are loaded in your preamble
%%   \usepackage{pgf}
%%
%% Also ensure that all the required font packages are loaded; for instance,
%% the lmodern package is sometimes necessary when using math font.
%%   \usepackage{lmodern}
%%
%% Figures using additional raster images can only be included by \input if
%% they are in the same directory as the main LaTeX file. For loading figures
%% from other directories you can use the `import` package
%%   \usepackage{import}
%%
%% and then include the figures with
%%   \import{<path to file>}{<filename>.pgf}
%%
%% Matplotlib used the following preamble
%%   \def\mathdefault#1{#1}
%%   \everymath=\expandafter{\the\everymath\displaystyle}
%%   \IfFileExists{scrextend.sty}{
%%     \usepackage[fontsize=10.000000pt]{scrextend}
%%   }{
%%     \renewcommand{\normalsize}{\fontsize{10.000000}{12.000000}\selectfont}
%%     \normalsize
%%   }
%%   
%%   \ifdefined\pdftexversion\else  % non-pdftex case.
%%     \usepackage{fontspec}
%%     \setmainfont{DejaVuSerif.ttf}[Path=\detokenize{/Library/Frameworks/Python.framework/Versions/3.13/lib/python3.13/site-packages/matplotlib/mpl-data/fonts/ttf/}]
%%     \setsansfont{DejaVuSans.ttf}[Path=\detokenize{/Library/Frameworks/Python.framework/Versions/3.13/lib/python3.13/site-packages/matplotlib/mpl-data/fonts/ttf/}]
%%     \setmonofont{DejaVuSansMono.ttf}[Path=\detokenize{/Library/Frameworks/Python.framework/Versions/3.13/lib/python3.13/site-packages/matplotlib/mpl-data/fonts/ttf/}]
%%   \fi
%%   \makeatletter\@ifpackageloaded{underscore}{}{\usepackage[strings]{underscore}}\makeatother
%%
\begingroup%
\makeatletter%
\begin{pgfpicture}%
\pgfpathrectangle{\pgfpointorigin}{\pgfqpoint{7.000000in}{4.000000in}}%
\pgfusepath{use as bounding box, clip}%
\begin{pgfscope}%
\pgfsetbuttcap%
\pgfsetmiterjoin%
\definecolor{currentfill}{rgb}{1.000000,1.000000,1.000000}%
\pgfsetfillcolor{currentfill}%
\pgfsetlinewidth{0.000000pt}%
\definecolor{currentstroke}{rgb}{1.000000,1.000000,1.000000}%
\pgfsetstrokecolor{currentstroke}%
\pgfsetdash{}{0pt}%
\pgfpathmoveto{\pgfqpoint{0.000000in}{0.000000in}}%
\pgfpathlineto{\pgfqpoint{7.000000in}{0.000000in}}%
\pgfpathlineto{\pgfqpoint{7.000000in}{4.000000in}}%
\pgfpathlineto{\pgfqpoint{0.000000in}{4.000000in}}%
\pgfpathlineto{\pgfqpoint{0.000000in}{0.000000in}}%
\pgfpathclose%
\pgfusepath{fill}%
\end{pgfscope}%
\begin{pgfscope}%
\pgfsetbuttcap%
\pgfsetmiterjoin%
\definecolor{currentfill}{rgb}{1.000000,1.000000,1.000000}%
\pgfsetfillcolor{currentfill}%
\pgfsetlinewidth{0.000000pt}%
\definecolor{currentstroke}{rgb}{0.000000,0.000000,0.000000}%
\pgfsetstrokecolor{currentstroke}%
\pgfsetstrokeopacity{0.000000}%
\pgfsetdash{}{0pt}%
\pgfpathmoveto{\pgfqpoint{0.603704in}{0.549691in}}%
\pgfpathlineto{\pgfqpoint{6.711111in}{0.549691in}}%
\pgfpathlineto{\pgfqpoint{6.711111in}{3.650926in}}%
\pgfpathlineto{\pgfqpoint{0.603704in}{3.650926in}}%
\pgfpathlineto{\pgfqpoint{0.603704in}{0.549691in}}%
\pgfpathclose%
\pgfusepath{fill}%
\end{pgfscope}%
\begin{pgfscope}%
\pgfpathrectangle{\pgfqpoint{0.603704in}{0.549691in}}{\pgfqpoint{6.107407in}{3.101235in}}%
\pgfusepath{clip}%
\pgfsetrectcap%
\pgfsetroundjoin%
\pgfsetlinewidth{0.803000pt}%
\definecolor{currentstroke}{rgb}{0.690196,0.690196,0.690196}%
\pgfsetstrokecolor{currentstroke}%
\pgfsetstrokeopacity{0.300000}%
\pgfsetdash{}{0pt}%
\pgfpathmoveto{\pgfqpoint{1.820295in}{0.549691in}}%
\pgfpathlineto{\pgfqpoint{1.820295in}{3.650926in}}%
\pgfusepath{stroke}%
\end{pgfscope}%
\begin{pgfscope}%
\pgfsetbuttcap%
\pgfsetroundjoin%
\definecolor{currentfill}{rgb}{0.000000,0.000000,0.000000}%
\pgfsetfillcolor{currentfill}%
\pgfsetlinewidth{0.803000pt}%
\definecolor{currentstroke}{rgb}{0.000000,0.000000,0.000000}%
\pgfsetstrokecolor{currentstroke}%
\pgfsetdash{}{0pt}%
\pgfsys@defobject{currentmarker}{\pgfqpoint{0.000000in}{-0.048611in}}{\pgfqpoint{0.000000in}{0.000000in}}{%
\pgfpathmoveto{\pgfqpoint{0.000000in}{0.000000in}}%
\pgfpathlineto{\pgfqpoint{0.000000in}{-0.048611in}}%
\pgfusepath{stroke,fill}%
}%
\begin{pgfscope}%
\pgfsys@transformshift{1.820295in}{0.549691in}%
\pgfsys@useobject{currentmarker}{}%
\end{pgfscope}%
\end{pgfscope}%
\begin{pgfscope}%
\definecolor{textcolor}{rgb}{0.000000,0.000000,0.000000}%
\pgfsetstrokecolor{textcolor}%
\pgfsetfillcolor{textcolor}%
\pgftext[x=1.820295in,y=0.452469in,,top]{\color{textcolor}{\rmfamily\fontsize{10.000000}{12.000000}\selectfont\catcode`\^=\active\def^{\ifmmode\sp\else\^{}\fi}\catcode`\%=\active\def%{\%}$\mathdefault{200}$}}%
\end{pgfscope}%
\begin{pgfscope}%
\pgfpathrectangle{\pgfqpoint{0.603704in}{0.549691in}}{\pgfqpoint{6.107407in}{3.101235in}}%
\pgfusepath{clip}%
\pgfsetrectcap%
\pgfsetroundjoin%
\pgfsetlinewidth{0.803000pt}%
\definecolor{currentstroke}{rgb}{0.690196,0.690196,0.690196}%
\pgfsetstrokecolor{currentstroke}%
\pgfsetstrokeopacity{0.300000}%
\pgfsetdash{}{0pt}%
\pgfpathmoveto{\pgfqpoint{3.042999in}{0.549691in}}%
\pgfpathlineto{\pgfqpoint{3.042999in}{3.650926in}}%
\pgfusepath{stroke}%
\end{pgfscope}%
\begin{pgfscope}%
\pgfsetbuttcap%
\pgfsetroundjoin%
\definecolor{currentfill}{rgb}{0.000000,0.000000,0.000000}%
\pgfsetfillcolor{currentfill}%
\pgfsetlinewidth{0.803000pt}%
\definecolor{currentstroke}{rgb}{0.000000,0.000000,0.000000}%
\pgfsetstrokecolor{currentstroke}%
\pgfsetdash{}{0pt}%
\pgfsys@defobject{currentmarker}{\pgfqpoint{0.000000in}{-0.048611in}}{\pgfqpoint{0.000000in}{0.000000in}}{%
\pgfpathmoveto{\pgfqpoint{0.000000in}{0.000000in}}%
\pgfpathlineto{\pgfqpoint{0.000000in}{-0.048611in}}%
\pgfusepath{stroke,fill}%
}%
\begin{pgfscope}%
\pgfsys@transformshift{3.042999in}{0.549691in}%
\pgfsys@useobject{currentmarker}{}%
\end{pgfscope}%
\end{pgfscope}%
\begin{pgfscope}%
\definecolor{textcolor}{rgb}{0.000000,0.000000,0.000000}%
\pgfsetstrokecolor{textcolor}%
\pgfsetfillcolor{textcolor}%
\pgftext[x=3.042999in,y=0.452469in,,top]{\color{textcolor}{\rmfamily\fontsize{10.000000}{12.000000}\selectfont\catcode`\^=\active\def^{\ifmmode\sp\else\^{}\fi}\catcode`\%=\active\def%{\%}$\mathdefault{400}$}}%
\end{pgfscope}%
\begin{pgfscope}%
\pgfpathrectangle{\pgfqpoint{0.603704in}{0.549691in}}{\pgfqpoint{6.107407in}{3.101235in}}%
\pgfusepath{clip}%
\pgfsetrectcap%
\pgfsetroundjoin%
\pgfsetlinewidth{0.803000pt}%
\definecolor{currentstroke}{rgb}{0.690196,0.690196,0.690196}%
\pgfsetstrokecolor{currentstroke}%
\pgfsetstrokeopacity{0.300000}%
\pgfsetdash{}{0pt}%
\pgfpathmoveto{\pgfqpoint{4.265703in}{0.549691in}}%
\pgfpathlineto{\pgfqpoint{4.265703in}{3.650926in}}%
\pgfusepath{stroke}%
\end{pgfscope}%
\begin{pgfscope}%
\pgfsetbuttcap%
\pgfsetroundjoin%
\definecolor{currentfill}{rgb}{0.000000,0.000000,0.000000}%
\pgfsetfillcolor{currentfill}%
\pgfsetlinewidth{0.803000pt}%
\definecolor{currentstroke}{rgb}{0.000000,0.000000,0.000000}%
\pgfsetstrokecolor{currentstroke}%
\pgfsetdash{}{0pt}%
\pgfsys@defobject{currentmarker}{\pgfqpoint{0.000000in}{-0.048611in}}{\pgfqpoint{0.000000in}{0.000000in}}{%
\pgfpathmoveto{\pgfqpoint{0.000000in}{0.000000in}}%
\pgfpathlineto{\pgfqpoint{0.000000in}{-0.048611in}}%
\pgfusepath{stroke,fill}%
}%
\begin{pgfscope}%
\pgfsys@transformshift{4.265703in}{0.549691in}%
\pgfsys@useobject{currentmarker}{}%
\end{pgfscope}%
\end{pgfscope}%
\begin{pgfscope}%
\definecolor{textcolor}{rgb}{0.000000,0.000000,0.000000}%
\pgfsetstrokecolor{textcolor}%
\pgfsetfillcolor{textcolor}%
\pgftext[x=4.265703in,y=0.452469in,,top]{\color{textcolor}{\rmfamily\fontsize{10.000000}{12.000000}\selectfont\catcode`\^=\active\def^{\ifmmode\sp\else\^{}\fi}\catcode`\%=\active\def%{\%}$\mathdefault{600}$}}%
\end{pgfscope}%
\begin{pgfscope}%
\pgfpathrectangle{\pgfqpoint{0.603704in}{0.549691in}}{\pgfqpoint{6.107407in}{3.101235in}}%
\pgfusepath{clip}%
\pgfsetrectcap%
\pgfsetroundjoin%
\pgfsetlinewidth{0.803000pt}%
\definecolor{currentstroke}{rgb}{0.690196,0.690196,0.690196}%
\pgfsetstrokecolor{currentstroke}%
\pgfsetstrokeopacity{0.300000}%
\pgfsetdash{}{0pt}%
\pgfpathmoveto{\pgfqpoint{5.488407in}{0.549691in}}%
\pgfpathlineto{\pgfqpoint{5.488407in}{3.650926in}}%
\pgfusepath{stroke}%
\end{pgfscope}%
\begin{pgfscope}%
\pgfsetbuttcap%
\pgfsetroundjoin%
\definecolor{currentfill}{rgb}{0.000000,0.000000,0.000000}%
\pgfsetfillcolor{currentfill}%
\pgfsetlinewidth{0.803000pt}%
\definecolor{currentstroke}{rgb}{0.000000,0.000000,0.000000}%
\pgfsetstrokecolor{currentstroke}%
\pgfsetdash{}{0pt}%
\pgfsys@defobject{currentmarker}{\pgfqpoint{0.000000in}{-0.048611in}}{\pgfqpoint{0.000000in}{0.000000in}}{%
\pgfpathmoveto{\pgfqpoint{0.000000in}{0.000000in}}%
\pgfpathlineto{\pgfqpoint{0.000000in}{-0.048611in}}%
\pgfusepath{stroke,fill}%
}%
\begin{pgfscope}%
\pgfsys@transformshift{5.488407in}{0.549691in}%
\pgfsys@useobject{currentmarker}{}%
\end{pgfscope}%
\end{pgfscope}%
\begin{pgfscope}%
\definecolor{textcolor}{rgb}{0.000000,0.000000,0.000000}%
\pgfsetstrokecolor{textcolor}%
\pgfsetfillcolor{textcolor}%
\pgftext[x=5.488407in,y=0.452469in,,top]{\color{textcolor}{\rmfamily\fontsize{10.000000}{12.000000}\selectfont\catcode`\^=\active\def^{\ifmmode\sp\else\^{}\fi}\catcode`\%=\active\def%{\%}$\mathdefault{800}$}}%
\end{pgfscope}%
\begin{pgfscope}%
\pgfpathrectangle{\pgfqpoint{0.603704in}{0.549691in}}{\pgfqpoint{6.107407in}{3.101235in}}%
\pgfusepath{clip}%
\pgfsetrectcap%
\pgfsetroundjoin%
\pgfsetlinewidth{0.803000pt}%
\definecolor{currentstroke}{rgb}{0.690196,0.690196,0.690196}%
\pgfsetstrokecolor{currentstroke}%
\pgfsetstrokeopacity{0.300000}%
\pgfsetdash{}{0pt}%
\pgfpathmoveto{\pgfqpoint{6.711111in}{0.549691in}}%
\pgfpathlineto{\pgfqpoint{6.711111in}{3.650926in}}%
\pgfusepath{stroke}%
\end{pgfscope}%
\begin{pgfscope}%
\pgfsetbuttcap%
\pgfsetroundjoin%
\definecolor{currentfill}{rgb}{0.000000,0.000000,0.000000}%
\pgfsetfillcolor{currentfill}%
\pgfsetlinewidth{0.803000pt}%
\definecolor{currentstroke}{rgb}{0.000000,0.000000,0.000000}%
\pgfsetstrokecolor{currentstroke}%
\pgfsetdash{}{0pt}%
\pgfsys@defobject{currentmarker}{\pgfqpoint{0.000000in}{-0.048611in}}{\pgfqpoint{0.000000in}{0.000000in}}{%
\pgfpathmoveto{\pgfqpoint{0.000000in}{0.000000in}}%
\pgfpathlineto{\pgfqpoint{0.000000in}{-0.048611in}}%
\pgfusepath{stroke,fill}%
}%
\begin{pgfscope}%
\pgfsys@transformshift{6.711111in}{0.549691in}%
\pgfsys@useobject{currentmarker}{}%
\end{pgfscope}%
\end{pgfscope}%
\begin{pgfscope}%
\definecolor{textcolor}{rgb}{0.000000,0.000000,0.000000}%
\pgfsetstrokecolor{textcolor}%
\pgfsetfillcolor{textcolor}%
\pgftext[x=6.711111in,y=0.452469in,,top]{\color{textcolor}{\rmfamily\fontsize{10.000000}{12.000000}\selectfont\catcode`\^=\active\def^{\ifmmode\sp\else\^{}\fi}\catcode`\%=\active\def%{\%}$\mathdefault{1000}$}}%
\end{pgfscope}%
\begin{pgfscope}%
\definecolor{textcolor}{rgb}{0.000000,0.000000,0.000000}%
\pgfsetstrokecolor{textcolor}%
\pgfsetfillcolor{textcolor}%
\pgftext[x=3.657407in,y=0.273457in,,top]{\color{textcolor}{\rmfamily\fontsize{10.000000}{12.000000}\selectfont\catcode`\^=\active\def^{\ifmmode\sp\else\^{}\fi}\catcode`\%=\active\def%{\%}Time step $t$}}%
\end{pgfscope}%
\begin{pgfscope}%
\pgfpathrectangle{\pgfqpoint{0.603704in}{0.549691in}}{\pgfqpoint{6.107407in}{3.101235in}}%
\pgfusepath{clip}%
\pgfsetrectcap%
\pgfsetroundjoin%
\pgfsetlinewidth{0.803000pt}%
\definecolor{currentstroke}{rgb}{0.690196,0.690196,0.690196}%
\pgfsetstrokecolor{currentstroke}%
\pgfsetstrokeopacity{0.300000}%
\pgfsetdash{}{0pt}%
\pgfpathmoveto{\pgfqpoint{0.603704in}{0.549691in}}%
\pgfpathlineto{\pgfqpoint{6.711111in}{0.549691in}}%
\pgfusepath{stroke}%
\end{pgfscope}%
\begin{pgfscope}%
\pgfsetbuttcap%
\pgfsetroundjoin%
\definecolor{currentfill}{rgb}{0.000000,0.000000,0.000000}%
\pgfsetfillcolor{currentfill}%
\pgfsetlinewidth{0.803000pt}%
\definecolor{currentstroke}{rgb}{0.000000,0.000000,0.000000}%
\pgfsetstrokecolor{currentstroke}%
\pgfsetdash{}{0pt}%
\pgfsys@defobject{currentmarker}{\pgfqpoint{-0.048611in}{0.000000in}}{\pgfqpoint{-0.000000in}{0.000000in}}{%
\pgfpathmoveto{\pgfqpoint{-0.000000in}{0.000000in}}%
\pgfpathlineto{\pgfqpoint{-0.048611in}{0.000000in}}%
\pgfusepath{stroke,fill}%
}%
\begin{pgfscope}%
\pgfsys@transformshift{0.603704in}{0.549691in}%
\pgfsys@useobject{currentmarker}{}%
\end{pgfscope}%
\end{pgfscope}%
\begin{pgfscope}%
\definecolor{textcolor}{rgb}{0.000000,0.000000,0.000000}%
\pgfsetstrokecolor{textcolor}%
\pgfsetfillcolor{textcolor}%
\pgftext[x=0.329012in, y=0.501466in, left, base]{\color{textcolor}{\rmfamily\fontsize{10.000000}{12.000000}\selectfont\catcode`\^=\active\def^{\ifmmode\sp\else\^{}\fi}\catcode`\%=\active\def%{\%}$\mathdefault{0.0}$}}%
\end{pgfscope}%
\begin{pgfscope}%
\pgfpathrectangle{\pgfqpoint{0.603704in}{0.549691in}}{\pgfqpoint{6.107407in}{3.101235in}}%
\pgfusepath{clip}%
\pgfsetrectcap%
\pgfsetroundjoin%
\pgfsetlinewidth{0.803000pt}%
\definecolor{currentstroke}{rgb}{0.690196,0.690196,0.690196}%
\pgfsetstrokecolor{currentstroke}%
\pgfsetstrokeopacity{0.300000}%
\pgfsetdash{}{0pt}%
\pgfpathmoveto{\pgfqpoint{0.603704in}{1.169938in}}%
\pgfpathlineto{\pgfqpoint{6.711111in}{1.169938in}}%
\pgfusepath{stroke}%
\end{pgfscope}%
\begin{pgfscope}%
\pgfsetbuttcap%
\pgfsetroundjoin%
\definecolor{currentfill}{rgb}{0.000000,0.000000,0.000000}%
\pgfsetfillcolor{currentfill}%
\pgfsetlinewidth{0.803000pt}%
\definecolor{currentstroke}{rgb}{0.000000,0.000000,0.000000}%
\pgfsetstrokecolor{currentstroke}%
\pgfsetdash{}{0pt}%
\pgfsys@defobject{currentmarker}{\pgfqpoint{-0.048611in}{0.000000in}}{\pgfqpoint{-0.000000in}{0.000000in}}{%
\pgfpathmoveto{\pgfqpoint{-0.000000in}{0.000000in}}%
\pgfpathlineto{\pgfqpoint{-0.048611in}{0.000000in}}%
\pgfusepath{stroke,fill}%
}%
\begin{pgfscope}%
\pgfsys@transformshift{0.603704in}{1.169938in}%
\pgfsys@useobject{currentmarker}{}%
\end{pgfscope}%
\end{pgfscope}%
\begin{pgfscope}%
\definecolor{textcolor}{rgb}{0.000000,0.000000,0.000000}%
\pgfsetstrokecolor{textcolor}%
\pgfsetfillcolor{textcolor}%
\pgftext[x=0.329012in, y=1.121713in, left, base]{\color{textcolor}{\rmfamily\fontsize{10.000000}{12.000000}\selectfont\catcode`\^=\active\def^{\ifmmode\sp\else\^{}\fi}\catcode`\%=\active\def%{\%}$\mathdefault{0.2}$}}%
\end{pgfscope}%
\begin{pgfscope}%
\pgfpathrectangle{\pgfqpoint{0.603704in}{0.549691in}}{\pgfqpoint{6.107407in}{3.101235in}}%
\pgfusepath{clip}%
\pgfsetrectcap%
\pgfsetroundjoin%
\pgfsetlinewidth{0.803000pt}%
\definecolor{currentstroke}{rgb}{0.690196,0.690196,0.690196}%
\pgfsetstrokecolor{currentstroke}%
\pgfsetstrokeopacity{0.300000}%
\pgfsetdash{}{0pt}%
\pgfpathmoveto{\pgfqpoint{0.603704in}{1.790185in}}%
\pgfpathlineto{\pgfqpoint{6.711111in}{1.790185in}}%
\pgfusepath{stroke}%
\end{pgfscope}%
\begin{pgfscope}%
\pgfsetbuttcap%
\pgfsetroundjoin%
\definecolor{currentfill}{rgb}{0.000000,0.000000,0.000000}%
\pgfsetfillcolor{currentfill}%
\pgfsetlinewidth{0.803000pt}%
\definecolor{currentstroke}{rgb}{0.000000,0.000000,0.000000}%
\pgfsetstrokecolor{currentstroke}%
\pgfsetdash{}{0pt}%
\pgfsys@defobject{currentmarker}{\pgfqpoint{-0.048611in}{0.000000in}}{\pgfqpoint{-0.000000in}{0.000000in}}{%
\pgfpathmoveto{\pgfqpoint{-0.000000in}{0.000000in}}%
\pgfpathlineto{\pgfqpoint{-0.048611in}{0.000000in}}%
\pgfusepath{stroke,fill}%
}%
\begin{pgfscope}%
\pgfsys@transformshift{0.603704in}{1.790185in}%
\pgfsys@useobject{currentmarker}{}%
\end{pgfscope}%
\end{pgfscope}%
\begin{pgfscope}%
\definecolor{textcolor}{rgb}{0.000000,0.000000,0.000000}%
\pgfsetstrokecolor{textcolor}%
\pgfsetfillcolor{textcolor}%
\pgftext[x=0.329012in, y=1.741960in, left, base]{\color{textcolor}{\rmfamily\fontsize{10.000000}{12.000000}\selectfont\catcode`\^=\active\def^{\ifmmode\sp\else\^{}\fi}\catcode`\%=\active\def%{\%}$\mathdefault{0.4}$}}%
\end{pgfscope}%
\begin{pgfscope}%
\pgfpathrectangle{\pgfqpoint{0.603704in}{0.549691in}}{\pgfqpoint{6.107407in}{3.101235in}}%
\pgfusepath{clip}%
\pgfsetrectcap%
\pgfsetroundjoin%
\pgfsetlinewidth{0.803000pt}%
\definecolor{currentstroke}{rgb}{0.690196,0.690196,0.690196}%
\pgfsetstrokecolor{currentstroke}%
\pgfsetstrokeopacity{0.300000}%
\pgfsetdash{}{0pt}%
\pgfpathmoveto{\pgfqpoint{0.603704in}{2.410432in}}%
\pgfpathlineto{\pgfqpoint{6.711111in}{2.410432in}}%
\pgfusepath{stroke}%
\end{pgfscope}%
\begin{pgfscope}%
\pgfsetbuttcap%
\pgfsetroundjoin%
\definecolor{currentfill}{rgb}{0.000000,0.000000,0.000000}%
\pgfsetfillcolor{currentfill}%
\pgfsetlinewidth{0.803000pt}%
\definecolor{currentstroke}{rgb}{0.000000,0.000000,0.000000}%
\pgfsetstrokecolor{currentstroke}%
\pgfsetdash{}{0pt}%
\pgfsys@defobject{currentmarker}{\pgfqpoint{-0.048611in}{0.000000in}}{\pgfqpoint{-0.000000in}{0.000000in}}{%
\pgfpathmoveto{\pgfqpoint{-0.000000in}{0.000000in}}%
\pgfpathlineto{\pgfqpoint{-0.048611in}{0.000000in}}%
\pgfusepath{stroke,fill}%
}%
\begin{pgfscope}%
\pgfsys@transformshift{0.603704in}{2.410432in}%
\pgfsys@useobject{currentmarker}{}%
\end{pgfscope}%
\end{pgfscope}%
\begin{pgfscope}%
\definecolor{textcolor}{rgb}{0.000000,0.000000,0.000000}%
\pgfsetstrokecolor{textcolor}%
\pgfsetfillcolor{textcolor}%
\pgftext[x=0.329012in, y=2.362207in, left, base]{\color{textcolor}{\rmfamily\fontsize{10.000000}{12.000000}\selectfont\catcode`\^=\active\def^{\ifmmode\sp\else\^{}\fi}\catcode`\%=\active\def%{\%}$\mathdefault{0.6}$}}%
\end{pgfscope}%
\begin{pgfscope}%
\pgfpathrectangle{\pgfqpoint{0.603704in}{0.549691in}}{\pgfqpoint{6.107407in}{3.101235in}}%
\pgfusepath{clip}%
\pgfsetrectcap%
\pgfsetroundjoin%
\pgfsetlinewidth{0.803000pt}%
\definecolor{currentstroke}{rgb}{0.690196,0.690196,0.690196}%
\pgfsetstrokecolor{currentstroke}%
\pgfsetstrokeopacity{0.300000}%
\pgfsetdash{}{0pt}%
\pgfpathmoveto{\pgfqpoint{0.603704in}{3.030679in}}%
\pgfpathlineto{\pgfqpoint{6.711111in}{3.030679in}}%
\pgfusepath{stroke}%
\end{pgfscope}%
\begin{pgfscope}%
\pgfsetbuttcap%
\pgfsetroundjoin%
\definecolor{currentfill}{rgb}{0.000000,0.000000,0.000000}%
\pgfsetfillcolor{currentfill}%
\pgfsetlinewidth{0.803000pt}%
\definecolor{currentstroke}{rgb}{0.000000,0.000000,0.000000}%
\pgfsetstrokecolor{currentstroke}%
\pgfsetdash{}{0pt}%
\pgfsys@defobject{currentmarker}{\pgfqpoint{-0.048611in}{0.000000in}}{\pgfqpoint{-0.000000in}{0.000000in}}{%
\pgfpathmoveto{\pgfqpoint{-0.000000in}{0.000000in}}%
\pgfpathlineto{\pgfqpoint{-0.048611in}{0.000000in}}%
\pgfusepath{stroke,fill}%
}%
\begin{pgfscope}%
\pgfsys@transformshift{0.603704in}{3.030679in}%
\pgfsys@useobject{currentmarker}{}%
\end{pgfscope}%
\end{pgfscope}%
\begin{pgfscope}%
\definecolor{textcolor}{rgb}{0.000000,0.000000,0.000000}%
\pgfsetstrokecolor{textcolor}%
\pgfsetfillcolor{textcolor}%
\pgftext[x=0.329012in, y=2.982454in, left, base]{\color{textcolor}{\rmfamily\fontsize{10.000000}{12.000000}\selectfont\catcode`\^=\active\def^{\ifmmode\sp\else\^{}\fi}\catcode`\%=\active\def%{\%}$\mathdefault{0.8}$}}%
\end{pgfscope}%
\begin{pgfscope}%
\pgfpathrectangle{\pgfqpoint{0.603704in}{0.549691in}}{\pgfqpoint{6.107407in}{3.101235in}}%
\pgfusepath{clip}%
\pgfsetrectcap%
\pgfsetroundjoin%
\pgfsetlinewidth{0.803000pt}%
\definecolor{currentstroke}{rgb}{0.690196,0.690196,0.690196}%
\pgfsetstrokecolor{currentstroke}%
\pgfsetstrokeopacity{0.300000}%
\pgfsetdash{}{0pt}%
\pgfpathmoveto{\pgfqpoint{0.603704in}{3.650926in}}%
\pgfpathlineto{\pgfqpoint{6.711111in}{3.650926in}}%
\pgfusepath{stroke}%
\end{pgfscope}%
\begin{pgfscope}%
\pgfsetbuttcap%
\pgfsetroundjoin%
\definecolor{currentfill}{rgb}{0.000000,0.000000,0.000000}%
\pgfsetfillcolor{currentfill}%
\pgfsetlinewidth{0.803000pt}%
\definecolor{currentstroke}{rgb}{0.000000,0.000000,0.000000}%
\pgfsetstrokecolor{currentstroke}%
\pgfsetdash{}{0pt}%
\pgfsys@defobject{currentmarker}{\pgfqpoint{-0.048611in}{0.000000in}}{\pgfqpoint{-0.000000in}{0.000000in}}{%
\pgfpathmoveto{\pgfqpoint{-0.000000in}{0.000000in}}%
\pgfpathlineto{\pgfqpoint{-0.048611in}{0.000000in}}%
\pgfusepath{stroke,fill}%
}%
\begin{pgfscope}%
\pgfsys@transformshift{0.603704in}{3.650926in}%
\pgfsys@useobject{currentmarker}{}%
\end{pgfscope}%
\end{pgfscope}%
\begin{pgfscope}%
\definecolor{textcolor}{rgb}{0.000000,0.000000,0.000000}%
\pgfsetstrokecolor{textcolor}%
\pgfsetfillcolor{textcolor}%
\pgftext[x=0.329012in, y=3.602701in, left, base]{\color{textcolor}{\rmfamily\fontsize{10.000000}{12.000000}\selectfont\catcode`\^=\active\def^{\ifmmode\sp\else\^{}\fi}\catcode`\%=\active\def%{\%}$\mathdefault{1.0}$}}%
\end{pgfscope}%
\begin{pgfscope}%
\definecolor{textcolor}{rgb}{0.000000,0.000000,0.000000}%
\pgfsetstrokecolor{textcolor}%
\pgfsetfillcolor{textcolor}%
\pgftext[x=0.273457in,y=2.100309in,,bottom,rotate=90.000000]{\color{textcolor}{\rmfamily\fontsize{10.000000}{12.000000}\selectfont\catcode`\^=\active\def^{\ifmmode\sp\else\^{}\fi}\catcode`\%=\active\def%{\%}Empirical frequency}}%
\end{pgfscope}%
\begin{pgfscope}%
\pgfpathrectangle{\pgfqpoint{0.603704in}{0.549691in}}{\pgfqpoint{6.107407in}{3.101235in}}%
\pgfusepath{clip}%
\pgfsetrectcap%
\pgfsetroundjoin%
\pgfsetlinewidth{1.505625pt}%
\definecolor{currentstroke}{rgb}{0.121569,0.466667,0.705882}%
\pgfsetstrokecolor{currentstroke}%
\pgfsetdash{}{0pt}%
\pgfpathmoveto{\pgfqpoint{0.603704in}{3.650926in}}%
\pgfpathlineto{\pgfqpoint{0.634272in}{3.650926in}}%
\pgfpathlineto{\pgfqpoint{0.640385in}{3.207893in}}%
\pgfpathlineto{\pgfqpoint{0.646499in}{2.875617in}}%
\pgfpathlineto{\pgfqpoint{0.652612in}{2.617181in}}%
\pgfpathlineto{\pgfqpoint{0.664839in}{2.241274in}}%
\pgfpathlineto{\pgfqpoint{0.677066in}{1.981030in}}%
\pgfpathlineto{\pgfqpoint{0.689293in}{1.790185in}}%
\pgfpathlineto{\pgfqpoint{0.701520in}{1.644245in}}%
\pgfpathlineto{\pgfqpoint{0.713748in}{1.529029in}}%
\pgfpathlineto{\pgfqpoint{0.725975in}{1.435758in}}%
\pgfpathlineto{\pgfqpoint{0.732088in}{1.395483in}}%
\pgfpathlineto{\pgfqpoint{0.744315in}{1.583436in}}%
\pgfpathlineto{\pgfqpoint{0.756542in}{1.742474in}}%
\pgfpathlineto{\pgfqpoint{0.768769in}{1.878792in}}%
\pgfpathlineto{\pgfqpoint{0.780996in}{1.790185in}}%
\pgfpathlineto{\pgfqpoint{0.799337in}{1.677413in}}%
\pgfpathlineto{\pgfqpoint{0.817677in}{1.583436in}}%
\pgfpathlineto{\pgfqpoint{0.836018in}{1.503917in}}%
\pgfpathlineto{\pgfqpoint{0.848245in}{1.457370in}}%
\pgfpathlineto{\pgfqpoint{0.866586in}{1.606930in}}%
\pgfpathlineto{\pgfqpoint{0.872699in}{1.583436in}}%
\pgfpathlineto{\pgfqpoint{0.891040in}{1.712654in}}%
\pgfpathlineto{\pgfqpoint{0.909380in}{1.826670in}}%
\pgfpathlineto{\pgfqpoint{0.927721in}{1.928018in}}%
\pgfpathlineto{\pgfqpoint{0.946061in}{2.018697in}}%
\pgfpathlineto{\pgfqpoint{0.964402in}{1.945247in}}%
\pgfpathlineto{\pgfqpoint{0.988856in}{1.858025in}}%
\pgfpathlineto{\pgfqpoint{1.001083in}{1.818378in}}%
\pgfpathlineto{\pgfqpoint{1.019424in}{1.898054in}}%
\pgfpathlineto{\pgfqpoint{1.043878in}{1.824171in}}%
\pgfpathlineto{\pgfqpoint{1.056105in}{1.790185in}}%
\pgfpathlineto{\pgfqpoint{1.074445in}{1.861752in}}%
\pgfpathlineto{\pgfqpoint{1.092786in}{1.813157in}}%
\pgfpathlineto{\pgfqpoint{1.098899in}{1.835569in}}%
\pgfpathlineto{\pgfqpoint{1.105013in}{1.820077in}}%
\pgfpathlineto{\pgfqpoint{1.123353in}{1.883943in}}%
\pgfpathlineto{\pgfqpoint{1.147807in}{1.824643in}}%
\pgfpathlineto{\pgfqpoint{1.172262in}{1.770390in}}%
\pgfpathlineto{\pgfqpoint{1.178375in}{1.757541in}}%
\pgfpathlineto{\pgfqpoint{1.196716in}{1.815501in}}%
\pgfpathlineto{\pgfqpoint{1.202829in}{1.802715in}}%
\pgfpathlineto{\pgfqpoint{1.227283in}{1.874491in}}%
\pgfpathlineto{\pgfqpoint{1.233397in}{1.861752in}}%
\pgfpathlineto{\pgfqpoint{1.251737in}{1.911916in}}%
\pgfpathlineto{\pgfqpoint{1.270078in}{1.874764in}}%
\pgfpathlineto{\pgfqpoint{1.276191in}{1.890766in}}%
\pgfpathlineto{\pgfqpoint{1.306759in}{1.832961in}}%
\pgfpathlineto{\pgfqpoint{1.337327in}{1.779933in}}%
\pgfpathlineto{\pgfqpoint{1.367894in}{1.731114in}}%
\pgfpathlineto{\pgfqpoint{1.398462in}{1.804389in}}%
\pgfpathlineto{\pgfqpoint{1.422916in}{1.859102in}}%
\pgfpathlineto{\pgfqpoint{1.435143in}{1.839986in}}%
\pgfpathlineto{\pgfqpoint{1.441256in}{1.853109in}}%
\pgfpathlineto{\pgfqpoint{1.447370in}{1.843732in}}%
\pgfpathlineto{\pgfqpoint{1.459597in}{1.869366in}}%
\pgfpathlineto{\pgfqpoint{1.465710in}{1.860072in}}%
\pgfpathlineto{\pgfqpoint{1.471824in}{1.872596in}}%
\pgfpathlineto{\pgfqpoint{1.477938in}{1.863409in}}%
\pgfpathlineto{\pgfqpoint{1.490165in}{1.887895in}}%
\pgfpathlineto{\pgfqpoint{1.508505in}{1.860952in}}%
\pgfpathlineto{\pgfqpoint{1.545186in}{1.930241in}}%
\pgfpathlineto{\pgfqpoint{1.551300in}{1.941271in}}%
\pgfpathlineto{\pgfqpoint{1.587981in}{1.889731in}}%
\pgfpathlineto{\pgfqpoint{1.600208in}{1.873389in}}%
\pgfpathlineto{\pgfqpoint{1.612435in}{1.894805in}}%
\pgfpathlineto{\pgfqpoint{1.618548in}{1.886751in}}%
\pgfpathlineto{\pgfqpoint{1.624662in}{1.897252in}}%
\pgfpathlineto{\pgfqpoint{1.643003in}{1.873610in}}%
\pgfpathlineto{\pgfqpoint{1.667457in}{1.914235in}}%
\pgfpathlineto{\pgfqpoint{1.710251in}{1.861752in}}%
\pgfpathlineto{\pgfqpoint{1.728592in}{1.840475in}}%
\pgfpathlineto{\pgfqpoint{1.734705in}{1.850209in}}%
\pgfpathlineto{\pgfqpoint{1.777500in}{1.803040in}}%
\pgfpathlineto{\pgfqpoint{1.801954in}{1.777591in}}%
\pgfpathlineto{\pgfqpoint{1.826408in}{1.814872in}}%
\pgfpathlineto{\pgfqpoint{1.856976in}{1.784163in}}%
\pgfpathlineto{\pgfqpoint{1.869203in}{1.802113in}}%
\pgfpathlineto{\pgfqpoint{1.918111in}{1.755727in}}%
\pgfpathlineto{\pgfqpoint{1.924224in}{1.750169in}}%
\pgfpathlineto{\pgfqpoint{1.930338in}{1.758888in}}%
\pgfpathlineto{\pgfqpoint{1.954792in}{1.737101in}}%
\pgfpathlineto{\pgfqpoint{1.960906in}{1.745683in}}%
\pgfpathlineto{\pgfqpoint{2.009814in}{1.704264in}}%
\pgfpathlineto{\pgfqpoint{2.058722in}{1.665617in}}%
\pgfpathlineto{\pgfqpoint{2.083176in}{1.647248in}}%
\pgfpathlineto{\pgfqpoint{2.132084in}{1.711110in}}%
\pgfpathlineto{\pgfqpoint{2.138198in}{1.718808in}}%
\pgfpathlineto{\pgfqpoint{2.193219in}{1.678493in}}%
\pgfpathlineto{\pgfqpoint{2.217673in}{1.661455in}}%
\pgfpathlineto{\pgfqpoint{2.236014in}{1.683725in}}%
\pgfpathlineto{\pgfqpoint{2.260468in}{1.667048in}}%
\pgfpathlineto{\pgfqpoint{2.272695in}{1.681529in}}%
\pgfpathlineto{\pgfqpoint{2.278809in}{1.677413in}}%
\pgfpathlineto{\pgfqpoint{2.284922in}{1.684563in}}%
\pgfpathlineto{\pgfqpoint{2.297149in}{1.676399in}}%
\pgfpathlineto{\pgfqpoint{2.303263in}{1.683476in}}%
\pgfpathlineto{\pgfqpoint{2.327717in}{1.667451in}}%
\pgfpathlineto{\pgfqpoint{2.364398in}{1.708630in}}%
\pgfpathlineto{\pgfqpoint{2.425533in}{1.669870in}}%
\pgfpathlineto{\pgfqpoint{2.486668in}{1.633618in}}%
\pgfpathlineto{\pgfqpoint{2.529463in}{1.609607in}}%
\pgfpathlineto{\pgfqpoint{2.535577in}{1.616046in}}%
\pgfpathlineto{\pgfqpoint{2.566144in}{1.599488in}}%
\pgfpathlineto{\pgfqpoint{2.584485in}{1.618424in}}%
\pgfpathlineto{\pgfqpoint{2.596712in}{1.611888in}}%
\pgfpathlineto{\pgfqpoint{2.602825in}{1.618104in}}%
\pgfpathlineto{\pgfqpoint{2.627279in}{1.605232in}}%
\pgfpathlineto{\pgfqpoint{2.639506in}{1.617482in}}%
\pgfpathlineto{\pgfqpoint{2.700642in}{1.586441in}}%
\pgfpathlineto{\pgfqpoint{2.755663in}{1.639077in}}%
\pgfpathlineto{\pgfqpoint{2.767890in}{1.632939in}}%
\pgfpathlineto{\pgfqpoint{2.774004in}{1.638608in}}%
\pgfpathlineto{\pgfqpoint{2.816798in}{1.617610in}}%
\pgfpathlineto{\pgfqpoint{2.829025in}{1.628751in}}%
\pgfpathlineto{\pgfqpoint{2.835139in}{1.625803in}}%
\pgfpathlineto{\pgfqpoint{2.853480in}{1.642267in}}%
\pgfpathlineto{\pgfqpoint{2.877934in}{1.630551in}}%
\pgfpathlineto{\pgfqpoint{2.884047in}{1.635953in}}%
\pgfpathlineto{\pgfqpoint{2.908501in}{1.624458in}}%
\pgfpathlineto{\pgfqpoint{2.914615in}{1.629805in}}%
\pgfpathlineto{\pgfqpoint{2.951296in}{1.612972in}}%
\pgfpathlineto{\pgfqpoint{2.975750in}{1.633928in}}%
\pgfpathlineto{\pgfqpoint{3.055226in}{1.598865in}}%
\pgfpathlineto{\pgfqpoint{3.122474in}{1.570921in}}%
\pgfpathlineto{\pgfqpoint{3.128588in}{1.575945in}}%
\pgfpathlineto{\pgfqpoint{3.140815in}{1.571011in}}%
\pgfpathlineto{\pgfqpoint{3.165269in}{1.590820in}}%
\pgfpathlineto{\pgfqpoint{3.201950in}{1.576156in}}%
\pgfpathlineto{\pgfqpoint{3.220291in}{1.590665in}}%
\pgfpathlineto{\pgfqpoint{3.269199in}{1.571608in}}%
\pgfpathlineto{\pgfqpoint{3.281426in}{1.581081in}}%
\pgfpathlineto{\pgfqpoint{3.360902in}{1.551418in}}%
\pgfpathlineto{\pgfqpoint{3.373129in}{1.560666in}}%
\pgfpathlineto{\pgfqpoint{3.409810in}{1.547480in}}%
\pgfpathlineto{\pgfqpoint{3.415923in}{1.552043in}}%
\pgfpathlineto{\pgfqpoint{3.440377in}{1.543420in}}%
\pgfpathlineto{\pgfqpoint{3.452605in}{1.552446in}}%
\pgfpathlineto{\pgfqpoint{3.483172in}{1.541824in}}%
\pgfpathlineto{\pgfqpoint{3.507626in}{1.559547in}}%
\pgfpathlineto{\pgfqpoint{3.611556in}{1.524724in}}%
\pgfpathlineto{\pgfqpoint{3.617670in}{1.529029in}}%
\pgfpathlineto{\pgfqpoint{3.666578in}{1.513422in}}%
\pgfpathlineto{\pgfqpoint{3.672691in}{1.517671in}}%
\pgfpathlineto{\pgfqpoint{3.703259in}{1.508144in}}%
\pgfpathlineto{\pgfqpoint{3.715486in}{1.516547in}}%
\pgfpathlineto{\pgfqpoint{3.733826in}{1.510893in}}%
\pgfpathlineto{\pgfqpoint{3.752167in}{1.523335in}}%
\pgfpathlineto{\pgfqpoint{3.862210in}{1.490515in}}%
\pgfpathlineto{\pgfqpoint{3.874437in}{1.487005in}}%
\pgfpathlineto{\pgfqpoint{3.917232in}{1.514901in}}%
\pgfpathlineto{\pgfqpoint{4.008935in}{1.488954in}}%
\pgfpathlineto{\pgfqpoint{4.021162in}{1.496675in}}%
\pgfpathlineto{\pgfqpoint{4.039502in}{1.491629in}}%
\pgfpathlineto{\pgfqpoint{4.045616in}{1.495458in}}%
\pgfpathlineto{\pgfqpoint{4.125092in}{1.474149in}}%
\pgfpathlineto{\pgfqpoint{4.143432in}{1.485409in}}%
\pgfpathlineto{\pgfqpoint{4.210681in}{1.467993in}}%
\pgfpathlineto{\pgfqpoint{4.216795in}{1.471680in}}%
\pgfpathlineto{\pgfqpoint{4.222908in}{1.470125in}}%
\pgfpathlineto{\pgfqpoint{4.247362in}{1.484737in}}%
\pgfpathlineto{\pgfqpoint{4.259589in}{1.481615in}}%
\pgfpathlineto{\pgfqpoint{4.277930in}{1.492425in}}%
\pgfpathlineto{\pgfqpoint{4.302384in}{1.486203in}}%
\pgfpathlineto{\pgfqpoint{4.394087in}{1.538491in}}%
\pgfpathlineto{\pgfqpoint{4.400200in}{1.541887in}}%
\pgfpathlineto{\pgfqpoint{4.467449in}{1.524645in}}%
\pgfpathlineto{\pgfqpoint{4.510243in}{1.547901in}}%
\pgfpathlineto{\pgfqpoint{4.595833in}{1.526533in}}%
\pgfpathlineto{\pgfqpoint{4.632514in}{1.545845in}}%
\pgfpathlineto{\pgfqpoint{4.638627in}{1.544338in}}%
\pgfpathlineto{\pgfqpoint{4.663081in}{1.557010in}}%
\pgfpathlineto{\pgfqpoint{4.711990in}{1.545036in}}%
\pgfpathlineto{\pgfqpoint{4.748671in}{1.563644in}}%
\pgfpathlineto{\pgfqpoint{4.760898in}{1.560666in}}%
\pgfpathlineto{\pgfqpoint{4.767011in}{1.563731in}}%
\pgfpathlineto{\pgfqpoint{4.785352in}{1.559290in}}%
\pgfpathlineto{\pgfqpoint{4.803692in}{1.568411in}}%
\pgfpathlineto{\pgfqpoint{4.870941in}{1.552379in}}%
\pgfpathlineto{\pgfqpoint{4.877055in}{1.555377in}}%
\pgfpathlineto{\pgfqpoint{4.913736in}{1.546830in}}%
\pgfpathlineto{\pgfqpoint{4.919849in}{1.549807in}}%
\pgfpathlineto{\pgfqpoint{4.962644in}{1.540001in}}%
\pgfpathlineto{\pgfqpoint{4.968757in}{1.542954in}}%
\pgfpathlineto{\pgfqpoint{4.980985in}{1.540183in}}%
\pgfpathlineto{\pgfqpoint{4.993212in}{1.546055in}}%
\pgfpathlineto{\pgfqpoint{5.036006in}{1.536448in}}%
\pgfpathlineto{\pgfqpoint{5.054347in}{1.545149in}}%
\pgfpathlineto{\pgfqpoint{5.127709in}{1.529029in}}%
\pgfpathlineto{\pgfqpoint{5.139936in}{1.534740in}}%
\pgfpathlineto{\pgfqpoint{5.146050in}{1.533416in}}%
\pgfpathlineto{\pgfqpoint{5.188844in}{1.553153in}}%
\pgfpathlineto{\pgfqpoint{5.201071in}{1.550488in}}%
\pgfpathlineto{\pgfqpoint{5.219412in}{1.558823in}}%
\pgfpathlineto{\pgfqpoint{5.298888in}{1.541764in}}%
\pgfpathlineto{\pgfqpoint{5.323342in}{1.552678in}}%
\pgfpathlineto{\pgfqpoint{5.329455in}{1.551382in}}%
\pgfpathlineto{\pgfqpoint{5.335569in}{1.554091in}}%
\pgfpathlineto{\pgfqpoint{5.372250in}{1.546375in}}%
\pgfpathlineto{\pgfqpoint{5.396704in}{1.557099in}}%
\pgfpathlineto{\pgfqpoint{5.402817in}{1.555817in}}%
\pgfpathlineto{\pgfqpoint{5.500634in}{1.597615in}}%
\pgfpathlineto{\pgfqpoint{5.616791in}{1.573363in}}%
\pgfpathlineto{\pgfqpoint{5.622904in}{1.575891in}}%
\pgfpathlineto{\pgfqpoint{5.641245in}{1.572159in}}%
\pgfpathlineto{\pgfqpoint{5.659585in}{1.579691in}}%
\pgfpathlineto{\pgfqpoint{5.732947in}{1.564976in}}%
\pgfpathlineto{\pgfqpoint{5.757402in}{1.574862in}}%
\pgfpathlineto{\pgfqpoint{5.806310in}{1.565236in}}%
\pgfpathlineto{\pgfqpoint{5.836877in}{1.577405in}}%
\pgfpathlineto{\pgfqpoint{5.855218in}{1.573820in}}%
\pgfpathlineto{\pgfqpoint{5.898013in}{1.590590in}}%
\pgfpathlineto{\pgfqpoint{5.904126in}{1.589391in}}%
\pgfpathlineto{\pgfqpoint{5.922467in}{1.596491in}}%
\pgfpathlineto{\pgfqpoint{5.965261in}{1.588146in}}%
\pgfpathlineto{\pgfqpoint{6.001942in}{1.602146in}}%
\pgfpathlineto{\pgfqpoint{6.038623in}{1.595051in}}%
\pgfpathlineto{\pgfqpoint{6.063078in}{1.604250in}}%
\pgfpathlineto{\pgfqpoint{6.069191in}{1.603072in}}%
\pgfpathlineto{\pgfqpoint{6.081418in}{1.607638in}}%
\pgfpathlineto{\pgfqpoint{6.130326in}{1.598286in}}%
\pgfpathlineto{\pgfqpoint{6.148667in}{1.605067in}}%
\pgfpathlineto{\pgfqpoint{6.203689in}{1.594709in}}%
\pgfpathlineto{\pgfqpoint{6.209802in}{1.596949in}}%
\pgfpathlineto{\pgfqpoint{6.350413in}{1.571352in}}%
\pgfpathlineto{\pgfqpoint{6.362640in}{1.575763in}}%
\pgfpathlineto{\pgfqpoint{6.429889in}{1.563932in}}%
\pgfpathlineto{\pgfqpoint{6.442116in}{1.568298in}}%
\pgfpathlineto{\pgfqpoint{6.454343in}{1.566171in}}%
\pgfpathlineto{\pgfqpoint{6.472683in}{1.572679in}}%
\pgfpathlineto{\pgfqpoint{6.509365in}{1.566332in}}%
\pgfpathlineto{\pgfqpoint{6.521592in}{1.570634in}}%
\pgfpathlineto{\pgfqpoint{6.643862in}{1.549988in}}%
\pgfpathlineto{\pgfqpoint{6.649975in}{1.552111in}}%
\pgfpathlineto{\pgfqpoint{6.656089in}{1.551099in}}%
\pgfpathlineto{\pgfqpoint{6.674430in}{1.557437in}}%
\pgfpathlineto{\pgfqpoint{6.711111in}{1.551390in}}%
\pgfpathlineto{\pgfqpoint{6.711111in}{1.551390in}}%
\pgfusepath{stroke}%
\end{pgfscope}%
\begin{pgfscope}%
\pgfpathrectangle{\pgfqpoint{0.603704in}{0.549691in}}{\pgfqpoint{6.107407in}{3.101235in}}%
\pgfusepath{clip}%
\pgfsetbuttcap%
\pgfsetroundjoin%
\pgfsetlinewidth{1.505625pt}%
\definecolor{currentstroke}{rgb}{0.121569,0.466667,0.705882}%
\pgfsetstrokecolor{currentstroke}%
\pgfsetstrokeopacity{0.700000}%
\pgfsetdash{{5.550000pt}{2.400000pt}}{0.000000pt}%
\pgfpathmoveto{\pgfqpoint{0.603704in}{1.583436in}}%
\pgfpathlineto{\pgfqpoint{6.711111in}{1.583436in}}%
\pgfusepath{stroke}%
\end{pgfscope}%
\begin{pgfscope}%
\pgfpathrectangle{\pgfqpoint{0.603704in}{0.549691in}}{\pgfqpoint{6.107407in}{3.101235in}}%
\pgfusepath{clip}%
\pgfsetrectcap%
\pgfsetroundjoin%
\pgfsetlinewidth{1.505625pt}%
\definecolor{currentstroke}{rgb}{1.000000,0.498039,0.054902}%
\pgfsetstrokecolor{currentstroke}%
\pgfsetdash{}{0pt}%
\pgfpathmoveto{\pgfqpoint{0.603704in}{0.549691in}}%
\pgfpathlineto{\pgfqpoint{0.634272in}{0.549691in}}%
\pgfpathlineto{\pgfqpoint{0.640385in}{0.992725in}}%
\pgfpathlineto{\pgfqpoint{0.646499in}{0.937346in}}%
\pgfpathlineto{\pgfqpoint{0.652612in}{1.238854in}}%
\pgfpathlineto{\pgfqpoint{0.664839in}{1.113552in}}%
\pgfpathlineto{\pgfqpoint{0.670953in}{1.066564in}}%
\pgfpathlineto{\pgfqpoint{0.677066in}{1.265361in}}%
\pgfpathlineto{\pgfqpoint{0.683180in}{1.214241in}}%
\pgfpathlineto{\pgfqpoint{0.689293in}{1.376687in}}%
\pgfpathlineto{\pgfqpoint{0.701520in}{1.279393in}}%
\pgfpathlineto{\pgfqpoint{0.707634in}{1.238854in}}%
\pgfpathlineto{\pgfqpoint{0.713748in}{1.365806in}}%
\pgfpathlineto{\pgfqpoint{0.725975in}{1.288080in}}%
\pgfpathlineto{\pgfqpoint{0.732088in}{1.395483in}}%
\pgfpathlineto{\pgfqpoint{0.744315in}{1.325000in}}%
\pgfpathlineto{\pgfqpoint{0.756542in}{1.265361in}}%
\pgfpathlineto{\pgfqpoint{0.768769in}{1.214241in}}%
\pgfpathlineto{\pgfqpoint{0.774883in}{1.298265in}}%
\pgfpathlineto{\pgfqpoint{0.787110in}{1.249970in}}%
\pgfpathlineto{\pgfqpoint{0.799337in}{1.207529in}}%
\pgfpathlineto{\pgfqpoint{0.805450in}{1.279393in}}%
\pgfpathlineto{\pgfqpoint{0.817677in}{1.238854in}}%
\pgfpathlineto{\pgfqpoint{0.823791in}{1.304046in}}%
\pgfpathlineto{\pgfqpoint{0.836018in}{1.265361in}}%
\pgfpathlineto{\pgfqpoint{0.848245in}{1.381730in}}%
\pgfpathlineto{\pgfqpoint{0.866586in}{1.325000in}}%
\pgfpathlineto{\pgfqpoint{0.872699in}{1.376687in}}%
\pgfpathlineto{\pgfqpoint{0.891040in}{1.325000in}}%
\pgfpathlineto{\pgfqpoint{0.909380in}{1.279393in}}%
\pgfpathlineto{\pgfqpoint{0.927721in}{1.238854in}}%
\pgfpathlineto{\pgfqpoint{0.946061in}{1.202583in}}%
\pgfpathlineto{\pgfqpoint{0.964402in}{1.325000in}}%
\pgfpathlineto{\pgfqpoint{0.988856in}{1.276543in}}%
\pgfpathlineto{\pgfqpoint{0.994969in}{1.265361in}}%
\pgfpathlineto{\pgfqpoint{1.001083in}{1.301506in}}%
\pgfpathlineto{\pgfqpoint{1.025537in}{1.258545in}}%
\pgfpathlineto{\pgfqpoint{1.049991in}{1.220228in}}%
\pgfpathlineto{\pgfqpoint{1.074445in}{1.185842in}}%
\pgfpathlineto{\pgfqpoint{1.092786in}{1.277141in}}%
\pgfpathlineto{\pgfqpoint{1.098899in}{1.268270in}}%
\pgfpathlineto{\pgfqpoint{1.105013in}{1.296977in}}%
\pgfpathlineto{\pgfqpoint{1.123353in}{1.270909in}}%
\pgfpathlineto{\pgfqpoint{1.129467in}{1.298265in}}%
\pgfpathlineto{\pgfqpoint{1.135580in}{1.289759in}}%
\pgfpathlineto{\pgfqpoint{1.141694in}{1.316289in}}%
\pgfpathlineto{\pgfqpoint{1.147807in}{1.307771in}}%
\pgfpathlineto{\pgfqpoint{1.153921in}{1.333520in}}%
\pgfpathlineto{\pgfqpoint{1.172262in}{1.308504in}}%
\pgfpathlineto{\pgfqpoint{1.178375in}{1.333161in}}%
\pgfpathlineto{\pgfqpoint{1.196716in}{1.309177in}}%
\pgfpathlineto{\pgfqpoint{1.202829in}{1.332831in}}%
\pgfpathlineto{\pgfqpoint{1.227283in}{1.302418in}}%
\pgfpathlineto{\pgfqpoint{1.233397in}{1.325000in}}%
\pgfpathlineto{\pgfqpoint{1.257851in}{1.296285in}}%
\pgfpathlineto{\pgfqpoint{1.270078in}{1.339096in}}%
\pgfpathlineto{\pgfqpoint{1.276191in}{1.331985in}}%
\pgfpathlineto{\pgfqpoint{1.306759in}{1.431939in}}%
\pgfpathlineto{\pgfqpoint{1.325100in}{1.487880in}}%
\pgfpathlineto{\pgfqpoint{1.331213in}{1.480062in}}%
\pgfpathlineto{\pgfqpoint{1.337327in}{1.498003in}}%
\pgfpathlineto{\pgfqpoint{1.361781in}{1.467657in}}%
\pgfpathlineto{\pgfqpoint{1.367894in}{1.484984in}}%
\pgfpathlineto{\pgfqpoint{1.404575in}{1.442471in}}%
\pgfpathlineto{\pgfqpoint{1.441256in}{1.403654in}}%
\pgfpathlineto{\pgfqpoint{1.471824in}{1.373796in}}%
\pgfpathlineto{\pgfqpoint{1.477938in}{1.389609in}}%
\pgfpathlineto{\pgfqpoint{1.490165in}{1.378103in}}%
\pgfpathlineto{\pgfqpoint{1.496278in}{1.393565in}}%
\pgfpathlineto{\pgfqpoint{1.532959in}{1.360472in}}%
\pgfpathlineto{\pgfqpoint{1.563527in}{1.334814in}}%
\pgfpathlineto{\pgfqpoint{1.569640in}{1.349381in}}%
\pgfpathlineto{\pgfqpoint{1.581867in}{1.339447in}}%
\pgfpathlineto{\pgfqpoint{1.587981in}{1.353715in}}%
\pgfpathlineto{\pgfqpoint{1.594094in}{1.348782in}}%
\pgfpathlineto{\pgfqpoint{1.600208in}{1.362820in}}%
\pgfpathlineto{\pgfqpoint{1.612435in}{1.353023in}}%
\pgfpathlineto{\pgfqpoint{1.618548in}{1.366783in}}%
\pgfpathlineto{\pgfqpoint{1.624662in}{1.361919in}}%
\pgfpathlineto{\pgfqpoint{1.643003in}{1.402077in}}%
\pgfpathlineto{\pgfqpoint{1.667457in}{1.382594in}}%
\pgfpathlineto{\pgfqpoint{1.685797in}{1.420825in}}%
\pgfpathlineto{\pgfqpoint{1.710251in}{1.401679in}}%
\pgfpathlineto{\pgfqpoint{1.728592in}{1.438153in}}%
\pgfpathlineto{\pgfqpoint{1.734705in}{1.433376in}}%
\pgfpathlineto{\pgfqpoint{1.753046in}{1.468576in}}%
\pgfpathlineto{\pgfqpoint{1.795841in}{1.435758in}}%
\pgfpathlineto{\pgfqpoint{1.801954in}{1.447003in}}%
\pgfpathlineto{\pgfqpoint{1.826408in}{1.429146in}}%
\pgfpathlineto{\pgfqpoint{1.856976in}{1.483073in}}%
\pgfpathlineto{\pgfqpoint{1.869203in}{1.474098in}}%
\pgfpathlineto{\pgfqpoint{1.875316in}{1.484513in}}%
\pgfpathlineto{\pgfqpoint{1.881430in}{1.480062in}}%
\pgfpathlineto{\pgfqpoint{1.911997in}{1.530547in}}%
\pgfpathlineto{\pgfqpoint{1.918111in}{1.526006in}}%
\pgfpathlineto{\pgfqpoint{1.924224in}{1.535798in}}%
\pgfpathlineto{\pgfqpoint{1.930338in}{1.531275in}}%
\pgfpathlineto{\pgfqpoint{1.942565in}{1.550544in}}%
\pgfpathlineto{\pgfqpoint{1.960906in}{1.537080in}}%
\pgfpathlineto{\pgfqpoint{1.967019in}{1.546517in}}%
\pgfpathlineto{\pgfqpoint{1.985360in}{1.533343in}}%
\pgfpathlineto{\pgfqpoint{2.009814in}{1.570011in}}%
\pgfpathlineto{\pgfqpoint{2.046495in}{1.544180in}}%
\pgfpathlineto{\pgfqpoint{2.058722in}{1.561810in}}%
\pgfpathlineto{\pgfqpoint{2.113744in}{1.525080in}}%
\pgfpathlineto{\pgfqpoint{2.138198in}{1.509597in}}%
\pgfpathlineto{\pgfqpoint{2.150425in}{1.526458in}}%
\pgfpathlineto{\pgfqpoint{2.162652in}{1.518827in}}%
\pgfpathlineto{\pgfqpoint{2.168765in}{1.527123in}}%
\pgfpathlineto{\pgfqpoint{2.174879in}{1.523335in}}%
\pgfpathlineto{\pgfqpoint{2.180992in}{1.531549in}}%
\pgfpathlineto{\pgfqpoint{2.199333in}{1.520307in}}%
\pgfpathlineto{\pgfqpoint{2.217673in}{1.544427in}}%
\pgfpathlineto{\pgfqpoint{2.236014in}{1.533292in}}%
\pgfpathlineto{\pgfqpoint{2.242128in}{1.541164in}}%
\pgfpathlineto{\pgfqpoint{2.248241in}{1.537492in}}%
\pgfpathlineto{\pgfqpoint{2.254355in}{1.545291in}}%
\pgfpathlineto{\pgfqpoint{2.272695in}{1.534390in}}%
\pgfpathlineto{\pgfqpoint{2.278809in}{1.542086in}}%
\pgfpathlineto{\pgfqpoint{2.291036in}{1.534921in}}%
\pgfpathlineto{\pgfqpoint{2.297149in}{1.542533in}}%
\pgfpathlineto{\pgfqpoint{2.303263in}{1.538974in}}%
\pgfpathlineto{\pgfqpoint{2.315490in}{1.554006in}}%
\pgfpathlineto{\pgfqpoint{2.364398in}{1.526205in}}%
\pgfpathlineto{\pgfqpoint{2.382739in}{1.548034in}}%
\pgfpathlineto{\pgfqpoint{2.413306in}{1.531227in}}%
\pgfpathlineto{\pgfqpoint{2.419420in}{1.538340in}}%
\pgfpathlineto{\pgfqpoint{2.425533in}{1.535033in}}%
\pgfpathlineto{\pgfqpoint{2.480555in}{1.596861in}}%
\pgfpathlineto{\pgfqpoint{2.486668in}{1.593473in}}%
\pgfpathlineto{\pgfqpoint{2.492782in}{1.600109in}}%
\pgfpathlineto{\pgfqpoint{2.498895in}{1.596732in}}%
\pgfpathlineto{\pgfqpoint{2.505009in}{1.603316in}}%
\pgfpathlineto{\pgfqpoint{2.511122in}{1.599950in}}%
\pgfpathlineto{\pgfqpoint{2.523349in}{1.612972in}}%
\pgfpathlineto{\pgfqpoint{2.590598in}{1.577094in}}%
\pgfpathlineto{\pgfqpoint{2.602825in}{1.570830in}}%
\pgfpathlineto{\pgfqpoint{2.608939in}{1.577152in}}%
\pgfpathlineto{\pgfqpoint{2.615052in}{1.574038in}}%
\pgfpathlineto{\pgfqpoint{2.627279in}{1.586550in}}%
\pgfpathlineto{\pgfqpoint{2.639506in}{1.580341in}}%
\pgfpathlineto{\pgfqpoint{2.663960in}{1.604845in}}%
\pgfpathlineto{\pgfqpoint{2.682301in}{1.595562in}}%
\pgfpathlineto{\pgfqpoint{2.688415in}{1.601572in}}%
\pgfpathlineto{\pgfqpoint{2.755663in}{1.568794in}}%
\pgfpathlineto{\pgfqpoint{2.767890in}{1.580524in}}%
\pgfpathlineto{\pgfqpoint{2.774004in}{1.577629in}}%
\pgfpathlineto{\pgfqpoint{2.792344in}{1.594954in}}%
\pgfpathlineto{\pgfqpoint{2.810685in}{1.586292in}}%
\pgfpathlineto{\pgfqpoint{2.816798in}{1.591980in}}%
\pgfpathlineto{\pgfqpoint{2.829025in}{1.586268in}}%
\pgfpathlineto{\pgfqpoint{2.835139in}{1.591909in}}%
\pgfpathlineto{\pgfqpoint{2.853480in}{1.583436in}}%
\pgfpathlineto{\pgfqpoint{2.859593in}{1.589024in}}%
\pgfpathlineto{\pgfqpoint{2.865707in}{1.586223in}}%
\pgfpathlineto{\pgfqpoint{2.877934in}{1.597293in}}%
\pgfpathlineto{\pgfqpoint{2.884047in}{1.594492in}}%
\pgfpathlineto{\pgfqpoint{2.902388in}{1.610856in}}%
\pgfpathlineto{\pgfqpoint{2.914615in}{1.605257in}}%
\pgfpathlineto{\pgfqpoint{2.920728in}{1.610640in}}%
\pgfpathlineto{\pgfqpoint{2.987977in}{1.580792in}}%
\pgfpathlineto{\pgfqpoint{3.006318in}{1.596555in}}%
\pgfpathlineto{\pgfqpoint{3.012431in}{1.593904in}}%
\pgfpathlineto{\pgfqpoint{3.030772in}{1.609410in}}%
\pgfpathlineto{\pgfqpoint{3.049112in}{1.601482in}}%
\pgfpathlineto{\pgfqpoint{3.073566in}{1.621723in}}%
\pgfpathlineto{\pgfqpoint{3.128588in}{1.598418in}}%
\pgfpathlineto{\pgfqpoint{3.134701in}{1.603364in}}%
\pgfpathlineto{\pgfqpoint{3.171383in}{1.588347in}}%
\pgfpathlineto{\pgfqpoint{3.189723in}{1.602941in}}%
\pgfpathlineto{\pgfqpoint{3.195837in}{1.600463in}}%
\pgfpathlineto{\pgfqpoint{3.201950in}{1.605276in}}%
\pgfpathlineto{\pgfqpoint{3.220291in}{1.597894in}}%
\pgfpathlineto{\pgfqpoint{3.263085in}{1.630856in}}%
\pgfpathlineto{\pgfqpoint{3.281426in}{1.623467in}}%
\pgfpathlineto{\pgfqpoint{3.287539in}{1.628075in}}%
\pgfpathlineto{\pgfqpoint{3.342561in}{1.606459in}}%
\pgfpathlineto{\pgfqpoint{3.348675in}{1.611003in}}%
\pgfpathlineto{\pgfqpoint{3.373129in}{1.601652in}}%
\pgfpathlineto{\pgfqpoint{3.385356in}{1.610640in}}%
\pgfpathlineto{\pgfqpoint{3.415923in}{1.599133in}}%
\pgfpathlineto{\pgfqpoint{3.440377in}{1.616783in}}%
\pgfpathlineto{\pgfqpoint{3.452605in}{1.612213in}}%
\pgfpathlineto{\pgfqpoint{3.470945in}{1.625226in}}%
\pgfpathlineto{\pgfqpoint{3.507626in}{1.611669in}}%
\pgfpathlineto{\pgfqpoint{3.519853in}{1.620201in}}%
\pgfpathlineto{\pgfqpoint{3.538194in}{1.613524in}}%
\pgfpathlineto{\pgfqpoint{3.574875in}{1.638626in}}%
\pgfpathlineto{\pgfqpoint{3.636010in}{1.616716in}}%
\pgfpathlineto{\pgfqpoint{3.648237in}{1.624869in}}%
\pgfpathlineto{\pgfqpoint{3.654351in}{1.622718in}}%
\pgfpathlineto{\pgfqpoint{3.660464in}{1.626767in}}%
\pgfpathlineto{\pgfqpoint{3.672691in}{1.622484in}}%
\pgfpathlineto{\pgfqpoint{3.678805in}{1.626509in}}%
\pgfpathlineto{\pgfqpoint{3.684918in}{1.624377in}}%
\pgfpathlineto{\pgfqpoint{3.691032in}{1.628382in}}%
\pgfpathlineto{\pgfqpoint{3.697145in}{1.626254in}}%
\pgfpathlineto{\pgfqpoint{3.703259in}{1.630240in}}%
\pgfpathlineto{\pgfqpoint{3.721599in}{1.623896in}}%
\pgfpathlineto{\pgfqpoint{3.733826in}{1.631798in}}%
\pgfpathlineto{\pgfqpoint{3.776621in}{1.617232in}}%
\pgfpathlineto{\pgfqpoint{3.807189in}{1.636600in}}%
\pgfpathlineto{\pgfqpoint{3.819416in}{1.632475in}}%
\pgfpathlineto{\pgfqpoint{3.825529in}{1.636298in}}%
\pgfpathlineto{\pgfqpoint{3.831643in}{1.634244in}}%
\pgfpathlineto{\pgfqpoint{3.849983in}{1.645616in}}%
\pgfpathlineto{\pgfqpoint{3.862210in}{1.641512in}}%
\pgfpathlineto{\pgfqpoint{3.874437in}{1.649010in}}%
\pgfpathlineto{\pgfqpoint{3.917232in}{1.634838in}}%
\pgfpathlineto{\pgfqpoint{3.923346in}{1.638544in}}%
\pgfpathlineto{\pgfqpoint{3.972254in}{1.622763in}}%
\pgfpathlineto{\pgfqpoint{3.990594in}{1.633726in}}%
\pgfpathlineto{\pgfqpoint{4.002821in}{1.629834in}}%
\pgfpathlineto{\pgfqpoint{4.008935in}{1.633456in}}%
\pgfpathlineto{\pgfqpoint{4.021162in}{1.629585in}}%
\pgfpathlineto{\pgfqpoint{4.027275in}{1.633189in}}%
\pgfpathlineto{\pgfqpoint{4.033389in}{1.631261in}}%
\pgfpathlineto{\pgfqpoint{4.039502in}{1.634848in}}%
\pgfpathlineto{\pgfqpoint{4.045616in}{1.632924in}}%
\pgfpathlineto{\pgfqpoint{4.070070in}{1.647135in}}%
\pgfpathlineto{\pgfqpoint{4.112865in}{1.633775in}}%
\pgfpathlineto{\pgfqpoint{4.125092in}{1.640767in}}%
\pgfpathlineto{\pgfqpoint{4.143432in}{1.635123in}}%
\pgfpathlineto{\pgfqpoint{4.149546in}{1.638593in}}%
\pgfpathlineto{\pgfqpoint{4.155659in}{1.636722in}}%
\pgfpathlineto{\pgfqpoint{4.167886in}{1.643620in}}%
\pgfpathlineto{\pgfqpoint{4.216795in}{1.628837in}}%
\pgfpathlineto{\pgfqpoint{4.222908in}{1.632247in}}%
\pgfpathlineto{\pgfqpoint{4.247362in}{1.624994in}}%
\pgfpathlineto{\pgfqpoint{4.259589in}{1.631758in}}%
\pgfpathlineto{\pgfqpoint{4.290157in}{1.622801in}}%
\pgfpathlineto{\pgfqpoint{4.302384in}{1.629494in}}%
\pgfpathlineto{\pgfqpoint{4.400200in}{1.601718in}}%
\pgfpathlineto{\pgfqpoint{4.406314in}{1.605007in}}%
\pgfpathlineto{\pgfqpoint{4.430768in}{1.598275in}}%
\pgfpathlineto{\pgfqpoint{4.449108in}{1.608049in}}%
\pgfpathlineto{\pgfqpoint{4.516357in}{1.589887in}}%
\pgfpathlineto{\pgfqpoint{4.559152in}{1.612151in}}%
\pgfpathlineto{\pgfqpoint{4.663081in}{1.584991in}}%
\pgfpathlineto{\pgfqpoint{4.669195in}{1.588093in}}%
\pgfpathlineto{\pgfqpoint{4.693649in}{1.581893in}}%
\pgfpathlineto{\pgfqpoint{4.711990in}{1.591116in}}%
\pgfpathlineto{\pgfqpoint{4.748671in}{1.581914in}}%
\pgfpathlineto{\pgfqpoint{4.760898in}{1.587990in}}%
\pgfpathlineto{\pgfqpoint{4.767011in}{1.586468in}}%
\pgfpathlineto{\pgfqpoint{4.785352in}{1.595509in}}%
\pgfpathlineto{\pgfqpoint{4.803692in}{1.590949in}}%
\pgfpathlineto{\pgfqpoint{4.815919in}{1.596920in}}%
\pgfpathlineto{\pgfqpoint{4.822033in}{1.595404in}}%
\pgfpathlineto{\pgfqpoint{4.846487in}{1.607235in}}%
\pgfpathlineto{\pgfqpoint{4.864828in}{1.602689in}}%
\pgfpathlineto{\pgfqpoint{4.870941in}{1.605620in}}%
\pgfpathlineto{\pgfqpoint{4.877055in}{1.604111in}}%
\pgfpathlineto{\pgfqpoint{4.913736in}{1.621506in}}%
\pgfpathlineto{\pgfqpoint{4.925963in}{1.618478in}}%
\pgfpathlineto{\pgfqpoint{4.944303in}{1.627054in}}%
\pgfpathlineto{\pgfqpoint{4.950417in}{1.625541in}}%
\pgfpathlineto{\pgfqpoint{4.962644in}{1.631214in}}%
\pgfpathlineto{\pgfqpoint{4.968757in}{1.629702in}}%
\pgfpathlineto{\pgfqpoint{4.980985in}{1.635340in}}%
\pgfpathlineto{\pgfqpoint{5.011552in}{1.627821in}}%
\pgfpathlineto{\pgfqpoint{5.029893in}{1.636193in}}%
\pgfpathlineto{\pgfqpoint{5.054347in}{1.630231in}}%
\pgfpathlineto{\pgfqpoint{5.060460in}{1.632999in}}%
\pgfpathlineto{\pgfqpoint{5.084914in}{1.627096in}}%
\pgfpathlineto{\pgfqpoint{5.091028in}{1.629849in}}%
\pgfpathlineto{\pgfqpoint{5.103255in}{1.626918in}}%
\pgfpathlineto{\pgfqpoint{5.109368in}{1.629661in}}%
\pgfpathlineto{\pgfqpoint{5.121595in}{1.626742in}}%
\pgfpathlineto{\pgfqpoint{5.127709in}{1.629473in}}%
\pgfpathlineto{\pgfqpoint{5.139936in}{1.626567in}}%
\pgfpathlineto{\pgfqpoint{5.146050in}{1.629288in}}%
\pgfpathlineto{\pgfqpoint{5.188844in}{1.619225in}}%
\pgfpathlineto{\pgfqpoint{5.194958in}{1.621927in}}%
\pgfpathlineto{\pgfqpoint{5.219412in}{1.616253in}}%
\pgfpathlineto{\pgfqpoint{5.237752in}{1.624296in}}%
\pgfpathlineto{\pgfqpoint{5.268320in}{1.617263in}}%
\pgfpathlineto{\pgfqpoint{5.274433in}{1.619921in}}%
\pgfpathlineto{\pgfqpoint{5.286661in}{1.617131in}}%
\pgfpathlineto{\pgfqpoint{5.298888in}{1.622420in}}%
\pgfpathlineto{\pgfqpoint{5.323342in}{1.616869in}}%
\pgfpathlineto{\pgfqpoint{5.329455in}{1.619497in}}%
\pgfpathlineto{\pgfqpoint{5.341682in}{1.616740in}}%
\pgfpathlineto{\pgfqpoint{5.347796in}{1.619358in}}%
\pgfpathlineto{\pgfqpoint{5.366136in}{1.615244in}}%
\pgfpathlineto{\pgfqpoint{5.372250in}{1.617850in}}%
\pgfpathlineto{\pgfqpoint{5.396704in}{1.612407in}}%
\pgfpathlineto{\pgfqpoint{5.402817in}{1.615001in}}%
\pgfpathlineto{\pgfqpoint{5.500634in}{1.593748in}}%
\pgfpathlineto{\pgfqpoint{5.506747in}{1.596310in}}%
\pgfpathlineto{\pgfqpoint{5.561769in}{1.584709in}}%
\pgfpathlineto{\pgfqpoint{5.604564in}{1.602369in}}%
\pgfpathlineto{\pgfqpoint{5.610677in}{1.601085in}}%
\pgfpathlineto{\pgfqpoint{5.616791in}{1.603582in}}%
\pgfpathlineto{\pgfqpoint{5.622904in}{1.602300in}}%
\pgfpathlineto{\pgfqpoint{5.641245in}{1.609750in}}%
\pgfpathlineto{\pgfqpoint{5.684039in}{1.600831in}}%
\pgfpathlineto{\pgfqpoint{5.690153in}{1.603292in}}%
\pgfpathlineto{\pgfqpoint{5.696266in}{1.602029in}}%
\pgfpathlineto{\pgfqpoint{5.702380in}{1.604482in}}%
\pgfpathlineto{\pgfqpoint{5.708493in}{1.603221in}}%
\pgfpathlineto{\pgfqpoint{5.732947in}{1.612972in}}%
\pgfpathlineto{\pgfqpoint{5.757402in}{1.607932in}}%
\pgfpathlineto{\pgfqpoint{5.794083in}{1.622354in}}%
\pgfpathlineto{\pgfqpoint{5.800196in}{1.621093in}}%
\pgfpathlineto{\pgfqpoint{5.806310in}{1.623476in}}%
\pgfpathlineto{\pgfqpoint{5.849104in}{1.614725in}}%
\pgfpathlineto{\pgfqpoint{5.855218in}{1.617093in}}%
\pgfpathlineto{\pgfqpoint{5.898013in}{1.608475in}}%
\pgfpathlineto{\pgfqpoint{5.904126in}{1.610828in}}%
\pgfpathlineto{\pgfqpoint{5.934694in}{1.604750in}}%
\pgfpathlineto{\pgfqpoint{5.940807in}{1.607092in}}%
\pgfpathlineto{\pgfqpoint{5.946921in}{1.605883in}}%
\pgfpathlineto{\pgfqpoint{5.965261in}{1.612871in}}%
\pgfpathlineto{\pgfqpoint{6.001942in}{1.605655in}}%
\pgfpathlineto{\pgfqpoint{6.008056in}{1.607966in}}%
\pgfpathlineto{\pgfqpoint{6.032510in}{1.603204in}}%
\pgfpathlineto{\pgfqpoint{6.038623in}{1.605505in}}%
\pgfpathlineto{\pgfqpoint{6.063078in}{1.600781in}}%
\pgfpathlineto{\pgfqpoint{6.069191in}{1.603072in}}%
\pgfpathlineto{\pgfqpoint{6.087532in}{1.599552in}}%
\pgfpathlineto{\pgfqpoint{6.111986in}{1.608649in}}%
\pgfpathlineto{\pgfqpoint{6.124213in}{1.606307in}}%
\pgfpathlineto{\pgfqpoint{6.130326in}{1.608566in}}%
\pgfpathlineto{\pgfqpoint{6.179234in}{1.599288in}}%
\pgfpathlineto{\pgfqpoint{6.203689in}{1.608237in}}%
\pgfpathlineto{\pgfqpoint{6.209802in}{1.607084in}}%
\pgfpathlineto{\pgfqpoint{6.222029in}{1.611527in}}%
\pgfpathlineto{\pgfqpoint{6.228143in}{1.610374in}}%
\pgfpathlineto{\pgfqpoint{6.246483in}{1.616999in}}%
\pgfpathlineto{\pgfqpoint{6.258710in}{1.614694in}}%
\pgfpathlineto{\pgfqpoint{6.264824in}{1.616891in}}%
\pgfpathlineto{\pgfqpoint{6.270937in}{1.615741in}}%
\pgfpathlineto{\pgfqpoint{6.283164in}{1.620117in}}%
\pgfpathlineto{\pgfqpoint{6.344299in}{1.608730in}}%
\pgfpathlineto{\pgfqpoint{6.350413in}{1.610900in}}%
\pgfpathlineto{\pgfqpoint{6.393208in}{1.603064in}}%
\pgfpathlineto{\pgfqpoint{6.399321in}{1.605222in}}%
\pgfpathlineto{\pgfqpoint{6.405435in}{1.604111in}}%
\pgfpathlineto{\pgfqpoint{6.411548in}{1.606263in}}%
\pgfpathlineto{\pgfqpoint{6.417662in}{1.605153in}}%
\pgfpathlineto{\pgfqpoint{6.429889in}{1.609442in}}%
\pgfpathlineto{\pgfqpoint{6.472683in}{1.601723in}}%
\pgfpathlineto{\pgfqpoint{6.484910in}{1.605979in}}%
\pgfpathlineto{\pgfqpoint{6.497137in}{1.603790in}}%
\pgfpathlineto{\pgfqpoint{6.509365in}{1.608024in}}%
\pgfpathlineto{\pgfqpoint{6.521592in}{1.605839in}}%
\pgfpathlineto{\pgfqpoint{6.527705in}{1.607948in}}%
\pgfpathlineto{\pgfqpoint{6.533819in}{1.606858in}}%
\pgfpathlineto{\pgfqpoint{6.539932in}{1.608961in}}%
\pgfpathlineto{\pgfqpoint{6.558273in}{1.605701in}}%
\pgfpathlineto{\pgfqpoint{6.564386in}{1.607797in}}%
\pgfpathlineto{\pgfqpoint{6.576613in}{1.605633in}}%
\pgfpathlineto{\pgfqpoint{6.588840in}{1.609807in}}%
\pgfpathlineto{\pgfqpoint{6.594954in}{1.608727in}}%
\pgfpathlineto{\pgfqpoint{6.607181in}{1.612882in}}%
\pgfpathlineto{\pgfqpoint{6.625521in}{1.609647in}}%
\pgfpathlineto{\pgfqpoint{6.637748in}{1.613779in}}%
\pgfpathlineto{\pgfqpoint{6.649975in}{1.611629in}}%
\pgfpathlineto{\pgfqpoint{6.656089in}{1.613687in}}%
\pgfpathlineto{\pgfqpoint{6.692770in}{1.607284in}}%
\pgfpathlineto{\pgfqpoint{6.711111in}{1.613415in}}%
\pgfpathlineto{\pgfqpoint{6.711111in}{1.613415in}}%
\pgfusepath{stroke}%
\end{pgfscope}%
\begin{pgfscope}%
\pgfpathrectangle{\pgfqpoint{0.603704in}{0.549691in}}{\pgfqpoint{6.107407in}{3.101235in}}%
\pgfusepath{clip}%
\pgfsetbuttcap%
\pgfsetroundjoin%
\pgfsetlinewidth{1.505625pt}%
\definecolor{currentstroke}{rgb}{1.000000,0.498039,0.054902}%
\pgfsetstrokecolor{currentstroke}%
\pgfsetstrokeopacity{0.700000}%
\pgfsetdash{{5.550000pt}{2.400000pt}}{0.000000pt}%
\pgfpathmoveto{\pgfqpoint{0.603704in}{1.583436in}}%
\pgfpathlineto{\pgfqpoint{6.711111in}{1.583436in}}%
\pgfusepath{stroke}%
\end{pgfscope}%
\begin{pgfscope}%
\pgfpathrectangle{\pgfqpoint{0.603704in}{0.549691in}}{\pgfqpoint{6.107407in}{3.101235in}}%
\pgfusepath{clip}%
\pgfsetrectcap%
\pgfsetroundjoin%
\pgfsetlinewidth{1.505625pt}%
\definecolor{currentstroke}{rgb}{0.172549,0.627451,0.172549}%
\pgfsetstrokecolor{currentstroke}%
\pgfsetdash{}{0pt}%
\pgfpathmoveto{\pgfqpoint{0.603704in}{0.549691in}}%
\pgfpathlineto{\pgfqpoint{0.640385in}{0.549691in}}%
\pgfpathlineto{\pgfqpoint{0.646499in}{0.937346in}}%
\pgfpathlineto{\pgfqpoint{0.652612in}{0.894273in}}%
\pgfpathlineto{\pgfqpoint{0.658726in}{1.169938in}}%
\pgfpathlineto{\pgfqpoint{0.670953in}{1.583436in}}%
\pgfpathlineto{\pgfqpoint{0.677066in}{1.503917in}}%
\pgfpathlineto{\pgfqpoint{0.683180in}{1.657275in}}%
\pgfpathlineto{\pgfqpoint{0.689293in}{1.583436in}}%
\pgfpathlineto{\pgfqpoint{0.701520in}{1.826670in}}%
\pgfpathlineto{\pgfqpoint{0.707634in}{1.928018in}}%
\pgfpathlineto{\pgfqpoint{0.713748in}{1.855474in}}%
\pgfpathlineto{\pgfqpoint{0.725975in}{2.026470in}}%
\pgfpathlineto{\pgfqpoint{0.738202in}{1.898054in}}%
\pgfpathlineto{\pgfqpoint{0.750429in}{1.790185in}}%
\pgfpathlineto{\pgfqpoint{0.762656in}{1.698297in}}%
\pgfpathlineto{\pgfqpoint{0.774883in}{1.619083in}}%
\pgfpathlineto{\pgfqpoint{0.787110in}{1.750169in}}%
\pgfpathlineto{\pgfqpoint{0.799337in}{1.865367in}}%
\pgfpathlineto{\pgfqpoint{0.805450in}{1.826670in}}%
\pgfpathlineto{\pgfqpoint{0.817677in}{1.928018in}}%
\pgfpathlineto{\pgfqpoint{0.823791in}{1.890766in}}%
\pgfpathlineto{\pgfqpoint{0.836018in}{1.981030in}}%
\pgfpathlineto{\pgfqpoint{0.854358in}{1.878792in}}%
\pgfpathlineto{\pgfqpoint{0.872699in}{1.790185in}}%
\pgfpathlineto{\pgfqpoint{0.891040in}{1.712654in}}%
\pgfpathlineto{\pgfqpoint{0.909380in}{1.644245in}}%
\pgfpathlineto{\pgfqpoint{0.927721in}{1.583436in}}%
\pgfpathlineto{\pgfqpoint{0.946061in}{1.529029in}}%
\pgfpathlineto{\pgfqpoint{0.964402in}{1.480062in}}%
\pgfpathlineto{\pgfqpoint{0.982742in}{1.583436in}}%
\pgfpathlineto{\pgfqpoint{0.994969in}{1.647051in}}%
\pgfpathlineto{\pgfqpoint{1.019424in}{1.583436in}}%
\pgfpathlineto{\pgfqpoint{1.043878in}{1.696723in}}%
\pgfpathlineto{\pgfqpoint{1.056105in}{1.748835in}}%
\pgfpathlineto{\pgfqpoint{1.080559in}{1.688119in}}%
\pgfpathlineto{\pgfqpoint{1.105013in}{1.633255in}}%
\pgfpathlineto{\pgfqpoint{1.129467in}{1.583436in}}%
\pgfpathlineto{\pgfqpoint{1.135580in}{1.606930in}}%
\pgfpathlineto{\pgfqpoint{1.141694in}{1.595051in}}%
\pgfpathlineto{\pgfqpoint{1.147807in}{1.617894in}}%
\pgfpathlineto{\pgfqpoint{1.153921in}{1.606156in}}%
\pgfpathlineto{\pgfqpoint{1.172262in}{1.671414in}}%
\pgfpathlineto{\pgfqpoint{1.202829in}{1.614762in}}%
\pgfpathlineto{\pgfqpoint{1.233397in}{1.563556in}}%
\pgfpathlineto{\pgfqpoint{1.251737in}{1.535130in}}%
\pgfpathlineto{\pgfqpoint{1.257851in}{1.554721in}}%
\pgfpathlineto{\pgfqpoint{1.288418in}{1.510251in}}%
\pgfpathlineto{\pgfqpoint{1.318986in}{1.469549in}}%
\pgfpathlineto{\pgfqpoint{1.325100in}{1.461819in}}%
\pgfpathlineto{\pgfqpoint{1.331213in}{1.480062in}}%
\pgfpathlineto{\pgfqpoint{1.337327in}{1.472373in}}%
\pgfpathlineto{\pgfqpoint{1.361781in}{1.542086in}}%
\pgfpathlineto{\pgfqpoint{1.398462in}{1.496633in}}%
\pgfpathlineto{\pgfqpoint{1.422916in}{1.468576in}}%
\pgfpathlineto{\pgfqpoint{1.435143in}{1.500435in}}%
\pgfpathlineto{\pgfqpoint{1.441256in}{1.493545in}}%
\pgfpathlineto{\pgfqpoint{1.447370in}{1.509066in}}%
\pgfpathlineto{\pgfqpoint{1.459597in}{1.495458in}}%
\pgfpathlineto{\pgfqpoint{1.465710in}{1.510637in}}%
\pgfpathlineto{\pgfqpoint{1.496278in}{1.477952in}}%
\pgfpathlineto{\pgfqpoint{1.508505in}{1.507119in}}%
\pgfpathlineto{\pgfqpoint{1.545186in}{1.470058in}}%
\pgfpathlineto{\pgfqpoint{1.551300in}{1.464158in}}%
\pgfpathlineto{\pgfqpoint{1.563527in}{1.491838in}}%
\pgfpathlineto{\pgfqpoint{1.569640in}{1.485913in}}%
\pgfpathlineto{\pgfqpoint{1.581867in}{1.512808in}}%
\pgfpathlineto{\pgfqpoint{1.587981in}{1.506862in}}%
\pgfpathlineto{\pgfqpoint{1.594094in}{1.520016in}}%
\pgfpathlineto{\pgfqpoint{1.636889in}{1.480062in}}%
\pgfpathlineto{\pgfqpoint{1.679684in}{1.443267in}}%
\pgfpathlineto{\pgfqpoint{1.685797in}{1.438247in}}%
\pgfpathlineto{\pgfqpoint{1.710251in}{1.486878in}}%
\pgfpathlineto{\pgfqpoint{1.753046in}{1.452167in}}%
\pgfpathlineto{\pgfqpoint{1.789727in}{1.519821in}}%
\pgfpathlineto{\pgfqpoint{1.795841in}{1.530694in}}%
\pgfpathlineto{\pgfqpoint{1.844749in}{1.492223in}}%
\pgfpathlineto{\pgfqpoint{1.875316in}{1.469675in}}%
\pgfpathlineto{\pgfqpoint{1.881430in}{1.480062in}}%
\pgfpathlineto{\pgfqpoint{1.911997in}{1.458425in}}%
\pgfpathlineto{\pgfqpoint{1.918111in}{1.468576in}}%
\pgfpathlineto{\pgfqpoint{1.942565in}{1.451869in}}%
\pgfpathlineto{\pgfqpoint{1.954792in}{1.471680in}}%
\pgfpathlineto{\pgfqpoint{1.967019in}{1.463448in}}%
\pgfpathlineto{\pgfqpoint{1.985360in}{1.492357in}}%
\pgfpathlineto{\pgfqpoint{2.009814in}{1.476034in}}%
\pgfpathlineto{\pgfqpoint{2.046495in}{1.531095in}}%
\pgfpathlineto{\pgfqpoint{2.058722in}{1.522882in}}%
\pgfpathlineto{\pgfqpoint{2.083176in}{1.557912in}}%
\pgfpathlineto{\pgfqpoint{2.138198in}{1.521904in}}%
\pgfpathlineto{\pgfqpoint{2.150425in}{1.514248in}}%
\pgfpathlineto{\pgfqpoint{2.162652in}{1.530941in}}%
\pgfpathlineto{\pgfqpoint{2.168765in}{1.527123in}}%
\pgfpathlineto{\pgfqpoint{2.174879in}{1.535355in}}%
\pgfpathlineto{\pgfqpoint{2.180992in}{1.531549in}}%
\pgfpathlineto{\pgfqpoint{2.199333in}{1.555817in}}%
\pgfpathlineto{\pgfqpoint{2.242128in}{1.529635in}}%
\pgfpathlineto{\pgfqpoint{2.248241in}{1.537492in}}%
\pgfpathlineto{\pgfqpoint{2.254355in}{1.533847in}}%
\pgfpathlineto{\pgfqpoint{2.260468in}{1.541630in}}%
\pgfpathlineto{\pgfqpoint{2.284922in}{1.527254in}}%
\pgfpathlineto{\pgfqpoint{2.291036in}{1.534921in}}%
\pgfpathlineto{\pgfqpoint{2.315490in}{1.520896in}}%
\pgfpathlineto{\pgfqpoint{2.327717in}{1.535950in}}%
\pgfpathlineto{\pgfqpoint{2.382739in}{1.505551in}}%
\pgfpathlineto{\pgfqpoint{2.413306in}{1.541669in}}%
\pgfpathlineto{\pgfqpoint{2.419420in}{1.538340in}}%
\pgfpathlineto{\pgfqpoint{2.425533in}{1.545405in}}%
\pgfpathlineto{\pgfqpoint{2.480555in}{1.516310in}}%
\pgfpathlineto{\pgfqpoint{2.486668in}{1.523218in}}%
\pgfpathlineto{\pgfqpoint{2.492782in}{1.520078in}}%
\pgfpathlineto{\pgfqpoint{2.498895in}{1.526929in}}%
\pgfpathlineto{\pgfqpoint{2.505009in}{1.523797in}}%
\pgfpathlineto{\pgfqpoint{2.511122in}{1.530593in}}%
\pgfpathlineto{\pgfqpoint{2.523349in}{1.524365in}}%
\pgfpathlineto{\pgfqpoint{2.529463in}{1.531095in}}%
\pgfpathlineto{\pgfqpoint{2.535577in}{1.527999in}}%
\pgfpathlineto{\pgfqpoint{2.566144in}{1.560963in}}%
\pgfpathlineto{\pgfqpoint{2.584485in}{1.551629in}}%
\pgfpathlineto{\pgfqpoint{2.596712in}{1.564468in}}%
\pgfpathlineto{\pgfqpoint{2.608939in}{1.558299in}}%
\pgfpathlineto{\pgfqpoint{2.615052in}{1.564641in}}%
\pgfpathlineto{\pgfqpoint{2.663960in}{1.540618in}}%
\pgfpathlineto{\pgfqpoint{2.682301in}{1.559184in}}%
\pgfpathlineto{\pgfqpoint{2.688415in}{1.556232in}}%
\pgfpathlineto{\pgfqpoint{2.700642in}{1.568411in}}%
\pgfpathlineto{\pgfqpoint{2.774004in}{1.534072in}}%
\pgfpathlineto{\pgfqpoint{2.792344in}{1.525846in}}%
\pgfpathlineto{\pgfqpoint{2.810685in}{1.543457in}}%
\pgfpathlineto{\pgfqpoint{2.859593in}{1.521970in}}%
\pgfpathlineto{\pgfqpoint{2.865707in}{1.527709in}}%
\pgfpathlineto{\pgfqpoint{2.902388in}{1.512143in}}%
\pgfpathlineto{\pgfqpoint{2.908501in}{1.517802in}}%
\pgfpathlineto{\pgfqpoint{2.920728in}{1.512706in}}%
\pgfpathlineto{\pgfqpoint{2.951296in}{1.540475in}}%
\pgfpathlineto{\pgfqpoint{2.975750in}{1.530287in}}%
\pgfpathlineto{\pgfqpoint{2.987977in}{1.541135in}}%
\pgfpathlineto{\pgfqpoint{3.006318in}{1.533586in}}%
\pgfpathlineto{\pgfqpoint{3.012431in}{1.538946in}}%
\pgfpathlineto{\pgfqpoint{3.030772in}{1.531489in}}%
\pgfpathlineto{\pgfqpoint{3.049112in}{1.547345in}}%
\pgfpathlineto{\pgfqpoint{3.073566in}{1.537492in}}%
\pgfpathlineto{\pgfqpoint{3.122474in}{1.578430in}}%
\pgfpathlineto{\pgfqpoint{3.134701in}{1.573472in}}%
\pgfpathlineto{\pgfqpoint{3.140815in}{1.578466in}}%
\pgfpathlineto{\pgfqpoint{3.165269in}{1.568668in}}%
\pgfpathlineto{\pgfqpoint{3.171383in}{1.573614in}}%
\pgfpathlineto{\pgfqpoint{3.189723in}{1.566370in}}%
\pgfpathlineto{\pgfqpoint{3.195837in}{1.571274in}}%
\pgfpathlineto{\pgfqpoint{3.263085in}{1.545501in}}%
\pgfpathlineto{\pgfqpoint{3.269199in}{1.550318in}}%
\pgfpathlineto{\pgfqpoint{3.287539in}{1.543496in}}%
\pgfpathlineto{\pgfqpoint{3.342561in}{1.585738in}}%
\pgfpathlineto{\pgfqpoint{3.348675in}{1.583436in}}%
\pgfpathlineto{\pgfqpoint{3.360902in}{1.592584in}}%
\pgfpathlineto{\pgfqpoint{3.385356in}{1.583436in}}%
\pgfpathlineto{\pgfqpoint{3.409810in}{1.601414in}}%
\pgfpathlineto{\pgfqpoint{3.470945in}{1.579037in}}%
\pgfpathlineto{\pgfqpoint{3.483172in}{1.587816in}}%
\pgfpathlineto{\pgfqpoint{3.519853in}{1.574786in}}%
\pgfpathlineto{\pgfqpoint{3.538194in}{1.587734in}}%
\pgfpathlineto{\pgfqpoint{3.574875in}{1.574945in}}%
\pgfpathlineto{\pgfqpoint{3.611556in}{1.600211in}}%
\pgfpathlineto{\pgfqpoint{3.617670in}{1.598084in}}%
\pgfpathlineto{\pgfqpoint{3.636010in}{1.610476in}}%
\pgfpathlineto{\pgfqpoint{3.648237in}{1.606224in}}%
\pgfpathlineto{\pgfqpoint{3.654351in}{1.610314in}}%
\pgfpathlineto{\pgfqpoint{3.660464in}{1.608197in}}%
\pgfpathlineto{\pgfqpoint{3.666578in}{1.612266in}}%
\pgfpathlineto{\pgfqpoint{3.678805in}{1.608049in}}%
\pgfpathlineto{\pgfqpoint{3.684918in}{1.612094in}}%
\pgfpathlineto{\pgfqpoint{3.691032in}{1.609995in}}%
\pgfpathlineto{\pgfqpoint{3.697145in}{1.614020in}}%
\pgfpathlineto{\pgfqpoint{3.715486in}{1.607760in}}%
\pgfpathlineto{\pgfqpoint{3.721599in}{1.611758in}}%
\pgfpathlineto{\pgfqpoint{3.752167in}{1.601467in}}%
\pgfpathlineto{\pgfqpoint{3.776621in}{1.617232in}}%
\pgfpathlineto{\pgfqpoint{3.807189in}{1.607065in}}%
\pgfpathlineto{\pgfqpoint{3.819416in}{1.614821in}}%
\pgfpathlineto{\pgfqpoint{3.825529in}{1.612804in}}%
\pgfpathlineto{\pgfqpoint{3.831643in}{1.616657in}}%
\pgfpathlineto{\pgfqpoint{3.849983in}{1.610640in}}%
\pgfpathlineto{\pgfqpoint{3.862210in}{1.618281in}}%
\pgfpathlineto{\pgfqpoint{3.923346in}{1.598638in}}%
\pgfpathlineto{\pgfqpoint{3.972254in}{1.628382in}}%
\pgfpathlineto{\pgfqpoint{3.990594in}{1.622551in}}%
\pgfpathlineto{\pgfqpoint{4.002821in}{1.629834in}}%
\pgfpathlineto{\pgfqpoint{4.027275in}{1.622132in}}%
\pgfpathlineto{\pgfqpoint{4.033389in}{1.625742in}}%
\pgfpathlineto{\pgfqpoint{4.070070in}{1.614376in}}%
\pgfpathlineto{\pgfqpoint{4.112865in}{1.639168in}}%
\pgfpathlineto{\pgfqpoint{4.149546in}{1.627917in}}%
\pgfpathlineto{\pgfqpoint{4.155659in}{1.631393in}}%
\pgfpathlineto{\pgfqpoint{4.167886in}{1.627689in}}%
\pgfpathlineto{\pgfqpoint{4.210681in}{1.651653in}}%
\pgfpathlineto{\pgfqpoint{4.277930in}{1.631517in}}%
\pgfpathlineto{\pgfqpoint{4.290157in}{1.638204in}}%
\pgfpathlineto{\pgfqpoint{4.406314in}{1.605007in}}%
\pgfpathlineto{\pgfqpoint{4.430768in}{1.618059in}}%
\pgfpathlineto{\pgfqpoint{4.449108in}{1.612972in}}%
\pgfpathlineto{\pgfqpoint{4.467449in}{1.622630in}}%
\pgfpathlineto{\pgfqpoint{4.510243in}{1.610895in}}%
\pgfpathlineto{\pgfqpoint{4.516357in}{1.614078in}}%
\pgfpathlineto{\pgfqpoint{4.559152in}{1.602580in}}%
\pgfpathlineto{\pgfqpoint{4.595833in}{1.621372in}}%
\pgfpathlineto{\pgfqpoint{4.632514in}{1.611629in}}%
\pgfpathlineto{\pgfqpoint{4.638627in}{1.614714in}}%
\pgfpathlineto{\pgfqpoint{4.669195in}{1.606719in}}%
\pgfpathlineto{\pgfqpoint{4.693649in}{1.618923in}}%
\pgfpathlineto{\pgfqpoint{4.815919in}{1.587931in}}%
\pgfpathlineto{\pgfqpoint{4.822033in}{1.590916in}}%
\pgfpathlineto{\pgfqpoint{4.846487in}{1.584924in}}%
\pgfpathlineto{\pgfqpoint{4.864828in}{1.593803in}}%
\pgfpathlineto{\pgfqpoint{4.919849in}{1.580512in}}%
\pgfpathlineto{\pgfqpoint{4.925963in}{1.583436in}}%
\pgfpathlineto{\pgfqpoint{4.944303in}{1.579074in}}%
\pgfpathlineto{\pgfqpoint{4.950417in}{1.581984in}}%
\pgfpathlineto{\pgfqpoint{4.993212in}{1.571934in}}%
\pgfpathlineto{\pgfqpoint{5.011552in}{1.580573in}}%
\pgfpathlineto{\pgfqpoint{5.029893in}{1.576307in}}%
\pgfpathlineto{\pgfqpoint{5.036006in}{1.579164in}}%
\pgfpathlineto{\pgfqpoint{5.060460in}{1.573524in}}%
\pgfpathlineto{\pgfqpoint{5.084914in}{1.584845in}}%
\pgfpathlineto{\pgfqpoint{5.091028in}{1.583436in}}%
\pgfpathlineto{\pgfqpoint{5.103255in}{1.589047in}}%
\pgfpathlineto{\pgfqpoint{5.109368in}{1.587638in}}%
\pgfpathlineto{\pgfqpoint{5.121595in}{1.593215in}}%
\pgfpathlineto{\pgfqpoint{5.194958in}{1.576563in}}%
\pgfpathlineto{\pgfqpoint{5.201071in}{1.579318in}}%
\pgfpathlineto{\pgfqpoint{5.237752in}{1.571178in}}%
\pgfpathlineto{\pgfqpoint{5.268320in}{1.584789in}}%
\pgfpathlineto{\pgfqpoint{5.274433in}{1.583436in}}%
\pgfpathlineto{\pgfqpoint{5.286661in}{1.588827in}}%
\pgfpathlineto{\pgfqpoint{5.335569in}{1.578101in}}%
\pgfpathlineto{\pgfqpoint{5.341682in}{1.580772in}}%
\pgfpathlineto{\pgfqpoint{5.347796in}{1.579445in}}%
\pgfpathlineto{\pgfqpoint{5.366136in}{1.587412in}}%
\pgfpathlineto{\pgfqpoint{5.506747in}{1.557689in}}%
\pgfpathlineto{\pgfqpoint{5.561769in}{1.580890in}}%
\pgfpathlineto{\pgfqpoint{5.604564in}{1.572076in}}%
\pgfpathlineto{\pgfqpoint{5.610677in}{1.574612in}}%
\pgfpathlineto{\pgfqpoint{5.659585in}{1.564709in}}%
\pgfpathlineto{\pgfqpoint{5.684039in}{1.574739in}}%
\pgfpathlineto{\pgfqpoint{5.690153in}{1.573508in}}%
\pgfpathlineto{\pgfqpoint{5.696266in}{1.575999in}}%
\pgfpathlineto{\pgfqpoint{5.702380in}{1.574770in}}%
\pgfpathlineto{\pgfqpoint{5.708493in}{1.577253in}}%
\pgfpathlineto{\pgfqpoint{5.794083in}{1.560329in}}%
\pgfpathlineto{\pgfqpoint{5.800196in}{1.562786in}}%
\pgfpathlineto{\pgfqpoint{5.836877in}{1.555693in}}%
\pgfpathlineto{\pgfqpoint{5.849104in}{1.560571in}}%
\pgfpathlineto{\pgfqpoint{5.922467in}{1.546644in}}%
\pgfpathlineto{\pgfqpoint{5.934694in}{1.551465in}}%
\pgfpathlineto{\pgfqpoint{5.940807in}{1.550318in}}%
\pgfpathlineto{\pgfqpoint{5.946921in}{1.552719in}}%
\pgfpathlineto{\pgfqpoint{6.008056in}{1.541386in}}%
\pgfpathlineto{\pgfqpoint{6.032510in}{1.550877in}}%
\pgfpathlineto{\pgfqpoint{6.081418in}{1.541948in}}%
\pgfpathlineto{\pgfqpoint{6.087532in}{1.544297in}}%
\pgfpathlineto{\pgfqpoint{6.111986in}{1.539886in}}%
\pgfpathlineto{\pgfqpoint{6.124213in}{1.544556in}}%
\pgfpathlineto{\pgfqpoint{6.148667in}{1.540174in}}%
\pgfpathlineto{\pgfqpoint{6.179234in}{1.551733in}}%
\pgfpathlineto{\pgfqpoint{6.222029in}{1.544109in}}%
\pgfpathlineto{\pgfqpoint{6.228143in}{1.546396in}}%
\pgfpathlineto{\pgfqpoint{6.246483in}{1.543160in}}%
\pgfpathlineto{\pgfqpoint{6.258710in}{1.547713in}}%
\pgfpathlineto{\pgfqpoint{6.264824in}{1.546636in}}%
\pgfpathlineto{\pgfqpoint{6.270937in}{1.548904in}}%
\pgfpathlineto{\pgfqpoint{6.283164in}{1.546755in}}%
\pgfpathlineto{\pgfqpoint{6.344299in}{1.569140in}}%
\pgfpathlineto{\pgfqpoint{6.362640in}{1.565896in}}%
\pgfpathlineto{\pgfqpoint{6.393208in}{1.576893in}}%
\pgfpathlineto{\pgfqpoint{6.399321in}{1.575811in}}%
\pgfpathlineto{\pgfqpoint{6.405435in}{1.577995in}}%
\pgfpathlineto{\pgfqpoint{6.411548in}{1.576914in}}%
\pgfpathlineto{\pgfqpoint{6.417662in}{1.579093in}}%
\pgfpathlineto{\pgfqpoint{6.442116in}{1.574786in}}%
\pgfpathlineto{\pgfqpoint{6.454343in}{1.579120in}}%
\pgfpathlineto{\pgfqpoint{6.484910in}{1.573775in}}%
\pgfpathlineto{\pgfqpoint{6.497137in}{1.578080in}}%
\pgfpathlineto{\pgfqpoint{6.527705in}{1.572779in}}%
\pgfpathlineto{\pgfqpoint{6.533819in}{1.574919in}}%
\pgfpathlineto{\pgfqpoint{6.539932in}{1.573864in}}%
\pgfpathlineto{\pgfqpoint{6.558273in}{1.580255in}}%
\pgfpathlineto{\pgfqpoint{6.564386in}{1.579199in}}%
\pgfpathlineto{\pgfqpoint{6.576613in}{1.583436in}}%
\pgfpathlineto{\pgfqpoint{6.588840in}{1.581326in}}%
\pgfpathlineto{\pgfqpoint{6.594954in}{1.583436in}}%
\pgfpathlineto{\pgfqpoint{6.607181in}{1.581333in}}%
\pgfpathlineto{\pgfqpoint{6.625521in}{1.587630in}}%
\pgfpathlineto{\pgfqpoint{6.637748in}{1.585529in}}%
\pgfpathlineto{\pgfqpoint{6.643862in}{1.587617in}}%
\pgfpathlineto{\pgfqpoint{6.674430in}{1.582396in}}%
\pgfpathlineto{\pgfqpoint{6.692770in}{1.588620in}}%
\pgfpathlineto{\pgfqpoint{6.711111in}{1.585504in}}%
\pgfpathlineto{\pgfqpoint{6.711111in}{1.585504in}}%
\pgfusepath{stroke}%
\end{pgfscope}%
\begin{pgfscope}%
\pgfpathrectangle{\pgfqpoint{0.603704in}{0.549691in}}{\pgfqpoint{6.107407in}{3.101235in}}%
\pgfusepath{clip}%
\pgfsetbuttcap%
\pgfsetroundjoin%
\pgfsetlinewidth{1.505625pt}%
\definecolor{currentstroke}{rgb}{0.172549,0.627451,0.172549}%
\pgfsetstrokecolor{currentstroke}%
\pgfsetstrokeopacity{0.700000}%
\pgfsetdash{{5.550000pt}{2.400000pt}}{0.000000pt}%
\pgfpathmoveto{\pgfqpoint{0.603704in}{1.583436in}}%
\pgfpathlineto{\pgfqpoint{6.711111in}{1.583436in}}%
\pgfusepath{stroke}%
\end{pgfscope}%
\begin{pgfscope}%
\pgfsetrectcap%
\pgfsetmiterjoin%
\pgfsetlinewidth{0.803000pt}%
\definecolor{currentstroke}{rgb}{0.000000,0.000000,0.000000}%
\pgfsetstrokecolor{currentstroke}%
\pgfsetdash{}{0pt}%
\pgfpathmoveto{\pgfqpoint{0.603704in}{0.549691in}}%
\pgfpathlineto{\pgfqpoint{0.603704in}{3.650926in}}%
\pgfusepath{stroke}%
\end{pgfscope}%
\begin{pgfscope}%
\pgfsetrectcap%
\pgfsetmiterjoin%
\pgfsetlinewidth{0.803000pt}%
\definecolor{currentstroke}{rgb}{0.000000,0.000000,0.000000}%
\pgfsetstrokecolor{currentstroke}%
\pgfsetdash{}{0pt}%
\pgfpathmoveto{\pgfqpoint{6.711111in}{0.549691in}}%
\pgfpathlineto{\pgfqpoint{6.711111in}{3.650926in}}%
\pgfusepath{stroke}%
\end{pgfscope}%
\begin{pgfscope}%
\pgfsetrectcap%
\pgfsetmiterjoin%
\pgfsetlinewidth{0.803000pt}%
\definecolor{currentstroke}{rgb}{0.000000,0.000000,0.000000}%
\pgfsetstrokecolor{currentstroke}%
\pgfsetdash{}{0pt}%
\pgfpathmoveto{\pgfqpoint{0.603704in}{0.549691in}}%
\pgfpathlineto{\pgfqpoint{6.711111in}{0.549691in}}%
\pgfusepath{stroke}%
\end{pgfscope}%
\begin{pgfscope}%
\pgfsetrectcap%
\pgfsetmiterjoin%
\pgfsetlinewidth{0.803000pt}%
\definecolor{currentstroke}{rgb}{0.000000,0.000000,0.000000}%
\pgfsetstrokecolor{currentstroke}%
\pgfsetdash{}{0pt}%
\pgfpathmoveto{\pgfqpoint{0.603704in}{3.650926in}}%
\pgfpathlineto{\pgfqpoint{6.711111in}{3.650926in}}%
\pgfusepath{stroke}%
\end{pgfscope}%
\begin{pgfscope}%
\definecolor{textcolor}{rgb}{0.000000,0.000000,0.000000}%
\pgfsetstrokecolor{textcolor}%
\pgfsetfillcolor{textcolor}%
\pgftext[x=3.657407in,y=3.734260in,,base]{\color{textcolor}{\rmfamily\fontsize{12.000000}{14.400000}\selectfont\catcode`\^=\active\def^{\ifmmode\sp\else\^{}\fi}\catcode`\%=\active\def%{\%}Convergence of empirical state frequencies to the stationary distribution}}%
\end{pgfscope}%
\begin{pgfscope}%
\pgfsetbuttcap%
\pgfsetmiterjoin%
\definecolor{currentfill}{rgb}{1.000000,1.000000,1.000000}%
\pgfsetfillcolor{currentfill}%
\pgfsetfillopacity{0.800000}%
\pgfsetlinewidth{1.003750pt}%
\definecolor{currentstroke}{rgb}{0.800000,0.800000,0.800000}%
\pgfsetstrokecolor{currentstroke}%
\pgfsetstrokeopacity{0.800000}%
\pgfsetdash{}{0pt}%
\pgfpathmoveto{\pgfqpoint{4.098702in}{2.544621in}}%
\pgfpathlineto{\pgfqpoint{6.630125in}{2.544621in}}%
\pgfpathquadraticcurveto{\pgfqpoint{6.653263in}{2.544621in}}{\pgfqpoint{6.653263in}{2.567760in}}%
\pgfpathlineto{\pgfqpoint{6.653263in}{3.569940in}}%
\pgfpathquadraticcurveto{\pgfqpoint{6.653263in}{3.593079in}}{\pgfqpoint{6.630125in}{3.593079in}}%
\pgfpathlineto{\pgfqpoint{4.098702in}{3.593079in}}%
\pgfpathquadraticcurveto{\pgfqpoint{4.075563in}{3.593079in}}{\pgfqpoint{4.075563in}{3.569940in}}%
\pgfpathlineto{\pgfqpoint{4.075563in}{2.567760in}}%
\pgfpathquadraticcurveto{\pgfqpoint{4.075563in}{2.544621in}}{\pgfqpoint{4.098702in}{2.544621in}}%
\pgfpathlineto{\pgfqpoint{4.098702in}{2.544621in}}%
\pgfpathclose%
\pgfusepath{stroke,fill}%
\end{pgfscope}%
\begin{pgfscope}%
\pgfsetrectcap%
\pgfsetroundjoin%
\pgfsetlinewidth{1.505625pt}%
\definecolor{currentstroke}{rgb}{0.121569,0.466667,0.705882}%
\pgfsetstrokecolor{currentstroke}%
\pgfsetdash{}{0pt}%
\pgfpathmoveto{\pgfqpoint{4.121841in}{3.503961in}}%
\pgfpathlineto{\pgfqpoint{4.237536in}{3.503961in}}%
\pgfpathlineto{\pgfqpoint{4.353230in}{3.503961in}}%
\pgfusepath{stroke}%
\end{pgfscope}%
\begin{pgfscope}%
\definecolor{textcolor}{rgb}{0.000000,0.000000,0.000000}%
\pgfsetstrokecolor{textcolor}%
\pgfsetfillcolor{textcolor}%
\pgftext[x=4.445786in,y=3.463468in,left,base]{\color{textcolor}{\rmfamily\fontsize{8.330000}{9.996000}\selectfont\catcode`\^=\active\def^{\ifmmode\sp\else\^{}\fi}\catcode`\%=\active\def%{\%}State 1 (low engagement) (empirical)}}%
\end{pgfscope}%
\begin{pgfscope}%
\pgfsetbuttcap%
\pgfsetroundjoin%
\pgfsetlinewidth{1.505625pt}%
\definecolor{currentstroke}{rgb}{0.121569,0.466667,0.705882}%
\pgfsetstrokecolor{currentstroke}%
\pgfsetstrokeopacity{0.700000}%
\pgfsetdash{{5.550000pt}{2.400000pt}}{0.000000pt}%
\pgfpathmoveto{\pgfqpoint{4.121841in}{3.335003in}}%
\pgfpathlineto{\pgfqpoint{4.353230in}{3.335003in}}%
\pgfusepath{stroke}%
\end{pgfscope}%
\begin{pgfscope}%
\definecolor{textcolor}{rgb}{0.000000,0.000000,0.000000}%
\pgfsetstrokecolor{textcolor}%
\pgfsetfillcolor{textcolor}%
\pgftext[x=4.445786in,y=3.294510in,left,base]{\color{textcolor}{\rmfamily\fontsize{8.330000}{9.996000}\selectfont\catcode`\^=\active\def^{\ifmmode\sp\else\^{}\fi}\catcode`\%=\active\def%{\%}$\pi_1 = 0.33$ (theoretical)}}%
\end{pgfscope}%
\begin{pgfscope}%
\pgfsetrectcap%
\pgfsetroundjoin%
\pgfsetlinewidth{1.505625pt}%
\definecolor{currentstroke}{rgb}{1.000000,0.498039,0.054902}%
\pgfsetstrokecolor{currentstroke}%
\pgfsetdash{}{0pt}%
\pgfpathmoveto{\pgfqpoint{4.121841in}{3.166044in}}%
\pgfpathlineto{\pgfqpoint{4.237536in}{3.166044in}}%
\pgfpathlineto{\pgfqpoint{4.353230in}{3.166044in}}%
\pgfusepath{stroke}%
\end{pgfscope}%
\begin{pgfscope}%
\definecolor{textcolor}{rgb}{0.000000,0.000000,0.000000}%
\pgfsetstrokecolor{textcolor}%
\pgfsetfillcolor{textcolor}%
\pgftext[x=4.445786in,y=3.125551in,left,base]{\color{textcolor}{\rmfamily\fontsize{8.330000}{9.996000}\selectfont\catcode`\^=\active\def^{\ifmmode\sp\else\^{}\fi}\catcode`\%=\active\def%{\%}State 2 (medium engagement) (empirical)}}%
\end{pgfscope}%
\begin{pgfscope}%
\pgfsetbuttcap%
\pgfsetroundjoin%
\pgfsetlinewidth{1.505625pt}%
\definecolor{currentstroke}{rgb}{1.000000,0.498039,0.054902}%
\pgfsetstrokecolor{currentstroke}%
\pgfsetstrokeopacity{0.700000}%
\pgfsetdash{{5.550000pt}{2.400000pt}}{0.000000pt}%
\pgfpathmoveto{\pgfqpoint{4.121841in}{2.997086in}}%
\pgfpathlineto{\pgfqpoint{4.353230in}{2.997086in}}%
\pgfusepath{stroke}%
\end{pgfscope}%
\begin{pgfscope}%
\definecolor{textcolor}{rgb}{0.000000,0.000000,0.000000}%
\pgfsetstrokecolor{textcolor}%
\pgfsetfillcolor{textcolor}%
\pgftext[x=4.445786in,y=2.956593in,left,base]{\color{textcolor}{\rmfamily\fontsize{8.330000}{9.996000}\selectfont\catcode`\^=\active\def^{\ifmmode\sp\else\^{}\fi}\catcode`\%=\active\def%{\%}$\pi_2 = 0.33$ (theoretical)}}%
\end{pgfscope}%
\begin{pgfscope}%
\pgfsetrectcap%
\pgfsetroundjoin%
\pgfsetlinewidth{1.505625pt}%
\definecolor{currentstroke}{rgb}{0.172549,0.627451,0.172549}%
\pgfsetstrokecolor{currentstroke}%
\pgfsetdash{}{0pt}%
\pgfpathmoveto{\pgfqpoint{4.121841in}{2.828128in}}%
\pgfpathlineto{\pgfqpoint{4.237536in}{2.828128in}}%
\pgfpathlineto{\pgfqpoint{4.353230in}{2.828128in}}%
\pgfusepath{stroke}%
\end{pgfscope}%
\begin{pgfscope}%
\definecolor{textcolor}{rgb}{0.000000,0.000000,0.000000}%
\pgfsetstrokecolor{textcolor}%
\pgfsetfillcolor{textcolor}%
\pgftext[x=4.445786in,y=2.787635in,left,base]{\color{textcolor}{\rmfamily\fontsize{8.330000}{9.996000}\selectfont\catcode`\^=\active\def^{\ifmmode\sp\else\^{}\fi}\catcode`\%=\active\def%{\%}State 3 (high engagement) (empirical)}}%
\end{pgfscope}%
\begin{pgfscope}%
\pgfsetbuttcap%
\pgfsetroundjoin%
\pgfsetlinewidth{1.505625pt}%
\definecolor{currentstroke}{rgb}{0.172549,0.627451,0.172549}%
\pgfsetstrokecolor{currentstroke}%
\pgfsetstrokeopacity{0.700000}%
\pgfsetdash{{5.550000pt}{2.400000pt}}{0.000000pt}%
\pgfpathmoveto{\pgfqpoint{4.121841in}{2.659169in}}%
\pgfpathlineto{\pgfqpoint{4.353230in}{2.659169in}}%
\pgfusepath{stroke}%
\end{pgfscope}%
\begin{pgfscope}%
\definecolor{textcolor}{rgb}{0.000000,0.000000,0.000000}%
\pgfsetstrokecolor{textcolor}%
\pgfsetfillcolor{textcolor}%
\pgftext[x=4.445786in,y=2.618676in,left,base]{\color{textcolor}{\rmfamily\fontsize{8.330000}{9.996000}\selectfont\catcode`\^=\active\def^{\ifmmode\sp\else\^{}\fi}\catcode`\%=\active\def%{\%}$\pi_3 = 0.33$ (theoretical)}}%
\end{pgfscope}%
\end{pgfpicture}%
\makeatother%
\endgroup%

  }
  \caption{\label{fig:markov}Simulation of the three–state Markov chain (t=0,\ldots, 1000)}
\end{figure}

\begin{figure}
  \centering
  \resizebox{\textwidth}{!}{%
    %% Creator: Matplotlib, PGF backend
%%
%% To include the figure in your LaTeX document, write
%%   \input{<filename>.pgf}
%%
%% Make sure the required packages are loaded in your preamble
%%   \usepackage{pgf}
%%
%% Also ensure that all the required font packages are loaded; for instance,
%% the lmodern package is sometimes necessary when using math font.
%%   \usepackage{lmodern}
%%
%% Figures using additional raster images can only be included by \input if
%% they are in the same directory as the main LaTeX file. For loading figures
%% from other directories you can use the `import` package
%%   \usepackage{import}
%%
%% and then include the figures with
%%   \import{<path to file>}{<filename>.pgf}
%%
%% Matplotlib used the following preamble
%%   \def\mathdefault#1{#1}
%%   \everymath=\expandafter{\the\everymath\displaystyle}
%%   \IfFileExists{scrextend.sty}{
%%     \usepackage[fontsize=10.000000pt]{scrextend}
%%   }{
%%     \renewcommand{\normalsize}{\fontsize{10.000000}{12.000000}\selectfont}
%%     \normalsize
%%   }
%%   
%%   \ifdefined\pdftexversion\else  % non-pdftex case.
%%     \usepackage{fontspec}
%%     \setmainfont{DejaVuSerif.ttf}[Path=\detokenize{/Library/Frameworks/Python.framework/Versions/3.13/lib/python3.13/site-packages/matplotlib/mpl-data/fonts/ttf/}]
%%     \setsansfont{DejaVuSans.ttf}[Path=\detokenize{/Library/Frameworks/Python.framework/Versions/3.13/lib/python3.13/site-packages/matplotlib/mpl-data/fonts/ttf/}]
%%     \setmonofont{DejaVuSansMono.ttf}[Path=\detokenize{/Library/Frameworks/Python.framework/Versions/3.13/lib/python3.13/site-packages/matplotlib/mpl-data/fonts/ttf/}]
%%   \fi
%%   \makeatletter\@ifpackageloaded{underscore}{}{\usepackage[strings]{underscore}}\makeatother
%%
\begingroup%
\makeatletter%
\begin{pgfpicture}%
\pgfpathrectangle{\pgfpointorigin}{\pgfqpoint{7.000000in}{4.000000in}}%
\pgfusepath{use as bounding box, clip}%
\begin{pgfscope}%
\pgfsetbuttcap%
\pgfsetmiterjoin%
\definecolor{currentfill}{rgb}{1.000000,1.000000,1.000000}%
\pgfsetfillcolor{currentfill}%
\pgfsetlinewidth{0.000000pt}%
\definecolor{currentstroke}{rgb}{1.000000,1.000000,1.000000}%
\pgfsetstrokecolor{currentstroke}%
\pgfsetdash{}{0pt}%
\pgfpathmoveto{\pgfqpoint{0.000000in}{0.000000in}}%
\pgfpathlineto{\pgfqpoint{7.000000in}{0.000000in}}%
\pgfpathlineto{\pgfqpoint{7.000000in}{4.000000in}}%
\pgfpathlineto{\pgfqpoint{0.000000in}{4.000000in}}%
\pgfpathlineto{\pgfqpoint{0.000000in}{0.000000in}}%
\pgfpathclose%
\pgfusepath{fill}%
\end{pgfscope}%
\begin{pgfscope}%
\pgfsetbuttcap%
\pgfsetmiterjoin%
\definecolor{currentfill}{rgb}{1.000000,1.000000,1.000000}%
\pgfsetfillcolor{currentfill}%
\pgfsetlinewidth{0.000000pt}%
\definecolor{currentstroke}{rgb}{0.000000,0.000000,0.000000}%
\pgfsetstrokecolor{currentstroke}%
\pgfsetstrokeopacity{0.000000}%
\pgfsetdash{}{0pt}%
\pgfpathmoveto{\pgfqpoint{0.603704in}{0.549691in}}%
\pgfpathlineto{\pgfqpoint{6.711111in}{0.549691in}}%
\pgfpathlineto{\pgfqpoint{6.711111in}{3.650926in}}%
\pgfpathlineto{\pgfqpoint{0.603704in}{3.650926in}}%
\pgfpathlineto{\pgfqpoint{0.603704in}{0.549691in}}%
\pgfpathclose%
\pgfusepath{fill}%
\end{pgfscope}%
\begin{pgfscope}%
\pgfpathrectangle{\pgfqpoint{0.603704in}{0.549691in}}{\pgfqpoint{6.107407in}{3.101235in}}%
\pgfusepath{clip}%
\pgfsetrectcap%
\pgfsetroundjoin%
\pgfsetlinewidth{0.803000pt}%
\definecolor{currentstroke}{rgb}{0.690196,0.690196,0.690196}%
\pgfsetstrokecolor{currentstroke}%
\pgfsetstrokeopacity{0.300000}%
\pgfsetdash{}{0pt}%
\pgfpathmoveto{\pgfqpoint{1.820295in}{0.549691in}}%
\pgfpathlineto{\pgfqpoint{1.820295in}{3.650926in}}%
\pgfusepath{stroke}%
\end{pgfscope}%
\begin{pgfscope}%
\pgfsetbuttcap%
\pgfsetroundjoin%
\definecolor{currentfill}{rgb}{0.000000,0.000000,0.000000}%
\pgfsetfillcolor{currentfill}%
\pgfsetlinewidth{0.803000pt}%
\definecolor{currentstroke}{rgb}{0.000000,0.000000,0.000000}%
\pgfsetstrokecolor{currentstroke}%
\pgfsetdash{}{0pt}%
\pgfsys@defobject{currentmarker}{\pgfqpoint{0.000000in}{-0.048611in}}{\pgfqpoint{0.000000in}{0.000000in}}{%
\pgfpathmoveto{\pgfqpoint{0.000000in}{0.000000in}}%
\pgfpathlineto{\pgfqpoint{0.000000in}{-0.048611in}}%
\pgfusepath{stroke,fill}%
}%
\begin{pgfscope}%
\pgfsys@transformshift{1.820295in}{0.549691in}%
\pgfsys@useobject{currentmarker}{}%
\end{pgfscope}%
\end{pgfscope}%
\begin{pgfscope}%
\definecolor{textcolor}{rgb}{0.000000,0.000000,0.000000}%
\pgfsetstrokecolor{textcolor}%
\pgfsetfillcolor{textcolor}%
\pgftext[x=1.820295in,y=0.452469in,,top]{\color{textcolor}{\rmfamily\fontsize{10.000000}{12.000000}\selectfont\catcode`\^=\active\def^{\ifmmode\sp\else\^{}\fi}\catcode`\%=\active\def%{\%}$\mathdefault{1200}$}}%
\end{pgfscope}%
\begin{pgfscope}%
\pgfpathrectangle{\pgfqpoint{0.603704in}{0.549691in}}{\pgfqpoint{6.107407in}{3.101235in}}%
\pgfusepath{clip}%
\pgfsetrectcap%
\pgfsetroundjoin%
\pgfsetlinewidth{0.803000pt}%
\definecolor{currentstroke}{rgb}{0.690196,0.690196,0.690196}%
\pgfsetstrokecolor{currentstroke}%
\pgfsetstrokeopacity{0.300000}%
\pgfsetdash{}{0pt}%
\pgfpathmoveto{\pgfqpoint{3.042999in}{0.549691in}}%
\pgfpathlineto{\pgfqpoint{3.042999in}{3.650926in}}%
\pgfusepath{stroke}%
\end{pgfscope}%
\begin{pgfscope}%
\pgfsetbuttcap%
\pgfsetroundjoin%
\definecolor{currentfill}{rgb}{0.000000,0.000000,0.000000}%
\pgfsetfillcolor{currentfill}%
\pgfsetlinewidth{0.803000pt}%
\definecolor{currentstroke}{rgb}{0.000000,0.000000,0.000000}%
\pgfsetstrokecolor{currentstroke}%
\pgfsetdash{}{0pt}%
\pgfsys@defobject{currentmarker}{\pgfqpoint{0.000000in}{-0.048611in}}{\pgfqpoint{0.000000in}{0.000000in}}{%
\pgfpathmoveto{\pgfqpoint{0.000000in}{0.000000in}}%
\pgfpathlineto{\pgfqpoint{0.000000in}{-0.048611in}}%
\pgfusepath{stroke,fill}%
}%
\begin{pgfscope}%
\pgfsys@transformshift{3.042999in}{0.549691in}%
\pgfsys@useobject{currentmarker}{}%
\end{pgfscope}%
\end{pgfscope}%
\begin{pgfscope}%
\definecolor{textcolor}{rgb}{0.000000,0.000000,0.000000}%
\pgfsetstrokecolor{textcolor}%
\pgfsetfillcolor{textcolor}%
\pgftext[x=3.042999in,y=0.452469in,,top]{\color{textcolor}{\rmfamily\fontsize{10.000000}{12.000000}\selectfont\catcode`\^=\active\def^{\ifmmode\sp\else\^{}\fi}\catcode`\%=\active\def%{\%}$\mathdefault{1400}$}}%
\end{pgfscope}%
\begin{pgfscope}%
\pgfpathrectangle{\pgfqpoint{0.603704in}{0.549691in}}{\pgfqpoint{6.107407in}{3.101235in}}%
\pgfusepath{clip}%
\pgfsetrectcap%
\pgfsetroundjoin%
\pgfsetlinewidth{0.803000pt}%
\definecolor{currentstroke}{rgb}{0.690196,0.690196,0.690196}%
\pgfsetstrokecolor{currentstroke}%
\pgfsetstrokeopacity{0.300000}%
\pgfsetdash{}{0pt}%
\pgfpathmoveto{\pgfqpoint{4.265703in}{0.549691in}}%
\pgfpathlineto{\pgfqpoint{4.265703in}{3.650926in}}%
\pgfusepath{stroke}%
\end{pgfscope}%
\begin{pgfscope}%
\pgfsetbuttcap%
\pgfsetroundjoin%
\definecolor{currentfill}{rgb}{0.000000,0.000000,0.000000}%
\pgfsetfillcolor{currentfill}%
\pgfsetlinewidth{0.803000pt}%
\definecolor{currentstroke}{rgb}{0.000000,0.000000,0.000000}%
\pgfsetstrokecolor{currentstroke}%
\pgfsetdash{}{0pt}%
\pgfsys@defobject{currentmarker}{\pgfqpoint{0.000000in}{-0.048611in}}{\pgfqpoint{0.000000in}{0.000000in}}{%
\pgfpathmoveto{\pgfqpoint{0.000000in}{0.000000in}}%
\pgfpathlineto{\pgfqpoint{0.000000in}{-0.048611in}}%
\pgfusepath{stroke,fill}%
}%
\begin{pgfscope}%
\pgfsys@transformshift{4.265703in}{0.549691in}%
\pgfsys@useobject{currentmarker}{}%
\end{pgfscope}%
\end{pgfscope}%
\begin{pgfscope}%
\definecolor{textcolor}{rgb}{0.000000,0.000000,0.000000}%
\pgfsetstrokecolor{textcolor}%
\pgfsetfillcolor{textcolor}%
\pgftext[x=4.265703in,y=0.452469in,,top]{\color{textcolor}{\rmfamily\fontsize{10.000000}{12.000000}\selectfont\catcode`\^=\active\def^{\ifmmode\sp\else\^{}\fi}\catcode`\%=\active\def%{\%}$\mathdefault{1600}$}}%
\end{pgfscope}%
\begin{pgfscope}%
\pgfpathrectangle{\pgfqpoint{0.603704in}{0.549691in}}{\pgfqpoint{6.107407in}{3.101235in}}%
\pgfusepath{clip}%
\pgfsetrectcap%
\pgfsetroundjoin%
\pgfsetlinewidth{0.803000pt}%
\definecolor{currentstroke}{rgb}{0.690196,0.690196,0.690196}%
\pgfsetstrokecolor{currentstroke}%
\pgfsetstrokeopacity{0.300000}%
\pgfsetdash{}{0pt}%
\pgfpathmoveto{\pgfqpoint{5.488407in}{0.549691in}}%
\pgfpathlineto{\pgfqpoint{5.488407in}{3.650926in}}%
\pgfusepath{stroke}%
\end{pgfscope}%
\begin{pgfscope}%
\pgfsetbuttcap%
\pgfsetroundjoin%
\definecolor{currentfill}{rgb}{0.000000,0.000000,0.000000}%
\pgfsetfillcolor{currentfill}%
\pgfsetlinewidth{0.803000pt}%
\definecolor{currentstroke}{rgb}{0.000000,0.000000,0.000000}%
\pgfsetstrokecolor{currentstroke}%
\pgfsetdash{}{0pt}%
\pgfsys@defobject{currentmarker}{\pgfqpoint{0.000000in}{-0.048611in}}{\pgfqpoint{0.000000in}{0.000000in}}{%
\pgfpathmoveto{\pgfqpoint{0.000000in}{0.000000in}}%
\pgfpathlineto{\pgfqpoint{0.000000in}{-0.048611in}}%
\pgfusepath{stroke,fill}%
}%
\begin{pgfscope}%
\pgfsys@transformshift{5.488407in}{0.549691in}%
\pgfsys@useobject{currentmarker}{}%
\end{pgfscope}%
\end{pgfscope}%
\begin{pgfscope}%
\definecolor{textcolor}{rgb}{0.000000,0.000000,0.000000}%
\pgfsetstrokecolor{textcolor}%
\pgfsetfillcolor{textcolor}%
\pgftext[x=5.488407in,y=0.452469in,,top]{\color{textcolor}{\rmfamily\fontsize{10.000000}{12.000000}\selectfont\catcode`\^=\active\def^{\ifmmode\sp\else\^{}\fi}\catcode`\%=\active\def%{\%}$\mathdefault{1800}$}}%
\end{pgfscope}%
\begin{pgfscope}%
\pgfpathrectangle{\pgfqpoint{0.603704in}{0.549691in}}{\pgfqpoint{6.107407in}{3.101235in}}%
\pgfusepath{clip}%
\pgfsetrectcap%
\pgfsetroundjoin%
\pgfsetlinewidth{0.803000pt}%
\definecolor{currentstroke}{rgb}{0.690196,0.690196,0.690196}%
\pgfsetstrokecolor{currentstroke}%
\pgfsetstrokeopacity{0.300000}%
\pgfsetdash{}{0pt}%
\pgfpathmoveto{\pgfqpoint{6.711111in}{0.549691in}}%
\pgfpathlineto{\pgfqpoint{6.711111in}{3.650926in}}%
\pgfusepath{stroke}%
\end{pgfscope}%
\begin{pgfscope}%
\pgfsetbuttcap%
\pgfsetroundjoin%
\definecolor{currentfill}{rgb}{0.000000,0.000000,0.000000}%
\pgfsetfillcolor{currentfill}%
\pgfsetlinewidth{0.803000pt}%
\definecolor{currentstroke}{rgb}{0.000000,0.000000,0.000000}%
\pgfsetstrokecolor{currentstroke}%
\pgfsetdash{}{0pt}%
\pgfsys@defobject{currentmarker}{\pgfqpoint{0.000000in}{-0.048611in}}{\pgfqpoint{0.000000in}{0.000000in}}{%
\pgfpathmoveto{\pgfqpoint{0.000000in}{0.000000in}}%
\pgfpathlineto{\pgfqpoint{0.000000in}{-0.048611in}}%
\pgfusepath{stroke,fill}%
}%
\begin{pgfscope}%
\pgfsys@transformshift{6.711111in}{0.549691in}%
\pgfsys@useobject{currentmarker}{}%
\end{pgfscope}%
\end{pgfscope}%
\begin{pgfscope}%
\definecolor{textcolor}{rgb}{0.000000,0.000000,0.000000}%
\pgfsetstrokecolor{textcolor}%
\pgfsetfillcolor{textcolor}%
\pgftext[x=6.711111in,y=0.452469in,,top]{\color{textcolor}{\rmfamily\fontsize{10.000000}{12.000000}\selectfont\catcode`\^=\active\def^{\ifmmode\sp\else\^{}\fi}\catcode`\%=\active\def%{\%}$\mathdefault{2000}$}}%
\end{pgfscope}%
\begin{pgfscope}%
\definecolor{textcolor}{rgb}{0.000000,0.000000,0.000000}%
\pgfsetstrokecolor{textcolor}%
\pgfsetfillcolor{textcolor}%
\pgftext[x=3.657407in,y=0.273457in,,top]{\color{textcolor}{\rmfamily\fontsize{10.000000}{12.000000}\selectfont\catcode`\^=\active\def^{\ifmmode\sp\else\^{}\fi}\catcode`\%=\active\def%{\%}Time step $t$}}%
\end{pgfscope}%
\begin{pgfscope}%
\pgfpathrectangle{\pgfqpoint{0.603704in}{0.549691in}}{\pgfqpoint{6.107407in}{3.101235in}}%
\pgfusepath{clip}%
\pgfsetrectcap%
\pgfsetroundjoin%
\pgfsetlinewidth{0.803000pt}%
\definecolor{currentstroke}{rgb}{0.690196,0.690196,0.690196}%
\pgfsetstrokecolor{currentstroke}%
\pgfsetstrokeopacity{0.300000}%
\pgfsetdash{}{0pt}%
\pgfpathmoveto{\pgfqpoint{0.603704in}{0.549691in}}%
\pgfpathlineto{\pgfqpoint{6.711111in}{0.549691in}}%
\pgfusepath{stroke}%
\end{pgfscope}%
\begin{pgfscope}%
\pgfsetbuttcap%
\pgfsetroundjoin%
\definecolor{currentfill}{rgb}{0.000000,0.000000,0.000000}%
\pgfsetfillcolor{currentfill}%
\pgfsetlinewidth{0.803000pt}%
\definecolor{currentstroke}{rgb}{0.000000,0.000000,0.000000}%
\pgfsetstrokecolor{currentstroke}%
\pgfsetdash{}{0pt}%
\pgfsys@defobject{currentmarker}{\pgfqpoint{-0.048611in}{0.000000in}}{\pgfqpoint{-0.000000in}{0.000000in}}{%
\pgfpathmoveto{\pgfqpoint{-0.000000in}{0.000000in}}%
\pgfpathlineto{\pgfqpoint{-0.048611in}{0.000000in}}%
\pgfusepath{stroke,fill}%
}%
\begin{pgfscope}%
\pgfsys@transformshift{0.603704in}{0.549691in}%
\pgfsys@useobject{currentmarker}{}%
\end{pgfscope}%
\end{pgfscope}%
\begin{pgfscope}%
\definecolor{textcolor}{rgb}{0.000000,0.000000,0.000000}%
\pgfsetstrokecolor{textcolor}%
\pgfsetfillcolor{textcolor}%
\pgftext[x=0.329012in, y=0.501466in, left, base]{\color{textcolor}{\rmfamily\fontsize{10.000000}{12.000000}\selectfont\catcode`\^=\active\def^{\ifmmode\sp\else\^{}\fi}\catcode`\%=\active\def%{\%}$\mathdefault{0.0}$}}%
\end{pgfscope}%
\begin{pgfscope}%
\pgfpathrectangle{\pgfqpoint{0.603704in}{0.549691in}}{\pgfqpoint{6.107407in}{3.101235in}}%
\pgfusepath{clip}%
\pgfsetrectcap%
\pgfsetroundjoin%
\pgfsetlinewidth{0.803000pt}%
\definecolor{currentstroke}{rgb}{0.690196,0.690196,0.690196}%
\pgfsetstrokecolor{currentstroke}%
\pgfsetstrokeopacity{0.300000}%
\pgfsetdash{}{0pt}%
\pgfpathmoveto{\pgfqpoint{0.603704in}{1.169938in}}%
\pgfpathlineto{\pgfqpoint{6.711111in}{1.169938in}}%
\pgfusepath{stroke}%
\end{pgfscope}%
\begin{pgfscope}%
\pgfsetbuttcap%
\pgfsetroundjoin%
\definecolor{currentfill}{rgb}{0.000000,0.000000,0.000000}%
\pgfsetfillcolor{currentfill}%
\pgfsetlinewidth{0.803000pt}%
\definecolor{currentstroke}{rgb}{0.000000,0.000000,0.000000}%
\pgfsetstrokecolor{currentstroke}%
\pgfsetdash{}{0pt}%
\pgfsys@defobject{currentmarker}{\pgfqpoint{-0.048611in}{0.000000in}}{\pgfqpoint{-0.000000in}{0.000000in}}{%
\pgfpathmoveto{\pgfqpoint{-0.000000in}{0.000000in}}%
\pgfpathlineto{\pgfqpoint{-0.048611in}{0.000000in}}%
\pgfusepath{stroke,fill}%
}%
\begin{pgfscope}%
\pgfsys@transformshift{0.603704in}{1.169938in}%
\pgfsys@useobject{currentmarker}{}%
\end{pgfscope}%
\end{pgfscope}%
\begin{pgfscope}%
\definecolor{textcolor}{rgb}{0.000000,0.000000,0.000000}%
\pgfsetstrokecolor{textcolor}%
\pgfsetfillcolor{textcolor}%
\pgftext[x=0.329012in, y=1.121713in, left, base]{\color{textcolor}{\rmfamily\fontsize{10.000000}{12.000000}\selectfont\catcode`\^=\active\def^{\ifmmode\sp\else\^{}\fi}\catcode`\%=\active\def%{\%}$\mathdefault{0.2}$}}%
\end{pgfscope}%
\begin{pgfscope}%
\pgfpathrectangle{\pgfqpoint{0.603704in}{0.549691in}}{\pgfqpoint{6.107407in}{3.101235in}}%
\pgfusepath{clip}%
\pgfsetrectcap%
\pgfsetroundjoin%
\pgfsetlinewidth{0.803000pt}%
\definecolor{currentstroke}{rgb}{0.690196,0.690196,0.690196}%
\pgfsetstrokecolor{currentstroke}%
\pgfsetstrokeopacity{0.300000}%
\pgfsetdash{}{0pt}%
\pgfpathmoveto{\pgfqpoint{0.603704in}{1.790185in}}%
\pgfpathlineto{\pgfqpoint{6.711111in}{1.790185in}}%
\pgfusepath{stroke}%
\end{pgfscope}%
\begin{pgfscope}%
\pgfsetbuttcap%
\pgfsetroundjoin%
\definecolor{currentfill}{rgb}{0.000000,0.000000,0.000000}%
\pgfsetfillcolor{currentfill}%
\pgfsetlinewidth{0.803000pt}%
\definecolor{currentstroke}{rgb}{0.000000,0.000000,0.000000}%
\pgfsetstrokecolor{currentstroke}%
\pgfsetdash{}{0pt}%
\pgfsys@defobject{currentmarker}{\pgfqpoint{-0.048611in}{0.000000in}}{\pgfqpoint{-0.000000in}{0.000000in}}{%
\pgfpathmoveto{\pgfqpoint{-0.000000in}{0.000000in}}%
\pgfpathlineto{\pgfqpoint{-0.048611in}{0.000000in}}%
\pgfusepath{stroke,fill}%
}%
\begin{pgfscope}%
\pgfsys@transformshift{0.603704in}{1.790185in}%
\pgfsys@useobject{currentmarker}{}%
\end{pgfscope}%
\end{pgfscope}%
\begin{pgfscope}%
\definecolor{textcolor}{rgb}{0.000000,0.000000,0.000000}%
\pgfsetstrokecolor{textcolor}%
\pgfsetfillcolor{textcolor}%
\pgftext[x=0.329012in, y=1.741960in, left, base]{\color{textcolor}{\rmfamily\fontsize{10.000000}{12.000000}\selectfont\catcode`\^=\active\def^{\ifmmode\sp\else\^{}\fi}\catcode`\%=\active\def%{\%}$\mathdefault{0.4}$}}%
\end{pgfscope}%
\begin{pgfscope}%
\pgfpathrectangle{\pgfqpoint{0.603704in}{0.549691in}}{\pgfqpoint{6.107407in}{3.101235in}}%
\pgfusepath{clip}%
\pgfsetrectcap%
\pgfsetroundjoin%
\pgfsetlinewidth{0.803000pt}%
\definecolor{currentstroke}{rgb}{0.690196,0.690196,0.690196}%
\pgfsetstrokecolor{currentstroke}%
\pgfsetstrokeopacity{0.300000}%
\pgfsetdash{}{0pt}%
\pgfpathmoveto{\pgfqpoint{0.603704in}{2.410432in}}%
\pgfpathlineto{\pgfqpoint{6.711111in}{2.410432in}}%
\pgfusepath{stroke}%
\end{pgfscope}%
\begin{pgfscope}%
\pgfsetbuttcap%
\pgfsetroundjoin%
\definecolor{currentfill}{rgb}{0.000000,0.000000,0.000000}%
\pgfsetfillcolor{currentfill}%
\pgfsetlinewidth{0.803000pt}%
\definecolor{currentstroke}{rgb}{0.000000,0.000000,0.000000}%
\pgfsetstrokecolor{currentstroke}%
\pgfsetdash{}{0pt}%
\pgfsys@defobject{currentmarker}{\pgfqpoint{-0.048611in}{0.000000in}}{\pgfqpoint{-0.000000in}{0.000000in}}{%
\pgfpathmoveto{\pgfqpoint{-0.000000in}{0.000000in}}%
\pgfpathlineto{\pgfqpoint{-0.048611in}{0.000000in}}%
\pgfusepath{stroke,fill}%
}%
\begin{pgfscope}%
\pgfsys@transformshift{0.603704in}{2.410432in}%
\pgfsys@useobject{currentmarker}{}%
\end{pgfscope}%
\end{pgfscope}%
\begin{pgfscope}%
\definecolor{textcolor}{rgb}{0.000000,0.000000,0.000000}%
\pgfsetstrokecolor{textcolor}%
\pgfsetfillcolor{textcolor}%
\pgftext[x=0.329012in, y=2.362207in, left, base]{\color{textcolor}{\rmfamily\fontsize{10.000000}{12.000000}\selectfont\catcode`\^=\active\def^{\ifmmode\sp\else\^{}\fi}\catcode`\%=\active\def%{\%}$\mathdefault{0.6}$}}%
\end{pgfscope}%
\begin{pgfscope}%
\pgfpathrectangle{\pgfqpoint{0.603704in}{0.549691in}}{\pgfqpoint{6.107407in}{3.101235in}}%
\pgfusepath{clip}%
\pgfsetrectcap%
\pgfsetroundjoin%
\pgfsetlinewidth{0.803000pt}%
\definecolor{currentstroke}{rgb}{0.690196,0.690196,0.690196}%
\pgfsetstrokecolor{currentstroke}%
\pgfsetstrokeopacity{0.300000}%
\pgfsetdash{}{0pt}%
\pgfpathmoveto{\pgfqpoint{0.603704in}{3.030679in}}%
\pgfpathlineto{\pgfqpoint{6.711111in}{3.030679in}}%
\pgfusepath{stroke}%
\end{pgfscope}%
\begin{pgfscope}%
\pgfsetbuttcap%
\pgfsetroundjoin%
\definecolor{currentfill}{rgb}{0.000000,0.000000,0.000000}%
\pgfsetfillcolor{currentfill}%
\pgfsetlinewidth{0.803000pt}%
\definecolor{currentstroke}{rgb}{0.000000,0.000000,0.000000}%
\pgfsetstrokecolor{currentstroke}%
\pgfsetdash{}{0pt}%
\pgfsys@defobject{currentmarker}{\pgfqpoint{-0.048611in}{0.000000in}}{\pgfqpoint{-0.000000in}{0.000000in}}{%
\pgfpathmoveto{\pgfqpoint{-0.000000in}{0.000000in}}%
\pgfpathlineto{\pgfqpoint{-0.048611in}{0.000000in}}%
\pgfusepath{stroke,fill}%
}%
\begin{pgfscope}%
\pgfsys@transformshift{0.603704in}{3.030679in}%
\pgfsys@useobject{currentmarker}{}%
\end{pgfscope}%
\end{pgfscope}%
\begin{pgfscope}%
\definecolor{textcolor}{rgb}{0.000000,0.000000,0.000000}%
\pgfsetstrokecolor{textcolor}%
\pgfsetfillcolor{textcolor}%
\pgftext[x=0.329012in, y=2.982454in, left, base]{\color{textcolor}{\rmfamily\fontsize{10.000000}{12.000000}\selectfont\catcode`\^=\active\def^{\ifmmode\sp\else\^{}\fi}\catcode`\%=\active\def%{\%}$\mathdefault{0.8}$}}%
\end{pgfscope}%
\begin{pgfscope}%
\pgfpathrectangle{\pgfqpoint{0.603704in}{0.549691in}}{\pgfqpoint{6.107407in}{3.101235in}}%
\pgfusepath{clip}%
\pgfsetrectcap%
\pgfsetroundjoin%
\pgfsetlinewidth{0.803000pt}%
\definecolor{currentstroke}{rgb}{0.690196,0.690196,0.690196}%
\pgfsetstrokecolor{currentstroke}%
\pgfsetstrokeopacity{0.300000}%
\pgfsetdash{}{0pt}%
\pgfpathmoveto{\pgfqpoint{0.603704in}{3.650926in}}%
\pgfpathlineto{\pgfqpoint{6.711111in}{3.650926in}}%
\pgfusepath{stroke}%
\end{pgfscope}%
\begin{pgfscope}%
\pgfsetbuttcap%
\pgfsetroundjoin%
\definecolor{currentfill}{rgb}{0.000000,0.000000,0.000000}%
\pgfsetfillcolor{currentfill}%
\pgfsetlinewidth{0.803000pt}%
\definecolor{currentstroke}{rgb}{0.000000,0.000000,0.000000}%
\pgfsetstrokecolor{currentstroke}%
\pgfsetdash{}{0pt}%
\pgfsys@defobject{currentmarker}{\pgfqpoint{-0.048611in}{0.000000in}}{\pgfqpoint{-0.000000in}{0.000000in}}{%
\pgfpathmoveto{\pgfqpoint{-0.000000in}{0.000000in}}%
\pgfpathlineto{\pgfqpoint{-0.048611in}{0.000000in}}%
\pgfusepath{stroke,fill}%
}%
\begin{pgfscope}%
\pgfsys@transformshift{0.603704in}{3.650926in}%
\pgfsys@useobject{currentmarker}{}%
\end{pgfscope}%
\end{pgfscope}%
\begin{pgfscope}%
\definecolor{textcolor}{rgb}{0.000000,0.000000,0.000000}%
\pgfsetstrokecolor{textcolor}%
\pgfsetfillcolor{textcolor}%
\pgftext[x=0.329012in, y=3.602701in, left, base]{\color{textcolor}{\rmfamily\fontsize{10.000000}{12.000000}\selectfont\catcode`\^=\active\def^{\ifmmode\sp\else\^{}\fi}\catcode`\%=\active\def%{\%}$\mathdefault{1.0}$}}%
\end{pgfscope}%
\begin{pgfscope}%
\definecolor{textcolor}{rgb}{0.000000,0.000000,0.000000}%
\pgfsetstrokecolor{textcolor}%
\pgfsetfillcolor{textcolor}%
\pgftext[x=0.273457in,y=2.100309in,,bottom,rotate=90.000000]{\color{textcolor}{\rmfamily\fontsize{10.000000}{12.000000}\selectfont\catcode`\^=\active\def^{\ifmmode\sp\else\^{}\fi}\catcode`\%=\active\def%{\%}Empirical frequency}}%
\end{pgfscope}%
\begin{pgfscope}%
\pgfpathrectangle{\pgfqpoint{0.603704in}{0.549691in}}{\pgfqpoint{6.107407in}{3.101235in}}%
\pgfusepath{clip}%
\pgfsetrectcap%
\pgfsetroundjoin%
\pgfsetlinewidth{1.505625pt}%
\definecolor{currentstroke}{rgb}{0.121569,0.466667,0.705882}%
\pgfsetstrokecolor{currentstroke}%
\pgfsetdash{}{0pt}%
\pgfpathmoveto{\pgfqpoint{0.603704in}{1.550389in}}%
\pgfpathlineto{\pgfqpoint{0.609818in}{1.549391in}}%
\pgfpathlineto{\pgfqpoint{0.622045in}{1.553577in}}%
\pgfpathlineto{\pgfqpoint{0.640385in}{1.550586in}}%
\pgfpathlineto{\pgfqpoint{0.670953in}{1.560963in}}%
\pgfpathlineto{\pgfqpoint{0.719861in}{1.553032in}}%
\pgfpathlineto{\pgfqpoint{0.725975in}{1.555087in}}%
\pgfpathlineto{\pgfqpoint{0.915494in}{1.525460in}}%
\pgfpathlineto{\pgfqpoint{0.927721in}{1.529493in}}%
\pgfpathlineto{\pgfqpoint{0.939948in}{1.527637in}}%
\pgfpathlineto{\pgfqpoint{0.946061in}{1.529646in}}%
\pgfpathlineto{\pgfqpoint{0.958288in}{1.527795in}}%
\pgfpathlineto{\pgfqpoint{0.988856in}{1.537773in}}%
\pgfpathlineto{\pgfqpoint{1.007196in}{1.534994in}}%
\pgfpathlineto{\pgfqpoint{1.025537in}{1.540927in}}%
\pgfpathlineto{\pgfqpoint{1.215056in}{1.513018in}}%
\pgfpathlineto{\pgfqpoint{1.227283in}{1.516894in}}%
\pgfpathlineto{\pgfqpoint{1.282305in}{1.509066in}}%
\pgfpathlineto{\pgfqpoint{1.294532in}{1.512911in}}%
\pgfpathlineto{\pgfqpoint{1.331213in}{1.507751in}}%
\pgfpathlineto{\pgfqpoint{1.343440in}{1.511572in}}%
\pgfpathlineto{\pgfqpoint{1.355667in}{1.509860in}}%
\pgfpathlineto{\pgfqpoint{1.361781in}{1.511763in}}%
\pgfpathlineto{\pgfqpoint{1.367894in}{1.510909in}}%
\pgfpathlineto{\pgfqpoint{1.380121in}{1.514703in}}%
\pgfpathlineto{\pgfqpoint{1.416802in}{1.509597in}}%
\pgfpathlineto{\pgfqpoint{1.429029in}{1.513367in}}%
\pgfpathlineto{\pgfqpoint{1.447370in}{1.510829in}}%
\pgfpathlineto{\pgfqpoint{1.471824in}{1.518318in}}%
\pgfpathlineto{\pgfqpoint{1.496278in}{1.514940in}}%
\pgfpathlineto{\pgfqpoint{1.502392in}{1.516801in}}%
\pgfpathlineto{\pgfqpoint{1.539073in}{1.511773in}}%
\pgfpathlineto{\pgfqpoint{1.545186in}{1.513625in}}%
\pgfpathlineto{\pgfqpoint{1.563527in}{1.511128in}}%
\pgfpathlineto{\pgfqpoint{1.587981in}{1.518493in}}%
\pgfpathlineto{\pgfqpoint{1.600208in}{1.516829in}}%
\pgfpathlineto{\pgfqpoint{1.624662in}{1.524137in}}%
\pgfpathlineto{\pgfqpoint{1.649116in}{1.520812in}}%
\pgfpathlineto{\pgfqpoint{1.655230in}{1.522628in}}%
\pgfpathlineto{\pgfqpoint{1.667457in}{1.520972in}}%
\pgfpathlineto{\pgfqpoint{1.673570in}{1.522783in}}%
\pgfpathlineto{\pgfqpoint{1.691911in}{1.520307in}}%
\pgfpathlineto{\pgfqpoint{1.746932in}{1.536448in}}%
\pgfpathlineto{\pgfqpoint{1.832522in}{1.524955in}}%
\pgfpathlineto{\pgfqpoint{1.881430in}{1.539011in}}%
\pgfpathlineto{\pgfqpoint{1.911997in}{1.534939in}}%
\pgfpathlineto{\pgfqpoint{1.936452in}{1.541883in}}%
\pgfpathlineto{\pgfqpoint{1.954792in}{1.539447in}}%
\pgfpathlineto{\pgfqpoint{1.960906in}{1.541173in}}%
\pgfpathlineto{\pgfqpoint{1.973133in}{1.539555in}}%
\pgfpathlineto{\pgfqpoint{1.991473in}{1.544713in}}%
\pgfpathlineto{\pgfqpoint{2.003700in}{1.543095in}}%
\pgfpathlineto{\pgfqpoint{2.009814in}{1.544807in}}%
\pgfpathlineto{\pgfqpoint{2.138198in}{1.528116in}}%
\pgfpathlineto{\pgfqpoint{2.174879in}{1.538241in}}%
\pgfpathlineto{\pgfqpoint{2.187106in}{1.536672in}}%
\pgfpathlineto{\pgfqpoint{2.211560in}{1.543362in}}%
\pgfpathlineto{\pgfqpoint{2.254355in}{1.537890in}}%
\pgfpathlineto{\pgfqpoint{2.260468in}{1.539551in}}%
\pgfpathlineto{\pgfqpoint{2.266582in}{1.538773in}}%
\pgfpathlineto{\pgfqpoint{2.303263in}{1.548682in}}%
\pgfpathlineto{\pgfqpoint{2.315490in}{1.547122in}}%
\pgfpathlineto{\pgfqpoint{2.333830in}{1.552037in}}%
\pgfpathlineto{\pgfqpoint{2.394966in}{1.544291in}}%
\pgfpathlineto{\pgfqpoint{2.431647in}{1.554014in}}%
\pgfpathlineto{\pgfqpoint{2.456101in}{1.550933in}}%
\pgfpathlineto{\pgfqpoint{2.492782in}{1.560552in}}%
\pgfpathlineto{\pgfqpoint{2.511122in}{1.558242in}}%
\pgfpathlineto{\pgfqpoint{2.566144in}{1.572489in}}%
\pgfpathlineto{\pgfqpoint{2.572258in}{1.571716in}}%
\pgfpathlineto{\pgfqpoint{2.608939in}{1.581103in}}%
\pgfpathlineto{\pgfqpoint{2.670074in}{1.573400in}}%
\pgfpathlineto{\pgfqpoint{2.688415in}{1.578044in}}%
\pgfpathlineto{\pgfqpoint{2.749550in}{1.570438in}}%
\pgfpathlineto{\pgfqpoint{2.780117in}{1.578104in}}%
\pgfpathlineto{\pgfqpoint{2.822912in}{1.572826in}}%
\pgfpathlineto{\pgfqpoint{2.829025in}{1.574348in}}%
\pgfpathlineto{\pgfqpoint{2.884047in}{1.567637in}}%
\pgfpathlineto{\pgfqpoint{2.902388in}{1.572175in}}%
\pgfpathlineto{\pgfqpoint{2.914615in}{1.570692in}}%
\pgfpathlineto{\pgfqpoint{2.939069in}{1.576709in}}%
\pgfpathlineto{\pgfqpoint{2.951296in}{1.575226in}}%
\pgfpathlineto{\pgfqpoint{2.969636in}{1.579712in}}%
\pgfpathlineto{\pgfqpoint{3.000204in}{1.576015in}}%
\pgfpathlineto{\pgfqpoint{3.018545in}{1.580474in}}%
\pgfpathlineto{\pgfqpoint{3.104134in}{1.570239in}}%
\pgfpathlineto{\pgfqpoint{3.146929in}{1.580518in}}%
\pgfpathlineto{\pgfqpoint{3.214177in}{1.572577in}}%
\pgfpathlineto{\pgfqpoint{3.244745in}{1.579829in}}%
\pgfpathlineto{\pgfqpoint{3.250858in}{1.579111in}}%
\pgfpathlineto{\pgfqpoint{3.263085in}{1.581996in}}%
\pgfpathlineto{\pgfqpoint{3.275312in}{1.580561in}}%
\pgfpathlineto{\pgfqpoint{3.281426in}{1.581999in}}%
\pgfpathlineto{\pgfqpoint{3.287539in}{1.581283in}}%
\pgfpathlineto{\pgfqpoint{3.293653in}{1.582719in}}%
\pgfpathlineto{\pgfqpoint{3.330334in}{1.578435in}}%
\pgfpathlineto{\pgfqpoint{3.360902in}{1.585572in}}%
\pgfpathlineto{\pgfqpoint{3.367015in}{1.584859in}}%
\pgfpathlineto{\pgfqpoint{3.385356in}{1.589116in}}%
\pgfpathlineto{\pgfqpoint{3.391469in}{1.588403in}}%
\pgfpathlineto{\pgfqpoint{3.422037in}{1.595456in}}%
\pgfpathlineto{\pgfqpoint{3.434264in}{1.594028in}}%
\pgfpathlineto{\pgfqpoint{3.452605in}{1.598234in}}%
\pgfpathlineto{\pgfqpoint{3.458718in}{1.597520in}}%
\pgfpathlineto{\pgfqpoint{3.501513in}{1.607265in}}%
\pgfpathlineto{\pgfqpoint{3.605443in}{1.595215in}}%
\pgfpathlineto{\pgfqpoint{3.629897in}{1.600711in}}%
\pgfpathlineto{\pgfqpoint{3.697145in}{1.593040in}}%
\pgfpathlineto{\pgfqpoint{3.715486in}{1.597128in}}%
\pgfpathlineto{\pgfqpoint{3.788848in}{1.588870in}}%
\pgfpathlineto{\pgfqpoint{3.813302in}{1.594275in}}%
\pgfpathlineto{\pgfqpoint{3.819416in}{1.593591in}}%
\pgfpathlineto{\pgfqpoint{3.831643in}{1.596282in}}%
\pgfpathlineto{\pgfqpoint{3.837756in}{1.595598in}}%
\pgfpathlineto{\pgfqpoint{3.874437in}{1.603626in}}%
\pgfpathlineto{\pgfqpoint{3.886664in}{1.602256in}}%
\pgfpathlineto{\pgfqpoint{3.898891in}{1.604917in}}%
\pgfpathlineto{\pgfqpoint{3.923346in}{1.602183in}}%
\pgfpathlineto{\pgfqpoint{3.935573in}{1.604833in}}%
\pgfpathlineto{\pgfqpoint{3.960027in}{1.602110in}}%
\pgfpathlineto{\pgfqpoint{3.984481in}{1.607384in}}%
\pgfpathlineto{\pgfqpoint{3.990594in}{1.606704in}}%
\pgfpathlineto{\pgfqpoint{4.070070in}{1.623652in}}%
\pgfpathlineto{\pgfqpoint{4.082297in}{1.622284in}}%
\pgfpathlineto{\pgfqpoint{4.118978in}{1.630007in}}%
\pgfpathlineto{\pgfqpoint{4.137319in}{1.627955in}}%
\pgfpathlineto{\pgfqpoint{4.143432in}{1.629235in}}%
\pgfpathlineto{\pgfqpoint{4.167886in}{1.626509in}}%
\pgfpathlineto{\pgfqpoint{4.174000in}{1.627786in}}%
\pgfpathlineto{\pgfqpoint{4.271816in}{1.617012in}}%
\pgfpathlineto{\pgfqpoint{4.290157in}{1.620816in}}%
\pgfpathlineto{\pgfqpoint{4.406314in}{1.608277in}}%
\pgfpathlineto{\pgfqpoint{4.418541in}{1.610791in}}%
\pgfpathlineto{\pgfqpoint{4.449108in}{1.607536in}}%
\pgfpathlineto{\pgfqpoint{4.467449in}{1.611290in}}%
\pgfpathlineto{\pgfqpoint{4.473562in}{1.610640in}}%
\pgfpathlineto{\pgfqpoint{4.485789in}{1.613134in}}%
\pgfpathlineto{\pgfqpoint{4.491903in}{1.612485in}}%
\pgfpathlineto{\pgfqpoint{4.504130in}{1.614972in}}%
\pgfpathlineto{\pgfqpoint{4.528584in}{1.612379in}}%
\pgfpathlineto{\pgfqpoint{4.546925in}{1.616094in}}%
\pgfpathlineto{\pgfqpoint{4.583606in}{1.612221in}}%
\pgfpathlineto{\pgfqpoint{4.595833in}{1.614686in}}%
\pgfpathlineto{\pgfqpoint{4.638627in}{1.610198in}}%
\pgfpathlineto{\pgfqpoint{4.669195in}{1.616322in}}%
\pgfpathlineto{\pgfqpoint{4.699763in}{1.613131in}}%
\pgfpathlineto{\pgfqpoint{4.711990in}{1.615567in}}%
\pgfpathlineto{\pgfqpoint{4.864828in}{1.599874in}}%
\pgfpathlineto{\pgfqpoint{4.901509in}{1.607096in}}%
\pgfpathlineto{\pgfqpoint{4.913736in}{1.605856in}}%
\pgfpathlineto{\pgfqpoint{4.932076in}{1.609446in}}%
\pgfpathlineto{\pgfqpoint{4.974871in}{1.605123in}}%
\pgfpathlineto{\pgfqpoint{5.060460in}{1.621679in}}%
\pgfpathlineto{\pgfqpoint{5.109368in}{1.616744in}}%
\pgfpathlineto{\pgfqpoint{5.164390in}{1.627224in}}%
\pgfpathlineto{\pgfqpoint{5.170504in}{1.626607in}}%
\pgfpathlineto{\pgfqpoint{5.176617in}{1.627765in}}%
\pgfpathlineto{\pgfqpoint{5.231639in}{1.622246in}}%
\pgfpathlineto{\pgfqpoint{5.249979in}{1.625702in}}%
\pgfpathlineto{\pgfqpoint{5.286661in}{1.622048in}}%
\pgfpathlineto{\pgfqpoint{5.305001in}{1.625487in}}%
\pgfpathlineto{\pgfqpoint{5.329455in}{1.623061in}}%
\pgfpathlineto{\pgfqpoint{5.347796in}{1.626485in}}%
\pgfpathlineto{\pgfqpoint{5.360023in}{1.625274in}}%
\pgfpathlineto{\pgfqpoint{5.372250in}{1.627549in}}%
\pgfpathlineto{\pgfqpoint{5.427271in}{1.622129in}}%
\pgfpathlineto{\pgfqpoint{5.433385in}{1.623262in}}%
\pgfpathlineto{\pgfqpoint{5.549542in}{1.611993in}}%
\pgfpathlineto{\pgfqpoint{5.555655in}{1.613119in}}%
\pgfpathlineto{\pgfqpoint{5.580109in}{1.610775in}}%
\pgfpathlineto{\pgfqpoint{5.647358in}{1.623065in}}%
\pgfpathlineto{\pgfqpoint{5.653472in}{1.622477in}}%
\pgfpathlineto{\pgfqpoint{5.684039in}{1.628014in}}%
\pgfpathlineto{\pgfqpoint{5.726834in}{1.623909in}}%
\pgfpathlineto{\pgfqpoint{5.757402in}{1.629405in}}%
\pgfpathlineto{\pgfqpoint{5.763515in}{1.628820in}}%
\pgfpathlineto{\pgfqpoint{5.769629in}{1.629915in}}%
\pgfpathlineto{\pgfqpoint{5.781856in}{1.628746in}}%
\pgfpathlineto{\pgfqpoint{5.787969in}{1.629840in}}%
\pgfpathlineto{\pgfqpoint{5.842991in}{1.624608in}}%
\pgfpathlineto{\pgfqpoint{5.855218in}{1.626787in}}%
\pgfpathlineto{\pgfqpoint{5.879672in}{1.624475in}}%
\pgfpathlineto{\pgfqpoint{5.910240in}{1.629897in}}%
\pgfpathlineto{\pgfqpoint{5.922467in}{1.628742in}}%
\pgfpathlineto{\pgfqpoint{5.934694in}{1.630901in}}%
\pgfpathlineto{\pgfqpoint{5.946921in}{1.629748in}}%
\pgfpathlineto{\pgfqpoint{5.953034in}{1.630825in}}%
\pgfpathlineto{\pgfqpoint{6.008056in}{1.625663in}}%
\pgfpathlineto{\pgfqpoint{6.032510in}{1.629952in}}%
\pgfpathlineto{\pgfqpoint{6.044737in}{1.628809in}}%
\pgfpathlineto{\pgfqpoint{6.056964in}{1.630946in}}%
\pgfpathlineto{\pgfqpoint{6.130326in}{1.624135in}}%
\pgfpathlineto{\pgfqpoint{6.142553in}{1.626260in}}%
\pgfpathlineto{\pgfqpoint{6.228143in}{1.618415in}}%
\pgfpathlineto{\pgfqpoint{6.240370in}{1.620528in}}%
\pgfpathlineto{\pgfqpoint{6.277051in}{1.617198in}}%
\pgfpathlineto{\pgfqpoint{6.313732in}{1.623504in}}%
\pgfpathlineto{\pgfqpoint{6.319845in}{1.622949in}}%
\pgfpathlineto{\pgfqpoint{6.338186in}{1.626087in}}%
\pgfpathlineto{\pgfqpoint{6.368754in}{1.623318in}}%
\pgfpathlineto{\pgfqpoint{6.387094in}{1.626442in}}%
\pgfpathlineto{\pgfqpoint{6.442116in}{1.621488in}}%
\pgfpathlineto{\pgfqpoint{6.448229in}{1.622525in}}%
\pgfpathlineto{\pgfqpoint{6.521592in}{1.615987in}}%
\pgfpathlineto{\pgfqpoint{6.527705in}{1.617020in}}%
\pgfpathlineto{\pgfqpoint{6.558273in}{1.614318in}}%
\pgfpathlineto{\pgfqpoint{6.594954in}{1.620486in}}%
\pgfpathlineto{\pgfqpoint{6.607181in}{1.619406in}}%
\pgfpathlineto{\pgfqpoint{6.631635in}{1.623496in}}%
\pgfpathlineto{\pgfqpoint{6.662203in}{1.620800in}}%
\pgfpathlineto{\pgfqpoint{6.692770in}{1.625883in}}%
\pgfpathlineto{\pgfqpoint{6.711111in}{1.624269in}}%
\pgfpathlineto{\pgfqpoint{6.711111in}{1.624269in}}%
\pgfusepath{stroke}%
\end{pgfscope}%
\begin{pgfscope}%
\pgfpathrectangle{\pgfqpoint{0.603704in}{0.549691in}}{\pgfqpoint{6.107407in}{3.101235in}}%
\pgfusepath{clip}%
\pgfsetbuttcap%
\pgfsetroundjoin%
\pgfsetlinewidth{1.505625pt}%
\definecolor{currentstroke}{rgb}{0.121569,0.466667,0.705882}%
\pgfsetstrokecolor{currentstroke}%
\pgfsetstrokeopacity{0.700000}%
\pgfsetdash{{5.550000pt}{2.400000pt}}{0.000000pt}%
\pgfpathmoveto{\pgfqpoint{0.603704in}{1.583436in}}%
\pgfpathlineto{\pgfqpoint{6.711111in}{1.583436in}}%
\pgfusepath{stroke}%
\end{pgfscope}%
\begin{pgfscope}%
\pgfpathrectangle{\pgfqpoint{0.603704in}{0.549691in}}{\pgfqpoint{6.107407in}{3.101235in}}%
\pgfusepath{clip}%
\pgfsetrectcap%
\pgfsetroundjoin%
\pgfsetlinewidth{1.505625pt}%
\definecolor{currentstroke}{rgb}{1.000000,0.498039,0.054902}%
\pgfsetstrokecolor{currentstroke}%
\pgfsetdash{}{0pt}%
\pgfpathmoveto{\pgfqpoint{0.603704in}{1.615450in}}%
\pgfpathlineto{\pgfqpoint{0.609818in}{1.617482in}}%
\pgfpathlineto{\pgfqpoint{0.628158in}{1.614294in}}%
\pgfpathlineto{\pgfqpoint{0.640385in}{1.618339in}}%
\pgfpathlineto{\pgfqpoint{0.701520in}{1.607831in}}%
\pgfpathlineto{\pgfqpoint{0.713748in}{1.611841in}}%
\pgfpathlineto{\pgfqpoint{0.732088in}{1.608723in}}%
\pgfpathlineto{\pgfqpoint{0.738202in}{1.610720in}}%
\pgfpathlineto{\pgfqpoint{0.744315in}{1.609684in}}%
\pgfpathlineto{\pgfqpoint{0.756542in}{1.613663in}}%
\pgfpathlineto{\pgfqpoint{0.780996in}{1.609531in}}%
\pgfpathlineto{\pgfqpoint{0.793223in}{1.613487in}}%
\pgfpathlineto{\pgfqpoint{0.829904in}{1.607338in}}%
\pgfpathlineto{\pgfqpoint{0.854358in}{1.615183in}}%
\pgfpathlineto{\pgfqpoint{0.878813in}{1.611108in}}%
\pgfpathlineto{\pgfqpoint{0.884926in}{1.613056in}}%
\pgfpathlineto{\pgfqpoint{0.909380in}{1.609009in}}%
\pgfpathlineto{\pgfqpoint{0.915494in}{1.610950in}}%
\pgfpathlineto{\pgfqpoint{0.927721in}{1.608937in}}%
\pgfpathlineto{\pgfqpoint{0.933834in}{1.610872in}}%
\pgfpathlineto{\pgfqpoint{0.946061in}{1.608864in}}%
\pgfpathlineto{\pgfqpoint{0.958288in}{1.612721in}}%
\pgfpathlineto{\pgfqpoint{1.086672in}{1.592051in}}%
\pgfpathlineto{\pgfqpoint{1.123353in}{1.603426in}}%
\pgfpathlineto{\pgfqpoint{1.129467in}{1.602456in}}%
\pgfpathlineto{\pgfqpoint{1.147807in}{1.608094in}}%
\pgfpathlineto{\pgfqpoint{1.184489in}{1.602300in}}%
\pgfpathlineto{\pgfqpoint{1.190602in}{1.604168in}}%
\pgfpathlineto{\pgfqpoint{1.208943in}{1.601292in}}%
\pgfpathlineto{\pgfqpoint{1.215056in}{1.603153in}}%
\pgfpathlineto{\pgfqpoint{1.227283in}{1.601243in}}%
\pgfpathlineto{\pgfqpoint{1.233397in}{1.603100in}}%
\pgfpathlineto{\pgfqpoint{1.239510in}{1.602146in}}%
\pgfpathlineto{\pgfqpoint{1.245624in}{1.603999in}}%
\pgfpathlineto{\pgfqpoint{1.294532in}{1.596428in}}%
\pgfpathlineto{\pgfqpoint{1.306759in}{1.600109in}}%
\pgfpathlineto{\pgfqpoint{1.325100in}{1.597293in}}%
\pgfpathlineto{\pgfqpoint{1.331213in}{1.599127in}}%
\pgfpathlineto{\pgfqpoint{1.343440in}{1.597256in}}%
\pgfpathlineto{\pgfqpoint{1.355667in}{1.600910in}}%
\pgfpathlineto{\pgfqpoint{1.361781in}{1.599976in}}%
\pgfpathlineto{\pgfqpoint{1.367894in}{1.601798in}}%
\pgfpathlineto{\pgfqpoint{1.380121in}{1.599932in}}%
\pgfpathlineto{\pgfqpoint{1.398462in}{1.605372in}}%
\pgfpathlineto{\pgfqpoint{1.410689in}{1.603509in}}%
\pgfpathlineto{\pgfqpoint{1.416802in}{1.605314in}}%
\pgfpathlineto{\pgfqpoint{1.429029in}{1.603456in}}%
\pgfpathlineto{\pgfqpoint{1.447370in}{1.608849in}}%
\pgfpathlineto{\pgfqpoint{1.502392in}{1.600545in}}%
\pgfpathlineto{\pgfqpoint{1.508505in}{1.602330in}}%
\pgfpathlineto{\pgfqpoint{1.526846in}{1.599588in}}%
\pgfpathlineto{\pgfqpoint{1.539073in}{1.603144in}}%
\pgfpathlineto{\pgfqpoint{1.545186in}{1.602232in}}%
\pgfpathlineto{\pgfqpoint{1.551300in}{1.604004in}}%
\pgfpathlineto{\pgfqpoint{1.587981in}{1.598560in}}%
\pgfpathlineto{\pgfqpoint{1.600208in}{1.602086in}}%
\pgfpathlineto{\pgfqpoint{1.624662in}{1.598482in}}%
\pgfpathlineto{\pgfqpoint{1.630776in}{1.600238in}}%
\pgfpathlineto{\pgfqpoint{1.643003in}{1.598444in}}%
\pgfpathlineto{\pgfqpoint{1.649116in}{1.600195in}}%
\pgfpathlineto{\pgfqpoint{1.655230in}{1.599299in}}%
\pgfpathlineto{\pgfqpoint{1.667457in}{1.602791in}}%
\pgfpathlineto{\pgfqpoint{1.673570in}{1.601896in}}%
\pgfpathlineto{\pgfqpoint{1.685797in}{1.605375in}}%
\pgfpathlineto{\pgfqpoint{1.759159in}{1.594729in}}%
\pgfpathlineto{\pgfqpoint{1.808068in}{1.608460in}}%
\pgfpathlineto{\pgfqpoint{1.826408in}{1.605815in}}%
\pgfpathlineto{\pgfqpoint{1.832522in}{1.607517in}}%
\pgfpathlineto{\pgfqpoint{1.905884in}{1.597060in}}%
\pgfpathlineto{\pgfqpoint{1.911997in}{1.598751in}}%
\pgfpathlineto{\pgfqpoint{1.942565in}{1.594451in}}%
\pgfpathlineto{\pgfqpoint{1.954792in}{1.597817in}}%
\pgfpathlineto{\pgfqpoint{1.960906in}{1.596960in}}%
\pgfpathlineto{\pgfqpoint{1.973133in}{1.600314in}}%
\pgfpathlineto{\pgfqpoint{1.997587in}{1.596894in}}%
\pgfpathlineto{\pgfqpoint{2.003700in}{1.598564in}}%
\pgfpathlineto{\pgfqpoint{2.009814in}{1.597712in}}%
\pgfpathlineto{\pgfqpoint{2.070949in}{1.614257in}}%
\pgfpathlineto{\pgfqpoint{2.132084in}{1.605747in}}%
\pgfpathlineto{\pgfqpoint{2.138198in}{1.607381in}}%
\pgfpathlineto{\pgfqpoint{2.174879in}{1.602336in}}%
\pgfpathlineto{\pgfqpoint{2.187106in}{1.605588in}}%
\pgfpathlineto{\pgfqpoint{2.217673in}{1.601414in}}%
\pgfpathlineto{\pgfqpoint{2.223787in}{1.603033in}}%
\pgfpathlineto{\pgfqpoint{2.236014in}{1.601372in}}%
\pgfpathlineto{\pgfqpoint{2.254355in}{1.606209in}}%
\pgfpathlineto{\pgfqpoint{2.303263in}{1.599601in}}%
\pgfpathlineto{\pgfqpoint{2.315490in}{1.602804in}}%
\pgfpathlineto{\pgfqpoint{2.346057in}{1.598709in}}%
\pgfpathlineto{\pgfqpoint{2.358284in}{1.601896in}}%
\pgfpathlineto{\pgfqpoint{2.431647in}{1.592183in}}%
\pgfpathlineto{\pgfqpoint{2.456101in}{1.598498in}}%
\pgfpathlineto{\pgfqpoint{2.492782in}{1.593695in}}%
\pgfpathlineto{\pgfqpoint{2.511122in}{1.598395in}}%
\pgfpathlineto{\pgfqpoint{2.566144in}{1.591256in}}%
\pgfpathlineto{\pgfqpoint{2.572258in}{1.592813in}}%
\pgfpathlineto{\pgfqpoint{2.608939in}{1.588103in}}%
\pgfpathlineto{\pgfqpoint{2.627279in}{1.592749in}}%
\pgfpathlineto{\pgfqpoint{2.657847in}{1.588848in}}%
\pgfpathlineto{\pgfqpoint{2.670074in}{1.591928in}}%
\pgfpathlineto{\pgfqpoint{2.688415in}{1.589599in}}%
\pgfpathlineto{\pgfqpoint{2.694528in}{1.591133in}}%
\pgfpathlineto{\pgfqpoint{2.718982in}{1.588041in}}%
\pgfpathlineto{\pgfqpoint{2.731209in}{1.591099in}}%
\pgfpathlineto{\pgfqpoint{2.737323in}{1.590328in}}%
\pgfpathlineto{\pgfqpoint{2.749550in}{1.593376in}}%
\pgfpathlineto{\pgfqpoint{2.798458in}{1.587237in}}%
\pgfpathlineto{\pgfqpoint{2.804571in}{1.588753in}}%
\pgfpathlineto{\pgfqpoint{2.810685in}{1.587990in}}%
\pgfpathlineto{\pgfqpoint{2.822912in}{1.591015in}}%
\pgfpathlineto{\pgfqpoint{2.829025in}{1.590252in}}%
\pgfpathlineto{\pgfqpoint{2.847366in}{1.594771in}}%
\pgfpathlineto{\pgfqpoint{2.877934in}{1.590965in}}%
\pgfpathlineto{\pgfqpoint{2.884047in}{1.592464in}}%
\pgfpathlineto{\pgfqpoint{2.902388in}{1.590193in}}%
\pgfpathlineto{\pgfqpoint{2.908501in}{1.591688in}}%
\pgfpathlineto{\pgfqpoint{2.945182in}{1.587171in}}%
\pgfpathlineto{\pgfqpoint{2.951296in}{1.588661in}}%
\pgfpathlineto{\pgfqpoint{2.994091in}{1.583436in}}%
\pgfpathlineto{\pgfqpoint{3.000204in}{1.584920in}}%
\pgfpathlineto{\pgfqpoint{3.018545in}{1.582696in}}%
\pgfpathlineto{\pgfqpoint{3.036885in}{1.587131in}}%
\pgfpathlineto{\pgfqpoint{3.042999in}{1.586390in}}%
\pgfpathlineto{\pgfqpoint{3.049112in}{1.587863in}}%
\pgfpathlineto{\pgfqpoint{3.055226in}{1.587123in}}%
\pgfpathlineto{\pgfqpoint{3.067453in}{1.590063in}}%
\pgfpathlineto{\pgfqpoint{3.091907in}{1.587107in}}%
\pgfpathlineto{\pgfqpoint{3.104134in}{1.590035in}}%
\pgfpathlineto{\pgfqpoint{3.146929in}{1.584895in}}%
\pgfpathlineto{\pgfqpoint{3.159156in}{1.587807in}}%
\pgfpathlineto{\pgfqpoint{3.165269in}{1.587076in}}%
\pgfpathlineto{\pgfqpoint{3.195837in}{1.594318in}}%
\pgfpathlineto{\pgfqpoint{3.201950in}{1.593585in}}%
\pgfpathlineto{\pgfqpoint{3.214177in}{1.596467in}}%
\pgfpathlineto{\pgfqpoint{3.244745in}{1.592814in}}%
\pgfpathlineto{\pgfqpoint{3.250858in}{1.594249in}}%
\pgfpathlineto{\pgfqpoint{3.269199in}{1.592069in}}%
\pgfpathlineto{\pgfqpoint{3.275312in}{1.593500in}}%
\pgfpathlineto{\pgfqpoint{3.281426in}{1.592775in}}%
\pgfpathlineto{\pgfqpoint{3.287539in}{1.594204in}}%
\pgfpathlineto{\pgfqpoint{3.305880in}{1.592033in}}%
\pgfpathlineto{\pgfqpoint{3.330334in}{1.597724in}}%
\pgfpathlineto{\pgfqpoint{3.385356in}{1.591246in}}%
\pgfpathlineto{\pgfqpoint{3.391469in}{1.592660in}}%
\pgfpathlineto{\pgfqpoint{3.422037in}{1.589093in}}%
\pgfpathlineto{\pgfqpoint{3.434264in}{1.591909in}}%
\pgfpathlineto{\pgfqpoint{3.452605in}{1.589778in}}%
\pgfpathlineto{\pgfqpoint{3.458718in}{1.591182in}}%
\pgfpathlineto{\pgfqpoint{3.513740in}{1.584836in}}%
\pgfpathlineto{\pgfqpoint{3.544307in}{1.591807in}}%
\pgfpathlineto{\pgfqpoint{3.556534in}{1.590402in}}%
\pgfpathlineto{\pgfqpoint{3.562648in}{1.591790in}}%
\pgfpathlineto{\pgfqpoint{3.574875in}{1.590388in}}%
\pgfpathlineto{\pgfqpoint{3.587102in}{1.593156in}}%
\pgfpathlineto{\pgfqpoint{3.599329in}{1.591756in}}%
\pgfpathlineto{\pgfqpoint{3.605443in}{1.593136in}}%
\pgfpathlineto{\pgfqpoint{3.629897in}{1.590346in}}%
\pgfpathlineto{\pgfqpoint{3.636010in}{1.591723in}}%
\pgfpathlineto{\pgfqpoint{3.684918in}{1.586184in}}%
\pgfpathlineto{\pgfqpoint{3.691032in}{1.587555in}}%
\pgfpathlineto{\pgfqpoint{3.715486in}{1.584805in}}%
\pgfpathlineto{\pgfqpoint{3.739940in}{1.590264in}}%
\pgfpathlineto{\pgfqpoint{3.758281in}{1.588206in}}%
\pgfpathlineto{\pgfqpoint{3.764394in}{1.589565in}}%
\pgfpathlineto{\pgfqpoint{3.770508in}{1.588881in}}%
\pgfpathlineto{\pgfqpoint{3.788848in}{1.592945in}}%
\pgfpathlineto{\pgfqpoint{3.813302in}{1.590210in}}%
\pgfpathlineto{\pgfqpoint{3.819416in}{1.591560in}}%
\pgfpathlineto{\pgfqpoint{3.831643in}{1.590197in}}%
\pgfpathlineto{\pgfqpoint{3.837756in}{1.591544in}}%
\pgfpathlineto{\pgfqpoint{3.874437in}{1.587474in}}%
\pgfpathlineto{\pgfqpoint{3.886664in}{1.590158in}}%
\pgfpathlineto{\pgfqpoint{3.898891in}{1.588806in}}%
\pgfpathlineto{\pgfqpoint{3.905005in}{1.590144in}}%
\pgfpathlineto{\pgfqpoint{3.917232in}{1.588796in}}%
\pgfpathlineto{\pgfqpoint{3.923346in}{1.590131in}}%
\pgfpathlineto{\pgfqpoint{3.935573in}{1.588785in}}%
\pgfpathlineto{\pgfqpoint{3.941686in}{1.590118in}}%
\pgfpathlineto{\pgfqpoint{3.947800in}{1.589446in}}%
\pgfpathlineto{\pgfqpoint{3.960027in}{1.592106in}}%
\pgfpathlineto{\pgfqpoint{3.984481in}{1.589423in}}%
\pgfpathlineto{\pgfqpoint{3.990594in}{1.590749in}}%
\pgfpathlineto{\pgfqpoint{4.118978in}{1.576877in}}%
\pgfpathlineto{\pgfqpoint{4.125092in}{1.578192in}}%
\pgfpathlineto{\pgfqpoint{4.131205in}{1.577540in}}%
\pgfpathlineto{\pgfqpoint{4.137319in}{1.578853in}}%
\pgfpathlineto{\pgfqpoint{4.155659in}{1.576902in}}%
\pgfpathlineto{\pgfqpoint{4.167886in}{1.579520in}}%
\pgfpathlineto{\pgfqpoint{4.180113in}{1.578222in}}%
\pgfpathlineto{\pgfqpoint{4.204567in}{1.583436in}}%
\pgfpathlineto{\pgfqpoint{4.210681in}{1.582786in}}%
\pgfpathlineto{\pgfqpoint{4.229022in}{1.586679in}}%
\pgfpathlineto{\pgfqpoint{4.259589in}{1.583436in}}%
\pgfpathlineto{\pgfqpoint{4.271816in}{1.586019in}}%
\pgfpathlineto{\pgfqpoint{4.290157in}{1.584081in}}%
\pgfpathlineto{\pgfqpoint{4.302384in}{1.586655in}}%
\pgfpathlineto{\pgfqpoint{4.320724in}{1.584721in}}%
\pgfpathlineto{\pgfqpoint{4.332951in}{1.587286in}}%
\pgfpathlineto{\pgfqpoint{4.387973in}{1.581522in}}%
\pgfpathlineto{\pgfqpoint{4.406314in}{1.585347in}}%
\pgfpathlineto{\pgfqpoint{4.418541in}{1.584072in}}%
\pgfpathlineto{\pgfqpoint{4.424654in}{1.585343in}}%
\pgfpathlineto{\pgfqpoint{4.442995in}{1.583436in}}%
\pgfpathlineto{\pgfqpoint{4.449108in}{1.584705in}}%
\pgfpathlineto{\pgfqpoint{4.467449in}{1.582803in}}%
\pgfpathlineto{\pgfqpoint{4.473562in}{1.584069in}}%
\pgfpathlineto{\pgfqpoint{4.504130in}{1.580913in}}%
\pgfpathlineto{\pgfqpoint{4.510243in}{1.582175in}}%
\pgfpathlineto{\pgfqpoint{4.546925in}{1.578412in}}%
\pgfpathlineto{\pgfqpoint{4.559152in}{1.580927in}}%
\pgfpathlineto{\pgfqpoint{4.571379in}{1.579677in}}%
\pgfpathlineto{\pgfqpoint{4.583606in}{1.582185in}}%
\pgfpathlineto{\pgfqpoint{4.595833in}{1.580936in}}%
\pgfpathlineto{\pgfqpoint{4.620287in}{1.585930in}}%
\pgfpathlineto{\pgfqpoint{4.675309in}{1.580336in}}%
\pgfpathlineto{\pgfqpoint{4.687536in}{1.582817in}}%
\pgfpathlineto{\pgfqpoint{4.693649in}{1.582198in}}%
\pgfpathlineto{\pgfqpoint{4.699763in}{1.583436in}}%
\pgfpathlineto{\pgfqpoint{4.711990in}{1.582200in}}%
\pgfpathlineto{\pgfqpoint{4.718103in}{1.583436in}}%
\pgfpathlineto{\pgfqpoint{4.736444in}{1.581587in}}%
\pgfpathlineto{\pgfqpoint{4.748671in}{1.584052in}}%
\pgfpathlineto{\pgfqpoint{4.754784in}{1.583436in}}%
\pgfpathlineto{\pgfqpoint{4.760898in}{1.584666in}}%
\pgfpathlineto{\pgfqpoint{4.803692in}{1.580374in}}%
\pgfpathlineto{\pgfqpoint{4.809806in}{1.581600in}}%
\pgfpathlineto{\pgfqpoint{4.840374in}{1.578554in}}%
\pgfpathlineto{\pgfqpoint{4.852601in}{1.580998in}}%
\pgfpathlineto{\pgfqpoint{4.858714in}{1.580390in}}%
\pgfpathlineto{\pgfqpoint{4.864828in}{1.581610in}}%
\pgfpathlineto{\pgfqpoint{4.907622in}{1.577373in}}%
\pgfpathlineto{\pgfqpoint{4.913736in}{1.578589in}}%
\pgfpathlineto{\pgfqpoint{4.968757in}{1.573189in}}%
\pgfpathlineto{\pgfqpoint{4.974871in}{1.574400in}}%
\pgfpathlineto{\pgfqpoint{5.060460in}{1.566107in}}%
\pgfpathlineto{\pgfqpoint{5.066574in}{1.567312in}}%
\pgfpathlineto{\pgfqpoint{5.072687in}{1.566724in}}%
\pgfpathlineto{\pgfqpoint{5.078801in}{1.567927in}}%
\pgfpathlineto{\pgfqpoint{5.091028in}{1.566753in}}%
\pgfpathlineto{\pgfqpoint{5.109368in}{1.570351in}}%
\pgfpathlineto{\pgfqpoint{5.176617in}{1.563932in}}%
\pgfpathlineto{\pgfqpoint{5.201071in}{1.568694in}}%
\pgfpathlineto{\pgfqpoint{5.207185in}{1.568113in}}%
\pgfpathlineto{\pgfqpoint{5.213298in}{1.569299in}}%
\pgfpathlineto{\pgfqpoint{5.219412in}{1.568719in}}%
\pgfpathlineto{\pgfqpoint{5.225525in}{1.569904in}}%
\pgfpathlineto{\pgfqpoint{5.249979in}{1.567587in}}%
\pgfpathlineto{\pgfqpoint{5.256093in}{1.568769in}}%
\pgfpathlineto{\pgfqpoint{5.280547in}{1.566461in}}%
\pgfpathlineto{\pgfqpoint{5.286661in}{1.567640in}}%
\pgfpathlineto{\pgfqpoint{5.305001in}{1.565915in}}%
\pgfpathlineto{\pgfqpoint{5.317228in}{1.568268in}}%
\pgfpathlineto{\pgfqpoint{5.347796in}{1.565402in}}%
\pgfpathlineto{\pgfqpoint{5.353909in}{1.566575in}}%
\pgfpathlineto{\pgfqpoint{5.378363in}{1.564293in}}%
\pgfpathlineto{\pgfqpoint{5.402817in}{1.568966in}}%
\pgfpathlineto{\pgfqpoint{5.421158in}{1.567257in}}%
\pgfpathlineto{\pgfqpoint{5.427271in}{1.568421in}}%
\pgfpathlineto{\pgfqpoint{5.463953in}{1.565018in}}%
\pgfpathlineto{\pgfqpoint{5.470066in}{1.566178in}}%
\pgfpathlineto{\pgfqpoint{5.488407in}{1.564484in}}%
\pgfpathlineto{\pgfqpoint{5.500634in}{1.566800in}}%
\pgfpathlineto{\pgfqpoint{5.506747in}{1.566236in}}%
\pgfpathlineto{\pgfqpoint{5.512861in}{1.567391in}}%
\pgfpathlineto{\pgfqpoint{5.531201in}{1.565702in}}%
\pgfpathlineto{\pgfqpoint{5.537315in}{1.566855in}}%
\pgfpathlineto{\pgfqpoint{5.555655in}{1.565170in}}%
\pgfpathlineto{\pgfqpoint{5.573996in}{1.568620in}}%
\pgfpathlineto{\pgfqpoint{5.684039in}{1.558608in}}%
\pgfpathlineto{\pgfqpoint{5.702380in}{1.562029in}}%
\pgfpathlineto{\pgfqpoint{5.714607in}{1.560927in}}%
\pgfpathlineto{\pgfqpoint{5.720720in}{1.562064in}}%
\pgfpathlineto{\pgfqpoint{5.757402in}{1.558770in}}%
\pgfpathlineto{\pgfqpoint{5.763515in}{1.559904in}}%
\pgfpathlineto{\pgfqpoint{5.769629in}{1.559357in}}%
\pgfpathlineto{\pgfqpoint{5.781856in}{1.561620in}}%
\pgfpathlineto{\pgfqpoint{5.818537in}{1.558345in}}%
\pgfpathlineto{\pgfqpoint{5.830764in}{1.560600in}}%
\pgfpathlineto{\pgfqpoint{5.836877in}{1.560056in}}%
\pgfpathlineto{\pgfqpoint{5.842991in}{1.561181in}}%
\pgfpathlineto{\pgfqpoint{5.855218in}{1.560094in}}%
\pgfpathlineto{\pgfqpoint{5.861331in}{1.561217in}}%
\pgfpathlineto{\pgfqpoint{5.873558in}{1.560131in}}%
\pgfpathlineto{\pgfqpoint{5.879672in}{1.561253in}}%
\pgfpathlineto{\pgfqpoint{5.910240in}{1.558547in}}%
\pgfpathlineto{\pgfqpoint{5.922467in}{1.560783in}}%
\pgfpathlineto{\pgfqpoint{5.934694in}{1.559704in}}%
\pgfpathlineto{\pgfqpoint{5.940807in}{1.560820in}}%
\pgfpathlineto{\pgfqpoint{5.953034in}{1.559742in}}%
\pgfpathlineto{\pgfqpoint{5.977488in}{1.564191in}}%
\pgfpathlineto{\pgfqpoint{5.989715in}{1.563113in}}%
\pgfpathlineto{\pgfqpoint{6.008056in}{1.566436in}}%
\pgfpathlineto{\pgfqpoint{6.032510in}{1.564283in}}%
\pgfpathlineto{\pgfqpoint{6.044737in}{1.566490in}}%
\pgfpathlineto{\pgfqpoint{6.056964in}{1.565415in}}%
\pgfpathlineto{\pgfqpoint{6.063078in}{1.566516in}}%
\pgfpathlineto{\pgfqpoint{6.069191in}{1.565980in}}%
\pgfpathlineto{\pgfqpoint{6.081418in}{1.568178in}}%
\pgfpathlineto{\pgfqpoint{6.105872in}{1.566035in}}%
\pgfpathlineto{\pgfqpoint{6.130326in}{1.570413in}}%
\pgfpathlineto{\pgfqpoint{6.154780in}{1.568274in}}%
\pgfpathlineto{\pgfqpoint{6.160894in}{1.569364in}}%
\pgfpathlineto{\pgfqpoint{6.167007in}{1.568831in}}%
\pgfpathlineto{\pgfqpoint{6.173121in}{1.569920in}}%
\pgfpathlineto{\pgfqpoint{6.215916in}{1.566198in}}%
\pgfpathlineto{\pgfqpoint{6.228143in}{1.568369in}}%
\pgfpathlineto{\pgfqpoint{6.246483in}{1.566780in}}%
\pgfpathlineto{\pgfqpoint{6.264824in}{1.570025in}}%
\pgfpathlineto{\pgfqpoint{6.270937in}{1.569496in}}%
\pgfpathlineto{\pgfqpoint{6.277051in}{1.570575in}}%
\pgfpathlineto{\pgfqpoint{6.313732in}{1.567409in}}%
\pgfpathlineto{\pgfqpoint{6.319845in}{1.568485in}}%
\pgfpathlineto{\pgfqpoint{6.362640in}{1.564815in}}%
\pgfpathlineto{\pgfqpoint{6.368754in}{1.565888in}}%
\pgfpathlineto{\pgfqpoint{6.393208in}{1.563801in}}%
\pgfpathlineto{\pgfqpoint{6.442116in}{1.572338in}}%
\pgfpathlineto{\pgfqpoint{6.460456in}{1.570772in}}%
\pgfpathlineto{\pgfqpoint{6.466570in}{1.571833in}}%
\pgfpathlineto{\pgfqpoint{6.503251in}{1.568713in}}%
\pgfpathlineto{\pgfqpoint{6.509365in}{1.569772in}}%
\pgfpathlineto{\pgfqpoint{6.552159in}{1.566155in}}%
\pgfpathlineto{\pgfqpoint{6.558273in}{1.567210in}}%
\pgfpathlineto{\pgfqpoint{6.594954in}{1.564128in}}%
\pgfpathlineto{\pgfqpoint{6.601067in}{1.565181in}}%
\pgfpathlineto{\pgfqpoint{6.637748in}{1.562116in}}%
\pgfpathlineto{\pgfqpoint{6.643862in}{1.563167in}}%
\pgfpathlineto{\pgfqpoint{6.649975in}{1.562657in}}%
\pgfpathlineto{\pgfqpoint{6.662203in}{1.564754in}}%
\pgfpathlineto{\pgfqpoint{6.704997in}{1.561200in}}%
\pgfpathlineto{\pgfqpoint{6.711111in}{1.562244in}}%
\pgfpathlineto{\pgfqpoint{6.711111in}{1.562244in}}%
\pgfusepath{stroke}%
\end{pgfscope}%
\begin{pgfscope}%
\pgfpathrectangle{\pgfqpoint{0.603704in}{0.549691in}}{\pgfqpoint{6.107407in}{3.101235in}}%
\pgfusepath{clip}%
\pgfsetbuttcap%
\pgfsetroundjoin%
\pgfsetlinewidth{1.505625pt}%
\definecolor{currentstroke}{rgb}{1.000000,0.498039,0.054902}%
\pgfsetstrokecolor{currentstroke}%
\pgfsetstrokeopacity{0.700000}%
\pgfsetdash{{5.550000pt}{2.400000pt}}{0.000000pt}%
\pgfpathmoveto{\pgfqpoint{0.603704in}{1.583436in}}%
\pgfpathlineto{\pgfqpoint{6.711111in}{1.583436in}}%
\pgfusepath{stroke}%
\end{pgfscope}%
\begin{pgfscope}%
\pgfpathrectangle{\pgfqpoint{0.603704in}{0.549691in}}{\pgfqpoint{6.107407in}{3.101235in}}%
\pgfusepath{clip}%
\pgfsetrectcap%
\pgfsetroundjoin%
\pgfsetlinewidth{1.505625pt}%
\definecolor{currentstroke}{rgb}{0.172549,0.627451,0.172549}%
\pgfsetstrokecolor{currentstroke}%
\pgfsetdash{}{0pt}%
\pgfpathmoveto{\pgfqpoint{0.603704in}{1.584469in}}%
\pgfpathlineto{\pgfqpoint{0.622045in}{1.581377in}}%
\pgfpathlineto{\pgfqpoint{0.628158in}{1.583436in}}%
\pgfpathlineto{\pgfqpoint{0.670953in}{1.576286in}}%
\pgfpathlineto{\pgfqpoint{0.701520in}{1.586486in}}%
\pgfpathlineto{\pgfqpoint{0.713748in}{1.584451in}}%
\pgfpathlineto{\pgfqpoint{0.719861in}{1.586477in}}%
\pgfpathlineto{\pgfqpoint{0.725975in}{1.585461in}}%
\pgfpathlineto{\pgfqpoint{0.732088in}{1.587482in}}%
\pgfpathlineto{\pgfqpoint{0.738202in}{1.586468in}}%
\pgfpathlineto{\pgfqpoint{0.744315in}{1.588484in}}%
\pgfpathlineto{\pgfqpoint{0.756542in}{1.586459in}}%
\pgfpathlineto{\pgfqpoint{0.780996in}{1.594476in}}%
\pgfpathlineto{\pgfqpoint{0.793223in}{1.592451in}}%
\pgfpathlineto{\pgfqpoint{0.829904in}{1.604350in}}%
\pgfpathlineto{\pgfqpoint{0.854358in}{1.600301in}}%
\pgfpathlineto{\pgfqpoint{0.878813in}{1.608143in}}%
\pgfpathlineto{\pgfqpoint{0.884926in}{1.607132in}}%
\pgfpathlineto{\pgfqpoint{0.909380in}{1.614911in}}%
\pgfpathlineto{\pgfqpoint{0.933834in}{1.610872in}}%
\pgfpathlineto{\pgfqpoint{0.939948in}{1.612804in}}%
\pgfpathlineto{\pgfqpoint{0.988856in}{1.604811in}}%
\pgfpathlineto{\pgfqpoint{1.007196in}{1.610563in}}%
\pgfpathlineto{\pgfqpoint{1.025537in}{1.607589in}}%
\pgfpathlineto{\pgfqpoint{1.086672in}{1.626509in}}%
\pgfpathlineto{\pgfqpoint{1.123353in}{1.620560in}}%
\pgfpathlineto{\pgfqpoint{1.129467in}{1.622427in}}%
\pgfpathlineto{\pgfqpoint{1.147807in}{1.619475in}}%
\pgfpathlineto{\pgfqpoint{1.184489in}{1.630596in}}%
\pgfpathlineto{\pgfqpoint{1.190602in}{1.629611in}}%
\pgfpathlineto{\pgfqpoint{1.208943in}{1.635123in}}%
\pgfpathlineto{\pgfqpoint{1.233397in}{1.631191in}}%
\pgfpathlineto{\pgfqpoint{1.239510in}{1.633018in}}%
\pgfpathlineto{\pgfqpoint{1.245624in}{1.632039in}}%
\pgfpathlineto{\pgfqpoint{1.282305in}{1.642932in}}%
\pgfpathlineto{\pgfqpoint{1.306759in}{1.639014in}}%
\pgfpathlineto{\pgfqpoint{1.325100in}{1.644408in}}%
\pgfpathlineto{\pgfqpoint{1.398462in}{1.632793in}}%
\pgfpathlineto{\pgfqpoint{1.410689in}{1.636355in}}%
\pgfpathlineto{\pgfqpoint{1.471824in}{1.626848in}}%
\pgfpathlineto{\pgfqpoint{1.496278in}{1.633907in}}%
\pgfpathlineto{\pgfqpoint{1.508505in}{1.632019in}}%
\pgfpathlineto{\pgfqpoint{1.526846in}{1.637277in}}%
\pgfpathlineto{\pgfqpoint{1.551300in}{1.633514in}}%
\pgfpathlineto{\pgfqpoint{1.563527in}{1.636998in}}%
\pgfpathlineto{\pgfqpoint{1.630776in}{1.626767in}}%
\pgfpathlineto{\pgfqpoint{1.643003in}{1.630224in}}%
\pgfpathlineto{\pgfqpoint{1.685797in}{1.623803in}}%
\pgfpathlineto{\pgfqpoint{1.691911in}{1.625522in}}%
\pgfpathlineto{\pgfqpoint{1.746932in}{1.617372in}}%
\pgfpathlineto{\pgfqpoint{1.759159in}{1.620790in}}%
\pgfpathlineto{\pgfqpoint{1.808068in}{1.613637in}}%
\pgfpathlineto{\pgfqpoint{1.826408in}{1.618726in}}%
\pgfpathlineto{\pgfqpoint{1.881430in}{1.610775in}}%
\pgfpathlineto{\pgfqpoint{1.905884in}{1.617497in}}%
\pgfpathlineto{\pgfqpoint{1.936452in}{1.613117in}}%
\pgfpathlineto{\pgfqpoint{1.942565in}{1.614787in}}%
\pgfpathlineto{\pgfqpoint{1.991473in}{1.607849in}}%
\pgfpathlineto{\pgfqpoint{1.997587in}{1.609511in}}%
\pgfpathlineto{\pgfqpoint{2.070949in}{1.599263in}}%
\pgfpathlineto{\pgfqpoint{2.132084in}{1.615663in}}%
\pgfpathlineto{\pgfqpoint{2.211560in}{1.604700in}}%
\pgfpathlineto{\pgfqpoint{2.217673in}{1.606317in}}%
\pgfpathlineto{\pgfqpoint{2.223787in}{1.605483in}}%
\pgfpathlineto{\pgfqpoint{2.236014in}{1.608709in}}%
\pgfpathlineto{\pgfqpoint{2.260468in}{1.605379in}}%
\pgfpathlineto{\pgfqpoint{2.266582in}{1.606986in}}%
\pgfpathlineto{\pgfqpoint{2.333830in}{1.597928in}}%
\pgfpathlineto{\pgfqpoint{2.346057in}{1.601121in}}%
\pgfpathlineto{\pgfqpoint{2.358284in}{1.599488in}}%
\pgfpathlineto{\pgfqpoint{2.394966in}{1.609000in}}%
\pgfpathlineto{\pgfqpoint{2.627279in}{1.578780in}}%
\pgfpathlineto{\pgfqpoint{2.657847in}{1.586529in}}%
\pgfpathlineto{\pgfqpoint{2.694528in}{1.581897in}}%
\pgfpathlineto{\pgfqpoint{2.718982in}{1.588041in}}%
\pgfpathlineto{\pgfqpoint{2.731209in}{1.586501in}}%
\pgfpathlineto{\pgfqpoint{2.737323in}{1.588031in}}%
\pgfpathlineto{\pgfqpoint{2.780117in}{1.582674in}}%
\pgfpathlineto{\pgfqpoint{2.798458in}{1.587237in}}%
\pgfpathlineto{\pgfqpoint{2.804571in}{1.586474in}}%
\pgfpathlineto{\pgfqpoint{2.810685in}{1.587990in}}%
\pgfpathlineto{\pgfqpoint{2.847366in}{1.583436in}}%
\pgfpathlineto{\pgfqpoint{2.877934in}{1.590965in}}%
\pgfpathlineto{\pgfqpoint{2.908501in}{1.587187in}}%
\pgfpathlineto{\pgfqpoint{2.914615in}{1.588684in}}%
\pgfpathlineto{\pgfqpoint{2.939069in}{1.585679in}}%
\pgfpathlineto{\pgfqpoint{2.945182in}{1.587171in}}%
\pgfpathlineto{\pgfqpoint{2.969636in}{1.584181in}}%
\pgfpathlineto{\pgfqpoint{2.994091in}{1.590120in}}%
\pgfpathlineto{\pgfqpoint{3.036885in}{1.584914in}}%
\pgfpathlineto{\pgfqpoint{3.042999in}{1.586390in}}%
\pgfpathlineto{\pgfqpoint{3.049112in}{1.585650in}}%
\pgfpathlineto{\pgfqpoint{3.055226in}{1.587123in}}%
\pgfpathlineto{\pgfqpoint{3.067453in}{1.585645in}}%
\pgfpathlineto{\pgfqpoint{3.091907in}{1.591512in}}%
\pgfpathlineto{\pgfqpoint{3.159156in}{1.583436in}}%
\pgfpathlineto{\pgfqpoint{3.165269in}{1.584892in}}%
\pgfpathlineto{\pgfqpoint{3.195837in}{1.581260in}}%
\pgfpathlineto{\pgfqpoint{3.201950in}{1.582711in}}%
\pgfpathlineto{\pgfqpoint{3.263085in}{1.575517in}}%
\pgfpathlineto{\pgfqpoint{3.269199in}{1.576962in}}%
\pgfpathlineto{\pgfqpoint{3.293653in}{1.574110in}}%
\pgfpathlineto{\pgfqpoint{3.305880in}{1.576989in}}%
\pgfpathlineto{\pgfqpoint{3.360902in}{1.570621in}}%
\pgfpathlineto{\pgfqpoint{3.367015in}{1.572053in}}%
\pgfpathlineto{\pgfqpoint{3.501513in}{1.556804in}}%
\pgfpathlineto{\pgfqpoint{3.513740in}{1.559640in}}%
\pgfpathlineto{\pgfqpoint{3.544307in}{1.556232in}}%
\pgfpathlineto{\pgfqpoint{3.556534in}{1.559055in}}%
\pgfpathlineto{\pgfqpoint{3.562648in}{1.558376in}}%
\pgfpathlineto{\pgfqpoint{3.574875in}{1.561190in}}%
\pgfpathlineto{\pgfqpoint{3.587102in}{1.559832in}}%
\pgfpathlineto{\pgfqpoint{3.599329in}{1.562636in}}%
\pgfpathlineto{\pgfqpoint{3.636010in}{1.558577in}}%
\pgfpathlineto{\pgfqpoint{3.684918in}{1.569699in}}%
\pgfpathlineto{\pgfqpoint{3.691032in}{1.569021in}}%
\pgfpathlineto{\pgfqpoint{3.697145in}{1.570403in}}%
\pgfpathlineto{\pgfqpoint{3.739940in}{1.565684in}}%
\pgfpathlineto{\pgfqpoint{3.758281in}{1.569807in}}%
\pgfpathlineto{\pgfqpoint{3.764394in}{1.569135in}}%
\pgfpathlineto{\pgfqpoint{3.770508in}{1.570506in}}%
\pgfpathlineto{\pgfqpoint{3.905005in}{1.555932in}}%
\pgfpathlineto{\pgfqpoint{3.917232in}{1.558648in}}%
\pgfpathlineto{\pgfqpoint{3.941686in}{1.556039in}}%
\pgfpathlineto{\pgfqpoint{3.947800in}{1.557392in}}%
\pgfpathlineto{\pgfqpoint{4.070070in}{1.544539in}}%
\pgfpathlineto{\pgfqpoint{4.082297in}{1.547222in}}%
\pgfpathlineto{\pgfqpoint{4.125092in}{1.542794in}}%
\pgfpathlineto{\pgfqpoint{4.131205in}{1.544130in}}%
\pgfpathlineto{\pgfqpoint{4.143432in}{1.542871in}}%
\pgfpathlineto{\pgfqpoint{4.155659in}{1.545537in}}%
\pgfpathlineto{\pgfqpoint{4.174000in}{1.543652in}}%
\pgfpathlineto{\pgfqpoint{4.180113in}{1.544980in}}%
\pgfpathlineto{\pgfqpoint{4.204567in}{1.542476in}}%
\pgfpathlineto{\pgfqpoint{4.210681in}{1.543802in}}%
\pgfpathlineto{\pgfqpoint{4.229022in}{1.541931in}}%
\pgfpathlineto{\pgfqpoint{4.259589in}{1.548525in}}%
\pgfpathlineto{\pgfqpoint{4.302384in}{1.544172in}}%
\pgfpathlineto{\pgfqpoint{4.320724in}{1.548100in}}%
\pgfpathlineto{\pgfqpoint{4.332951in}{1.546860in}}%
\pgfpathlineto{\pgfqpoint{4.387973in}{1.558550in}}%
\pgfpathlineto{\pgfqpoint{4.424654in}{1.554827in}}%
\pgfpathlineto{\pgfqpoint{4.442995in}{1.558687in}}%
\pgfpathlineto{\pgfqpoint{4.485789in}{1.554370in}}%
\pgfpathlineto{\pgfqpoint{4.491903in}{1.555651in}}%
\pgfpathlineto{\pgfqpoint{4.510243in}{1.553811in}}%
\pgfpathlineto{\pgfqpoint{4.528584in}{1.557640in}}%
\pgfpathlineto{\pgfqpoint{4.559152in}{1.554582in}}%
\pgfpathlineto{\pgfqpoint{4.571379in}{1.557123in}}%
\pgfpathlineto{\pgfqpoint{4.620287in}{1.552262in}}%
\pgfpathlineto{\pgfqpoint{4.638627in}{1.556052in}}%
\pgfpathlineto{\pgfqpoint{4.669195in}{1.553032in}}%
\pgfpathlineto{\pgfqpoint{4.675309in}{1.554290in}}%
\pgfpathlineto{\pgfqpoint{4.687536in}{1.553087in}}%
\pgfpathlineto{\pgfqpoint{4.693649in}{1.554343in}}%
\pgfpathlineto{\pgfqpoint{4.718103in}{1.551942in}}%
\pgfpathlineto{\pgfqpoint{4.736444in}{1.555697in}}%
\pgfpathlineto{\pgfqpoint{4.748671in}{1.554499in}}%
\pgfpathlineto{\pgfqpoint{4.754784in}{1.555747in}}%
\pgfpathlineto{\pgfqpoint{4.760898in}{1.555148in}}%
\pgfpathlineto{\pgfqpoint{4.803692in}{1.563839in}}%
\pgfpathlineto{\pgfqpoint{4.809806in}{1.563239in}}%
\pgfpathlineto{\pgfqpoint{4.840374in}{1.569401in}}%
\pgfpathlineto{\pgfqpoint{4.852601in}{1.568198in}}%
\pgfpathlineto{\pgfqpoint{4.858714in}{1.569425in}}%
\pgfpathlineto{\pgfqpoint{4.901509in}{1.565236in}}%
\pgfpathlineto{\pgfqpoint{4.907622in}{1.566460in}}%
\pgfpathlineto{\pgfqpoint{4.932076in}{1.564080in}}%
\pgfpathlineto{\pgfqpoint{4.968757in}{1.571381in}}%
\pgfpathlineto{\pgfqpoint{5.066574in}{1.561937in}}%
\pgfpathlineto{\pgfqpoint{5.072687in}{1.563143in}}%
\pgfpathlineto{\pgfqpoint{5.078801in}{1.562558in}}%
\pgfpathlineto{\pgfqpoint{5.091028in}{1.564966in}}%
\pgfpathlineto{\pgfqpoint{5.164390in}{1.557992in}}%
\pgfpathlineto{\pgfqpoint{5.170504in}{1.559189in}}%
\pgfpathlineto{\pgfqpoint{5.201071in}{1.556310in}}%
\pgfpathlineto{\pgfqpoint{5.207185in}{1.557504in}}%
\pgfpathlineto{\pgfqpoint{5.213298in}{1.556930in}}%
\pgfpathlineto{\pgfqpoint{5.219412in}{1.558122in}}%
\pgfpathlineto{\pgfqpoint{5.225525in}{1.557548in}}%
\pgfpathlineto{\pgfqpoint{5.231639in}{1.558739in}}%
\pgfpathlineto{\pgfqpoint{5.256093in}{1.556448in}}%
\pgfpathlineto{\pgfqpoint{5.280547in}{1.561192in}}%
\pgfpathlineto{\pgfqpoint{5.317228in}{1.557768in}}%
\pgfpathlineto{\pgfqpoint{5.329455in}{1.560127in}}%
\pgfpathlineto{\pgfqpoint{5.353909in}{1.557854in}}%
\pgfpathlineto{\pgfqpoint{5.360023in}{1.559031in}}%
\pgfpathlineto{\pgfqpoint{5.372250in}{1.557897in}}%
\pgfpathlineto{\pgfqpoint{5.378363in}{1.559072in}}%
\pgfpathlineto{\pgfqpoint{5.402817in}{1.556811in}}%
\pgfpathlineto{\pgfqpoint{5.421158in}{1.560323in}}%
\pgfpathlineto{\pgfqpoint{5.433385in}{1.559194in}}%
\pgfpathlineto{\pgfqpoint{5.463953in}{1.565018in}}%
\pgfpathlineto{\pgfqpoint{5.470066in}{1.564453in}}%
\pgfpathlineto{\pgfqpoint{5.488407in}{1.567930in}}%
\pgfpathlineto{\pgfqpoint{5.500634in}{1.566800in}}%
\pgfpathlineto{\pgfqpoint{5.506747in}{1.567956in}}%
\pgfpathlineto{\pgfqpoint{5.512861in}{1.567391in}}%
\pgfpathlineto{\pgfqpoint{5.531201in}{1.570850in}}%
\pgfpathlineto{\pgfqpoint{5.537315in}{1.570286in}}%
\pgfpathlineto{\pgfqpoint{5.549542in}{1.572585in}}%
\pgfpathlineto{\pgfqpoint{5.573996in}{1.570329in}}%
\pgfpathlineto{\pgfqpoint{5.580109in}{1.571475in}}%
\pgfpathlineto{\pgfqpoint{5.647358in}{1.565320in}}%
\pgfpathlineto{\pgfqpoint{5.653472in}{1.566462in}}%
\pgfpathlineto{\pgfqpoint{5.702380in}{1.562029in}}%
\pgfpathlineto{\pgfqpoint{5.714607in}{1.564303in}}%
\pgfpathlineto{\pgfqpoint{5.720720in}{1.563751in}}%
\pgfpathlineto{\pgfqpoint{5.726834in}{1.564886in}}%
\pgfpathlineto{\pgfqpoint{5.787969in}{1.559396in}}%
\pgfpathlineto{\pgfqpoint{5.818537in}{1.565036in}}%
\pgfpathlineto{\pgfqpoint{5.830764in}{1.563942in}}%
\pgfpathlineto{\pgfqpoint{5.836877in}{1.565066in}}%
\pgfpathlineto{\pgfqpoint{5.861331in}{1.562883in}}%
\pgfpathlineto{\pgfqpoint{5.873558in}{1.565125in}}%
\pgfpathlineto{\pgfqpoint{5.940807in}{1.559165in}}%
\pgfpathlineto{\pgfqpoint{5.946921in}{1.560280in}}%
\pgfpathlineto{\pgfqpoint{5.977488in}{1.557593in}}%
\pgfpathlineto{\pgfqpoint{5.989715in}{1.559817in}}%
\pgfpathlineto{\pgfqpoint{6.063078in}{1.553417in}}%
\pgfpathlineto{\pgfqpoint{6.069191in}{1.554524in}}%
\pgfpathlineto{\pgfqpoint{6.081418in}{1.553465in}}%
\pgfpathlineto{\pgfqpoint{6.105872in}{1.557878in}}%
\pgfpathlineto{\pgfqpoint{6.142553in}{1.554706in}}%
\pgfpathlineto{\pgfqpoint{6.154780in}{1.556902in}}%
\pgfpathlineto{\pgfqpoint{6.160894in}{1.556375in}}%
\pgfpathlineto{\pgfqpoint{6.167007in}{1.557471in}}%
\pgfpathlineto{\pgfqpoint{6.173121in}{1.556944in}}%
\pgfpathlineto{\pgfqpoint{6.215916in}{1.564582in}}%
\pgfpathlineto{\pgfqpoint{6.240370in}{1.562471in}}%
\pgfpathlineto{\pgfqpoint{6.246483in}{1.563556in}}%
\pgfpathlineto{\pgfqpoint{6.264824in}{1.561978in}}%
\pgfpathlineto{\pgfqpoint{6.270937in}{1.563062in}}%
\pgfpathlineto{\pgfqpoint{6.338186in}{1.557313in}}%
\pgfpathlineto{\pgfqpoint{6.362640in}{1.561623in}}%
\pgfpathlineto{\pgfqpoint{6.387094in}{1.559544in}}%
\pgfpathlineto{\pgfqpoint{6.393208in}{1.560617in}}%
\pgfpathlineto{\pgfqpoint{6.448229in}{1.555968in}}%
\pgfpathlineto{\pgfqpoint{6.460456in}{1.558107in}}%
\pgfpathlineto{\pgfqpoint{6.466570in}{1.557593in}}%
\pgfpathlineto{\pgfqpoint{6.503251in}{1.563981in}}%
\pgfpathlineto{\pgfqpoint{6.509365in}{1.563465in}}%
\pgfpathlineto{\pgfqpoint{6.521592in}{1.565586in}}%
\pgfpathlineto{\pgfqpoint{6.527705in}{1.565070in}}%
\pgfpathlineto{\pgfqpoint{6.552159in}{1.569297in}}%
\pgfpathlineto{\pgfqpoint{6.601067in}{1.565181in}}%
\pgfpathlineto{\pgfqpoint{6.607181in}{1.566233in}}%
\pgfpathlineto{\pgfqpoint{6.631635in}{1.564187in}}%
\pgfpathlineto{\pgfqpoint{6.637748in}{1.565236in}}%
\pgfpathlineto{\pgfqpoint{6.643862in}{1.564726in}}%
\pgfpathlineto{\pgfqpoint{6.649975in}{1.565774in}}%
\pgfpathlineto{\pgfqpoint{6.692770in}{1.562213in}}%
\pgfpathlineto{\pgfqpoint{6.704997in}{1.564302in}}%
\pgfpathlineto{\pgfqpoint{6.711111in}{1.563795in}}%
\pgfpathlineto{\pgfqpoint{6.711111in}{1.563795in}}%
\pgfusepath{stroke}%
\end{pgfscope}%
\begin{pgfscope}%
\pgfpathrectangle{\pgfqpoint{0.603704in}{0.549691in}}{\pgfqpoint{6.107407in}{3.101235in}}%
\pgfusepath{clip}%
\pgfsetbuttcap%
\pgfsetroundjoin%
\pgfsetlinewidth{1.505625pt}%
\definecolor{currentstroke}{rgb}{0.172549,0.627451,0.172549}%
\pgfsetstrokecolor{currentstroke}%
\pgfsetstrokeopacity{0.700000}%
\pgfsetdash{{5.550000pt}{2.400000pt}}{0.000000pt}%
\pgfpathmoveto{\pgfqpoint{0.603704in}{1.583436in}}%
\pgfpathlineto{\pgfqpoint{6.711111in}{1.583436in}}%
\pgfusepath{stroke}%
\end{pgfscope}%
\begin{pgfscope}%
\pgfsetrectcap%
\pgfsetmiterjoin%
\pgfsetlinewidth{0.803000pt}%
\definecolor{currentstroke}{rgb}{0.000000,0.000000,0.000000}%
\pgfsetstrokecolor{currentstroke}%
\pgfsetdash{}{0pt}%
\pgfpathmoveto{\pgfqpoint{0.603704in}{0.549691in}}%
\pgfpathlineto{\pgfqpoint{0.603704in}{3.650926in}}%
\pgfusepath{stroke}%
\end{pgfscope}%
\begin{pgfscope}%
\pgfsetrectcap%
\pgfsetmiterjoin%
\pgfsetlinewidth{0.803000pt}%
\definecolor{currentstroke}{rgb}{0.000000,0.000000,0.000000}%
\pgfsetstrokecolor{currentstroke}%
\pgfsetdash{}{0pt}%
\pgfpathmoveto{\pgfqpoint{6.711111in}{0.549691in}}%
\pgfpathlineto{\pgfqpoint{6.711111in}{3.650926in}}%
\pgfusepath{stroke}%
\end{pgfscope}%
\begin{pgfscope}%
\pgfsetrectcap%
\pgfsetmiterjoin%
\pgfsetlinewidth{0.803000pt}%
\definecolor{currentstroke}{rgb}{0.000000,0.000000,0.000000}%
\pgfsetstrokecolor{currentstroke}%
\pgfsetdash{}{0pt}%
\pgfpathmoveto{\pgfqpoint{0.603704in}{0.549691in}}%
\pgfpathlineto{\pgfqpoint{6.711111in}{0.549691in}}%
\pgfusepath{stroke}%
\end{pgfscope}%
\begin{pgfscope}%
\pgfsetrectcap%
\pgfsetmiterjoin%
\pgfsetlinewidth{0.803000pt}%
\definecolor{currentstroke}{rgb}{0.000000,0.000000,0.000000}%
\pgfsetstrokecolor{currentstroke}%
\pgfsetdash{}{0pt}%
\pgfpathmoveto{\pgfqpoint{0.603704in}{3.650926in}}%
\pgfpathlineto{\pgfqpoint{6.711111in}{3.650926in}}%
\pgfusepath{stroke}%
\end{pgfscope}%
\begin{pgfscope}%
\definecolor{textcolor}{rgb}{0.000000,0.000000,0.000000}%
\pgfsetstrokecolor{textcolor}%
\pgfsetfillcolor{textcolor}%
\pgftext[x=3.657407in,y=3.734260in,,base]{\color{textcolor}{\rmfamily\fontsize{12.000000}{14.400000}\selectfont\catcode`\^=\active\def^{\ifmmode\sp\else\^{}\fi}\catcode`\%=\active\def%{\%}Convergence of empirical state frequencies to the stationary distribution}}%
\end{pgfscope}%
\begin{pgfscope}%
\pgfsetbuttcap%
\pgfsetmiterjoin%
\definecolor{currentfill}{rgb}{1.000000,1.000000,1.000000}%
\pgfsetfillcolor{currentfill}%
\pgfsetfillopacity{0.800000}%
\pgfsetlinewidth{1.003750pt}%
\definecolor{currentstroke}{rgb}{0.800000,0.800000,0.800000}%
\pgfsetstrokecolor{currentstroke}%
\pgfsetstrokeopacity{0.800000}%
\pgfsetdash{}{0pt}%
\pgfpathmoveto{\pgfqpoint{4.098702in}{2.544621in}}%
\pgfpathlineto{\pgfqpoint{6.630125in}{2.544621in}}%
\pgfpathquadraticcurveto{\pgfqpoint{6.653263in}{2.544621in}}{\pgfqpoint{6.653263in}{2.567760in}}%
\pgfpathlineto{\pgfqpoint{6.653263in}{3.569940in}}%
\pgfpathquadraticcurveto{\pgfqpoint{6.653263in}{3.593079in}}{\pgfqpoint{6.630125in}{3.593079in}}%
\pgfpathlineto{\pgfqpoint{4.098702in}{3.593079in}}%
\pgfpathquadraticcurveto{\pgfqpoint{4.075563in}{3.593079in}}{\pgfqpoint{4.075563in}{3.569940in}}%
\pgfpathlineto{\pgfqpoint{4.075563in}{2.567760in}}%
\pgfpathquadraticcurveto{\pgfqpoint{4.075563in}{2.544621in}}{\pgfqpoint{4.098702in}{2.544621in}}%
\pgfpathlineto{\pgfqpoint{4.098702in}{2.544621in}}%
\pgfpathclose%
\pgfusepath{stroke,fill}%
\end{pgfscope}%
\begin{pgfscope}%
\pgfsetrectcap%
\pgfsetroundjoin%
\pgfsetlinewidth{1.505625pt}%
\definecolor{currentstroke}{rgb}{0.121569,0.466667,0.705882}%
\pgfsetstrokecolor{currentstroke}%
\pgfsetdash{}{0pt}%
\pgfpathmoveto{\pgfqpoint{4.121841in}{3.503961in}}%
\pgfpathlineto{\pgfqpoint{4.237536in}{3.503961in}}%
\pgfpathlineto{\pgfqpoint{4.353230in}{3.503961in}}%
\pgfusepath{stroke}%
\end{pgfscope}%
\begin{pgfscope}%
\definecolor{textcolor}{rgb}{0.000000,0.000000,0.000000}%
\pgfsetstrokecolor{textcolor}%
\pgfsetfillcolor{textcolor}%
\pgftext[x=4.445786in,y=3.463468in,left,base]{\color{textcolor}{\rmfamily\fontsize{8.330000}{9.996000}\selectfont\catcode`\^=\active\def^{\ifmmode\sp\else\^{}\fi}\catcode`\%=\active\def%{\%}State 1 (low engagement) (empirical)}}%
\end{pgfscope}%
\begin{pgfscope}%
\pgfsetbuttcap%
\pgfsetroundjoin%
\pgfsetlinewidth{1.505625pt}%
\definecolor{currentstroke}{rgb}{0.121569,0.466667,0.705882}%
\pgfsetstrokecolor{currentstroke}%
\pgfsetstrokeopacity{0.700000}%
\pgfsetdash{{5.550000pt}{2.400000pt}}{0.000000pt}%
\pgfpathmoveto{\pgfqpoint{4.121841in}{3.335003in}}%
\pgfpathlineto{\pgfqpoint{4.353230in}{3.335003in}}%
\pgfusepath{stroke}%
\end{pgfscope}%
\begin{pgfscope}%
\definecolor{textcolor}{rgb}{0.000000,0.000000,0.000000}%
\pgfsetstrokecolor{textcolor}%
\pgfsetfillcolor{textcolor}%
\pgftext[x=4.445786in,y=3.294510in,left,base]{\color{textcolor}{\rmfamily\fontsize{8.330000}{9.996000}\selectfont\catcode`\^=\active\def^{\ifmmode\sp\else\^{}\fi}\catcode`\%=\active\def%{\%}$\pi_1 = 0.33$ (theoretical)}}%
\end{pgfscope}%
\begin{pgfscope}%
\pgfsetrectcap%
\pgfsetroundjoin%
\pgfsetlinewidth{1.505625pt}%
\definecolor{currentstroke}{rgb}{1.000000,0.498039,0.054902}%
\pgfsetstrokecolor{currentstroke}%
\pgfsetdash{}{0pt}%
\pgfpathmoveto{\pgfqpoint{4.121841in}{3.166044in}}%
\pgfpathlineto{\pgfqpoint{4.237536in}{3.166044in}}%
\pgfpathlineto{\pgfqpoint{4.353230in}{3.166044in}}%
\pgfusepath{stroke}%
\end{pgfscope}%
\begin{pgfscope}%
\definecolor{textcolor}{rgb}{0.000000,0.000000,0.000000}%
\pgfsetstrokecolor{textcolor}%
\pgfsetfillcolor{textcolor}%
\pgftext[x=4.445786in,y=3.125551in,left,base]{\color{textcolor}{\rmfamily\fontsize{8.330000}{9.996000}\selectfont\catcode`\^=\active\def^{\ifmmode\sp\else\^{}\fi}\catcode`\%=\active\def%{\%}State 2 (medium engagement) (empirical)}}%
\end{pgfscope}%
\begin{pgfscope}%
\pgfsetbuttcap%
\pgfsetroundjoin%
\pgfsetlinewidth{1.505625pt}%
\definecolor{currentstroke}{rgb}{1.000000,0.498039,0.054902}%
\pgfsetstrokecolor{currentstroke}%
\pgfsetstrokeopacity{0.700000}%
\pgfsetdash{{5.550000pt}{2.400000pt}}{0.000000pt}%
\pgfpathmoveto{\pgfqpoint{4.121841in}{2.997086in}}%
\pgfpathlineto{\pgfqpoint{4.353230in}{2.997086in}}%
\pgfusepath{stroke}%
\end{pgfscope}%
\begin{pgfscope}%
\definecolor{textcolor}{rgb}{0.000000,0.000000,0.000000}%
\pgfsetstrokecolor{textcolor}%
\pgfsetfillcolor{textcolor}%
\pgftext[x=4.445786in,y=2.956593in,left,base]{\color{textcolor}{\rmfamily\fontsize{8.330000}{9.996000}\selectfont\catcode`\^=\active\def^{\ifmmode\sp\else\^{}\fi}\catcode`\%=\active\def%{\%}$\pi_2 = 0.33$ (theoretical)}}%
\end{pgfscope}%
\begin{pgfscope}%
\pgfsetrectcap%
\pgfsetroundjoin%
\pgfsetlinewidth{1.505625pt}%
\definecolor{currentstroke}{rgb}{0.172549,0.627451,0.172549}%
\pgfsetstrokecolor{currentstroke}%
\pgfsetdash{}{0pt}%
\pgfpathmoveto{\pgfqpoint{4.121841in}{2.828128in}}%
\pgfpathlineto{\pgfqpoint{4.237536in}{2.828128in}}%
\pgfpathlineto{\pgfqpoint{4.353230in}{2.828128in}}%
\pgfusepath{stroke}%
\end{pgfscope}%
\begin{pgfscope}%
\definecolor{textcolor}{rgb}{0.000000,0.000000,0.000000}%
\pgfsetstrokecolor{textcolor}%
\pgfsetfillcolor{textcolor}%
\pgftext[x=4.445786in,y=2.787635in,left,base]{\color{textcolor}{\rmfamily\fontsize{8.330000}{9.996000}\selectfont\catcode`\^=\active\def^{\ifmmode\sp\else\^{}\fi}\catcode`\%=\active\def%{\%}State 3 (high engagement) (empirical)}}%
\end{pgfscope}%
\begin{pgfscope}%
\pgfsetbuttcap%
\pgfsetroundjoin%
\pgfsetlinewidth{1.505625pt}%
\definecolor{currentstroke}{rgb}{0.172549,0.627451,0.172549}%
\pgfsetstrokecolor{currentstroke}%
\pgfsetstrokeopacity{0.700000}%
\pgfsetdash{{5.550000pt}{2.400000pt}}{0.000000pt}%
\pgfpathmoveto{\pgfqpoint{4.121841in}{2.659169in}}%
\pgfpathlineto{\pgfqpoint{4.353230in}{2.659169in}}%
\pgfusepath{stroke}%
\end{pgfscope}%
\begin{pgfscope}%
\definecolor{textcolor}{rgb}{0.000000,0.000000,0.000000}%
\pgfsetstrokecolor{textcolor}%
\pgfsetfillcolor{textcolor}%
\pgftext[x=4.445786in,y=2.618676in,left,base]{\color{textcolor}{\rmfamily\fontsize{8.330000}{9.996000}\selectfont\catcode`\^=\active\def^{\ifmmode\sp\else\^{}\fi}\catcode`\%=\active\def%{\%}$\pi_3 = 0.33$ (theoretical)}}%
\end{pgfscope}%
\end{pgfpicture}%
\makeatother%
\endgroup%

  }
  \caption{\label{fig:markov_2}Simulation of the three–state Markov chain  (t=1000,\ldots, 2000)}
\end{figure}




\clearpage
\bibliographystyle{dcudoi}
\bibliography{transpor}

\end{document}


