\documentclass[12pt,a4paper]{article}

% Package to include code
\usepackage{listings}
\usepackage{inconsolata}
\usepackage{color}
\newcommand*{\boldone}{\text{\usefont{U}{bbold}{m}{n}1}}
\usepackage{tikz}
\usepackage{varioref}
\usepackage{pgfplots}
\usepackage{lscape}
\lstset{language=Python}
\lstset{numbers=none, basicstyle=\ttfamily\footnotesize,
  numberstyle=\tiny,keywordstyle=\color{blue},stringstyle=\ttfamily,showstringspaces=false}
\lstset{backgroundcolor=\color[rgb]{0.95 0.95 0.95}}
\lstdefinestyle{numbers}{numbers=left, stepnumber=1,
  numberstyle=\tiny,basicstyle=\footnotesize, numbersep=10pt}
\lstdefinestyle{nonumbers}{numbers=none}
\lstset{
  breaklines=true,
  breakatwhitespace=true,
}


% Font selection: uncomment the next line to use the ``beton'' font
%\usepackage{beton}

% Font selection: uncomment the next line to use the ``times'' font
%\usepackage{times}

% Font for equations
\usepackage{euler}


%Package to define the headers and footers of the pages
\usepackage{fancyhdr}


%Package to include an index
\usepackage{index}

%Package to display boxes around texts. Used especially for the internal notes.
\usepackage{framed}

%PSTricks is a collection of PostScript-based TEX macros that is compatible
% with most TEX macro packages
\usepackage{pstricks}
\usepackage{pst-node}
\usepackage{pst-plot}
\usepackage{pst-tree}

%Package to display boxes around a minipage. Used especially to
%describe the biography of people.
\usepackage{boxedminipage}

%Package to include postscript figures
\usepackage{epsfig}

%Package for the bibliography
% \cite{XXX} produces Ben-Akiva et. al., 2010
% \citeasnoun{XXX} produces Ben-Akiva et al. (2010)
% \citeasnoun*{XXX} produces Ben-Akiva, Bierlaire, Bolduc and Walker (2010)
\usepackage[dcucite,abbr]{harvard}
\harvardparenthesis{none}\harvardyearparenthesis{round}

%Packages for advanced mathematics typesetting
\usepackage{amsmath,amsfonts,amssymb}

%Package to display trees easily
%\usepackage{xyling}

%Package to include smart references (on the next page, on the
%previous page, etc.)
%%

%% Remove as it is not working when the book will be procesed by the
%% publisher.
%\usepackage{varioref}

%Package to display the euro sign
\usepackage[right,official]{eurosym}

%Rotate material, especially large table (defines sidewaystable)
\usepackage[figuresright]{rotating}

%Defines the subfigure environment, to obtain refs like Figure 1(a)
%and Figure 1(b).
\usepackage{subfigure}

%Package for appendices. Allows subappendices, in particular
\usepackage{appendix}

%Package controling the fonts for the captions
\usepackage[font={small,sf}]{caption}

%Defines new types of columns for tabular ewnvironment
\usepackage{dcolumn}
\newcolumntype{d}{D{.}{.}{-1}}
\newcolumntype{P}[1]{>{#1\hspace{0pt}\arraybackslash}}
\newcolumntype{.}{D{.}{.}{9.3}}

%Allows multi-row cells in tables
\usepackage{multirow}

%Tables spaning more than one page
\usepackage{longtable}


%%
%%  Macros by Michel
%%

\newcommand{\PBIOGEME}{PythonBiogeme}
\newcommand{\PDBIOGEME}{Biogeme}
\newcommand{\BIOGEME}{Biogeme}
\newcommand{\BBIOGEME}{BisonBiogeme}


%Internal notes
\newcommand{\note}[1]{
  \begin{framed}{}%
    \textbf{\underline{Internal note}:} #1
\end{framed}}

%Use this version to turn off the notes
%\newcommand{\note}[1]{}


%Include a postscript figure . Note that the label is prefixed with
%``fig:''. Remember it when you refer to it.
%Three arguments:
% #1 label
% #2 file (without extension)
% #3 Caption
\newcommand{\afigure}[3]{%
  \begin{figure}[!tbp]%
    \begin{center}%
      \epsfig{figure=#2,width=0.8\textwidth}%
    \end{center}
    \caption{\label{fig:#1} #3}%
\end{figure}}






%Include two postscript figures side by side.
% #1 label of the first figure
% #2 file for the first figure
% #3 Caption for the first figure
% #4 label of the second figure
% #5 file for the second figure
% #6 Caption for the first figure
% #7 Caption for the set of two figures
\newcommand{\twofigures}[7]{%
  \begin{figure}[htb]%
    \begin{center}%
      \subfigure[\label{fig:#1}#3]{\epsfig{figure=#2,width=0.45\textwidth}}%
      \hfill
      \subfigure[\label{fig:#4}#6]{\epsfig{figure=#5,width=0.45\textwidth}}%
    \end{center}
    \caption{#7}%
\end{figure}}

%Include a figure generated by gnuplot using the epslatex output. Note that the label is prefixed with
%``fig:''. Remember it when you refer to it.

%Three arguments:
% #1 label
% #2 file (without extension)
% #3 Caption
\newcommand{\agnuplotfigure}[3]{%
  \begin{figure}[!tbp]%
    \begin{center}%
      \input{#2}%
    \end{center}
    \caption{\label{fig:#1} #3}%
\end{figure}}

%Three arguments:
% #1 label
% #2 file (without extension)
% #3 Caption
\newcommand{\asidewaysgnuplotfigure}[3]{%
  \begin{sidewaysfigure}[!tbp]%
    \begin{center}%
      \input{#2}%
    \end{center}
    \caption{\label{fig:#1} #3}%
\end{sidewaysfigure}}


%Include two postscript figures side by side.
% #1 label of the first figure
% #2 file for the first figure
% #3 Caption for the first figure
% #4 label of the second figure
% #5 file for the second figure
% #6 Caption for the second figure
% #7 Caption for the set of two figures
% #8 label for the whole figure
\newcommand{\twognuplotfigures}[7]{%
  \begin{figure}[htb]%
    \begin{center}%
      \subfigure[\label{fig:#1}#3]{\input{#2}}%
      \hfill
      \subfigure[\label{fig:#4}#6]{\input{#5}}%
    \end{center}
    \caption{#7}%
\end{figure}}



%Include the description of somebody. Four arguments:
% #1 label
% #2 Name
% #3 file (without extension)
% #4 description
\newcommand{\people}[4]{
  \begin{figure}[tbf]
    \begin{boxedminipage}{\textwidth}
      \parbox{0.40\textwidth}{\epsfig{figure=#3,width = 0.39\textwidth}}%\hfill
      \parbox{0.59\textwidth}{%
        #4%
      }%
    \end{boxedminipage}
    \caption{\label{fig:#1} #2}
  \end{figure}
}

%Default command for a definition
% #1 label (prefix def:)
% #2 concept to be defined
% #3 definition
\newtheorem{definition}{Definition}
\newcommand{\mydef}[3]{%
  \begin{definition}%
    \index{#2|textbf}%
    \label{def:#1}%
    \textbf{#2} \slshape #3\end{definition}}

%Reference to a definitoin. Prefix 'def:' is assumed
\newcommand{\refdef}[1]{definition~\ref{def:#1}}


%Default command for a theorem, with proof
% #1: label (prefix thm:)
% #2: name of the theorem
% #3: statement
% #4: proof
\newtheorem{theorem}{Theorem}
\newcommand{\mytheorem}[4]{%
  \begin{theorem}%
    \index{#2|textbf}%
    \index{Theorems!#2}%
    \label{thm:#1}%
    \textbf{#2} \sffamily \slshape #3
  \end{theorem} \bpr #4 \epr \par}


%Default command for a theorem, without proof
% #1: label (prefix thm:)
% #2: name of the theorem
% #3: statement
\newcommand{\mytheoremsp}[3]{%
  \begin{theorem}%
    \index{#2|textbf}%
    \index{Theorems!#2}%
    \label{thm:#1}%
    \textbf{#2} \sffamily \slshape #3
\end{theorem}}



%Put parentheses around the reference, as standard for equations
\newcommand{\req}[1]{(\ref{#1})}

%Short cut to make a column vector in math environment (centered)
\newcommand{\cvect}[1]{\left(\begin{array}{c} #1 \end{array} \right) }

%Short cut to make a column vector in math environment (right justified)
\newcommand{\rvect}[1]{\left(\begin{array}{r} #1 \end{array} \right) }

%A reference to a theorem. Prefix thm: is assumed for the label.
\newcommand{\refthm}[1]{Theorem~\ref{thm:#1}}

%Reference to a figure. Prefix fig: is assumed for the label.
\newcommand{\reffig}[1]{Figure~\ref{fig:#1}}

%Smart reference to a figure. Prefix fig: is assumed for the label.
%\newcommand{\vreffig}[1]{Figure~\vref{fig:#1}}

%C in mathcal font for the choice set
\newcommand{\C}{\mathcal{C}}

%R in bold font for the set of real numbers
\newcommand{\R}{\mathbb{R}}

%N in bold font for the set of natural numbers
\newcommand{\N}{\mathbb{N}}

%C in mathcal font for the log likelihood
\renewcommand{\L}{\mathcal{L}}

%S in mathcal font for the subset S
\renewcommand{\S}{\mathcal{S}}

%To write an half in math envionment
\newcommand{\half}{\frac{1}{2}}

%Probability
\newcommand{\prob}{\operatorname{Pr}}

%Expectation
\newcommand{\expect}{\operatorname{E}}

%Variance
\newcommand{\var}{\operatorname{Var}}

%Covariance
\newcommand{\cov}{\operatorname{Cov}}

%Correlation
\newcommand{\corr}{\operatorname{Corr}}

%Span
\newcommand{\myspan}{\operatorname{span}}

%plim
\newcommand{\plim}{\operatorname{plim}}

%Displays n in bold (for the normal distribution?)
\newcommand{\n}{{\bf n}}

%Includes footnote in a table environment. Warning: the footmark is
%always 1.
\newcommand{\tablefootnote}[1]{\begin{flushright}
    \rule{5cm}{1pt}\\
    \protect\footnotemark[1]{\footnotesize #1}
  \end{flushright}
}
\renewcommand*{\thefootnote}{\alph{footnote}}

%Defines the ``th'' as in ``19th'' to be a superscript
\renewcommand{\th}{\textsuperscript{th}}

%Begin and end of a proof
\newcommand{\bpr}{{\bf Proof.} \hspace{1 em}}
\newcommand{\epr}{$\Box$}


\title{Assisted specification with Biogeme 3.2.12}
\author{Michel Bierlaire \and Nicola Ortelli}
\date{August 16, 2023}

\newcommand*{\examplesPath}{../../examples}


\begin{document}


\begin{titlepage}
  \pagestyle{empty}

  \maketitle
  \vspace{2cm}

  \begin{center}
    \small Report TRANSP-OR 230816 \\ Transport and Mobility Laboratory \\ School of Architecture, Civil and Environmental Engineering \\ Ecole Polytechnique F\'ed\'erale de Lausanne \\ \verb+transp-or.epfl.ch+
    \begin{center}
      \textsc{Series on Biogeme}
    \end{center}
  \end{center}
    \begin{center}
\emph{This document is an updated version of \citeasnoun{BierOrte22}, adapted to version 3.2.12 of \PDBIOGEME.}
    \end{center}


  \clearpage
\end{titlepage}



The package Biogeme (\texttt{biogeme.epfl.ch}) is designed to estimate
the parameters of various models using maximum likelihood
estimation. It is particularly designed for discrete choice
models. It is a Python package written in Python and C++, that relies on the
Pandas library for the management of the data. 

This document describes how to obtain assistance from \PDBIOGEME\ for the model specification. In particular, it shows how to apply the
algorithm  described by
\citeasnoun{OrteHillCamadeLaBier21}. In a nutshell, an optimization
algorithm is used to generate models based on a minimal
number of inputs provided by the analyst. These inputs are used to
build a space of possible specifications that may contain any form of
variable interaction, nonlinear transformation, segmentation of the
population and potential choice models; the space is
then explored by an algorithm that sequentially introduces small
modifications to an initial set of promising specifications.

We assume that the reader is already familiar with discrete choice
models and \PDBIOGEME.\@   This document has
been written using \PDBIOGEME\ 3.2.12.

We use the Swissmetro example throughout the document. The Python
scripts are available on GitHub in the biogeme repository, in the directory examples/assisted. They are also reported in the Appendix.

\clearpage

\section{Catalogs}\label{sec:catalogs}

The philosophy of the assisted specification is that the analyst may
have several specifications in mind, but does not know a priori which
one is the most appropriate. \PDBIOGEME\ can then accept as input a
``catalog'' of different specifications, and estimate all
specifications in the catalog, and provide a comparative report of the
estimation results. It provides a great flexibility to the analyst who
can replace any expression of the model by such a catalog, as illustrated with the examples in this document.

In some cases, the number of possible specifications is so high that
an exhaustive enumeration is not feasible. In that case, the algorithm
proposed by \citeasnoun{OrteHillCamadeLaBier21} is applied in order to
investigate a subset of potentially promising specifications.

 The code used to generate the examples presented in this Section is
 available in Appendix~\ref{sec:simple}.


Each catalog is associated with a unique name, and a list of different
valid expressions, each of them also associated with a name. For
instance, suppose that we want to define a catalog that contains both
a logit and a nested logit models.


We first define each of the models, like in a regular Biogeme script:
\begin{lstlisting}
logprob_logit = models.loglogit(V, av, CHOICE)
\end{lstlisting}
and
\begin{lstlisting}
logprob_nested = models.lognested(V, av, nests, CHOICE)
\end{lstlisting}

The catalog can then be defined using the following syntax, that is self-explanatory:
\begin{lstlisting}
model_catalog = Catalog.from_dict(
    catalog_name='model_catalog',
    dict_of_expressions={
        'logit': logprob_logit,
        'nested': logprob_nested
    },
)
\end{lstlisting}
Note that the \lstinline@Catalog@ class must first be imported using the following syntax:
\begin{lstlisting}
from biogeme.catalog import Catalog
\end{lstlisting}

A \textbf{catalog} is a regular Biogeme expression, that can be used
in another expression. At each given point in time, exactly one of the
expressions of the catalog is active, and used for the evaluation of
the expression. For instance, if we print the catalog above, it
corresponds to the logit specification by default:
\begin{lstlisting}
print(model_catalog)
[model_catalog: logit]...
\end{lstlisting}
where the ellipsis is the actual expression of the logit model (which is too long to report in this document). 
In order to modify the configuration of a catalog, Biogeme uses a
\textbf{controller}, that is accessible using the
\lstinline@controlled_by@ attribute of the catalog. For instance, in
order to activate the nested logit specification, we need to write
\begin{lstlisting}
model_catalog.controlled_by.set_name('nested')  
\end{lstlisting}
Now, if we print the catalog again, we obtain
\begin{lstlisting}
print(model_catalog)
[model_catalog: nested]...
\end{lstlisting}
where the ellipsis is the actual expression of the nested logit model. 

In general, there is no need to explicitly access to the controller,
as Biogeme provides high level access to the catalog. The simplest one
is an iterator:
\begin{lstlisting}
for specification in model_catalog:
    print(specification)
\end{lstlisting}
provides the following output:
\begin{lstlisting}
[model_catalog: nested]...
[model_catalog: logit]...
\end{lstlisting}
For the sake of this document, instead of listing the expressions themselves (which can be long and complicated), we report the configuration identifiers of the controller, that identifies all possible specifications associated with a catalog. This is usually not needed by regular users. The function used to do that is described in Appendix~\ref{sec:print}.
For the \lstinline@model_catalog@, it gives the following output:
\begin{lstlisting}
model_catalog:logit
model_catalog:nested
\end{lstlisting}
Also, the Biogeme object has a function called
\lstinline@estimate_catalog@, that iterates on all specifications in a
catalog (if possible), and estimate the corresponding models. If there
are too many specifications to be enumerated, it launches the assisted
specification algorithm. This function is illustrated in
Section~\ref{sec:estimation}.

\subsection{Synchronized catalogs}

A catalog can be used for alternative nonlinear specifications of a variable. Here, we use the example of the train travel time, in the Swissmetro example. Again, we first define each specification separately:
\begin{enumerate}
\item the linear specification:
  \begin{lstlisting}
linear_train_tt = TRAIN_TT
  \end{lstlisting}
\item the Box-Cox transform:
\begin{lstlisting}
ell_travel_time = Beta('lambda_travel_time', 1, -10, 10, 0)
boxcox_train_tt = boxcox(TRAIN_TT_SCALED, ell_travel_time)
\end{lstlisting}
\item the squared variable:
  \begin{lstlisting}
squared_train_tt = TRAIN_TT * TRAIN_TT
  \end{lstlisting}
\end{enumerate}
Note that the \lstinline@boxcox@ function must first be imported as follows:
\begin{lstlisting}
from biogeme.models import boxcox
\end{lstlisting}
The catalog can  be defined, using the same syntax as above:
\begin{lstlisting}
train_tt_catalog = Catalog.from_dict(
    catalog_name='train_tt_catalog',
    dict_of_expressions={
        'linear': linear_train_tt,
        'boxcox': boxcox_train_tt,
        'squared': squared_train_tt,
    },
)  
\end{lstlisting}
The catalog can  be used as a regular expression in the definition of the utility function, for instance:
\begin{lstlisting}
V_TRAIN = ASC_TRAIN + B_TIME * train_tt_catalog + ...
\end{lstlisting}
Note that, because \lstinline@V_TRAIN@ contains a catalog, it is possible to iterate through its specifications as well:
\begin{lstlisting}
for specification in V_TRAIN:
    print(specification)
\end{lstlisting}
generates the following output:
\begin{lstlisting}
(ASC_TRAIN(init=0) + (B_TIME(init=0) * [train_tt_catalog: boxcox]...
(ASC_TRAIN(init=0) + (B_TIME(init=0) * [train_tt_catalog: linear]TRAIN_TT))
(ASC_TRAIN(init=0) + (B_TIME(init=0) * [train_tt_catalog: squared](TRAIN_TT * TRAIN_TT)))
\end{lstlisting}
where the ellipsis is replaced by the complete specification of the Box-Cox model.

Now, we would like to specify a similar catalog for the car travel time, in the same model. We apply the exact same syntax as above:
\begin{lstlisting}
CAR_TT = Variable('CAR_TT')
linear_car_tt = CAR_TT
boxcox_car_tt = boxcox(CAR_TT, ell_travel_time)
squared_car_tt = CAR_TT * CAR_TT
car_tt_catalog = Catalog.from_dict(
    catalog_name='car_tt_catalog',
    dict_of_expressions={
        'linear': linear_car_tt,
        'boxcox': boxcox_car_tt,
        'squared': squared_car_tt,
    },
)
\end{lstlisting}
In order to illustrate how those catalogs are combined, we build a dummy expression that calculates their sum:
\begin{lstlisting}
dummy_expression = train_tt_catalog + car_tt_catalog  
\end{lstlisting}
If we print all possible configurations, we obtain nine combinations (the order in which they appear is irrelevant):
\begin{lstlisting}
car_tt_catalog:linear;train_tt_catalog:linear
car_tt_catalog:linear;train_tt_catalog:boxcox
car_tt_catalog:linear;train_tt_catalog:squared
car_tt_catalog:boxcox;train_tt_catalog:squared
car_tt_catalog:boxcox;train_tt_catalog:boxcox
car_tt_catalog:boxcox;train_tt_catalog:linear
car_tt_catalog:squared;train_tt_catalog:linear
car_tt_catalog:squared;train_tt_catalog:boxcox
car_tt_catalog:squared;train_tt_catalog:squared
\end{lstlisting}

Indeed, the combination of three configurations for one variable and
three configurations for the other one gives nine
specifications. However, this is not always the desired effect. It is
actually often desirable that the same nonlinear transform is applied
to both variables. In that case, we need to synchronize the two
catalogs. It means that they must be controlled by the same
controller. This is achieved by constructing the second catalog as
follows:
\begin{lstlisting}
  car_tt_catalog = Catalog.from_dict(
    catalog_name='car_tt_catalog',
    dict_of_expressions={
        'linear': linear_car_tt,
        'boxcox': boxcox_car_tt,
        'squared': squared_car_tt,
    },
    controlled_by=train_tt_catalog.controlled_by
)
\end{lstlisting}
The \lstinline@controlled_by@ argument allows to explicitly specify a
controller for the catalog. In this case, we provide the controller of
the \lstinline@train_tt_catalog@. Note that it is required that
synchronized catalogs have exactly the same set of labels to identify
their entries. If we now report the specifications  of the dummy expression defined
above, we obtain only three specifications, where both variables are
associated with the same transformation:
\begin{lstlisting}
train_tt_catalog:linear
train_tt_catalog:squared
train_tt_catalog:boxcox
\end{lstlisting}
Note that only the controller of the train travel time catalog is involved, as it is used also for the car travel time.

\subsection{Alternative-specific coefficient}\label{sec:alt_spec}
In discrete choice models, it is typical to test a specification where
the coefficient of a variable is generic, that is, the same for all
alternatives, or alternative-specific. For example, we are considering a catalog containing
 specifications where the cost coefficient and the time coefficient
should be both generic, or both alternative-specific. In order to build such a catalog, we need the function \lstinline@generic_alt_specific_catalogs@ that can be imported as follows:
\begin{lstlisting}
from biogeme.catalog import generic_alt_specific_catalogs
\end{lstlisting}
The following syntax is used:
\begin{lstlisting}
(B_TIME_catalog_dict, B_COST_catalog_dict) = generic_alt_specific_catalogs(
    generic_name='coefficients',
    beta_parameters=[B_TIME, B_COST],
    alternatives=('TRAIN', 'CAR')
)
\end{lstlisting}
The function takes three\footnote{As discussed later, it actually takes five arguments, but two of them have default values.} arguments:
\begin{enumerate}
\item a generic name that identifies the catalogs,
\item a list of parameters, defined with \lstinline@Beta@,
\item a tuple containing the names identifying the alternatives. 
\end{enumerate}
It returns a tuple of dictionaries where the keys are the name of the alternatives, and the values are the corresponding catalogs. They are used as follows:
\begin{lstlisting}
V_TRAIN = (
    B_TIME_catalog_dict['TRAIN'] * TRAIN_TT +
    B_COST_catalog_dict['TRAIN'] * TRAIN_COST
)
V_CAR = (
    B_TIME_catalog_dict['CAR'] * CAR_TT +
    B_COST_catalog_dict['CAR'] * CAR_COST
)
\end{lstlisting}
In order to illustrate the catalogs, we build again a dummy expression:
\begin{lstlisting}
dummy_expression = V_TRAIN + V_CAR
\end{lstlisting}
There are two possible configurations for this expression, one where both coefficients are alternative-specific, and one where both are generic.
\begin{lstlisting}
coefficients_gen_altspec:generic
coefficients_gen_altspec:altspec
\end{lstlisting}

If it is not desirable to have both coefficients synchronized, two different calls to the function must be performed:
\begin{lstlisting}
(B_TIME_catalog_dict, ) = generic_alt_specific_catalogs(
    generic_name='time_coefficient',
    beta_parameters=[B_TIME],
    alternatives=('TRAIN', 'CAR')
)

(B_COST_catalog_dict, ) = generic_alt_specific_catalogs(
    generic_name='cost_coefficient',
    beta_parameters=[B_COST],
    alternatives=('TRAIN', 'CAR')
)
\end{lstlisting}
Note that the function returns a tuple. And if the tuple contains only one entry (as in this example), a comma must be explicitly mentioned in order to obtain this single entry. An equivalent syntax would be
\begin{lstlisting}
B_TIME_catalog_dict_tuple = generic_alt_specific_catalogs(
    generic_name='time_coefficient',
    beta_parameters=[B_TIME],
    alternatives=('TRAIN', 'CAR')
)
B_TIME_catalog_dict = B_TIME_catalog_dict_tuple[0]
\end{lstlisting}
As the two specifications are now independent, iterating on the dummy
expression provides four specifications:
\begin{lstlisting}
cost_coefficient_gen_altspec:generic;time_coefficient_gen_altspec:generic
cost_coefficient_gen_altspec:generic;time_coefficient_gen_altspec:altspec
cost_coefficient_gen_altspec:altspec;time_coefficient_gen_altspec:generic
cost_coefficient_gen_altspec:altspec;time_coefficient_gen_altspec:altspec
\end{lstlisting}

\subsection{Segmentations}

In order to capture potential taste heterogeneity, specifications
where a coefficient takes different values for different segments of the
population can be investigated. The population is segmented using
discrete socio-economic characteristics. If such a discrete variable
takes $L$ values, they correspond to $L$ segments in the
population. But several such variables can be combined to define a
segmentation. If $K$ socio-economic characteristics are considered,
each of them with $L_k$ discrete values, a total of $\prod_{k=1}^K
L_k$ segments can potentially be defined, and a different coefficient
associated with each of them. However, the number of segments defined
in this way grows exponentially with $K$. It is statistically impossible to estimate a different coefficient for each segment when $K$ is high. Therefore, we consider a simplified  segmentation method that proceeds as follows:
\begin{itemize}
\item Define a reference coefficient $\beta_\text{ref}$.
\item For each socio-economic characteristic $x_k$, select one value
  that corresponds to the reference. Without loss of generality,
  assume that it is the first one.
\item Introduce a parameter $\beta_k^\ell$, for each other value $\ell=2, \ldots, L_k$.
\item The value of the coefficient as a function of the socio-economic characteristics is defined as
  \[
\beta(x_1, \ldots, x_K) = \beta_\text{ref} + \sum_{k=1}^K \sum_{\ell=2}^{L_k} \beta_k^\ell \;\boldone[x_k = \ell],
\]
where $\boldone[x_k = \ell]$ is 1 if the condition within the brackets is true, and 0 otherwise.
\end{itemize}
The number of parameters is therefore $1 - K + \sum_{k=1}^K L_k$, which grows linearly with $K$.

Let's take an example with $K=2$, where the first socio-economic
characteristic segments the population between individuals who are
commuters from those who are not, and the second segments the
population into individuals without luggage, those carrying one piece
of luggage, and those carrying more than one piece of
luggage. Therefore, $L_1=2$ and $L_2=3$. This segmentation is
associated with $1-2+2+3=4$ coefficients:
\begin{itemize}
\item $\beta_\text{ref}$,
\item $\beta_1^\text{commuters}$,
\item $\beta_2^\text{one\_luggage}$,
\item $\beta_2^\text{several\_luggages}$,
\end{itemize}
where the values ``non commuters'' and ``no luggage'' are used as reference for each variable, respectively.
Now, note that the number of segments is  $2\cdot3=6$. The value of the coefficient associated with each of them can be reconstructed from the above coefficients as follows:
\[
  \begin{array}{lll}
    \text{Commuter} & \text{Luggages} & \text{Coefficient}\\
    \hline
       \text{yes} &    0  &  \beta_\text{ref} + \beta_1^\text{commuters}\\
       \text{yes} &    1  &  \beta_\text{ref} + \beta_1^\text{commuters} + \beta_2^\text{one\_luggage}\\
       \text{yes} &    >1  & \beta_\text{ref} + \beta_1^\text{commuters} + \beta_2^\text{several\_luggages}\\
       \text{no} &    0  & \beta_\text{ref}  \\
       \text{no} &    1  &  \beta_\text{ref} + \beta_2^\text{one\_luggage}\\
       \text{no} &    >1  & \beta_\text{ref} + \beta_2^\text{several\_luggages} \\
  \end{array}
  \]
This simplified procedure makes the implicit assumption that the combined effects of two socio-economic characteristics is the sum of two specific effects. This is the price to pay to deal with the curse of dimensionality.  

In this context, we would like to construct a catalog that contains
the following specifications:
\begin{itemize}
\item no segmentation, that is, the same coefficient for the whole population,
\item a segmentation with the first variable only, that is $1-1+2=2$ coefficients: $\beta_\text{ref}$ and $\beta_1^\text{commuters}$,
\item a segmentation with the second variable only, that is $1-1+3=3$ coefficients: $\beta_\text{ref}$, $\beta_2^\text{one\_luggage}$ and $\beta_2^\text{several\_luggages}$,
\item a segmentation with both variables, involving 4 coefficients as described above.
\end{itemize}

And we would like to apply these segmentations to two alternative-specific constants, that must be segmented in the same way. To do that with \PDBIOGEME, we first need to define the segmentations, using the following syntax:
\begin{lstlisting}
segmentation_purpose = database.generate_segmentation(
    variable='COMMUTERS',
    mapping={
        0: 'non_commuters',
        1: 'commuters'
    },
    reference='non_commuters'
    
)
segmentation_luggage = database.generate_segmentation(
    variable='LUGGAGE',
    mapping={
        0: 'no_lugg',
        1: 'one_lugg',
        3: 'several_lugg'
    },
    reference='no_lugg'
)
\end{lstlisting}
where the function \lstinline@generate_segmentation@ takes the following two arguments:
\begin{itemize}
\item the name of the discrete socio-economic characteristic in the database,
\item a dictionary mapping the values of the variables in the database, and a name identifying what they mean,
\item the name of the reference level.
\end{itemize}
Note that the name of the reference level can be omitted. One of the
levels will then be arbitrarily chosen as the reference.
We can now create the catalogs themselves:
\begin{lstlisting}
ASC_TRAIN_catalog, ASC_CAR_catalog = segmentation_catalogs(
    generic_name='ASC',
    beta_parameters=[ASC_TRAIN, ASC_CAR],
    potential_segmentations=(
        segmentation_purpose,
        segmentation_luggage,
    ),
    maximum_number=2,
)
\end{lstlisting}
where the function \lstinline@segmentation_catalogs@ can be imported using the following statement
\begin{lstlisting}
from biogeme.catalog import segmentation_catalogs
\end{lstlisting}
It takes four arguments:
\begin{enumerate}
\item a generic name that applies to all specifications,
\item a list of parameters to be segmented,
\item a list of potential segmentations,
\item the maximum number of segmentations that can be activated at the same time. 
\end{enumerate}
If we report the configurations of the dummy expression defined as the sum of the two catalogs, we obtain the following four configurations:
\begin{lstlisting}
ASC:no_seg
ASC:LUGGAGE
ASC:COMMUTERS
ASC:COMMUTERS-LUGGAGE
\end{lstlisting}
If we call the same function with the parameter  \lstinline@maximum_number@ set to 1, we obtain 
\begin{lstlisting}
ASC:no_seg
ASC:LUGGAGE
ASC:COMMUTERS
\end{lstlisting}
as the interaction with both variables is not allowed anymore.

\subsection{Alternative-specific and segmented coefficients}

It is also possible to segment alternative-specific coefficients, and
generate catalogs that provide specifications with or without
segmentation, and with generic or alternative-specific coefficients. This is done using the following syntax:
\begin{lstlisting}
(B_TIME_catalog_dict,) = generic_alt_specific_catalogs(
    generic_name='B_TIME',
    beta_parameters=[B_TIME],
    alternatives=['TRAIN', 'CAR'],
    potential_segmentations=(
        segmentation_purpose,
        segmentation_luggage,
    ),
    maximum_number=1,
)
\end{lstlisting}
where the function \lstinline@generic_alt_specific_catalogs@ is the same as in Section~\ref{sec:alt_spec}, and can be imported as follows:
\begin{lstlisting}
from biogeme.catalog import generic_alt_specific_catalogs
\end{lstlisting}
The function takes five arguments:
\begin{enumerate}
\item a generic name that identifies the catalogs,
\item a list of parameters, defined with \lstinline@Beta@,
\item a tuple containing the names identifying the alternatives,
\item a list of potential segmentations (set to \lstinline@None@ by default),
\item the maximum number of segmentations that can be activated at the same time (set to 5 by default). 
\end{enumerate}
This function creates a dictionary with two catalogs \lstinline@B_TIME_catalog['TRAIN']@ and \lstinline@B_TIME_catalog['CAR']@, synchronized, and therefore controlled by the same controller.
There are six possible configurations:
\begin{lstlisting}
B_TIME:no_seg;B_TIME_gen_altspec:generic
B_TIME:no_seg;B_TIME_gen_altspec:altspec
B_TIME:LUGGAGE;B_TIME_gen_altspec:generic
B_TIME:LUGGAGE;B_TIME_gen_altspec:altspec
B_TIME:COMMUTERS;B_TIME_gen_altspec:generic
B_TIME:COMMUTERS;B_TIME_gen_altspec:altspec
\end{lstlisting}
If we allow to segment the population with two socio-economic characteristics instead of just one, we obtain a total of eight configurations, as the double segmentation can be considered with generic or alternative-specific coefficients:
\begin{lstlisting}
B_TIME:no_seg;B_TIME_gen_altspec:generic
B_TIME:no_seg;B_TIME_gen_altspec:altspec
B_TIME:LUGGAGE;B_TIME_gen_altspec:generic
B_TIME:LUGGAGE;B_TIME_gen_altspec:altspec
B_TIME:COMMUTERS;B_TIME_gen_altspec:generic
B_TIME:COMMUTERS;B_TIME_gen_altspec:altspec
B_TIME:COMMUTERS-LUGGAGE;B_TIME_gen_altspec:generic
B_TIME:COMMUTERS-LUGGAGE;B_TIME_gen_altspec:altspec
\end{lstlisting}

\section{Comparing models}

The use of catalogs generates a great deal of potential
specifications. And we would like to focus of the best ones. One
possibility would be to focus on one criterion, such as the Akaike
Information Criterion (AIC), and decide that the best model is the one
with the lowest AIC. While it is a valid idea, the outcome of the
estimation will be exactly one model. And if, for some reasons, that
model happens not to be acceptable, no other model will be proposed to
the analyst. Instead, we would like to combine several indicators to
identify good models. In particular, we would like to keep models that
fit the data well (that is, associated with a high log likelihood),
and models that are parsimonious (that is, with a low number of
parameters). If we consider those two indicators simultaneously, we need to use the
concept of dominance and Pareto optimality (formally defined in
Appendix~\ref{sec:pareto}). Consider a model $M_1$ with $K_1$ parameters and final log likelihood $\L_1$,  and $M_2$ with $K_2$ parameters and final log likelihood $\L_2$.
We say that $M_1$ dominates $M_2$ if it is no worse than $M_2$ in any objective, and strictly better in at least one objective, that is:
\[
\L_1 \geq \L_2 \text{ and } K_1 < K_2,
\]
or
\[
\L_1 > \L_2 \text{ and } K_1 \leq K_2.
\]
In this context, we will keep only models that are not dominated. Such
models are said to be Pareto optimal.


\section{Estimating parameters using catalogs}\label{sec:estimation}

We illustrate the concept of catalogs by estimating several
specifications. We build on the examples from
Section~\vref{sec:catalogs}.

\subsection{Various choice models}
We consider first a catalog that includes a logit and two nested logit
models, each with a different nest definition. The catalog is constructed as described above:
\begin{lstlisting}
model_catalog = Catalog.from_dict(
    catalog_name='model_catalog',
    dict_of_expressions={
        'logit': logprob_logit,
        'nested existing': logprob_nested_existing,
        'nested public': logprob_nested_public,
    },
)
\end{lstlisting}
and is provided to the Biogeme object:
\begin{lstlisting}
the_biogeme = bio.BIOGEME(database, model_catalog)
\end{lstlisting}
The various specifications can  be estimated using the \lstinline@estimate_catalog@ function:
\begin{lstlisting}
dict_of_results = the_biogeme.estimate_catalog()
\end{lstlisting}
The complete code is available in Appendix~\ref{sec:b01model}. The
output of estimation is a dictionary, where each key is the name of a
model, and each value is an object containing the estimation
results. In this document, we process this dictionary using the code
presented in Appendix~\ref{sec:reporting}. The output of the script
contains two parts. The first part contains the complete set of
results (see Figure~\ref{fig:b01model}). Each column is associated with
a model name, each name being associated with a specification below:
\begin{lstlisting}
Model_000000	model_catalog:nested public
Model_000001	model_catalog:nested existing
Model_000002	model_catalog:logit
\end{lstlisting}
\begin{landscape}
  \begin{figure}[p]
\begin{lstlisting}
A total of 3 models have been estimated
== Estimation results ==
                                   Model_000000     Model_000001     Model_000002
Number of estimated parameters                5                5                4
Sample size                                6768             6768             6768
Final log likelihood               -5331.252007     -5236.900014     -5331.252007
Akaike Information Criterion       10672.504014     10483.800028     10670.504014
Bayesian Information Criterion     10706.603818     10517.899832     10697.783857
ASC_CAR (t-test)                -0.155  (-2.03)  -0.167  (-3.07)  -0.155  (-2.66)
ASC_TRAIN (t-test)              -0.701  (-5.22)  -0.512  (-6.47)  -0.701  (-8.49)
B_COST (t-test)                  -1.08  (-14.4)  -0.857  (-14.3)   -1.08  (-15.9)
B_TIME (t-test)                  -1.28  (-10.5)  -0.899  (-8.39)   -1.28  (-12.3)
MU_public (t-test)                    1  (8.78)
MU_existing (t-test)                                2.05  (12.5)
Model_000000	model_catalog:nested public
Model_000001	model_catalog:nested existing
Model_000002	model_catalog:logit
\end{lstlisting}
\caption{\label{fig:b01model}Different choice models: complete estimation report}
  \end{figure}
  
\end{landscape}

It can be seen that the models \lstinline@model_catalog:nested public@
and \lstinline@model_catalog:logit@ achieve the same final log
likelihood. The nest parameter of the nested logit model is actually
1. Therefore, model \lstinline@model_catalog:nested public@
is dominated by model \lstinline@model_catalog:logit@, and should be rejected. This is how the second part of the output is generated, keeping only non dominated models, as reported in Figure~\vref{fig:b01model_2}. Note that the logit model is better in terms of parsimony, and the nested logit model is better in terms of fit. 


\begin{landscape}
  \begin{figure}[p]
\begin{lstlisting}
                                   Model_000000     Model_000001
Number of estimated parameters                5                4
Sample size                                6768             6768
Final log likelihood               -5236.900014     -5331.252007
Akaike Information Criterion       10483.800028     10670.504014
Bayesian Information Criterion     10517.899832     10697.783857
ASC_CAR (t-test)                -0.167  (-3.07)  -0.155  (-2.66)
ASC_TRAIN (t-test)              -0.512  (-6.47)  -0.701  (-8.49)
B_COST (t-test)                 -0.857  (-14.3)   -1.08  (-15.9)
B_TIME (t-test)                 -0.899  (-8.39)   -1.28  (-12.3)
MU_existing (t-test)               2.05  (12.5)
Model_000000	model_catalog:nested existing
Model_000001	model_catalog:logit
\end{lstlisting}
\caption{\label{fig:b01model_2}Different choice models: Pareto optimal models}
  \end{figure}
\end{landscape}


\subsection{Nonlinear specifications}

We consider a catalog that includes various specifications for the travel time variables:
\begin{itemize}
\item a linear specification,
\item a Box-Cox transform,
\item a power series of degree 3.
\end{itemize}
If $x_t$ is the travel time variable, the catalog contains the following specifications:
\[
  x_t, \;
  \frac{x_t^\lambda - 1}{\lambda}, \text{ and }
  x_t + \beta_\text{square} x_t^2 + \beta_\text{cube} x_t^3.
\]
It can be seen that some of these specifications involve additional
parameters, some not. We use synchronized catalogs, so that the travel
time variable is involved in the same way in all alternatives. The
full specification is available in
Appendix~\ref{sec:b02nonlinear}. The results associated with each of
the three specifications are reported in
Figure~\vref{fig:b02nonlinear}. It is interesting to note that none of
these model is dominated by another one.

\begin{landscape}
  \begin{figure}[p]
\begin{lstlisting}
A total of 3 models have been estimated
== Estimation results ==
                                   Model_000000         Model_000001      Model_000002
Number of estimated parameters                4                    5                 6
Sample size                                6768                 6768              6768
Final log likelihood               -5331.252007         -5292.095411      -5236.262942
Akaike Information Criterion       10670.504014         10594.190822      10484.525883
Bayesian Information Criterion     10697.783857         10628.290626      10525.445649
ASC_CAR (t-test)                -0.155  (-2.66)  -0.00462  (-0.0963)   0.0434  (0.965)
ASC_TRAIN (t-test)              -0.701  (-8.49)      -0.485  (-7.53)    -0.409  (-6.8)
B_COST (t-test)                  -1.08  (-15.9)       -1.08  (-15.9)      -1.11  (-16)
B_TIME (t-test)                  -1.28  (-12.3)       -1.67  (-21.9)    -2.32  (-22.6)
lambda_travel_time (t-test)                              0.51  (6.6)
cube_tt_coef (t-test)                                                 0.000193  (7.38)
square_tt_coef (t-test)                                                -0.105  (-21.2)
Model_000000	train_tt_catalog:linear
Model_000001	train_tt_catalog:boxcox
Model_000002	train_tt_catalog:power
\end{lstlisting}
\caption{\label{fig:b02nonlinear}Nonlinear specifications: complete estimation report}
  \end{figure}
\end{landscape}

\subsection{Alternative-specific coefficients}

We consider a catalog that considers both generic and alternative-specific specifications for both the cost coefficient and the travel time coefficient.  The
full specification is available in
Appendix~\ref{sec:b03alt_spec}. The results associated with each of
the four specifications are reported in
Figure~\vref{fig:b03alt_spec}. Note that the model where the cost coefficient is generic and the time coefficient is alternative-specific is dominated by the model where the cost coefficient is alternative-specific and the time coefficient is generic. Indeed, both models involve 6 parameters, but the latter has a better fit.

\begin{landscape}
  \begin{figure}[p]
    \begin{lstlisting}[basicstyle=\scriptsize]
A total of 4 models have been estimated
== Estimation results ==
                                   Model_000000       Model_000001     Model_000002     Model_000003
Number of estimated parameters                6                  8                6                4
Sample size                                6768               6768             6768             6768
Final log likelihood               -5312.894223       -5075.704346     -5083.499937     -5331.252007
Akaike Information Criterion       10637.788446       10167.408692     10178.999875     10670.504014
Bayesian Information Criterion     10678.708211       10221.968379      10219.91964     10697.783857
ASC_CAR (t-test)                -0.271  (-2.29)    -0.367  (-3.32)  -0.427  (-5.55)  -0.155  (-2.66)
ASC_TRAIN (t-test)              -0.202  (-1.82)  -0.0754  (-0.712)    0.189  (2.06)  -0.701  (-8.49)
B_COST (t-test)                    -1.07  (-16)                                       -1.08  (-15.9)
B_TIME_CAR (t-test)              -1.12  (-10.3)     -1.29  (-7.92)
B_TIME_SM (t-test)               -1.17  (-6.42)     -1.11  (-6.25)
B_TIME_TRAIN (t-test)            -1.57  (-14.4)    -0.889  (-7.51)
B_COST_CAR (t-test)                                -0.786  (-5.27)   -0.939  (-8.1)
B_COST_SM (t-test)                                  -1.12  (-14.2)   -1.09  (-15.5)
B_COST_TRAIN (t-test)                                 -3.08  (-16)   -2.93  (-17.4)
B_TIME (t-test)                                                       -1.12  (-9.3)   -1.28  (-12.3)
Model_000000	B_COST_gen_altspec:generic;B_TIME_gen_altspec:altspec
Model_000001	B_COST_gen_altspec:altspec;B_TIME_gen_altspec:altspec
Model_000002	B_COST_gen_altspec:altspec;B_TIME_gen_altspec:generic
Model_000003	B_COST_gen_altspec:generic;B_TIME_gen_altspec:generic
\end{lstlisting}
\caption{\label{fig:b03alt_spec}Alternative-specific coefficients: complete estimation report}
  \end{figure}
\end{landscape}

\subsection{Segmentations}

We consider a catalog that considers potential segmentations of the parameters. The alternative-specific constants are potentially interacted with the variables GA (identifying if the traveler owns a yearly subscription, with 2 levels) and LUGGAGES (identifying if the traveler is carrying luggages, with 3 levels), or both. The travel time coefficient is potentially interacted with the variables FIRST (identifying if the traveler is traveling first class, with 2 levels) or PURPOSE (identifying if the traveler is a commuter or not, with 2 levels). Maximum one such interaction is allowed.

Therefore, we have 4 specifications for the constants:
\begin{itemize}
\item not segmented,
\item segmented by GA (yearly subscription to public transport),
\item segmented by luggage,
\item segmented both by GA and luggage,
\end{itemize}
and 3 specifications for the time coefficients:
\begin{itemize}
\item not segmented,
\item segmented with first class,
\item segmented with trip purpose,
\end{itemize}
so that we obtain a total of 12 specifications.

The
full specification is available in
Appendix~\ref{sec:b04segmentation}. Among the 12 estimated models, 5 are Pareto optimal. The estimation results are reported in Figure~\vref{fig:b04segmentation}. 

\begin{landscape}
  \begin{figure}[p]
%    \begin{lstlisting}[basicstyle=\fontsize{6}{7}\selectfont]
    \begin{lstlisting}[basicstyle=\scriptsize]
                                    Model_000000     Model_000001     Model_000002     Model_000003     Model_000004
Number of estimated parameters                 6                7               11                5                4
Sample size                                 6768             6768             6768             6768             6768
Final log likelihood                -5050.677696     -4976.118641     -4952.546476     -5234.708233     -5331.252007
Akaike Information Criterion        10113.355391      9966.237282      9927.092951     10479.416466     10670.504014
Bayesian Information Criterion      10154.275157     10013.977009     10002.112521      10513.51627     10697.783857
ASC_CAR (t-test)                 -0.249  (-3.97)  -0.281  (-4.53)  -0.298  (-4.12)  -0.187  (-3.23)  -0.155  (-2.66)
ASC_CAR_GA (t-test)              -0.301  (-1.56)  -0.231  (-1.19)  -0.206  (-1.05)
ASC_TRAIN (t-test)                  -1.28  (-14)   -1.37  (-14.7)   -1.79  (-15.4)  -0.814  (-9.45)  -0.701  (-8.49)
ASC_TRAIN_GA (t-test)               1.97  (22.3)     1.91  (21.5)     1.75  (19.1)
B_COST (t-test)                    -1.1  (-14.8)   -1.26  (-15.3)   -1.25  (-15.3)   -1.23  (-16.6)   -1.08  (-15.9)
B_TIME (t-test)                   -1.18  (-11.3)  -0.621  (-4.46)  -0.622  (-4.42)  -0.647  (-4.69)   -1.28  (-12.3)
B_TIME_1st_class (t-test)                          -0.914  (-8.6)  -0.891  (-8.26)   -1.02  (-9.87)
ASC_CAR_one_lugg (t-test)                                          0.0324  (0.486)
ASC_CAR_several_lugg (t-test)                                      -0.437  (-1.82)
ASC_TRAIN_one_lugg (t-test)                                           0.635  (6.4)
ASC_TRAIN_several_lugg (t-test)                                         0.431  (2)
Model_000000	ASC:GA;B_TIME:no_seg
Model_000001	ASC:GA;B_TIME:FIRST
Model_000002	ASC:GA-LUGGAGE;B_TIME:FIRST
Model_000003	ASC:no_seg;B_TIME:FIRST
Model_000004	ASC:no_seg;B_TIME:no_seg
\end{lstlisting}
\caption{\label{fig:b04segmentation}Segmentation: Pareto optimal models}
  \end{figure}
\end{landscape}

\subsection{Segmentations and alternative-specific coefficients}

We consider a catalog that considers potential segmentations of the parameters as well as alternative-specific coefficients.
We consider 4 specifications for the constants:
\begin{itemize}
\item not segmented,
\item segmented by GA (yearly subscription to public transport),
\item segmented by luggage,
\item segmented both by GA and luggage.
\end{itemize}
We consider 6 specifications for the time coefficients:
\begin{itemize}
\item generic and not segmented,
\item generic and segmented with first class,
\item generic and segmented with trip purpose,
\item alternative-specific and not segmented,
\item alternative-specific and segmented with first class,
\item alternative-specific and segmented with trip purpose.
\end{itemize}
Finally, We consider 2 specifications for the cost coefficients:
\begin{itemize}
\item generic,
\item alternative-specific.
\end{itemize}
In total, we obtain 48 specifications. 
The
full specification is available in
Appendix~\ref{sec:b05alt_spec_segmentation}. Among the 48 estimated models, 8 are Pareto optimal. The estimation results are reported in Figure~\vref{fig:b05alt_spec_segmentation} and Figure~\vref{fig:b05alt_spec_segmentation_2}. 

\begin{landscape}
  \begin{figure}[p]
%    \begin{lstlisting}[basicstyle=\fontsize{6}{7}\selectfont]
    \begin{lstlisting}[basicstyle=\scriptsize]
                                    Model_000000      Model_000001     Model_000002       Model_000003
Number of estimated parameters                 7                17                6                  9
Sample size                                 6768              6768             6768               6768
Final log likelihood                -4976.118641      -4865.971435     -5050.677696        -4945.30006
Akaike Information Criterion         9966.237282        9765.94287     10113.355391         9908.60012
Bayesian Information Criterion      10013.977009       9881.882206     10154.275157        9969.979768
ASC_CAR (t-test)                 -0.281  (-4.53)   -0.446  (-3.68)  -0.249  (-3.97)    -0.662  (-7.79)
ASC_CAR_GA (t-test)              -0.231  (-1.19)  -0.145  (-0.739)  -0.301  (-1.56)  -0.0761  (-0.389)
ASC_TRAIN (t-test)                -1.37  (-14.7)    -1.07  (-6.72)     -1.28  (-14)    -0.938  (-6.76)
ASC_TRAIN_GA (t-test)               1.91  (21.5)      1.26  (8.67)     1.97  (22.3)       1.52  (11.1)
B_COST (t-test)                   -1.26  (-15.3)                      -1.1  (-14.8)                   
B_TIME (t-test)                  -0.621  (-4.46)                     -1.18  (-11.3)     -0.69  (-4.56)
B_TIME_1st_class (t-test)         -0.914  (-8.6)                                       -0.925  (-8.62)
ASC_CAR_one_lugg (t-test)                          0.0264  (0.394)
ASC_CAR_several_lugg (t-test)                      -0.299  (-1.23)
ASC_TRAIN_one_lugg (t-test)                           0.674  (6.7)
ASC_TRAIN_several_lugg (t-test)                       0.495  (2.3)
B_COST_CAR (t-test)                                -0.836  (-5.28)                     -0.848  (-7.25)
B_COST_SM (t-test)                                    -1.15  (-14)                       -1.3  (-16.1)
B_COST_TRAIN (t-test)                               -2.03  (-9.61)                      -1.83  (-10.3)
B_TIME_CAR (t-test)                                 -1.55  (-11.3)                                    
B_TIME_CAR_commuters (t-test)                        0.682  (3.48)                                    
B_TIME_SM (t-test)                                  -1.73  (-15.3)                                    
B_TIME_SM_commuters (t-test)                           1.6  (8.06)                                    
B_TIME_TRAIN (t-test)                               -1.34  (-12.7)                                    
B_TIME_TRAIN_commuters (t-test)                     0.116  (0.848)                                                               
Model_000000	ASC:GA;B_COST_gen_altspec:generic;B_TIME:FIRST;B_TIME_gen_altspec:generic
Model_000001	ASC:GA-LUGGAGE;B_COST_gen_altspec:altspec;B_TIME:COMMUTERS;B_TIME_gen_altspec:altspec
Model_000002	ASC:GA;B_COST_gen_altspec:generic;B_TIME:no_seg;B_TIME_gen_altspec:generic
Model_000003	ASC:GA;B_COST_gen_altspec:altspec;B_TIME:FIRST;B_TIME_gen_altspec:generic
\end{lstlisting}
\caption{\label{fig:b05alt_spec_segmentation}Segmentation and alternative-specific coefficients: Pareto optimal models (part 1)}
  \end{figure}
\end{landscape}

\begin{landscape}
  \begin{figure}[p]
%    \begin{lstlisting}[basicstyle=\fontsize{6}{7}\selectfont]
    \begin{lstlisting}[basicstyle=\scriptsize]
                                    Model_000004     Model_000005     Model_000006      Model_000007
Number of estimated parameters                 4                5               11                13
Sample size                                 6768             6768             6768              6768
Final log likelihood                -5331.252007     -5234.708233     -4928.268572      -4890.815071
Akaike Information Criterion        10670.504014     10479.416466      9878.537145       9807.630143
Bayesian Information Criterion      10697.783857      10513.51627      9953.556715       9896.289635
ASC_CAR (t-test)                 -0.155  (-2.66)  -0.187  (-3.23)  -0.383  (-2.95)   -0.434  (-3.72)
ASC_CAR_GA (t-test)              -0.                               -0.217  (-1.14)  -0.173  (-0.891)
ASC_TRAIN (t-test)                -1701  (-8.49)  -0.814  (-9.45)  -0.965  (-7.29)   -0.593  (-4.28)
ASC_TRAIN_GA (t-test)                                                 2.05  (21.8)       1.38  (9.3)
B_COST (t-test)                   -1.08  (-15.9)   -1.23  (-16.6)     -1.13  (-15)
B_TIME (t-test)                  -0..28  (-12.3)  -0.647  (-4.69)
B_TIME_1st_class (t-test)         -0               -1.02  (-9.87)
ASC_CAR_one_lugg (t-test)           
ASC_CAR_several_lugg (t-test)       
ASC_TRAIN_one_lugg (t-test)         
ASC_TRAIN_several_lugg (t-test)     
B_COST_CAR (t-test)                                                                  -0.845  (-5.37)
B_COST_SM (t-test)                                                                    -1.15  (-14.1)
B_COST_TRAIN (t-test)                                                                 -2.09  (-9.76)
B_TIME_CAR (t-test)                                                  -1.4  (-16.8)    -1.55  (-11.4)
B_TIME_CAR_commuters (t-test)                                        0.699  (3.61)     0.692  (3.54)
B_TIME_SM (t-test)                                                   -1.8  (-16.6)    -1.74  (-15.4)
B_TIME_SM_commuters (t-test)                                          1.66  (8.62)      1.62  (8.15)
B_TIME_TRAIN (t-test)                                               -1.61  (-17.3)    -1.35  (-12.8)
B_TIME_TRAIN_commuters (t-test)                                       0.178  (1.3)     0.13  (0.956)
Model_000004	ASC:no_seg;B_COST_gen_altspec:generic;B_TIME:no_seg;B_TIME_gen_altspec:generic
Model_000005	ASC:no_seg;B_COST_gen_altspec:generic;B_TIME:FIRST;B_TIME_gen_altspec:generic
Model_000006	ASC:GA;B_COST_gen_altspec:generic;B_TIME:COMMUTERS;B_TIME_gen_altspec:altspec
Model_000007	ASC:GA;B_COST_gen_altspec:altspec;B_TIME:COMMUTERS;B_TIME_gen_altspec:altspec
\end{lstlisting}
\caption{\label{fig:b05alt_spec_segmentation_2}Segmentation and alternative-specific coefficients: Pareto optimal models (part 2)}
  \end{figure}
\end{landscape}


\subsection{Combining several specifications}

We consider now a combination of the various specifications considered so far:
\begin{itemize}
\item  3 models:
  \begin{itemize}
    \item logit,
    \item  nested logit with two nests: public and private transportation,
    \item  nested logit with two nests existing and future modes,
  \end{itemize}
\item 3 functional forms for the travel time variables:
  \begin{itemize}
    \item linear specification,
    \item  Box-Cox transform,
    \item power series,
  \end{itemize}
\item 2 specifications for the cost coefficients:
  \begin{itemize}
    \item generic,
    \item  alternative-specific,
  \end{itemize}
\item  2 specification for the travel time coefficients:
  \begin{itemize}
    \item generic,
    \item alternative-specific,
  \end{itemize}
\item 4 segmentations for the constants:
  \begin{itemize}
    \item not segmented,
    \item  segmented by GA (yearly subscription to public transport),
    \item segmented by luggage,
    \item segmented both by GA and luggage,
  \end{itemize}
\item  3 segmentations for the time coefficients:
  \begin{itemize}
    \item not segmented,
    \item segmented with first class,
    \item segmented with trip purpose.
  \end{itemize}
\end{itemize}
This leads to a total of 432 specifications. The script with the specification is available in
Appendix~\ref{sec:everything_spec}. If it is attempted to estimate all specifications of this catalog, the following exception will be raised:
\begin{lstlisting}
There are too many [432] different specifications for the log likelihood function. This is above the maximum number: 100. Simplify the specification, change the value of the parameter maximum_number_catalog_expressions, or consider using the AssistedSpecification object in the "biogeme.assisted" module.
\end{lstlisting}

\section{Assisted specification}

When the systematic estimation of all possible specifications is
infeasible, it is possible to rely on the assisted specification
algorithm, inspired by the work of
\citeasnoun{OrteHillCamadeLaBier21}.

This is done by first creating the object, using the following syntax:
\begin{lstlisting}
assisted_specification = AssistedSpecification(
    biogeme_object=the_biogeme,
    multi_objectives=loglikelihood_dimension,
    pareto_file_name=PARETO_FILE_NAME,
)
\end{lstlisting}
where the class \lstinline@AssistedSpecification@ must be imported  as follows:
\begin{lstlisting}
from biogeme.assisted import AssistedSpecification
\end{lstlisting}
Its constructor takes three arguments:
\begin{enumerate}
\item the biogeme object,
\item a function providing all the indicators used to exclude dominated models,
\item the name of a file that will collect all the models that have been estimated,
\item a function verifying the validity of the results (optional).
\end{enumerate}

The biogeme object is constructed as before, from the database and the catalog:
\begin{lstlisting}
the_biogeme = bio.BIOGEME(database, model_catalog)
\end{lstlisting}
The function must take the estimation results as argument, and return
a list of indicators. The convention is that, the lower the value of
the indicator, the better the model.  Here is an example of such a function:
\begin{lstlisting}
def loglikelihood_dimension(results):
    """Function returning the negative log likelihood and the number
    of parameters, designed for multi-objective optimization

    :param results: estimation results
    :type results: biogeme.results.bioResults
    """
    return [-results.data.logLike, results.data.nparam]
\end{lstlisting}
The two indicators in this case are
\begin{itemize}
\item the opposite of the final log likelihood (opposite, because of the above mentioned convention),
  \item the number of estimated parameters.
\end{itemize}
Another example involving three indicators is as follows:
\begin{lstlisting}
def AIC_BIC_dimension(results):
    """Function returning the AIC, BIC and the number
    of parameters, designed for multi-objective optimization

    :param results: estimation results
    :type results: biogeme.results.bioResults
    """
    return [results.data.akaike, results.data.bayesian, results.data.nparam]
\end{lstlisting}
The three indicators are the Akaike Information Criterion (AIC), the Bayesian Information Criterion (BIC) and the number of estimated parameters. Those two examples can actually be directly imported  from biogeme:
\begin{lstlisting}
from biogeme.multiobjectives import loglikelihood_dimension, AIC_BIC_dimension
\end{lstlisting}

The ``pareto file'' is the memory of the process. It stores all models that have been estimated by the algorithm, together with the relevant indicators. It is organized in three sections:
\begin{enumerate}
\item The \lstinline@[Pareto]@ section contains all models that are not dominated.
\item The \lstinline@[Considered]@ section contains all models that have been estimated.
\item The \lstinline@[Removed]@ section contains all models that have been Pareto optimal at some point during the algorithm, but that have been rejected by a dominating model.
\item The \lstinline@[Invalid]@ section contains all models that have been identified as invalid.
\end{enumerate}
If the file exists when the algorithm is started, its content is used to initialize the algorithm. This allows to interrupt the algorithm and to relaunch it without losing what has been found so far.

Like the function calculating the indicators, the function verifying the validity of the results also takes estimation results as argument, and returns a tuple with two values:
\begin{enumerate}
\item a boolean that is True if the results are valid, and False otherwise,
\item a string explaining why the results are invalid, or None if they are valid.
\end{enumerate}
Here is an example of such a function, where the results are reported invalid if any coefficient of time or cost is non negative:
\begin{lstlisting}
def validity(results):
    """Function verifying that the estimation results are valid.

    The results are not valid if any of the time or cost coefficient is non negative.
    """
    for beta in results.data.betas:
        if 'TIME' in beta.name and beta.value >= 0:
            return False, f'{beta.name} = {beta.value}'
        if 'COST' in beta.name and beta.value >= 0:
            return False, f'{beta.name} = {beta.value}'
    return True, None
\end{lstlisting}

The algorithm is executed using the following statement:
\begin{lstlisting}
non_dominated_models = assisted_specification.run()
\end{lstlisting}
Similarly to the \lstinline@estimate_catalog@ function, it returns a dictionary with all Pareto optimal models. The code is reported in Appendix~\ref{sec:b07everything_assisted}. 

Before looking at the results in the next section, we note that the
concept of ``valid'' models can be dealt with in several ways. In
particular, the sign of a coefficient can be constrained using the
bounds appearing in the definition of the Beta expression. For
instance, if the time and cost coefficients are constrained to be non
positive, all models will be ``valid'' by design, and the above
function will always return ``True''. This may be a good alternative if
there is a high rate of rejected invalid models, that may decrease the capability
of the algorithm to explore the space of possible specifications.


\section{Using the Pareto file}

As mentioned above, the Pareto file contains the description of all
models that have been estimated by the algorithm, as well as the
requested indicators. In this Section, we describe some
post-processing methods that allow to exploit it.

\subsection{Selecting one model}

Each model in the file is characterized by an ID. For instance:
\begin{lstlisting}
SPEC_ID = (
    'ASC:GA-LUGGAGE;'
    'B_COST_gen_altspec:generic;'
    'B_TIME:FIRST;'
    'B_TIME_gen_altspec:generic;'
    'model_catalog:logit;'
    'train_tt_catalog:power'
)
\end{lstlisting}
corresponds to a model where
\begin{itemize}
\item the constants are segmented both by GA and luggages,
\item the cost coefficient is generic,
\item the time coefficient is segmented by first class,
\item the time coefficient is generic,
\item the model is logit,
  \item the travel time variable is transformed using a power series.
\end{itemize}

The Biogeme object corresponding to this specification can be obtained using the following constructor:
\begin{lstlisting}
the_biogeme = bio.BIOGEME.from_configuration(
    config_id=SPEC_ID,
    expression=model_catalog,
    database=database,
)
\end{lstlisting}

It can  be used, either for re-estimation, or for applications.

\subsection{Post processing}

The post processing object accepts as input the Biogeme object as well as the Pareto file:
\begin{lstlisting}
post_processing = ParetoPostProcessing(
    biogeme_object=the_biogeme, pareto_file_name=PARETO_FILE_NAME
)
\end{lstlisting}
where the class itself is imported as follows:
\begin{lstlisting}
from biogeme.assisted import ParetoPostProcessing
\end{lstlisting}

The main purpose of this object is to re-estimate all models that are Pareto optimal. This can be done using the statement:
\begin{lstlisting}
post_processing.reestimate(recycle=True)
\end{lstlisting}
The option ``recycle=True'' does not re-estimate a model if the pickle
file is already present. Instead, it reads the results from this
file. This may be useful when you interrupt the process. The next time
you run it, it does not need to re-estimate the models that have
already been processed. If you set it to False, the models are re-estimated, irrespectively of the presence of the pickle file. Note that no output file is overwritten. If an HTML file or a pickle file for a model already exist, a version number is inserted in the name of the file. For instance, if \lstinline@my_model.html@ already exists, the results will be saved in the file \lstinline@my_model~00.html@.

Finally, it is possible to obtain an illustration of the amount of
models that have been estimated by the algorithm and saved in the
Pareto file. This can be done using the following statements:
\begin{lstlisting}
_ = post_processing.plot(
    label_x='Negative log likelihood',
    label_y='Nbr of parameters',
)
    plt.show()
\end{lstlisting}

It generates a figure with two axes, corresponding to two objectives. Each model is represented by a point with coordinates calculated using the corresponding objectives. The shape of the point represents the status of the model:
\begin{itemize}
\item A circle represents a Pareto optimal model.
\item A cross represents a model that has been Pareto optimal at some point during the course of the algorithm, and later dominated by another model.
\item A star represents a model that has been deemed invalid.
\item A small dot represents all other models that have been considered.
\end{itemize}
An example of this illustration is available in Figure~\vref{fig:pareto}.

Note that, when more than two objectives have been used by the algorithm, the first two are used by default for the plot. But other objectives can be selected using the parameters \lstinline@objective_x@ and \lstinline@objective_y@. This can also be used to swap the position of the axes, as illustrated by the following statement, that generates the picture in Figure~\vref{fig:pareto_2}:
\begin{lstlisting}
    _ = post_processing.plot(
        label_x='Nbr of parameters',
        label_y='Negative log likelihood',
        objective_x=1,
        objective_y=0,
    )
\end{lstlisting}

\begin{figure}
  \begin{center}
    \epsfig{figure=pareto, width=0.7\textwidth}
  \end{center}
  \caption{\label{fig:pareto}Models in the Pareto file}
\end{figure}

\begin{figure}
  \begin{center}
    \epsfig{figure=pareto_2, width=0.7\textwidth}
  \end{center}
  \caption{\label{fig:pareto_2}Models in the Pareto file (swapped axes)}
\end{figure}

\section{Conclusion}

This report describes several functionalities of Biogeme that happened
to be useful to the authors in the context of model development. It is
important to emphasize that they are not designed to replace the
analyst and the modeler. Instead, they are designed to assist her, in
order to facilitate the investigation of many possible specifications.

These features are experimental, and are likely to be improved in the future. 

\bibliographystyle{dcu}
\bibliography{../dca}

\clearpage
\appendix
\renewcommand{\thesection}{\arabic{section}}

\section*{Appendix}

\section{Dominance and Pareto optimality}\label{sec:pareto}

We consider a vector $x \in \R^n$, which is associated with $P$
indicators: $f_1(x)$, \ldots, $f_P(x)$. Each of these indicators is such
that lower values are better than higher values. As there are multiple
indicators, it is not necessarily straightforward to decide which
between two vectors $x$ and $y$ is better, as one can be better for
some indicators, and the other one for other indicators. In order to formalize this, we introduce the concept of dominance.

Consider two vectors $x, y \in \R^n$. We say that $x$ is \textbf{dominating} $y$, and use the  notation $x \prec y$, if
\begin{enumerate}
\item $x$ is no worse in any objective
  \[
  \forall i \in \{1, \ldots, P\}, f_i(x) \leq f_i(y),
  \]
\item $x$ is strictly better in at least one objective
  \[
  \exists i \in \{1, \ldots, P\}, f_i(x) < f_i(y).
  \]
\end{enumerate}
The dominance relation has the following properties:
\begin{itemize}
\item Not reflexive: $x \nprec x$.
\item Not symmetric: $x \prec y \not\Rightarrow y \prec x$.
\item Instead: $x \prec y \Rightarrow y \nprec x$.
\item Transitive: $x \prec y$ and $y \prec z$ $\Rightarrow x
  \prec z$.
  \item Not complete: $\exists x, y$: $x \nprec y$ and $y \nprec x$.
\end{itemize}

Consider now a set $\mathcal{F} \subseteq \R^n$. The vector $x^* \in
\mathcal{F}$ is \textbf{Pareto optimal} if it is not dominated by any
solution in $\mathcal{F}$:
\[
\nexists x \in \mathcal{F} \text{ such that } x \prec x^*.
\]
Intuitively, $x^*$ is Pareto optimal if no objective can be improved without
degrading at least one of the others.

As the relation is not complete, there may be more than one Pareto optimal solution in a
set. The \text{Pareto optimal set} is defined as
\[
P^* = \{ x^* \in \mathcal{F} | \nexists x \in \mathcal{F}: x \prec x^* \}.
\]




\section{Function printing the configurations of an expression}\label{sec:print}
\begin{lstlisting}
def print_all_configurations(expression: Expression) -> None:
    """Prints all configurations that an expression can take
    """
    expression.set_central_controller()
    total = expression.central_controller.number_of_configurations()
    print(f'Total: {total} configurations')
    for config_id in expression.central_controller.all_configurations_ids:
        print(config_id)
\end{lstlisting}

\section{Illustrations of the catalogs}\label{sec:simple}
This is the code used to generate the examples in Section~\ref{sec:catalogs}.

\lstinputlisting[style=numbers]{\examplesPath/assisted/simple_example.py}

\section{Data}\label{sec:data}

\lstinputlisting[style=numbers]{\examplesPath/assisted/swissmetro_data.py}

\section{Reporting}\label{sec:reporting}

\lstinputlisting[style=numbers]{\examplesPath/assisted/results_analysis.py}

\section{Estimation of a catalog with two models}\label{sec:b01model}

\lstinputlisting[style=numbers]{\examplesPath/assisted/b01model.py}

\section{Estimation of a catalog with nonlinear specifications}\label{sec:b02nonlinear}

\lstinputlisting[style=numbers]{\examplesPath/assisted/b02nonlinear.py}

\section{Estimation of a catalog with alternative-specific coefficients}\label{sec:b03alt_spec}

\lstinputlisting[style=numbers]{\examplesPath/assisted/b03alt_spec.py}

\section{Estimation of a catalog with segmentations}\label{sec:b04segmentation}

\lstinputlisting[style=numbers]{\examplesPath/assisted/b04segmentation.py}

\section{Estimation of a catalog with segmentations and alternative-specific coefficients}\label{sec:b05alt_spec_segmentation}

\lstinputlisting[style=numbers]{\examplesPath/assisted/b05alt_spec_segmentation.py}

\section{Specification of a catalog with 432 configurations}\label{sec:everything_spec}

\lstinputlisting[style=numbers]{\examplesPath/assisted/everything_spec.py}

\section{Assisted Specification of a catalog with 432 configurations}\label{sec:b07everything_assisted}

\lstinputlisting[style=numbers]{\examplesPath/assisted/b07everything_assisted.py}

\section{Postprocessing}\label{sec:b09post_processing}

\lstinputlisting[style=numbers]{\examplesPath/assisted/b09post_processing.py}



\end{document}
